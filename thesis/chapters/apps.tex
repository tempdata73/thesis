\appendix

\chapter{Algoritmo de Ramificación y Acotamiento}
\label{app:bb}

\begin{algorithm}[ht]
	\LinesNumbered
	\KwData{
		Problema de maximización lineal $S_0$.
		}
	\KwResult{
		Solución óptima entera $\vec{x}^*$ y valor óptimo $\optilp{z}$.
	}
	\Begin{
		$\mathcal{L} \leftarrow \braces{S_0}$\;
		$\vec{x}^* \leftarrow -\vec{\infty}$\;
		$\optilp{z} \leftarrow -\infty$\;
		\While{$\mathcal{L} \neq \emptyset$}{
			elegir de $\mathcal{L}$ subproblema $S_i$\; \nllabel{p1c9:alg:BB_loop}
			obtener de $S_i$ valor óptimo $z^*_i$ y solución óptima $\vec{x}^i$\;
			$\mathcal{L} \leftarrow \mathcal{L} \setminus \braces{S_i}$\;
			\If{$S_i = \emptyset$ o $z^*_i \leq \optilp{z}$}{
				ir al paso \ref{p1c9:alg:BB_loop}\;
			}
			\If{$\vec{x}^i \in \Z^n$}{
				$\vec{x}^* \leftarrow \vec{x}^i$\;
				$\optilp{z} \leftarrow z^*_i$\;
				ir al paso \ref{p1c9:alg:BB_loop}\;
			}
		    elegir $x^i_j \not \in \Z$ y generar subproblemas $S_{i0}$ y $S_{i1}$ con
			regiones factibles
			$S_{i} \cup \braces{x_j \leq \floor{x^i_j}}$ y
			$S_{i} \cup \braces{x_j \geq \ceil{x^i_j}}$, respectivamente\;
			$\mathcal{L} \leftarrow \mathcal{L} \cup \braces{S_{i1}, S_{i2}}$.
		}
		\Return{$(\vec{x}^*, \optilp{z})$}
	}
	\caption{Ramificación y Acotamiento (adaptado de \cite{fabs})} \label{p1c9:alg:BB}
	\label{algo:bb}
\end{algorithm}

\chapter{Algoritmo extendido del caso infinito}
\label{app:inf:ext}

% TODO: implement algorithm
\begin{algorithm}[ht]
	\LinesNumbered
	\SetKwProg{Fn}{Fn}{\string:}{}
	\SetKwFunction{switch}{switch}
	\SetKwFunction{NonNegativeIntSol}{NonNegativeIntSol}
	\SetKwFunction{FindNegEntry}{FindNegEntry}
	\SetKwFunction{Length}{Length}
	\SetKwFunction{Dioph}{Dioph}
	\Fn{\Dioph{$\vec{q}$, $\eta$}}{
		\KwData{\\
			Vector coprimo $\vec{q}$ tal que $q_i < 0$ para alguna $i \in \braces{1, \ldots, n}$. \\
			Lado derecho $\eta$.
			}
		\KwResult{\\
			Solución entera no negativa $\vec{x}$ a la ecuación lineal diofantina $\vec{q}^T\vec{x} =
			\eta$.
		}
		$\vec{x} \leftarrow \vec{0}$\;
		$\vec{\sigma} \leftarrow \left(i \colon q_i \neq 0\right)$\;
		$\vec{\tilde{q}} \leftarrow \left( q_i \colon q_i \neq 0 \right)$\;

		$m \leftarrow$ \Length{$\vec{\tilde{q}}$}\;
		$j \leftarrow m$\;
		\If{$\tilde{q}_m < 0$ y $\tilde{q}_{m-1} > 0$}{
			$j \leftarrow m - 1$\;
		}
		\ElseIf{$\tilde{q}_m < 0$ y $\tilde{q}_{m-1} < 0$}{
			$j \leftarrow 1$\;
		}
		\ElseIf{$\tilde{q}_m > 0$ y $\tilde{q}_{m-1} > 0$}{
			$j \leftarrow$ \FindNegEntry{$\vec{\tilde{q}}$}\;
		}
		\switch{$\vec{\tilde{q}}$, $j$, $m$}\;
		$\vec{\tilde{x}} \leftarrow$ \NonNegativeIntSol{$\vec{\tilde{q}}$, $\eta$}\;
		\switch{$\vec{\tilde{x}}$, $j$, $m$}\;

		\For{$i \leftarrow 1$ \KwTo $m$}{
			$x_{\sigma_i} \leftarrow \tilde{x}_i$\;
		}
		\Return{$\vec{x}$}
	}
	\caption{Algoritmo para obtener soluciones enteras no negativas a la ecuación lineal diofantina
	$\vec{q}^T\vec{x} = \eta$ para el caso infinito.}
	\label{algo:inf:ext}
\end{algorithm}

\begin{theorem}
	El Algoritmo \ref{algo:inf:ext} es correcto.
\end{theorem}
\begin{proof}
\end{proof}
