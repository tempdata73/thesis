\chapter*{Introducción}
\noindent
Esta tesis analiza a profundidad el programa lineal entero
\begin{subequations}
	\label{eq:base}
	\begin{align}
		\max_{\vec{x} \in \Z^n} \quad
			& \vec{p}^T\vec{x}, \label{eq:base:1} \\
		\text{s.a.} \quad
			& \vec{p}^T\vec{x} \leq u, \label{eq:base:2} \\
			& \vec{x} \geq \vec{0}. \nonumber
	\end{align}
\end{subequations}
donde $\vec{p} \in \R^n \setminus \braces{\vec{0}}$ pertenece a una clase de vectores que
definiremos en el primer capítulo, y $u \in \R$ es un escalar. Son dos las razones por las que nos
dedicamos a estudiar casi exclusivamente este problema: en primer lugar, el hecho de que
tenga una sola restricción facilita su análisis y su interpretación geométrica; y, en segundo
lugar, las simetrías introducidas por el hecho de que el vector $\vec{p}$ define tanto la función
objetivo \eqref{eq:base:1} como la restricción \eqref{eq:base:2} causan ineficiencias en el
método de R\&A al momento de resolver el problema. De esta manera, el programa
lineal entero \eqref{eq:base} es una especie de ``ejemplo minimal'' que motiva el diseño de
algoritmos alternativos a Ramificación y Acotamiento.

% Se ha observado que esto último ocurre siempre que el vector objetivo es ortogonal a una de las
% restricciones activas del programa lineal entero. En este caso, el programa relajado cuenta con una
% infinidad de soluciones, por lo que todo subproblema tendrá al menos una solución, lo cual implica
% que las políticas resultan ser ineficientes. La instancia más simple que exhibe estas ineficiencias
% en las políticas de poda es En este caso, la restricción $\vec{p}^T\vec{x} \leq u$ es ortogonal al
% vector objetivo.

% De manera resumida, mostramos que existe una equivalencia entre resolver problemas del tipo
% \eqref{eq:base} y resolver ecuaciones lineales diofantinas en $n$ incógnitas. Los coeficientes de
% las ecuaciones lineales diofantinas estarán dadas por las entradas de un vector $\vec{q}$ definido a
% partir de $\vec{p}$. Así también, el número de ecuaciones que deberemos resolver depende, en gran
% medida, de los signos en las entradas de $\vec{q}$. El teorema \ref{theory:th:feasibility} muestra
% que si alguna entrada $q_i$ es negativa, entonces es necesario resolver una sola ecuación y, en caso
% de que todas las entradas de $\vec{q}$ sean positivas, el número de ecuaciones a resolver es finito.

En el capítulo 1 se presentan los prerrequisitos necesarios para obtener los resultados que se
encuentran a lo largo de esta tesis. En particular, se define una clase de vectores a la cual
supondremos que el vector objetivo $\vec{p}$ pertenece, y también se obtienen varias de sus
propiedades. Esta clase de vectores contiene cualquier vector representable en aritmética finita por
lo que, en la práctica, este supuesto es razonable. Entre los resultados originales del autor que
más destacan en esta parte de la tesis son el teorema \ref{theory:th:feasibility}, el cual separa en
dos subclases las instancias de \eqref{eq:base} y que son tratadas respectivamente en los capítulos
2 y 3; y el teorema \ref{th:lattice}, el cual da inicio a una clasificación de programas lineales
enteros.

En el capítulo 2 se analiza el caso en el que dos entradas del vector $\vec{p}$ tienen signos
distintos. Bajo esta hipótesis adicional, la solución del problema \eqref{eq:base} se obtiene al
resolver una sola ecuación lineal diofantina. Por un lado, mostramos que el valor objetivo del
problema \eqref{eq:base} se puede determinar de manera inmediata sin tener conocimiento de la
solución óptima. Por el otro lado, presentamos un algoritmo que construye la solución óptima y cuya
complejidad es polinomialmente acotada en la dimensión del vector $\vec{p}$. Finalmente, realizamos
una serie de experimentos numéricos que permiten comparar los tiempos de terminación de este nuevo
algoritmo con los de Ramificación y Acotamiento.

En el capítulo 3 se analiza el caso en el que todas las entradas del vector $\vec{p}$ tienen el mismo
signo. Bajo esta hipótesis solamente podemos asegurar la finitud del número de ecuaciones lineales
diofantinas que debemos resolver para encontrar la solución de \eqref{eq:base}. No obstante,
mostramos que si el lado derecho de la restricción \eqref{eq:base:2} es suficientemente grande,
entonces sí basta con resolver una sola ecuación lineal diofantina para obtener el óptimo. Además,
presentamos un algoritmo que construye la solución óptima de \eqref{eq:base}. Es en este capítulo
que, de manera simultánea, encontramos cotas superiores para el número de Frobenius mencionado en la
motivación de esta tesis. Al igual que en el capítulo 2, se realizan experimentos numéricos
que permiten comparar los tiempos de terminación de este nuevo algoritmo con los de Ramificación y
Acotamiento.

En el capítulo 4 se introduce el caso de múltiples restricciones, y resulta ser que la división en
casos del teorema \ref{theory:th:feasibility} deja de ser vigente. Por lo tanto, se desarrolla un
nuevo método que permite resolver este tipo de problemas más generales bajo la perspectiva de
sistemas de ecuaciones lineales diofantinas. Puesto que un análisis detallado sería demasiado
extenso para añadirlo a esta tesis de licenciatura, la discusión en este capítulo es más
superficial, pero no por ello menos formal.

Finalmente, en el capítulo 5 se presenta una recopilación de los resultados originales y más
destacables obtenidos a lo largo de esta tesis. Así también, se presenta una recopilación de
problemas mencionados en esta tesis pero que no fueron tratados, bien porque eran tangenciales al
objetivo central, bien porque sus respectivos análisis serían demasiado extensos para ser añadidos.
Ciertamente, cada uno de estos problemas sirve como directriz inicial para la realización de futuras
investigaciones.
