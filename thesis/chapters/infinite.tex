\chapter{El caso infinito}

% TODO: resumen del capítulo

\noindent
Recordemos del Teorema \ref{theory:th:feasibility} que si $\vec{q}_i < 0$ para alguna $i \in \lbrace
2, \ldots, n \rbrace$, entonces la $\eta$-ésima capa entera contiene un número infinito de puntos
factibles. A partir de esto somos capaces de resolver el automáticamente el problema de decisión de
determinar si un escalar $k^*$ es el valor óptimo del programa (\ref{theory:formulation}).
\begin{corollary}
	Si $\vec{q}_i < 0$ para algún $i \in \lbrace 2, \ldots, n \rbrace$, entonces el valor óptimo del
	programa (\ref{theory:formulation}) es $m\eta$. Además, si $m > 0$, entonces  $\eta$ es el
	múltiplo de $m$ más grande que satisface $m\eta \leq u$.
\end{corollary}
\begin{proof}
	Por el Teorema \ref{theory:th:feasibility} sabemos que existen una infinidad de soluciones en la
	$\eta$-ésima capa entera, así que sea $\vec{x}^*$ una de ellas. Entonces $\vec{q}^T\vec{x}^* =
	\eta$, pero $\vec{p} = m\vec{q}$, por lo que obtenemos $\vec{p}^T\vec{x}^* = m\eta$.

	Ahora bien, recordemos que $\eta = \lfloor u/m \rfloor$ si $m > 0$ por el Lema
	\ref{phase-1:lemma:eta}. Supongamos que $\xi \in \Z$ satisface $m\xi \leq u$ y también $\lfloor
	u/m \rfloor < \xi$. Luego,
	\begin{equation*}
		m\left\lfloor \frac{u}{m} \right\rfloor < m\xi \leq u
		\implies \left\lfloor \frac{u}{m} \right\rfloor < \xi \leq \frac{u}{m},
	\end{equation*}
	pero esto contradice las propiedades de la función piso.
\end{proof}

\begin{observation}
	En el capítulo anterior mencionamos que siempre supondremos que $m > 0$, es decir, que la primera
	entrada de nuestro vector esencialmente entero $\vec{p}_1$ es no negativo. Sin embargo, en caso de
	que $m < 0$, es posible demostrar también que $\eta \coloneq \lceil u/m \rceil$ es ahora el múltiplo
	más chico de $m$ que satisface $m\eta \geq u$. Este es uno de los muchos casos en los que las
	desigualdades se invierten y la función piso se reemplaza por la función techo en caso de que $m$
	sea negativo.
\end{observation}

Una vez resuelto el problema de decisión, podemos preguntarnos concretamente cómo obtener el punto
óptimo. Del capítulo anterior sabemos que debemos resolver la ecuación lineal diofantina
$\vec{q}^T\vec{x} = \eta$. Pero
\begin{equation*}
	\vec{q}^T\vec{x} = \eta\vec{q}^T\vec{\omega} + \vec{q}^TM\vec{t} = \eta,
\end{equation*}
para toda $\vec{t} \in \Z^{n-1}$. Así que debemos encontrar condiciones suficientes en $\vec{t}$
para asegurar la no negatividad de $\vec{x}$. Recordemos que $\vec{t}_i$ debe satisfacer
(\ref{eq:param-lb}). En términos del vector $\vec{\omega}$, tenemos
\begin{equation}
	\label{eq:vect-param-lb}
	\vec{t}_i \geq \left\lceil -\frac{\vec{\omega}_i}{g_{i+1}} \right\rceil,
\end{equation}
para todo $i \in \lbrace 2, \ldots, n - 2\rbrace$.

Ahora bien, recuperamos de (\ref{eq:last-solution}) que las últimas dos soluciones de la ecuación
$\vec{q}^T\vec{x}$ están dadas por
\begin{equation}
	\begin{cases}
		\vec{x}_{n-1} = \omega_{n-1}x_{n-1}' + \frac{\vec{q}_n}{\prod_{j=1}^{n-1}g_j}\vec{t}_{n-1}, \\
		\vec{x}_n = \omega_{n-1}x_n' - \frac{\vec{q}_{n-1}}{\prod_{j=1}^{n-1}g_j}\vec{t}_{n-1},
	\end{cases}
\end{equation}
Para que se satisfagan las condiciones de no negatividad de $\vec{x}_{n-1}$ y de $\vec{x}_n$,
encontramos que la variable libre $\vec{t}_{n-1} \in \Z$ debe cumplir ciertas desigualdades según los
signos de $\vec{q}_{n-1}$ y de $\vec{q}_n$. Definamos, por conveniencia,
\begin{subequations}
	\label{eq:lr-bounds}
	\begin{align}
		b_1 &\coloneq -\frac{\omega_{n-1}x_{n-1}'}{\vec{q}_n} \cdot \prod_{j=1}^{n-1}g_j
		= -\frac{\vec{\omega}_{n-1}}{\vec{q}_n} \cdot \prod_{j=1}^{n-1}g_j, \\
		b_2 &\coloneq \frac{\omega_{n-1}x_{n}'}{\vec{q}_{n-1}} \cdot \prod_{j=1}^{n-1}g_j
		= \frac{\vec{\omega}_{n}}{\vec{q}_n} \cdot \prod_{j=1}^{n-1}g_j,
	\end{align}
\end{subequations}
Entonces se verifica que
\begin{equation}
	\label{eq:feasible-param}
	t_{n-1} \in 
	\begin{cases}
		\big[ \lceil b_1 \rceil, \lfloor b_2 \rfloor \big] &  0 < \vec{q}_{n-1}, \vec{q}_n, \\
		\big[ \lceil b_2 \rceil, \lfloor b_1 \rfloor \big] &  \vec{q}_{n-1}, \vec{q}_n < 0, \\
		\big[ \lceil \max\lbrace b_1 ,  b_2 \rbrace \rceil, \infty \big) &  \vec{q}_{n-1}
		< 0 < \vec{q}_n, \\
		\big( -\infty, \lfloor \min\lbrace b_1, b_2\rbrace \rfloor \big] &  \vec{q}_n < 0
		< \vec{q}_{n-1}.
	\end{cases}
\end{equation}

\begin{lemma}
	\label{lemma:t-existence}
	Existe un vector $\vec{t} \in \Z^{n-1}$ que satisface ambos (\ref{eq:vect-param-lb}) y
	(\ref{eq:feasible-param}).
\end{lemma}
\begin{proof}
	Tenemos cuatro casos, pero observemos que los dos en donde $\vec{q}_{n - 1}$ y $\vec{q}_n$
	tienen signo distinto no son difíciles: si $\vec{q}_{n - 1} <0 < \vec{q}_n$, entonces el vector
	$\vec{t} \in \Z^{n-1}$ dado por
	\begin{equation*}
		\vec{t}_i \coloneq \begin{cases}
			\left\lceil -\frac{\vec{\omega}_i}{g_{i + 1}} \right\rceil, & i < n - 1, \\
			\lceil \max\lbrace b_1, b_2 \rbrace \rceil, & i = n - 1,
		\end{cases}
	\end{equation*}
	satisface ambos (\ref{eq:vect-param-lb}) y (\ref{eq:feasible-param}). El caso $\vec{q}_n < 0 <
	\vec{q}_{n - 1}$ es completamente similar.

	Ahora bien, supongamos que $0 < \vec{q}_{n - 1}, \vec{q}_n$. Podemos suponer sin pérdida de
	generalidad que $\vec{q}_{n - 2} < 0$. En efecto, como $\vec{q}_i < 0$ para alguna $i \in
	\lbrace 2, \ldots, n - 2\rbrace$, somos capaces permutar las entradas $i$ y $n - 2$ de $\vec{q}$ en
	el problema (\ref{theory:formulation}). Observemos que
	\begin{align*}
		b_2 - 1 &\leq \lfloor b_2 \rfloor \leq b_2, \\
		b_1 &\leq \lceil b_1 \rceil \leq b_1 + 1.
	\end{align*}
	De donde obtenemos
	\begin{equation*}
		b_2 - b_1 - 2 \leq \lfloor b_2 \rfloor - \lceil b_1 \rceil \leq b_2 - b_1.
	\end{equation*}
	Así pues, para que el intervalo $[\lceil b_1 \rceil, \lfloor b_2 \rfloor]$ esté bien definido,
	es suficiente con mostrar que existe un escalar $\omega_{n - 1}$ que satisfaga $b_2 - b_1 \geq
	2$. Tenemos
	\begin{equation}
		\label{proof:b-sub}
		b_2 - b_1 = \omega_{n - 1}\prod_{j = 1}^{n-1}g_j \cdot
			\left(\frac{x_{n-1}'}{\vec{q}_n} + \frac{x_n'}{\vec{q}_{n - 1}}\right)
	\end{equation}
	Como $x_{n - 1}'$ y $x_n'$ son coeficientes de Bézout asociados a los dos coeficientes en
	(\ref{eq:last-equation}) que son coprimos, se cumple
	\begin{equation*}
		\frac{\vec{q}_{n - 1}}{\prod_{j = 1}^{n-1}g_j}x_{n-1}' +
		\frac{\vec{q}_{n}}{\prod_{j = 1}^{n-1}g_j}x_{n}' = 1,
	\end{equation*}
	lo que implica que
	\begin{equation*}
		\frac{x_{n-1}'}{\vec{q}_n} + \frac{x_n'}{\vec{q}_{n - 1}} = \frac{\prod_{j =
		1}^{n-1}g_j}{\vec{q}_{n-1}\vec{q}_n}.
	\end{equation*}
	Sustituyendo en (\ref{proof:b-sub}),
	\begin{equation}
		\label{proof:omega-sub}
		b_2 - b_1 = \omega_{n-1}\cdot \frac{\prod_{j=1}^{n-1}g_j^2}{\vec{q}_{n-1}\vec{q}_n} \geq 2
		\iff \omega_{n-1} \geq 2\frac{\vec{q}_{n-1}\vec{q}_n}{\prod_{j=1}^{n-1}g_j^2}.
	\end{equation}
	De (\ref{eq:recurrence}) sabemos que
	\begin{equation*}
		\omega_{n-1} = \omega_{n-2}\omega_{n-1}' -
		\frac{\vec{q}_{n-2}}{\prod_{j=1}^{n-2}g_j}\vec{t}_{n-2}.
	\end{equation*}
	Sustituyendo en (\ref{proof:omega-sub}), usando el hecho de que $\vec{q}_{n-2} < 0$ y despejando
	$t_{n-2}$, encontramos que $\lceil b_2 \rceil - \lfloor b_1 \rfloor \geq 0$ si
	\begin{equation*}
		\vec{t}_{n-2} \geq \frac{\omega_{n-2}\omega_{n-1}'}{\vec{q}_{n-2}}\prod_{j=1}^{n-2}g_j
		- 2\frac{\vec{q}_{n-1}\vec{q}_n}{\vec{q}_{n-2}g_{n-1}^2}
		\prod_{j=1}^{n-2}g_j^{-1}
	\end{equation*}
	Llamemos $c$ al lado derecho de esta desigualdad. Así pues, definimos el vector
	$\vec{t} \in \Z^{n-1}$ de manera que
	\begin{equation*}
		\vec{t}_i \coloneq \begin{cases}
			\left\lceil -\frac{\vec{\omega}_i}{\vec{q}_i} \right\rceil, & i < n - 2, \\[1em]
			\left\lceil \max\left\lbrace -\frac{\vec{\omega}_i}{\vec{q}_i}, c \right\rbrace
			\right\rceil, & i = n -2, \\[0.8em]
			\lceil b_1 \rceil, & i = n - 1.
		\end{cases}
	\end{equation*}
	Se verifica que $\vec{t}$ satisface ambos (\ref{eq:vect-param-lb}) y (\ref{eq:feasible-param}).
	Finalmente, el caso $\vec{q}_{n-1}, \vec{q}_n < 0$ es completamente similar.
\end{proof}

En síntesis, por el Lema \ref{lemma:t-existence} sabemos que existe un vector
$\vec{t} \in \Z^{n-1}$ que satisface ambos (\ref{eq:vect-param-lb}) y (\ref{eq:feasible-param}).
Al definir $\vec{x}^* \coloneq \eta\vec{\omega} + M\vec{t}$, entonces $\vec{x}$ es una solución no
negativa de $\vec{q}^T\vec{x} = \eta$. Por el Teorema \ref{theory:th:feasibility} se sigue que
$\vec{x}^*$ es el óptimo de (\ref{theory:formulation}).

En la práctica es mejor usar la relación de recurrencia (\ref{eq:recurrence}) y ``construir'' las
entradas $\vec{x}_i$ al mismo tiempo que definimos $\vec{t}_i$ de manera que satisfaga
(\ref{eq:param-lb}) y (\ref{eq:lr-bounds}). Si procedemos de esta forma no tenemos que encontrar primero
$\vec{\omega}$ y $M$, determinar $\vec{t}$ y luego recuperar $\vec{x}$. A partir de esto obtenemos
el siguiente resultado.
\begin{theorem}
	\label{infinite:th:complexity}
	El problema (\ref{theory:formulation}) se puede resolver a través de encontrar la solución de
	una ecuación lineal diofantina en $n$ incógnitas.
\end{theorem}

\section{Análisis de resultados}
\noindent
Una consecuencia del Teorema \ref{infinite:th:complexity} es que la complejidad algoritmítica del
problema (\ref{theory:formulation}) es lineal en la dimensión $n$ siempre y cuando $\vec{q}_i < 0$
para alguna $i \in \lbrace 2, \ldots, n\rbrace$. En esta sección describimos un algoritmo cuyo
tiempo de terminación es $\mathcal{O}(n)$. A través de los resultados obtenidos previamente, somos
capaces de mostrar que nuestro algoritmo es correcto. Finalmente, implementamos nuestro algoritmo en
el lenguaje de programación Python y comparamos sus tiempos de terminación con los de la
implementación de Ramificación y Acotamiento en la librería PuLP. 
