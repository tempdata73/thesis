\chapter{El caso infinito}
\noindent
Los puntos enteros que se encuentran en la $\eta$-ésima capa entera satisfacen la ecuación lineal
diofantina
\begin{equation}
	\label{phase-1:eq:dioph}
	\vec{q}^T\vec{x} = q_1x_1 + q_2x_2 + \cdots + q_nx_n = \eta.
\end{equation}
En la sección de Teoría de Números mostramos bajo qué condiciones existen soluciones a este tipo de
ecuaciones y también cómo construirlas cuando solamente tenemos dos incógnitas. Por conveniencia
definamos $g_1 \coloneq \gcd{q_1, \ldots, q_n} = 1$ y $\omega_1 \coloneq \eta$. De manera análoga a
una formulación de programación dinámica hacia adelante, podemos definir también
$g_2 \coloneq \gcd{\frac{q_2}{g_1}, \ldots, \frac{q_n}{g_1}}$, con lo que la ecuación anterior es
equivalente a la ecuación
\begin{equation}
	\label{phase-1:eq:forward}
	\frac{q_1}{g_1}x_1 + g_2
		\underbrace{
		\left(\frac{q_2}{g_2 \cdot g_1}x_2 + \cdots + \frac{q_n}{g_2 \cdot g_1}x_n\right)}_{\coloneq
		\omega_2}
	= \omega_1.
\end{equation}
No es difícil observar que $\gcd{\frac{q_1}{g_1}, g_2} = 1$ y por lo tanto existen soluciones
enteras para todo $\omega_1 \in \Z$. Como estos coeficientes son coprimos, encontramos que sus
coeficientes de Bézout asociados (c.f. Definición \ref{prerreq:def:bezout}) $x_1', \omega_2'$ son
soluciones particulares de la ecuación
\begin{equation*}
	\frac{q_1}{g_1}x_1 + g_2\omega_2 = 1,
\end{equation*}
con lo que las soluciones de la ecuación (\ref{phase-1:eq:forward}) están dadas por
\begin{equation*}
	\begin{cases}
		x_1 = \omega_1x_1' + g_2t_1, \\
		\omega_2 = \omega_1\omega_2' - \frac{q_1}{g_1}t_1,
	\end{cases}
\end{equation*}
donde $t_1 \in \Z$. Como la restricción de no negatividad se debe satisfacer ($x_1 \geq 0$),
encontramos que
\begin{equation*}
	t_1 \geq \left\lceil -\frac{\omega_1x_1'}{g_2} \right\rceil.
\end{equation*}
Siguiendo el razonamiento de la formulación dinámica en (\ref{phase-1:eq:forward}), fijamos $t_1$ y
resolvemos la ecuación
\begin{equation*}
	\frac{q_2}{g_2 \cdot g_1}x_2 +
	\frac{q_3}{g_2 \cdot g_1}x_3 +
	\cdots +
	\frac{q_n}{g_2 \cdot g_1}x_n
	= \omega_2.
\end{equation*}
Por construcción, los coeficientes de las incógnitas en el lado izquierdo de la ecuación son
coprimos y por lo tanto existen una infinidad de soluciones para todo $\omega_2 \in \Z$. Por el
mismo razonamiento que el anterior, encontramos que las soluciones son
\begin{equation*}
	\begin{cases}
		x_2 = \omega_2x_2' + g_3t_2, \\
		\omega_3 = \omega_2\omega_3' - \frac{q_2}{g_2 \cdot g_1}t_2,
	\end{cases}
\end{equation*}
donde $t_2 \in \Z$ es un parámetro, $\omega_3$ y $g_3$ están definidos de forma análoga al paso
anterior, y $x_2', \omega_3'$ son los coeficientes de Bézout asociados a $\frac{q_2}{g_2 \cdot g_2}$
y $g_3$, respectivamente. Por la restricción de no negatividad ($x_2 \geq 0$), se debe cumplir
\begin{equation*}
	t_2 \geq \left\lceil -\frac{\omega_2x_2'}{g_3} \right\rceil.
\end{equation*}

De manera general, en el $i$-ésimo paso de la formulación dinámica para $i \in \lbrace 1, \ldots, n
- 2 \rbrace$, encontramos
\begin{equation}
	\label{phase-1:eq:recursive}
	\begin{cases}
		x_i = \omega_ix_i' + g_{i + 1}t_i, \\
		\omega_{i + 1} = \omega_i\omega_{i + 1}' - \frac{q_i}{\prod_{j=1}^{i}g_j}t_i,
	\end{cases}
\end{equation}
donde $t_i \in \Z$ satisface, debido a la restricción de no negatividad,
\begin{equation}
	\label{phase-1:eq:param-bound}
	t_i \geq \left\lceil -\frac{\omega_ix_i'}{g_{i + 1}} \right\rceil.
\end{equation}
Finalmente, en el último paso, obtenemos la ecuación lineal diofantina
\begin{equation}
	\label{phase-1:eq:stopping}
	\frac{q_{n-1}}{\prod_{j=1}^{n-2}g_j}x_{n-1} +
	\frac{q_{n}}{\prod_{j=1}^{n-2}g_j}x_n
	= \omega_{n-1}.
\end{equation}
Nuevamente, los coeficientes de las incógnitas son coprimos. Por lo tanto, las soluciones están
dadas por
\begin{equation}
	\label{phase-1:eq:lastsol}
	\begin{cases}
		x_{n-1} = \omega_{n-1}x_{n-1}' + \frac{q_n}{\prod_{j=1}^{n-2}g_j}t_{n-1}, \\
		x_n = \omega_{n-1}x_n' - \frac{q_{n-1}}{\prod_{j=1}^{n-2}g_j}t_{n-1},
	\end{cases}
\end{equation}
Para que ahora se satisfagan las condiciones de no negatividad de $x_{n-1}$ y de $x_n$, encontramos
que $t_{n-1} \in \Z$ debe cumplir ciertas desigualdades según los signos de $q_{n-1}$ y de $q_n$.
Por conveniencia, definamos
\begin{equation*}
	b_1 \coloneq -\frac{\omega_{n-1}x_{n-1}'}{q_n} \cdot \prod_{j=1}^{n-2}g_j,
	\quad b_2 \coloneq \frac{\omega_{n-1}x_{n}'}{q_{n-1}} \cdot \prod_{j=1}^{n-2}g_j.
\end{equation*}
Entonces se verifica que
\begin{equation}
	\label{phase-1:eq:feasible-param}
	t_{n-1} \in 
	\begin{cases}
		\big[ \lceil b_1 \rceil, \lfloor b_2 \rfloor \big] & \text{si } 0 < q_{n-1}, q_n, \\
		\big[ \lceil b_2 \rceil, \lfloor b_1 \rfloor \big] & \text{si } q_{n-1}, q_n < 0, \\
		\big[ \lceil \max\lbrace b_1 ,  b_2 \rbrace \rceil, \infty \big) & \text{si } q_{n-1} < 0 < q_n, \\
		\big( -\infty, \lfloor \min\lbrace b_1, b_2\rbrace \rfloor \big] & \text{si } q_n < 0 < q_{n-1}.
	\end{cases}
\end{equation}

Como supusimos que el problema es factible, existe un vector de parámetros $\vec{t} \in \Z^{n-1}$
que satisface las desigualdades (\ref{phase-1:eq:param-bound}) y (\ref{phase-1:eq:feasible-param}) y
por lo tanto que provee una solución al problema (\ref{theory:formulation}). Para encontrar estos
parámetros, usamos la técnica de \textit{backtracking}, en donde hacemos una elección de $t_1,
\ldots, t_{n-2} \in \Z$ que satisfagan (\ref{phase-1:eq:param-bound}) y determinamos si existe
$t_{n-1} \in \Z$ que satisfaga (\ref{phase-1:eq:feasible-param}). En caso de que no exista tal
parámetro, cambiamos nuestra elección de $t_1, \ldots, t_{n-2}$. Ciertamente la decisión más simple
es realizar el cambio $t_{n-2} \leftarrow t_{n-2} - 1$. Cabe destacar que si los signos de $q_{n-1}$
y $q_n$ son distintos, entonces no es necesario realizar \textit{backtracking}, pues el algoritmo
termina en la primera iteración. Independientemente del caso, eventualmente obtendremos un vector de
parámetros $\vec{t}$ que satisfaga las desigualdades necesarias y, por lo tanto, eventualmente
obtendremos un punto entero óptimo.
