\chapter{El caso infinito}

% TODO: resumen del capítulo

\noindent
Sea $\vec{p} \in \R^n \setminus \braces{\vec{0}}$ un vector esencialmente entero y recordemos de la
Definición \ref{theory:def:rational} que tiene un único múltiplo coprimo $\vec{q} \in \Z^n$. Es
decir, existe un único escalar $m \in \R$ que satisface tres cosas: $\vec{p} = m\vec{q}$, las
entradas $q_1, \ldots, q_n$ son coprimas, y la primera entrada no nula $q_i$ es positiva. Al igual
que en el capítulo anterior, supondremos que $m$ es positivo. Equivalentemente, supondremos que la
primera entrada no nula $p_i$ es positiva\footnote{ El autor hace recordar que esta es una cuestión
	puramente de comodidad y no hay pérdida de generalidad. Cuando $m$ es negativo, los resultados
se mantienen pero es necesario voltear las desigualdades y cambiar las funciones piso por las
funciones techo, lo cual añadiría un número innecesario de casos a analizar.}.

Retomemos el entero $\eta \in \Z$ del Lema \ref{phase-1:lemma:eta} que parametriza la primera capa
entera que satisface el presupuesto \eqref{theory:constraint:budget}. A causa del Teorema
\ref{theory:th:feasibility} sabemos que si $q_i \leq 0$ para alguna $i \in \braces{1, \ldots, n}$,
entonces la $\eta$-ésima capa entera contiene un número infinito de puntos factibles. A partir de
esto último, somos capaces de resolver el automáticamente el problema de decisión de determinar si
un escalar $u^* \in \R$ es el valor óptimo del programa (\ref{theory:formulation}).
\begin{corollary}
	Supongamos que $q_i \leq 0$ para algún $i \in \braces{1, \ldots, n}$. Entonces el valor óptimo
	del programa lineal entero (\ref{theory:formulation}) es $m\eta$. Además, si $m$ es positivo,
	tenemos que $\eta$ es el múltiplo de $m$ más grande que satisface $m\eta \leq u$, donde $u$ es
	el lado derecho de la restricción presupuestaria \eqref{theory:constraint:budget}.
\end{corollary}
\begin{proof}
	Por el Teorema \ref{theory:th:feasibility} sabemos que existen una infinidad de soluciones en la
	$\eta$-ésima capa entera, así que sea $\vec{x}^*$ una de ellas. Entonces $\vec{q}^T\vec{x}^* =
	\eta$, pero $\vec{p} = m\vec{q}$ por la Definición \ref{theory:def:rational}, por lo que
	obtenemos $\vec{p}^T\vec{x}^* = m\vec{q}^T\vec{x}^* = m\eta$.

	Ahora bien, si $m$ es positivo, por el Lema \ref{phase-1:lemma:eta} tenemos que
	$\eta = \lfloor u/m \rfloor$
	. Supongamos que $\xi \in \Z$ satisface $m\xi \leq u$ y también $\lfloor
	u/m \rfloor < \xi$. Luego,
	\begin{equation*}
		m\left\lfloor \frac{u}{m} \right\rfloor < m\xi \leq u
		\implies \left\lfloor \frac{u}{m} \right\rfloor < \xi \leq \frac{u}{m},
	\end{equation*}
	pero esto contradice las propiedades de la función piso.
\end{proof}

\begin{observation}
	Para ilustrar la conveniencia de restringir $m$ a que sea positivo, consideremos el caso cuando
	$m < 0$. De una manera similar a la del Lema \ref{phase-1:lemma:eta}, podemos demostrar que
	$\eta \coloneq \lceil u/m \rceil$ parametriza también la primera capa entera que satisface el
	presupuesto, pues ahora tenemos de la restricción \eqref{theory:constraint:budget} que
	$\vec{p}^T\vec{x} \leq u$ si y solo si $\vec{q}^T\vec{x} \geq u/m$. Se sigue cumpliendo que el
	valor óptimo del problema \eqref{theory:formulation} es $m\eta$. Sin embargo, $\eta$ ahora es el
	múltiplo más chico de $m$ que satisface $m\eta \geq u$.
\end{observation}

Una vez resuelto el problema de decisión, podemos preguntarnos concretamente cómo obtener el punto
óptimo. Por el Teorema \ref{theory:th:feasibility} sabemos que debemos resolver la ecuación lineal
diofantina $\vec{q}^T\vec{x} = \eta$. Del Teorema \ref{th:lattice} sabemos que si $q_n \neq 0$,
entonces existen $k \in \Z$ y $\vec{t} \in \Z^{n-1}$ tales que
\begin{equation*}
	\vec{x} = k\vec{\omega} + M\vec{t},
\end{equation*}
donde $\vec{\omega}$ y $M$ están definidas por \eqref{eq:vec-omega} y \eqref{eq:mat-T},
respectivamente. De los Lemas \ref{lemma:iso1} y \ref{lemma:iso2} sabemos que
\begin{equation*}
	\vec{q}^T\left(\eta\vec{\omega} + M\vec{t}\right) = \eta\vec{q}^T\vec{\omega} + \vec{q}^TM\vec{t} = \eta
\end{equation*}
para todo $\vec{t} \in \Z^{n-1}$. Así pues, debe ser el caso que $k = \eta$ y debemos encontrar
condiciones suficientes en $\vec{t}$ para asegurar la no-negatividad de $\vec{x}$. En primer lugar,
sabemos que para todo $i \in \braces{1, \ldots, n - 2}$, la entrada $t_i$ debe satisfacer
\eqref{eq:param-lb}. En segundo lugar, recuperamos de \eqref{eq:last-solution} que las últimas dos
soluciones de la ecuación $\vec{q}^T\vec{x} = \eta$ están dadas por
\begin{equation*}
	\begin{cases}
		x_{n-1} = \omega_{n-1}x_{n-1}' + \frac{q_n}{\prod_{j=1}^{n-1}g_j}t_{n-1}, \\
		x_n = \omega_{n-1}x_n' - \frac{q_{n-1}}{\prod_{j=1}^{n-1}g_j}t_{n-1},
	\end{cases}
\end{equation*}
donde los enteros $g_i$ están definidos por \eqref{dummy:next-g} con $g_1 = 1$, $\omega_{n-1}$ está
definida a través de la relación de recurrencia \eqref{eq:omega-recurrence} con condición inicial
$\omega_1 = \eta$ (o bien a partir del Lema \ref{eq:omega-formula} con $k = \eta$), y $x_{n-1}',
x_n'$ son coeficientes de Bézout que satisfacen \eqref{eq:last-equation-bez}.

En lo que se encuentra a continuación supondremos que ninguna entrada $q_i$ es nula. Esto no
constituye problema alguno debido a un razonamiento similar al de la demostración del Teorema
\ref{theory:th:feasibility}. Definimos
\begin{equation*}
	I^\circ \coloneq \braces{i \colon q_i = 0},
\end{equation*}
y también definimos el vector $\tilde{\vec{q}}$ cuyas entradas son las entradas no nulas de
$\vec{q}$. A partir de lo que sigue vamos a determinar un vector entero no nulo $\tilde{\vec{x}}$
que satisfaga $\tilde{\vec{q}}^T\tilde{\vec{x}} = \eta$. Luego, encontramos que el vector $\vec{x}$
dado por
\begin{equation*}
	x_i \coloneq
	\begin{cases}
		\tilde{x}_i, & i \not \in I^\circ, \\
		0, & i \in I^\circ,
	\end{cases}
\end{equation*}
es entero, no negativo, y también satisface $\vec{q}^T\vec{x} = \eta$. Así pues, la suposición de
que $q_i \neq 0$ para todo $i \in \braces{1, \ldots, n}$ toma lugar sin pérdida de generalidad.

Para que se satisfagan las condiciones de no negatividad de $x_{n-1}$ y de $x_n$, encontramos que
$t_{n-1} \in \Z$ debe cumplir ciertas desigualdades según los signos de $q_{n-1}$ y de $q_n$.
Definamos, por conveniencia,
\begin{equation}
	\label{eq:lr-bounds}
	b_1 \coloneq -\frac{\omega_{n-1}x_{n-1}'}{q_n} \cdot \prod_{j=1}^{n-1}g_j, \quad
	b_2 \coloneq \frac{\omega_{n-1}x_{n}'}{q_{n-1}} \cdot \prod_{j=1}^{n-1}g_j.
\end{equation}
Entonces, para asegurar la no-negatividad de $x_{n - 1}$ y de $x_n$, debe ser el caso que
\begin{equation}
	\label{eq:feasible-param}
	t_{n-1} \in 
	\begin{cases}
		\big[ \lceil \max\lbrace b_1 ,  b_2 \rbrace \rceil, \infty \big), &  q_{n-1} < 0 < q_n, \\[0.5em]
		\big( -\infty, \lfloor \min\lbrace b_1, b_2\rbrace \rfloor \big], &  q_n < 0 < q_{n-1}, \\[0.5em]
		\big[ \lceil b_2 \rceil, \lfloor b_1 \rfloor \big], &  q_{n-1}, q_n < 0, \\[0.5em]
		\big[ \lceil b_1 \rceil, \lfloor b_2 \rfloor \big], &  0 < q_{n-1}, q_n.
	\end{cases}
\end{equation}

Podemos emplear la misma estrategia de permutar las entradas de $q_i$ de manera que colapsemos estos
cuatro casos distintos en uno solo. Como estamos en el caso infinito del Teorema
\ref{theory:th:feasibility}, naturalmente supondremos que $q_i < 0$ para alguna $i \in \braces{1,
\ldots n}$. Así pues, podemos permutar esta $i$-ésima entrada de $\vec{q}$ con $q_{n-1}$, con lo que
obtenemos $q_{n-1} < 0$. Luego, como $\vec{q}$ es el múltiplo coprimo de $\vec{p}$ y ninguna entrada
de $\vec{q}$ es nula, se sigue de la Definición \ref{theory:def:rational} que $q_1 > 0$. Así pues,
podemos permutar la primera y última entrada de $\vec{q}$, de donde se sigue que $q_n > 0$.
Juntándolo todo, obtenemos $q_{n-1} < 0 < q_n$. De esta manera, para asegurar la no negatividad de
$x_{n-1}$ y $x_n$, basta con que se satisfaga el primer caso:
\begin{equation}
	\label{eq:feasible-param:collapsed}
	t_{n-1} \geq \ceil{\max\lbrace b_1, b_2 \rbrace}.
\end{equation}

\begin{lemma}
	\label{lemma:t-existence}
	Sea $\vec{p} \in \R$ un vector cuyas entradas son todas distintas de cero, y sea $\vec{q} \in
	\Z^n$ su múltiplo coprimo. Entonces existe un vector $\vec{t} \in \Z^{n-1}$ que satisface ambos
	\eqref{eq:param-lb} y \eqref{eq:feasible-param}.
\end{lemma}
\begin{proof}
	Por la discusión anterior, podemos suponer sin pérdida de generalidad que $q_{n-1} < 0 < q_n$,
	así que basta mostrar la existencia de  $\vec{t} \in \Z^{n-1}$ que satisfaga \eqref{eq:param-lb} y
	\eqref{eq:feasible-param:collapsed}. Si definimos
	\begin{equation*}
		t_i \coloneq \begin{cases}
			\ceil{-\frac{\omega_ix_i'}{g_{i + 1}}}, & i < n - 1, \\[0.5em]
			\ceil{\max\lbrace b_1, b_2 \rbrace}, & i = n - 1,
		\end{cases}
	\end{equation*}
	entonces se verifica automáticamente que estas condiciones se satisfacen.
\end{proof}

% \begin{lemma}
% 	\label{lemma:t-existence}
% 	Sea $\vec{p} \in \R$ un vector cuyas entradas son todas distintas de cero, y sea $\vec{q} \in
% 	\Z^n$ su múltiplo coprimo. Entonces existe un vector $\vec{t} \in \Z^{n-1}$ que satisface ambos
% 	\eqref{eq:param-lb} y \eqref{eq:feasible-param}.
% \end{lemma}
% \begin{proof}
% 	Tenemos cuatro casos, pero observemos que los dos en donde $q_{n - 1}$ y $q_n$
% 	tienen signo distinto no son difíciles: si $q_{n - 1} <0 < q_n$, entonces el vector
% 	$\vec{t} \in \Z^{n-1}$ dado por
% 	\begin{equation*}
% 		t_i \coloneq \begin{cases}
% 			\left\lceil -\frac{\omega_ix_i'}{g_{i + 1}} \right\rceil, & i < n - 1, \\[0.5em]
% 			\lceil \max\lbrace b_1, b_2 \rbrace \rceil, & i = n - 1,
% 		\end{cases}
% 	\end{equation*}
% 	satisface ambos (\ref{eq:param-lb}) y (\ref{eq:feasible-param}). El caso $q_n < 0 <
% 	q_{n - 1}$ es completamente similar.
% 
% 	Consideremos el caso $q_{n - 1}, q_n < 0$. Como ninguna entrada de $\vec{p}$ es nula y el vector
% 	$\vec{q}$ es su múltiplo coprimo, se sigue de la Definición \ref{theory:def:rational} que $q_1 >
% 	0$. Podemos entonces permutar las entradas $q_1$ y $q_{n}$ para regresar al primer caso.
% 
% 	Finalmente, consideremos el caso $0 < q_{n - 1}, q_n$. Queremos encontrar condiciones
% 	suficientes para asegurar que el intervalo $[\ceil{b_1}, \floor{b_2}]$ esté bien definido, es
% 	decir, para asegurar que $\floor{b_2} - \ceil{b_1} \geq 0$. Podemos suponer sin pérdida de
% 	generalidad que $q_{n - 2} < 0$. En efecto, como $q_i < 0$ para alguna $i \in \braces{1, \ldots,
% 	n - 2}$, somos capaces permutar las entradas $i$ y $n - 2$ de $\vec{q}$. Observemos que
% 	\begin{align*}
% 		b_2 - 1 &\leq \lfloor b_2 \rfloor \leq b_2, \\
% 		b_1 &\leq \lceil b_1 \rceil \leq b_1 + 1.
% 	\end{align*}
% 	De donde obtenemos
% 	\begin{equation*}
% 		b_2 - b_1 - 2 \leq \lfloor b_2 \rfloor - \lceil b_1 \rceil \leq b_2 - b_1.
% 	\end{equation*}
% 	Así pues, para que el intervalo $[\lceil b_1 \rceil, \lfloor b_2 \rfloor]$ esté bien definido,
% 	es suficiente con mostrar que existe un escalar $\omega_{n - 1}$ que satisfaga $b_2 - b_1 \geq
% 	2$. De \eqref{eq:lr-bounds} tenemos
% 	\begin{equation}
% 		\label{proof:b-sub}
% 		b_2 - b_1 = \omega_{n - 1}\prod_{j = 1}^{n-1}g_j \cdot
% 			\left(\frac{x_{n-1}'}{q_n} + \frac{x_n'}{q_{n - 1}}\right)
% 	\end{equation}
% 	Pero $x_{n-1}'$ y $x_n'$ satisfacen \eqref{eq:last-equation-bez}, y entonces
% 	\begin{equation*}
% 		\frac{x_{n-1}'}{q_n} + \frac{x_n'}{q_{n - 1}} = \frac{1}{q_{n-1}q_n} \cdot \prod_{j =
% 		1}^{n-1}g_j.
% 	\end{equation*}
% 	Sustituyendo en (\ref{proof:b-sub}),
% 	\begin{equation*}
% 		b_2 - b_1 = \frac{\omega_{n-1}}{q_{n-1}q_n} \cdot \prod_{j=1}^{n-1}g_j^2,
% 	\end{equation*}
% 	por lo que $b_2 - b_1 \geq 2$ si y solo si
% 	\begin{equation}
% 		\label{proof:omega-sub}
% 		\omega_{n-1} \geq 2\frac{q_{n-1}q_n}{\prod_{j=1}^{n-1}g_j^2}.
% 	\end{equation}
% 	De (\ref{eq:recurrence}) sabemos que
% 	\begin{equation*}
% 		\omega_{n-1} = \omega_{n-2}\omega_{n-1}' -
% 		\frac{q_{n-2}}{\prod_{j=1}^{n-2}g_j}t_{n-2}.
% 	\end{equation*}
% 	Sustituyendo en (\ref{proof:omega-sub}), usando el hecho de que $q_{n-2} < 0$ y despejando
% 	$t_{n-2}$, encontramos que $\floor{b_2} - \ceil{b_1} \geq 0$ si
% 	\begin{equation*}
% 		t_{n-2} \geq \frac{\omega_{n-2}\omega_{n-1}'}{q_{n-2}}\cdot\prod_{j=1}^{n-2}g_j
% 		- 2\frac{q_{n-1}q_n}{q_{n-2}g_{n-1}^2}\cdot
% 		\prod_{j=1}^{n-2}g_j^{-1}
% 	\end{equation*}
% 	Llamemos $c$ al lado derecho de esta desigualdad. Así pues, definimos el vector
% 	$\vec{t} \in \Z^{n-1}$ de manera que
% 	\begin{equation*}
% 		t_i \coloneq \begin{cases}
% 			\left\lceil -\frac{\omega_ix_i'}{q_i} \right\rceil, & i < n - 2, \\[0.5em]
% 			\left\lceil \max\left\lbrace -\frac{\omega_ix_i'}{q_i}, c \right\rbrace
% 			\right\rceil, & i = n -2, \\[0.5em]
% 			\lceil b_1 \rceil, & i = n - 1.
% 		\end{cases}
% 	\end{equation*}
% 	Se verifica que $\vec{t}$ satisface ambos (\ref{eq:param-lb}) y (\ref{eq:feasible-param}).
% \end{proof}

En síntesis, por el Lema \ref{lemma:t-existence} sabemos que existe un vector
$\vec{t} \in \Z^{n-1}$ que satisface ambos (\ref{eq:param-lb}) y (\ref{eq:feasible-param}).
Al definir $\vec{x}^* \coloneq \eta\vec{\omega} + M\vec{t}$, encontramos que $\vec{x}^*$ es entero y
no negativo, y además por los Lemas \ref{lemma:iso1} y \ref{lemma:iso2} encontramos que
\begin{equation*}
	\vec{q}^T\vec{x}^* = \eta\vec{q}^T\vec{\omega} + \vec{q}^TM\vec{t} = \eta.
\end{equation*}
Por el Teorema \ref{theory:th:feasibility}, se sigue que $\vec{x}^*$ es la solución al problema
\eqref{theory:formulation}.

En la práctica es mejor usar la relación de recurrencia (\ref{eq:recurrence}) y ``construir'' las
entradas $x_i$ al mismo tiempo que definimos $t_i$ de manera que satisfaga (\ref{eq:param-lb}) y
(\ref{eq:feasible-param:collapsed}). Si procedemos de esta forma no tenemos que encontrar primero
$\vec{\omega}$ y $M$, determinar $\vec{t}$ y luego recuperar $\vec{x}$. El Algoritmo \ref{algo:inf}
muestra este procedimiento constructivo.

\begin{algorithm}[ht]
	\LinesNumbered
	\SetKwProg{Fn}{Fn}{\string:}{}
	\SetKwFunction{Bezout}{Bezout}
	\SetKwFunction{NonNegativeIntSol}{NonNegativeIntSol}
	\Fn{\NonNegativeIntSol{$\vec{q}$, $\eta$}}{
		\KwData{\\
			Vector coprimo $\vec{q}$ tal que $q_i \neq 0$ para todo $i \in \braces{1, \ldots, n}$ y
			$q_{n-1} < 0 < q_n$.\\
			Lado derecho $\eta$.
			}
		\KwResult{\\
			Solución entera no negativa $\vec{x}$ a la ecuación lineal diofantina $\vec{q}^T\vec{x} =
			\eta$.
		}
		\Begin{
			$\vec{x} \leftarrow \vec{0}$\;
			$\omega_1 \leftarrow \eta$\;
			\For{$i \leftarrow 1$ \KwTo $n - 2$}{
				$g_{i+1} \leftarrow \gcd{q_{i+1}, \ldots, q_n}$\;
				$x_i', \omega_{i+1}' \leftarrow$ \Bezout{$q_i$, $g_{i+1}$}\;
				$t_i \leftarrow \ceil{-\omega_i x_i' / g_{i+1}}$\;
				$x_i \leftarrow \omega_i x_i' + g_{i+1}t_i$\;
				$\omega_{i+1} \leftarrow \omega_i \omega_{i+1}' - q_i t_i$\;
				\For{$j \leftarrow i$ \KwTo $n - 2$}{
					$q_{j+1} \leftarrow q_{j+1}/g_{i+1}$\;
				}
			}

			$x_{n-1}', x_n' \leftarrow$ \Bezout{$q_{n-1}$, $q_n$}\;
			$b_1 \leftarrow -\omega_{n-1} x_{n-1}' / q_n$\;
			$b_2 \leftarrow -\omega_{n-1} x_n' / q_{n-1}$\;
			$t_{n-1} \leftarrow \ceil{\max\lbrace b_1, b_2\rbrace}$\;
			$x_{n-1} \leftarrow \omega_{n-1}x_{n-1}' + q_nt_{n-1}$\;
			$x_{n} \leftarrow \omega_{n-1}x_{n}' - q_{n-1}t_{n-1}$\;

			\Return{$\vec{x}$}
		}
	}
	\caption{Algoritmo para obtener soluciones enteras no negativas a la ecuación lineal diofantina
	$\vec{q}^T\vec{x} = \eta$, dados ciertos supuestos sobre $\vec{q}$.}
	\label{algo:inf}
\end{algorithm}

En el Algoritmo \ref{algo:inf} supusimos la existencia de una subrutina \texttt{Bezout} que, como su
nombre lo indica, calcula los coeficientes de Bézout entre dos enteros. Es la creencia del autor
que no es necesario escribir la subrutina en esta tesis, pero reitera, así como lo hizo en la Sección
\ref{section:number-theory}, que estos coeficientes se pueden calcular por medio del Algoritmo
Extendido de Euclides.

El Algoritmo \ref{algo:inf:ext} del Apéndice \ref{app:inf:ext} extiende el Algoritmo \ref{algo:inf}
para deshacerse del supuesto de que ninguna entrada $q_i$ es nula, y también del supuesto de que
$q_{n-1} < 0 < q_n$. Basta con mencionar en este capítulo que usamos la misma lógica expuesta antes
del Lema \ref{lemma:t-existence}.

\begin{theorem}
	El Algoritmo \ref{algo:inf} es correcto.
\end{theorem}
\begin{proof}
\end{proof}

\begin{corollary}
	\label{infinite:th:complexity}
	El problema (\ref{theory:formulation}) se puede resolver a través de encontrar la solución de
	una ecuación lineal diofantina en $n$ incógnitas.
\end{corollary}

\section{Análisis de resultados}
\noindent
Una consecuencia del Teorema \ref{infinite:th:complexity} es que la complejidad algoritmítica del
problema (\ref{theory:formulation}) es lineal en la dimensión $n$ siempre y cuando $q_i < 0$
para alguna $i \in \lbrace 2, \ldots, n\rbrace$. En esta sección describimos un algoritmo cuyo
tiempo de terminación es $\mathcal{O}(n)$. A través de los resultados obtenidos previamente, somos
capaces de mostrar que nuestro algoritmo es correcto. Finalmente, implementamos nuestro algoritmo en
el lenguaje de programación Python y comparamos sus tiempos de terminación con los de la
implementación de Ramificación y Acotamiento en la librería PuLP. 
