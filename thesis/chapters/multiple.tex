\chapter{Múltiples restricciones}
\noindent
En esta sección hacemos un análisis extensivo sobre lo resulta de agregar más restricciones al
problema (\ref{theory:formulation}). Sea $\vec{p} \in \R^n$ esencialmente entero y consideremos su
múltiplo coprimo $\vec{q} \in \Z^n$. Sea $A \in \Q^{m \times n}$ una matriz racional con renglones
linealmente independientes y sea $\vec{b} \in \Q^m$ un vector. Consideremos el problema
\begin{subequations}
	\label{formulation:multiple}
	\begin{align}
		\max_{\vec{x} \in \Z^n} \quad
			& \vec{q}^T\vec{x}, \label{formulation:multiple:objective} \\
		\text{s.a.} \quad
			& \vec{q}^T\vec{x} \leq u, \label{formulation:multiple:constraint:budget} \\
			& A\vec{x} = \vec{b}, \label{formulation:multiple:constraints} \\
			& \vec{x} \geq \vec{0}. \nonumber
	\end{align}
\end{subequations}
Ciertamente, la solución no se encuentra necesariamente en la $\eta$-ésima capa entera. Por ejemplo,
si dejamos que $A \coloneq \vec{q}^T$ y $b \coloneq u - m$, la solución se encontrará en la
$\xi$-ésima capa entera, donde
\begin{equation*}
	\xi \coloneq \left\lfloor \frac{u}{m} - 1 \right\rfloor < \eta.
\end{equation*}
No obstante, si el problema (\ref{formulation:multiple}) es factible, sabemos que la solución se
encontrará en alguna capa entera con parámetro $k \in \lbrace \eta, \eta - 1, \ldots \rbrace$, pues
todavía contamos con una restricción presupuestaria que se debe satisfacer.
\begin{observation}
	Recordemos del teorema \ref{theory:th:feasibility} que, si tenemos solamente la restricción
	presupuestaria, entonces la utilidad máxima es $\eta$ si $q_i < 0$ para alguna $i \in
	\lbrace 2, \ldots, n - 1\rbrace$. Al igual que en el caso finito, ahora no somos capaces de
	saber inmediatamente en qué capa entera se encuentra nuestra solución.
\end{observation}

Ahora bien, en el contexto del problema (\ref{formulation:multiple}), el parámetro $k \in \Z$ se
encarga de maximizar la utilidad (\ref{formulation:multiple:objective}), así como de respetar el
presupuesto (\ref{formulation:multiple:constraint:budget}) a través de $k \leq \eta$. Similarmente,
el vector $\vec{t} \in \Z^{n-1}$ se encarga de respetar las otras restricciones
(\ref{formulation:multiple:constraints}).
\begin{theorem}
	El problema (\ref{formulation:multiple}) es equivalente al problema de maximización
	\begin{subequations}
		\label{formulation:lattice}
		\begin{align}
			\max_{k \in \Z, \vec{t} \in \Z^{n-1}}
				& k, \\
			\text{s.a.} \quad
				& k \leq \eta, \label{lattice:c-layer} \\
				& AM\vec{t} = kA\vec{\nu} - \vec{b}, \label{lattice:constraints} \\
				& M\vec{t} \geq -k\vec{\nu}.
		\end{align}
	\end{subequations}
\end{theorem}
\begin{proof}
	Por el teorema \ref{th:lattice}, sabemos que la transformación lineal
	\begin{align*}
		(k, \vec{t}) &\mapsto \vec{x} \coloneq k\vec{\nu} + M\vec{t}
	\end{align*}
	es un isomorfismo entre las redes $\Z \oplus \Z^{n - 1}$ y $\Z^n$. Así, tenemos
	\begin{align*}
		A\vec{x} = \vec{b} &\iff AM\vec{t} = \vec{b} - kA\vec{\nu}, \\
		\vec{x} \geq \vec{0} &\iff M\vec{t} \geq -k\vec{\nu},
	\end{align*}
	y por lo tanto basta mostrar que si un vector es factible para un problema, entonces satisface
	la correspondiente restricción presupuestaria del otro problema. Para ello, es de utilidad
	recordar que $\eta$ parametriza la primera capa entera que satisface el presupuesto.

	Sea $\vec{x} \in \Z^n$ un vector factible de (\ref{formulation:multiple}) Como $\vec{x}$ es
	entero, entonces se debe cumplir $\vec{q}^T\vec{x} \leq \eta$. Ahora bien, existe $(k, \vec{t})
	\in \Z^n$ que satisface $\vec{x} = k\vec{\nu} + M\vec{t}$. Por el lema \ref{lemma:iso1} y el
	corolario \ref{lemma:iso2} encontramos que
	\begin{equation*}
		k = \vec{q}^T\vec{x} \leq \eta,
	\end{equation*}
	y entonces $(k, \vec{t})$ es factible. Como $\vec{x}$ fue arbitrario, se sigue que la solución
	del problema (\ref{formulation:multiple}) es una cota inferior del problema
	(\ref{formulation:lattice}). La demostración de que la solución de (\ref{formulation:lattice})
	es una cota inferior de (\ref{formulation:multiple}) es análoga.

	Finalmente, supongamos que $(k, \vec{t}) \in \Z^n$ es solución de (\ref{formulation:lattice}).
	Si existe $\hat{\vec{x}}$ factible para (\ref{formulation:multiple}) con utilidad
	$\vec{q}^T\hat{\vec{x}} = \hat{k}$ estrictamente mayor, entonces consideramos $(\hat{k},
	\hat{\vec{t}})$ tal que $\hat{\vec{x}} = \hat{k}\vec{\nu} + M\hat{\vec{t}}$. Este vector
	también es factible con utilidad $k < \hat{k} \leq \eta$, y entonces $(k, \vec{t})$ no era la
	solución de (\ref{formulation:lattice}). Obtenemos una contradicción.
\end{proof}

\begin{observation}
	El vector objetivo todavía es ortogonal a la restricción presupuestaria. No obstante, es más
	fácil de manejar en caso de usar cortes como en Ramificación y Acotamiento. Si $k^*$ no es
	entero en la solución al problema relajado, la única manera de ramificar es con el nuevo corte
	$k \leq \lfloor k^* \rfloor$, pues el otro corte $k \geq \lceil k^* \rceil$ generará un
	subproblema infactible. Evidentemente, en la sección de análisis de resultados haremos
	comparaciones de tiempo en los tiempos de terminación entre esta formulación y la original.
\end{observation}

La formulación del problema equivalente en el teorema anterior resulta ser más interesante. Podemos
desacoplar esta nueva formulación de manera que obtengamos un problema de maximización y otro de
factibilidad. Supongamos, sin pérdida de generalidad, que las entradas de $A$ y $\vec{b}$ son
enteras. Como los renglones de $A$ son linealmente independientes, de \cite{alex} sabemos que tiene
una única factorización de Hermite. Es decir, existe una matriz $U \in \Z^{n \times n}$ unimodular
que satisface $AU = [H, \vec{0}]$, donde $H \in \Z^{m \times m}$ es triangular inferior y no
singular.

Consideremos el subproblema de maximización
\begin{subequations}
	\label{subformulation:lattice}
	\begin{align}
		\max_{k \in \Z}
			& ~ k, \\
		\text{s.a.} \quad
		k &\leq \eta, \\
			A\tvec{y} &= kA\vec{\nu} - \vec{b},
	\end{align}
\end{subequations}
donde 
\begin{equation*}
	\tvec{y} \coloneq U \begin{pmatrix} \tvec{y}_m \\ \tvec{y}_{n-m} \end{pmatrix}
	= U_m\tvec{y}_m + U_{n-m}\tvec{y}_{n-m} \in \Z^n,
\end{equation*}
con $\tvec{y}_m \in \Z^m$ y $\tvec{y}_{n-m} \in \Z^{n-m}$. Así también, $U_m$ y
$U_{n-m}$ denotan las primeras $m$ columnas y últimas $n - m$ columnas de $U$, respectivamente.
Observemos que para toda $k \in \Z$ se cumple
\begin{equation}
	AU \begin{pmatrix} \inv{H}\left(kA\vec{\nu} - \vec{b}\right) \\ \tvec{y}_{n-m} \end{pmatrix}
	=
	[H, \vec{0}] \begin{pmatrix} \inv{H}\left(kA\vec{\nu} - \vec{b}\right) \\ \tvec{y}_{n-m} \end{pmatrix}
	= kA\vec{\nu} - \vec{b},
\end{equation}
lo cual sugiere definir $\tvec{y}_m \coloneq \inv{H}(\vec{b} - kA\vec{w})$. No obstante,
también debemos asegurarnos que este vector sea entero. Observemos que $\tvec{y}_{n-m}$ queda
libre, así que en realidad este subproblema tiene dimensión $m + 1$. Definimos el conjunto de
factibilidad
\begin{equation}
	\label{eq:feas-set}
	F \coloneq \lbrace k \in \Z \vcentcolon \inv{H}\left(kA\vec{\nu} - \vec{b}\right) \in \Z^m \rbrace
	\cap \lbrace k \in \Z \vcentcolon k \leq \eta \rbrace.
\end{equation}
\begin{observation}
	Para que $F$ sea no vacío, debe existir $k \in \Z$ tal que $\det(H) \mid (k\vec{a}_j^T
	\vec{\nu} - b_j)$ para todo $j \in \lbrace 1, \ldots, m \rbrace$, donde $\vec{a}^T_j$
	denota el $j$-ésimo vector renglón de $A$. Es decir, una condición suficiente y necesaria para
	la no vacuidad de $F$ es
	\begin{equation*}
		\det(H) \mid \gcd{k\vec{a}_1^T\vec{\nu} - b_1, \ldots, k\vec{a}_m^T\vec{\nu} - b_m}.
	\end{equation*}
	Ahora bien, $H$ es triangular inferior e invertible, por lo que $\det(H) \neq 0$ es el producto
	de los elementos $h_1, \ldots, h_m$ en su diagonal. Entonces $h_j \mid \det(H)$ para todo $j \in
	\lbrace 1, \ldots m \rbrace$ y una condición necesaria para la no vacuidad de $F$ es
	\begin{equation*}
		\lcm{h_1, \ldots, h_m} \mid \gcd{k\vec{a}_1^T\vec{\nu} - b_1, \ldots, k\vec{a}_m^T\vec{\nu} - b_m}.
	\end{equation*}
\end{observation}

Si $F$ es vacío, deducimos que este subproblema es infactible y por lo tanto
(\ref{formulation:lattice}) también lo es. Supongamos, pues, que $F \neq \emptyset$. No es difícil
observar que $F$ tiene un elemento maximal $k^*$ y que este elemento es la solución al subproblema
(\ref{subformulation:lattice}). Luego, dada esta solución $k^* \in \Z$, buscamos resolver el
subproblema de factibilidad
\begin{subequations}
	\label{subformulation:feasibility}
	\begin{align}
		M\vec{t} &= \tvec{y}, \\
		M\vec{t} &\geq -k^*\vec{\nu}.
	\end{align}
\end{subequations}
Observemos que tenemos un sistema de $n$ ecuaciones lineales con $2n - m - 1$ incógnitas, por lo que
tendremos que lidiar con $n - m - 1$ parámetros libres:
\begin{align}
	\label{eq:feasibility-eqs}
	M\vec{t} = \tvec{y} = U_m\tvec{y}_m + U_{n-m}\tvec{y}_{n-m}
   \iff [M, -U_{n-m}] \begin{pmatrix} \vec{t} \\ \tvec{y}_{n-m} \end{pmatrix} = U_m\tvec{y}_m.
\end{align}
Si consideramos ahora la forma normal de Smith de esta matriz por bloques, obtenemos dos matrices
unimodulares $S \in \Z^{n \times n}$ y $T \in \Z^{(2n - m - 1) \times (2n - m -1)}$ que satisfacen
\begin{equation*}
	S[M, -U_{n-m}]T = D \in \Z^{n \times (2n - m - 1)},
\end{equation*}
donde $D$ es una matriz diagonal cuyas $n$ primeras entradas son distintas de cero y las restantes
$n - m - 1$ son cero. Si multiplicamos $S$ por la izquierda en ambos lados de la ecuación
(\ref{eq:feasibility-eqs}), tenemos
\begin{equation*}
	D\inv{T}\begin{pmatrix} \vec{t} \\ \tvec{y}_{n-m} \end{pmatrix}
	= SU_m\tvec{y}_{m}.
\end{equation*}
Si $d_i$ no divide a $(SU_m\tvec{y}_{m})_i$ para alguna $i \in \lbrace 1, \ldots, n \rbrace$,
encontramos que la primera ecuación del subproblema (\ref{subformulation:feasibility}) no tiene
solución en los enteros, lo que implica que la elección de $k^*$ fue la incorrecta para asegurar
soluciones enteras a este subproblema. De ser este el caso, redefinimos $F \leftarrow F \setminus
\lbrace k^* \rbrace$. Si $F$ ahora es vacío, entonces (\ref{formulation:lattice}) es
infactible, de caso contrario escogemos el nuevo elemento de maximal de $F$ y repetimos el proceso.

Supongamos, pues que $d_i \mid (SU_m\tvec{y}_{m})_i$ para todo $i \in \lbrace 1, \ldots,
n\rbrace$, por lo que obtenemos $n$ soluciones enteras $\vec{r} \in \Z^n$ y $n - m - 1$ variables
libres $\vec{s} \in \Z^{n-m-1}$:
\begin{equation*}
	\inv{T}\begin{pmatrix} \vec{t} \\ \tvec{y}_{n-m} \end{pmatrix}
	=
	\begin{pmatrix} \vec{r} \\ \vec{s} \end{pmatrix}.
\end{equation*}
Por lo tanto, nuestro vector $\vec{t}$ es una función lineal de $\vec{s}$, es decir, $\vec{t} =
\vec{t}(\vec{s})$. Hasta este punto el proceso no ha sido complicado, pues nos hemos encargado de
resolver sistemas de ecuaciones lineales diofantinas. En términos del problema original
(\ref{formulation:multiple}), hemos encontrado los vectores $\vec{x}(\vec{s}) \coloneq
k^*\vec{\nu} + M\vec{t}(\vec{s})$ que maximizan la utilidad y que satisfacen todas las
restricciones excepto, posiblemente, las de no negatividad.

La dificultad entra en juego cuando queremos determinar el vector de variables libres $\vec{s} \in
\Z^{n-m-1}$ que hagan que $\vec{t}(\vec{s})$ satisfaga la desigualdad en el subproblema
(\ref{subformulation:feasibility}). Debilitando más esta condición, nos gustaría determinar si el
conjunto
\begin{equation*}
	\lbrace \vec{s} \in \Z^{n-m-1} \vcentcolon M\vec{t}(\vec{s}) \geq -k^*\vec{\nu} \rbrace
\end{equation*}
es vacío o no. En esta versión debilitada no nos interesa saber qué elementos contiene o tan
siquiera cuántos elementos contiene. Es sabido que los programas enteros tales como
(\ref{formulation:multiple}) o (\ref{formulation:lattice}) son problemas difíciles de resolver, en
el sentido de que no es conocido si se pueden resolver en tiempo polinomial. A lo largo de este
capítulo, no obstante, hemos resuelto todos los problemas en tiempo polinomial\footnote{En
	\cite{alex} se muestra que calcular el máximo común divisor, resolver ecuaciones lineales
	diofantinas, y calcular las factorizaciones tanto de Hermite como de Smith son operaciones
	acotadas por tiempo polinomial.}.
La única deducción posible, entonces, es que el problema de determinar las variables $\vec{s}$, o
bien de determinar cuántas hay, o bien de determinar su existencia, son todos problemas difíciles de
resolver.

A pesar de lo anterior, hay dos casos donde la dificultad se reduce drásticamente. El caso menos
interesante es cuando $m = n - 1$, de manera que no hay parámetros libres. Esto se debe a que el
politopo factible resultante es un semirrayo o un segmento de línea. Al momento de escoger la
$k^*$-ésima capa entera, estamos agregando la ecuación $k^* = k$, con lo que obtenemos un sistema
lineal entero de $n$ ecuaciones con $n$ incógnitas, y entonces la solución es única. Basta entonces
verificar que este único vector $\vec{t}$ satisface la desigualdad en el subproblema
(\ref{subformulation:feasibility}). El caso un poco más interesante se obtiene cuando $m = n - 2$.
De esta manera obtenemos un solo parámetro, con lo que podemos determinar rápidamente la existencia
o inexistencia de un intervalo de factibilidad.

\begin{example}
	\label{ex:two-var}
	Consideremos el problema con $n = 2$ variables y $m = 1$ restricciones
	\begin{align*}
		\max
			~& x - y, \\
		\text{s.a.} \quad
			& x - y \leq 12, \\
			& 3x + 5y = 25, \\
			& x, y \geq 0.
	\end{align*}
	En este caso tenemos $A = (3, 5), \vec{b} = 25$, y también $\vec{q} = (1, -1)^T$, al igual que
	$\eta = 12$. De (\ref{eq:vec-omega}) y (\ref{eq:mat-T}) obtenemos
	\begin{equation*}
		\vec{\nu} = \begin{pmatrix} 1 \\ 0 \end{pmatrix},
		M = \begin{pmatrix} -1 \\ -1 \end{pmatrix}.
	\end{equation*}
	De la forma normal de Hermite de $A$ tenemos
	\begin{equation*}
		H = 1, U = \begin{pmatrix} 2 & -5 \\ -1 & 3 \end{pmatrix},
	\end{equation*}
	y de la forma normal de Smith de $[M, -U_m]$,
	\begin{equation*}
		S = \begin{pmatrix} -1 & 0 \\ 1 & -1 \end{pmatrix},
		D = \begin{pmatrix} 1 & 0 \\ 0 & 8 \end{pmatrix},
		T = \begin{pmatrix} 1 & 5 \\ 0 & 1 \end{pmatrix}. 
	\end{equation*}

	Como $H = 1$, se sigue que $\inv{H} (\vec{b} - kA\vec{\nu}) = 25 - 3k$ es entero para todo $k
	\in \Z$. Así, el conjunto factible $F$ (c.f. \ref{eq:feas-set}) está dado por
	\begin{equation*}
		F = \Z \cap \lbrace k \in \Z \vcentcolon k \leq 12 \rbrace
		= \lbrace k \in \Z \vcentcolon k \leq \eta = 12 \rbrace.
	\end{equation*}
	Entonces escogemos $k^* = 12$ por ser el elemento maximal de $F$. Así, encontramos
	\begin{equation*}
		SU_m\tvec{y}_m = SU_m \left(\inv{H} (\vec{b} - k^*A\vec{\nu})\right)
		= \begin{pmatrix} 22 \\ 33 \end{pmatrix}
	\end{equation*}
	Observemos que la segunda entrada de $SU_m\tvec{y}_m$ no es divisible por $D_{22} = 8$.
	Así, el subproblema (\ref{subformulation:feasibility}) no es factible para la elección de $k^*$
	previa. Escogemos el segundo elemento de $F$ más grande, con lo que tenemos $k^* \leftarrow 11$.
	En este caso obtenemos $SU_m\tvec{y}_m = (-16, -24)^T$, por lo que sí hay soluciones
	enteras. Luego, se debe satisfacer,
	\begin{equation*}
		\inv{T} \begin{pmatrix} \vec{t} \\ \tvec{y}_{n-m} \end{pmatrix} =
		\begin{pmatrix} 16 \\ 24 \end{pmatrix},
	\end{equation*}
	de donde se sigue que $(\vec{t}, \tvec{y}_{n-m}) = (1, 3)$. Verificamos factibilidad:
	\begin{equation*}
		M\vec{t} + k^*\vec{\nu}
		= 1 \begin{pmatrix} -1 \\ -1 \end{pmatrix} + 11 \begin{pmatrix} 1 \\ 0 \end{pmatrix}
		= \begin{pmatrix} 10 \\ -1 \end{pmatrix} \not \geq \vec{0}.
	\end{equation*}
	Ahora la elección de $k^*$ dio un punto entero pero con una entrada negativa. Seguimos este
	procedimiento hasta llegar a $k^* \leftarrow 3$. En este caso obtenemos $(\vec{t},
	\tvec{y}_{n-m}) = (-2, -6)^T$, de donde
	\begin{equation*}
		M\vec{t} + k^*\vec{\nu}
		= -2 \begin{pmatrix} -1 \\ -1 \end{pmatrix} + 3 \begin{pmatrix} 1 \\ 0 \end{pmatrix}
		= \begin{pmatrix} 5 \\ 2 \end{pmatrix} \geq \vec{0}.
	\end{equation*}
	Concluimos diciendo que $(k^*, \vec{t}) \coloneq (3, -2)$ es el óptimo del programa
	(\ref{formulation:lattice}) y entonces $(x, y) = (5, 2)$ es el óptimo de
	(\ref{formulation:multiple}).
\end{example}
\begin{example}
	Ahora consideremos el problema con $n = 3$ variables y $m = 1$ restricciones
	\begin{align*}
		\max
			~& x - y + 2z, \\
		\text{s.a.} \quad
			& x - y  + 2z \leq 10 \\
			& 3x + 4y - z = 15 \\
			& x, y, z \geq 0.
	\end{align*}
	En este caso tenemos $A = (3, 4, -1), \vec{b} = 15$, y también $\vec{q} = (1, -1, 2)^T$, al igual que
	$\eta = 10$. De (\ref{eq:vec-omega}) y (\ref{eq:mat-T}) obtenemos
	\begin{equation*}
		\vec{\nu} = \begin{pmatrix} 1 \\ 0 \\ 0 \end{pmatrix},
		M = \begin{pmatrix} 1 & 0 \\ -1 & 2 \\ -1 & 1 \end{pmatrix}.
	\end{equation*}
	De la forma normal de Hermite de $A$ tenemos
	\begin{equation*}
		H = 1, U = \begin{pmatrix} 0 & 0 & 1 \\ 0 & 1 & 0 \\ -1 & 4 & 3 \end{pmatrix},
	\end{equation*}
	y de la forma normal de Smith de $[M, -U_m]$,
	\begin{equation*}
		S = \begin{pmatrix}
			1 & 0 & 0 \\
			-1 & -1 & 0 \\
			3 & 4 & -1
		\end{pmatrix},
		D = \begin{pmatrix}
			1 & 0 & 0 & 0 \\
			0 & 1 & 0 & 0 \\
			0 & 0 & 7 & 0
		\end{pmatrix},
		T = \begin{pmatrix}
			1 & 0 & 0 & 1 \\
			0 & 0 & 1 & 0 \\
			0 & 1 & 2 & -1 \\
			0 & 0 & 0 & 1
		\end{pmatrix}.
	\end{equation*}
	Nuevamente, observemos que $H = 1$ y por lo tanto $F = \lbrace k \in \Z \vcentcolon k \leq 10
	\rbrace$. Seguimos exactamente el mismo procedimiento que en el Ejemplo \ref{ex:two-var} hasta
	llegar a $k^* \leftarrow 5$. Encontramos que se satisface
	\begin{equation*}
		\inv{T} \begin{pmatrix} \vec{t} \\ \tvec{y}_{n-m} \end{pmatrix}
		=
		\begin{pmatrix} 0 \\ 0 \\ 0 \\ s \end{pmatrix}
		\implies
		\begin{pmatrix} \vec{t} \\ \tvec{y}_{n-m} \end{pmatrix}
		=
		s \begin{pmatrix} 1 \\ 0 \\ -1 \\ 1 \end{pmatrix},
	\end{equation*}
	donde $s \in \Z$ es la única variable libre. En este caso podemos determinar rápidamente un
	intervalo de existencia: tenemos $M\vec{t} \geq -k^*\vec{\nu}$ si y solo si
	\begin{equation*}
		s\begin{pmatrix} 1 \\ 0 \\ -1 \end{pmatrix} \geq
		\begin{pmatrix} -5 \\ 0 \\ 0 \end{pmatrix},
	\end{equation*}
	de donde se sigue inmediatamente que $s \in \lbrace -5, -4, \ldots, 0 \rbrace$. Sustituyendo en
	$\vec{t}$ y transformando a $\vec{x}$, encontramos que
	\begin{equation*}
		\left\lbrace
			\begin{pmatrix} 0 \\ 5 \\ 5 \end{pmatrix},
			\begin{pmatrix} 1 \\ 4 \\ 4 \end{pmatrix},
			\begin{pmatrix} 2 \\ 3 \\ 3 \end{pmatrix},
			\begin{pmatrix} 3 \\ 2 \\ 2 \end{pmatrix},
			\begin{pmatrix} 4 \\ 1 \\ 1 \end{pmatrix},
			\begin{pmatrix} 5 \\ 0 \\ 0 \end{pmatrix}
		\right\rbrace
	\end{equation*}
	son las seis soluciones del problema. Todas alcanzan un nivel de utilidad $k^* = 5$.
\end{example}

Si el programa (\ref{formulation:multiple}) es factible, entonces el programa
(\ref{formulation:lattice}) también lo es. A partir de nuestro procedimiento, eventualmente
encontraremos un par $(k^*, \vec{t}^*)$ que resuelva tanto el subproblema de maximización
(\ref{subformulation:lattice}) como el de factibilidad (\ref{subformulation:feasibility}).

Ahora bien, son dos las maneras en las que nuestro problema sea infactible. Puede que nuestro
conjunto de factibilidad $F$ sea vacío y por lo tanto el sistema de ecuaciones lineales
(\ref{formulation:multiple:constraints}) sea inconsistente. O bien, puede ser que $F$ tenga
cardinalidad infinita pero para ninguno de sus elementos se satisfaga el subproblema de
factibilidad.

% TODO: mostrar una imagen.
Esto último puede ocurrir cuando el sistema de ecuaciones siempre tiene solución pero todas ellas
son negativas. En efecto, si en el Ejemplo \ref{ex:two-var} reemplazamos el lado derecho de la
igualdad $\vec{b} = 25$ por $\vec{b} = -4$, nos encontramos en aquella situación.

En conclusión, para asegurar terminación en tiempo finito, cualquier algoritmo basado en este método debe
asegurarse primero que el conjunto de factibilidad $F$ tiene un número finito de puntos. Este caso
lo estudiamos en la siguiente sección.

\subsection{Eliminando la restricción presupuestaria}
\noindent
Consideremos ahora el problema
\begin{subequations}
	\label{formulation:last}
	\begin{align}
		\max_{\vec{x} \in \Z^n} \quad
			& \vec{q}^T\vec{x}, \label{formulation:last:objective} \\
		\text{s.a.} \quad
			& A\vec{x} = \vec{b}, \label{formulation:last:constraints} \\
			& \vec{x} \geq \vec{0}, \nonumber
	\end{align}
\end{subequations}
Evidentemente, si su programa relajado tiene un valor objetivo $u^*$ finito, podemos agregar la
restricción presupuestaria $\vec{q}^T\vec{x} \leq u^*$ a este problema de manera válida. Entonces
podemos suponer sin pérdida de generalidad que este programa es equivalente a
(\ref{formulation:multiple}) siempre que su valor objetivo sea finito. Consecuentemente, podemos
utilizar las herramientas desarrolladas en la sección pasada para resolver este problema entero.

Es más, supongamos que el politopo asociado al problema relajado es acotado y no vacío. Entonces
tanto el problema de maximización como de minimización tienen valores objetivos finitos. Llamemos a
estos valores $\ell^*$ y $u^*$, respectivamente. Ahora la restricción
\begin{equation*}
	\ell^* \leq \vec{q}^T\vec{x} \leq u^*
\end{equation*}
es válida para el problema (\ref{formulation:last}). De la misma manera que $\eta$ parametriza la
primera capa entera que satisface el presupuesto, podemos definir análogamente la última capa que
satisface el presupuesto. Usando el mismo razonamiento que en el lema \ref{phase-1:lemma:eta},
encontramos que esta capa está parametrizada por $\tau \coloneq \lceil \ell^*/m \rceil$ si $m$ es
positiva. Así pues, al definir nuestro conjunto de factibilidad $F$ como
\begin{equation*}
	F \coloneq \lbrace k \in \Z \vcentcolon \inv{H}\left(kA\vec{\nu} - \vec{b}\right) \in \Z^m \rbrace
	\cap \lbrace k \in \Z \vcentcolon \tau \leq k \leq \eta \rbrace,
\end{equation*}
podemos replicar las mismas técnicas que en la sección pasada. Pero además, $F$ es un conjunto
finito y por lo tanto tenemos terminación en tiempo finito para este caso. Es decir, cualquier
algoritmo basado en los métodos desarrollados en la sección pasada podrá decidir en tiempo finito si
el problema es factible o no. En caso de que sí lo sea entonces terminará con la solución óptima.

Existen varios algoritmos para resolver el problema relajado de (\ref{formulation:last}) en su
versión general. Es cierto que el método del simplex es el más utilizado, a pesar de tener una
complejidad algorítmica no acotada polinomialmente. También es cierto que existen métodos
polinomiales para resolver este problema, tales como el método elipsoidal o el algoritmo de
Karmarkar. Pero más interesante es el hecho de que ya existen cotas superiores para ciertas
instancias de estos problemas, por ejemplo, en el caso del Problema de la Mochila, \cite{martello}
provee una cota superior razonable, y ciertamente el valor de 0 es una cota inferior justa. Mucho
hablaremos de este problema en el Capítulo 3. No obstante, el autor considera prudente dedicar el
siguiente capítulo para el caso infinito.
