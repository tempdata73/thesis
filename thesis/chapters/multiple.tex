\chapter{Múltiples restricciones}
\noindent
En este último capítulo construimos un método que permite resolver programas lineales enteros
generales. Mostramos que la complejidad exponencial de este tipo de programas se reduce a resolver
sistemas de desigualdades lineales en los enteros. Además, encontramos una formulación alternativa a
la manera tradicional de introducir programas lineales enteros que simplifica el árbol de
subproblemas generado por el algoritmo de Ramificación y Acotamiento, lo cual podría resultar en
mejores tiempos de terminación.

En la exposición de este capítulo dependemos en gran medida de las formas normales de Hermite y de
Smith, las cuales son tratadas extensivamente en \cite{alex} y \cite{morris}. El estilo de nuestra
discusión es menos formal aunque no por ello menos rigurosa.

Sea $\vec{p} \in \R^n$ esencialmente entero y sea $\vec{q} \in \Z^n$, de manera que $\vec{p} =
m\vec{q}$ para algún escalar $m > 0$. Vimos en la subsección \ref{subsec:dioph-eq} que el problema
\eqref{theory:formulation} es equivalente a
\begin{equation*}
	\max_{\vec{x} \in \Z^n}
	\braces{\vec{q}^T\vec{x} \colon \vec{q}^T\vec{x} \in \braces{0, \ldots, \eta}, \vec{x}
	\geq \vec{0}},
\end{equation*}
donde $\eta \in \Z$ está definida en el lema \ref{phase-1:lemma:eta}. Por ello, introducimos el
problema con múltiples restricciones \eqref{formulation:multiple} en términos del vector $\vec{q}$ y
el parámetro $\eta$ en vez del vector $\vec{p}$ y el lado derecho $u$ de
\eqref{theory:constraint:budget}. Así pues, sea $A \in \Q^{m \times n}$ una matriz racional con
renglones linealmente independientes y sea $\vec{b} \in \Q^m$ un vector. Consideremos el problema
\begin{subequations}
	\label{formulation:multiple}
	\begin{align}
		\max_{\vec{x} \in \Z^n} \quad
			& \vec{q}^T\vec{x}, \label{formulation:multiple:objective} \\
		\text{s.a.} \quad
			& \vec{q}^T\vec{x} \leq \eta, \label{formulation:multiple:constraint:budget} \\
			& A\vec{x} = \vec{b}, \label{formulation:multiple:constraints} \\
			& \vec{x} \geq \vec{0}. \nonumber
	\end{align}
\end{subequations}

\begin{observation}
	En contraste con el primer caso del teorema \ref{theory:th:feasibility}, no podemos asegurar que
	la solución se encuentre sobre la $\eta$-ésima capa entera $\qlayer{q}{\eta}$ aún cuando
	$\vec{q}$ tenga una entrada negativa. En efecto, si $A = \vec{q}^T$ y $\vec{b} = \eta - 1$, la
	restricción \eqref{formulation:multiple:constraint:budget} se vuelve redundante y obtenemos un
	problema similar a \eqref{theory:formulation}. En caso de que $\vec{q}$ tenga al menos una
	entrada negativa, el primer caso del teorema \ref{theory:th:feasibility} nos indica que el
	problema es factible y que la solución se encuentra en la $(\eta - 1)$-ésima capa entera.
\end{observation}

Supongamos que el problema \eqref{formulation:multiple} es factible. Debido a la restricción
presupuestaria \eqref{formulation:multiple:constraint:budget}, sabemos que la solución se encuentra
en alguna capa entera $\qlayer{q}{k}$ con parámetro entero $k \leq \eta$. Esto motiva el siguiente
resultado.

\begin{theorem}
	\label{th:multeq}
	Sea $M \in \Z^{n \times (n - 1)}$ la matriz definida en \eqref{eq:mat-T} y $\vec{\nu} \in \Z^n$
	el vector definido en \eqref{eq:vec-omega}. Entonces el problema \eqref{formulation:multiple} es
	equivalente al problema
	\begin{subequations}
		\label{formulation:lattice}
		\begin{align}
			\max_{k \in \Z, \vec{t} \in \Z^{n-1}}
				& k, \\
			\text{s.a.} \quad
				& k \leq \eta, \label{lattice:c-layer} \\
				& AM\vec{t} = \vec{b} - kA\vec{\nu}, \label{lattice:constraints} \\
				& M\vec{t} \geq -k\vec{\nu}.
		\end{align}
	\end{subequations}
\end{theorem}
\begin{proof}
	Por el teorema \ref{th:lattice} y la discusión que le sucede, sabemos que la transformación
	lineal
	\begin{align*}
		(k, \vec{t}) &\mapsto \vec{x} \coloneq k\vec{\nu} + M\vec{t}
	\end{align*}
	es un isomorfismo entre las redes $\Lambda_p \oplus \Lambda_h$ definidas en \eqref{eq:dec} y
	$\Z^n$. Así, tenemos
	\begin{align*}
		A\vec{x} = \vec{b} &\iff AM\vec{t} = \vec{b} - kA\vec{\nu}, \\
		\vec{x} \geq \vec{0} &\iff M\vec{t} \geq -k\vec{\nu},
	\end{align*}
	y por lo tanto basta mostrar que si un vector es factible para un problema, entonces satisface
	la correspondiente restricción presupuestaria \eqref{formulation:multiple:constraint:budget} o
	\eqref{lattice:c-layer} del otro problema.

	Sea $\vec{x} \in \Z^n$ un vector factible de \eqref{formulation:multiple}, entonces existe $(k,
	\vec{t}) \in \Z^n$ que satisface $\vec{x} = k\vec{\nu} + M\vec{t}$. Por los lemas \ref{lemma:iso1}
	y \ref{lemma:iso2} encontramos que
	\begin{equation*}
		k = \vec{q}^T\vec{x} \leq \eta,
	\end{equation*}
	y entonces $(k, \vec{t})$ es factible. Como $\vec{x}$ fue arbitrario, se sigue que la solución
	del problema (\ref{formulation:multiple}) es una cota inferior del problema
	\eqref{formulation:lattice}. La demostración de que la solución de \eqref{formulation:lattice}
	es una cota inferior de \eqref{formulation:multiple} es análoga usando el mismo isomorfismo.

	Finalmente, supongamos que $(k, \vec{t}) \in \Z^n$ es solución de \eqref{formulation:lattice}.
	Si existe $\tvec{x}$ factible para \eqref{formulation:multiple} con utilidad $\vec{q}^T\tvec{x}
	= \tilde{k}$ estrictamente mayor, entonces consideramos ($\tilde{k}, \tvec{t})$ tal que
	$\tvec{x} = \tilde{k}\vec{\nu} + M\tvec{t}$. Este vector también es factible con utilidad $k <
	\tilde{k} \leq \eta$, y entonces $(k, \vec{t})$ no era la solución de
	\eqref{formulation:lattice}. Obtenemos una contradicción.
\end{proof}

Al inicio de esta tesis mencionamos que las políticas de poda de Ramificación y Acotamiento operan
ineficientemente cuando el vector objetivo es ortogonal a una de sus restricciones. Observemos
en los problemas \eqref{theory:formulation} y \eqref{formulation:multiple} que el vector objetivo
$\vec{q}$ es ortogonal a las restricciones presupuestarias \eqref{theory:constraint:budget} y
\eqref{formulation:multiple:constraint:budget}. Esto también es cierto para el problema
equivalente \eqref{formulation:lattice}, pues el vector objetivo $k\vec{e}_1 \in \Z^n$ es ortogonal
a la restricción \eqref{lattice:c-layer}. A pesar de lo anterior, este problema equivalente induce a
que las políticas de poda sean más eficientes.
\begin{theorem}
	\label{th:bbspeed}
	Sea $(\optr{k}, \optr{\vec{t}})$ el óptimo del problema relajado de \eqref{formulation:lattice}
	y supongamos que $\optr{k}$ no es entero. Entonces el subproblema generado al añadir la
	restricción $k \geq \ceil{\optr{k}}$ es infactible.
\end{theorem}
\begin{proof}
	Supongamos que el subproblema es factible. Puesto que $\optr{k}$ no es entero, existe $\tau \in
	\Z$ tal que $\tau - 1 < \optr{k} < \tau$. Al añadir la restricción $k \geq \ceil{\optr{k}} =
	\tau$ al problema \eqref{formulation:lattice}, encontramos que el valor óptimo de este
	subproblema es estrictamente mayor que $\optr{k}$. Pero esto es una contradicción ya que en
	problemas de maximización el valor óptimo de un problema es una cota superior del valor óptimo
	de cualesquiera de sus subproblemas.
\end{proof}
Debido a este teorema, siempre es mejor priorizar ramificaciones en $\optr{k}$ puesto que nos
deshacemos de manera inmediata subproblemas infactibles.

A continuación desacoplamos el problema \eqref{formulation:lattice} en un subproblema de
maximización y en otro de factibilidad. Supongamos, sin pérdida de generalidad, que las entradas de
$A$ y $\vec{b}$ son enteras. En el capítulo 2 de \cite{morris} es introducida la forma normal de
Hermite de la matriz $A$, la cual afirma que existe una matriz unimodular $U \in \Z^{n \times n}$
que satisface $AU = [H \mid \vec{0}]$, donde $H \in \Z^{m \times m}$ es triangular inferior y no
singular.

Con esto en mente, introducimos el subproblema de \eqref{formulation:lattice} como
\begin{subequations}
	\label{subformulation:lattice}
	\begin{align}
		\max_{k \in \Z, \tvec{y} \in \Z^n}
			& k, \\
		\text{s.a.} \quad
		k &\leq \eta, \\
			A\tvec{y} &= \vec{b} - kA\vec{\nu},
	\end{align}
\end{subequations}
donde 
\begin{equation}
	\label{eq:tvecy}
	\tvec{y} \coloneq U \begin{pmatrix} \tvec{y}_m \\ \tvec{y}_{n-m} \end{pmatrix}
	= U_m\tvec{y}_m + U_{n-m}\tvec{y}_{n-m} \in \Z^n,
\end{equation}
con $\tvec{y}_m \in \Z^m$ y $\tvec{y}_{n-m} \in \Z^{n-m}$. Denotamos por $U_m$ y
$U_{n-m}$ las primeras $m$ columnas y últimas $n - m$ columnas de $U$, respectivamente.
Observemos que para toda $k \in \Z$ se cumple
\begin{equation}
	AU \begin{pmatrix} \inv{H}\left(\vec{b} - kA\vec{\nu}\right) \\ \tvec{y}_{n-m} \end{pmatrix}
	=
	[H, \vec{0}] \begin{pmatrix} \inv{H}\left(\vec{b} - kA\vec{\nu}\right) \\ \tvec{y}_{n-m} \end{pmatrix}
	= \vec{b} - kA\vec{\nu},
\end{equation}
lo cual sugiere definir
\begin{equation}
	\label{eq:ym}
	\tvec{y}_m \coloneq \inv{H}(\vec{b} - kA\vec{\nu}).
\end{equation}
No obstante, debemos asegurarnos que este vector sea entero. Observemos que $\tvec{y}_{n-m}$ es un
vector libre, así que en realidad este subproblema tiene dimensión $m + 1$. Definimos el conjunto de
factibilidad
\begin{equation}
	\label{eq:feas-set}
	\mathcal{F} \coloneq \braces{
		k \in \Z \colon \inv{H}\left(\vec{b} - kA\vec{\nu}\right) \in \Z^m, k \leq \eta
	}
\end{equation}

Puesto que $H \in \Z^{m \times m}$ es no singular, para cada $k \in \Z$, existe una única solución
$\tvec{y}_m \in \R^m$ del sistema de ecuaciones $H\tvec{y}_m = \vec{b} - kA\vec{\nu}$. Como, además, $H$
es triangular inferior, podemos resolver rápidamente este sistema de ecuaciones y verificar si, para
cada $k \in \Z$, la correspondiente solución $\tvec{y}_m$ es entera o no.

% \begin{observation}
% 	Para que $F$ sea no vacío, debe existir $k \in \Z$ tal que $\det(H) \mid (k\vec{a}_j^T
% 	\vec{\nu} - b_j)$ para todo $j \in \lbrace 1, \ldots, m \rbrace$, donde $\vec{a}^T_j$
% 	denota el $j$-ésimo vector renglón de $A$. Es decir, una condición suficiente y necesaria para
% 	la no vacuidad de $F$ es
% 	\begin{equation*}
% 		\det(H) \mid \gcd{k\vec{a}_1^T\vec{\nu} - b_1, \ldots, k\vec{a}_m^T\vec{\nu} - b_m}.
% 	\end{equation*}
% 	Ahora bien, $H$ es triangular inferior e invertible, por lo que $\det(H) \neq 0$ es el producto
% 	de los elementos $h_1, \ldots, h_m$ en su diagonal. Entonces $h_j \mid \det(H)$ para todo $j \in
% 	\lbrace 1, \ldots m \rbrace$ y una condición necesaria para la no vacuidad de $F$ es
% 	\begin{equation*}
% 		\lcm{h_1, \ldots, h_m} \mid \gcd{k\vec{a}_1^T\vec{\nu} - b_1, \ldots, k\vec{a}_m^T\vec{\nu} - b_m}.
% 	\end{equation*}
% \end{observation}

Si $\mathcal{F}$ es vacío, deducimos que el subproblema \eqref{subformulation:lattice} es infactible
y por lo tanto \eqref{formulation:lattice} también lo es. Supongamos, pues, que $\mathcal{F} \neq
\emptyset$. No es difícil observar que $\mathcal{F}$ tiene un elemento maximal $k^*$ y que este
elemento es la solución al subproblema \eqref{subformulation:lattice}. Luego, dada esta solución
$k^* \in \Z$, buscamos resolver el subproblema de \eqref{formulation:lattice}
\begin{subequations}
	\label{subformulation:feasibility}
	\begin{align}
		M\vec{t} &= \tvec{y}, \label{subf:feas:eq} \\
		M\vec{t} &\geq -k^*\vec{\nu}. \label{subf:feas:ineq}
	\end{align}
\end{subequations}
Tenemos un sistema de $n$ ecuaciones lineales con $2n - m - 1$ incógnitas, por lo que tendremos que
lidiar con $n - m - 1$ variables libres. En efecto, sustituyendo \eqref{eq:tvecy} en
\eqref{subf:feas:eq}, obtenemos
\begin{align}
	\label{eq:feasibility-eqs}
	M\vec{t} = \tvec{y} &= U_m\tvec{y}_m + U_{n-m}\tvec{y}_{n-m} \nonumber \\
   &\iff [M \mid -U_{n-m}] \begin{pmatrix} \vec{t} \\ \tvec{y}_{n-m} \end{pmatrix} = U_m\tvec{y}_m.
\end{align}

En el capítulo 2 de \cite{morris} también se introduce la forma normal de Smith, de la cual
obtenemos dos matrices unimodulares $S \in \Z^{n \times n}$ y $T \in \Z^{(2n - m - 1) \times (2n - m
-1)}$ que satisfacen
\begin{equation*}
	S[M \mid -U_{n-m}]T = D \in \Z^{n \times (2n - m - 1)},
\end{equation*}
donde $D$ es una matriz diagonal cuyas $n$ primeras entradas son distintas de cero y las restantes
$n - m - 1$ son cero. Si multiplicamos $S$ por la izquierda en ambos lados de la ecuación
(\ref{eq:feasibility-eqs}), tenemos
\begin{equation}
	\label{eq:sum}
	D\inv{T}\begin{pmatrix} \vec{t} \\ \tvec{y}_{n-m} \end{pmatrix}
	= SU_m\tvec{y}_{m}.
\end{equation}
Si $d_i$ no divide a $(SU_m\tvec{y}_{m})_i$ para alguna $i \in \lbrace 1, \ldots, n \rbrace$,
encontramos que la primera ecuación del subproblema \eqref{subformulation:feasibility} no tiene
solución en los enteros, lo que implica que la elección de $k^*$ fue la incorrecta para asegurar
soluciones enteras a este subproblema. De ser este el caso, redefinimos nuestro conjunto de
factibilidad $\mathcal{F}$ (ver \eqref{eq:feas-set}) como $\mathcal{F} \leftarrow \mathcal{F}
\setminus \braces{k^*}$. Si $\mathcal{F}$ ahora es vacío, entonces \eqref{formulation:lattice} es
infactible, y en caso contrario escogemos el nuevo elemento maximal de $\mathcal{F}$ y repetimos el proceso.

Supongamos, pues, que $d_i \mid (SU_m\tvec{y}_{m})_i$ para todo $i \in \lbrace 1, \ldots,
n\rbrace$, por lo que obtenemos $n$ soluciones enteras $\vec{r} \in \Z^n$ y $n - m - 1$ variables
libres $\vec{s} \in \Z^{n-m-1}$:
\begin{equation*}
	\inv{T}\begin{pmatrix} \vec{t} \\ \tvec{y}_{n-m} \end{pmatrix}
	=
	\begin{pmatrix} \vec{r} \\ \vec{s} \end{pmatrix}.
\end{equation*}
Por lo tanto, nuestro vector $\vec{t}$ es una función afina de $\vec{s}$, es decir, $\vec{t} =
\vec{t}(\vec{s})$. En términos del problema original \eqref{formulation:multiple}, hemos encontrado,
hasta este punto, los vectores $\vec{x}(\vec{s}) \coloneq k^*\vec{\nu} + M\vec{t}(\vec{s})$ que
maximizan la utilidad y que satisfacen todas las restricciones excepto, posiblemente, las de no
negatividad.

Consideremos el conjunto de vectores $\vec{s} \in \Z^{n-m-1}$ que inducen a que $\vec{t}(\vec{s})$
satisfaga \eqref{subf:feas:ineq}:
\begin{equation*}
	\mathcal{S} \coloneq \lbrace \vec{s} \in \Z^{n-m-1} \vcentcolon M\vec{t}(\vec{s}) \geq -k^*\vec{\nu} \rbrace
\end{equation*}
Por un lado, es sabido que los programas enteros tales como \eqref{formulation:multiple} o
\eqref{formulation:lattice} son problemas difíciles de resolver, a excepción de cuando la matriz de
restricciones $A \in \Z^{m \times n}$ es totalmente unimodular. De manera superficial, decimos que
un problema es difícil de resolver si no es conocida la existencia de un algoritmo con complejidad
polinomial que lo pueda resolver.

Por el otro lado, a lo largo de este capítulo hemos resuelto todos los problemas en tiempo
polinomial. En efecto, obtener $M$ y $\vec{\nu}$ de \eqref{eq:mat-T} y \eqref{eq:vec-omega} se
reduce a multiplicar números y calcular coeficientes de Bézout, al igual que máximos común
divisores. En \cite{alex} y \cite{morris} se muestra que realizar este tipo de cálculos, así como de
obtener las formas normales de Hermite y de Smith, son problemas acotados en tiempo polinomial.

Entonces, la única deducción posible es que el problema de determinar si el conjunto $\mathcal{S}$
es vacío, o cuántos elementos tiene, o cuáles son los elementos que contiene, son todos problemas
difíciles de resolver. Esta complejidad se reduce drásticamente en dos casos especiales.

En primer lugar, si $m = n - 1$, entonces no hay parámetros libres. De manera gráfica, el poliedro
factible resultante es un semirrayo o un segmento de línea. Al momento de escoger la $k^*$-ésima
capa entera, estamos agregando la ecuación $k = k^*$, con lo que obtenemos un sistema lineal entero
de $n$ ecuaciones con $n$ incógnitas, y entonces la solución es única. Resta verificar que esta
solución es entera y satisface \eqref{subf:feas:ineq}. Este caso se ilustra en el ejemplo
\ref{ex:two-var}.

En segundo lugar, si $m = n - 2$, obtenemos un solo parámetro libre $s \in \Z$, con lo que podemos
determinar rápidamente la existencia o inexistencia de un conjunto de factibilidad en $s$ que induce
a que $\vec{t}(s)$ satisfaga \eqref{subf:feas:ineq}. Este caso se ilustra en el ejemplo
\ref{ex:three-var}.

A modo de resumen, mostramos en el pseudocódigo \ref{algo:multiple} la forma de resolver problemas
del tipo \eqref{formulation:lattice}. Por el teorema \ref{th:multeq}, este método también resuelve
problemas del tipo \eqref{formulation:multiple}. Después de presentar los ejemplos \ref{ex:two-var}
y \ref{ex:three-var}, mostramos una manera con la cual podemos deshacernos del ciclo infinito en la
línea \ref{algo:mult:infloop}.

\renewcommand{\algorithmcfname}{Pseudocódigo}
\begin{algorithm}[ht]
	\LinesNumbered
	\KwData{
		Vector coprimo $\vec{q} \in \Z^n$, $\eta \in \Z$, $A \in \Z^{m \times n}$ y $\vec{b} \in
		\Z^m$.
		}
	\KwResult{
		Solución óptima $\vec{x}^*$ de \eqref{formulation:multiple}.
	}
	\Begin{
		Calcular $M$ y $\vec{\nu}$ de \eqref{eq:mat-T} y \eqref{eq:vec-omega}\;
		Obtener $U$ y $H$ de la forma normal de Hermite de $A$\;
		Particionar $U$ en $U_m$ y $U_{n-m}$ tal que $[U_m \mid U_{n-m}] = U$\;
		Obtener $S$ y $T$ de la forma normal de Smith de $[M \mid -U_{n-m}]$\;

		$k \leftarrow \eta$\;
		\While{$1 + 1 = 2$\label{algo:mult:infloop}}{
			Obtener $\tvec{y}_m$ de $H\tvec{y}_m = \vec{b} - kA\vec{\nu}$\;
			\If{$\tvec{y}_m \in \Z^m$}{
				ir al paso \ref{algo:mult:z}\;
			}
			$k \leftarrow k - 1$\; \label{algo:multiple:k}
		}
		$\vec{z} \leftarrow SU_m\tvec{y}_m$\; \label{algo:mult:z}
		$\vec{r} \leftarrow \vec{0}_n$\;
		\For{$i \leftarrow 1$ \KwTo $n$}{
			\If{$D_{ii} \nmid z_i$}{
				ir al paso \ref{algo:multiple:k}\;
			}
			$r_i \leftarrow z_i / D_{ii}$\;
		}
		$(\vec{t}(\vec{s}), \tvec{y}_{n-m}(\vec{s})) \leftarrow T(\vec{r}, \vec{s})^T$\; \label{algo:mult:free}
		\If{existe $\vec{s}$ tal que $M\vec{t}(\vec{s}) \geq -k\vec{\nu}$\label{algo:mult:hard}}{
			$\vec{x}^* \leftarrow k\vec{\nu} + M\vec{t}(\vec{s})$\;
			\Return{$\vec{x}^*$}\;
		}
		ir al paso \ref{algo:multiple:k}\;
	}
	\caption{}
	\label{algo:multiple}
\end{algorithm}
\renewcommand{\algorithmcfname}{Algoritmo}

A fin de obtener las formas normales de Hermite y de Smith de la matriz de restricciones $A \in
\Z^{m \times n}$ de los siguiente ejemplos, el autor utilizó la librería \texttt{hsnf} de
Python\footnote{
	Véase \url{https://hsnf.readthedocs.io/en/latest/index.html}.
}.
\begin{example}
	\label{ex:two-var}
	Consideremos el problema con $n = 2$ variables y $m = 1$ restricciones de igualdad
	\begin{align*}
		\max
			~& x - y, \\
		\text{s.a.} \quad
			& x - y \leq 12, \\
			& 3x + 5y = 25, \\
			& x, y \geq 0.
	\end{align*}

	En este caso tenemos $A = (3, 5), \vec{b} = 25$, y también $\vec{q} = (1, -1)^T$, al igual que
	$\eta = 12$. De \eqref{eq:vec-omega} y \eqref{eq:mat-T} calculamos
	\begin{equation*}
		\vec{\nu} = \begin{pmatrix} 1 \\ 0 \end{pmatrix},\quad
		M = \begin{pmatrix} -1 \\ -1 \end{pmatrix}.
	\end{equation*}
	De la forma normal de Hermite de $A$ tenemos
	\begin{equation*}
		H = 1,\quad U = \begin{pmatrix} 2 & -5 \\ -1 & 3 \end{pmatrix},
	\end{equation*}
	y de la forma normal de Smith de $[M \mid -U_{n-m}]$,
	\begin{equation*}
		S = \begin{pmatrix} -1 & 0 \\ 1 & -1 \end{pmatrix},\quad
		D = \begin{pmatrix} 1 & 0 \\ 0 & 8 \end{pmatrix},\quad
		T = \begin{pmatrix} 1 & 5 \\ 0 & 1 \end{pmatrix}. 
	\end{equation*}

	Como $H = 1$, se sigue que $\inv{H} (\vec{b} - kA\vec{\nu}) = 25 - 3k$ es entero para todo $k
	\in \Z$. Así, el conjunto factible $\mathcal{F}$ definido en \ref{eq:feas-set} está dado por
	\begin{equation*}
		\mathcal{F} 
		= \lbrace k \in \Z \vcentcolon k \leq \eta = 12 \rbrace.
	\end{equation*}
	Entonces escogemos $k^* = 12$ por ser el elemento maximal de $\mathcal{F}$. Luego,
	\begin{equation*}
		\vec{z} \coloneq SU_m\tvec{y}_m = SU_m \left(\inv{H} (\vec{b} - k^*A\vec{\nu})\right)
		= \begin{pmatrix} 22 \\ -33 \end{pmatrix}.
	\end{equation*}

	Observemos que $D_{22} \nmid z_2$, y entonces el subproblema \eqref{subformulation:feasibility}
	no es factible para la elección de $k^* = 12$. Escogemos el segundo elemento de $\mathcal{F}$
	más grande, con lo que tenemos $k^* = 11$. Siguiendo con el mismo procedimiento, encontramos
	ahora que $\vec{z} = (16, -24)$. En este caso, la diagonal de $D$ sí divide, elemento a elemento,
	las entradas de $\vec{z}$, y entonces $\vec{r} = (16, -3)$. Puesto que $n - m - 1 = 0$, no hay
	variables libres. Tenemos de \eqref{eq:sum}:
	\begin{equation*}
		\begin{pmatrix} \vec{t} \\ \tvec{y}_{n-m} \end{pmatrix} = T
		\begin{pmatrix} 16 \\ -3 \end{pmatrix} = \begin{pmatrix} 1 \\ -3 \end{pmatrix},
	\end{equation*}
	y verificamos que se satisfaga \eqref{subf:feas:ineq}:
	\begin{equation*}
		M\vec{t} + k^*\vec{\nu}
		= 1 \begin{pmatrix} -1 \\ -1 \end{pmatrix} + 11 \begin{pmatrix} 1 \\ 0 \end{pmatrix}
		= \begin{pmatrix} 10 \\ -1 \end{pmatrix} \not \geq \vec{0}.
	\end{equation*}
	Ahora la elección de $k^* = 11$ dio un punto entero pero con una entrada negativa. Es decir, el
	subproblema \eqref{subformulation:feasibility} es infactible dada esta elección.

	Repetimos este procedimiento hasta llegar a $k^* = 3$. En este caso encontramos que $(\vec{t},
	\tvec{y}_{n-m}) = (-2, 6)^T$. Por lo tanto,
	\begin{equation*}
		M\vec{t} + k^*\vec{\nu}
		= -2 \begin{pmatrix} -1 \\ -1 \end{pmatrix} + 3 \begin{pmatrix} 1 \\ 0 \end{pmatrix}
		= \begin{pmatrix} 5 \\ 2 \end{pmatrix} \geq \vec{0}.
	\end{equation*}
	Luego, $(k^*, \vec{t}) \coloneq (3, -2)$ es el óptimo del programa
	\eqref{formulation:lattice}. Por el teorema \ref{th:multeq}, concluimos que $(x^*, y^*) = (5,
	2)$ es el óptimo de \eqref{formulation:multiple}.
\end{example}
\begin{example}
	\label{ex:three-var}
	Ahora consideremos el problema con $n = 3$ variables y $m = 1$ restricciones de igualdad
	\begin{align*}
		\max
			~& x - y + 2z, \\
		\text{s.a.} \quad
			& x - y  + 2z \leq 10 \\
			& 3x + 4y - z = 15 \\
			& x, y, z \geq 0.
	\end{align*}

	En este caso tenemos $A = (3, 4, -1), \vec{b} = 15$, y también $\vec{q} = (1, -1, 2)^T$, al igual que
	$\eta = 10$. De \eqref{eq:vec-omega} y \eqref{eq:mat-T} calculamos
	\begin{equation*}
		\vec{\nu} = \begin{pmatrix} 1 \\ 0 \\ 0 \end{pmatrix},\quad
		M = \begin{pmatrix} 1 & 0 \\ -1 & 2 \\ -1 & 1 \end{pmatrix}.
	\end{equation*}
	De la forma normal de Hermite de $A$ tenemos
	\begin{equation*}
		H = 1,\quad U = \begin{pmatrix} 0 & 0 & 1 \\ 0 & 1 & 0 \\ -1 & 4 & 3 \end{pmatrix},
	\end{equation*}
	y de la forma normal de Smith de $[M \mid -U_{n-m}]$,
	\begin{equation*}
		S = \begin{pmatrix}
			1 & 0 & 0 \\
			-1 & -1 & 0 \\
			3 & 4 & -1
		\end{pmatrix},\quad
		D = \begin{pmatrix}
			1 & 0 & 0 & 0 \\
			0 & 1 & 0 & 0 \\
			0 & 0 & 7 & 0
		\end{pmatrix},\quad
		T = \begin{pmatrix}
			1 & 0 & 0 & 1 \\
			0 & 0 & 1 & 0 \\
			0 & 1 & 2 & -1 \\
			0 & 0 & 0 & 1
		\end{pmatrix}.
	\end{equation*}
	Puesto que $H = 1$, tenemos de \eqref{eq:feas-set} que el conjunto factible es
	\begin{equation*}
		\mathcal{F} = \braces{k \in \Z \vcentcolon k \leq \eta = 10}.
	\end{equation*}

	Ahora bien, seguimos exactamente el mismo procedimiento que en el ejemplo \ref{ex:two-var} hasta
	llegar a $k^* = 5$. Llegando a la línea \ref{algo:mult:free} del pseudocódigo
	\ref{algo:multiple}, encontramos que
	\begin{equation*}
		\inv{T} \begin{pmatrix} \vec{t} \\ \tvec{y}_{n-m} \end{pmatrix}
		=
		\begin{pmatrix} 0 \\ 0 \\ 0 \\ s \end{pmatrix}
		\implies
		\begin{pmatrix} \vec{t} \\ \tvec{y}_{n-m} \end{pmatrix}
		=
		s \begin{pmatrix} 1 \\ 0 \\ -1 \\ 1 \end{pmatrix},
	\end{equation*}
	donde $s \in \Z$ es la única variable libre. En este caso podemos determinar rápidamente un
	intervalo de existencia: tenemos $M\vec{t}(s) \geq -k^*\vec{\nu}$ si y solo si
	\begin{equation*}
		s\begin{pmatrix} 1 \\ 0 \\ -1 \end{pmatrix} \geq
		\begin{pmatrix} -5 \\ 0 \\ 0 \end{pmatrix},
	\end{equation*}
	de donde se sigue inmediatamente que $s \in \lbrace -5, -4, \ldots, 0 \rbrace$. Sustituyendo
	cada posible valor de $s$ en $\vec{t}(s)$ y transformando a $\vec{x}^*(s) = k^*\vec{\nu} +
	M\vec{t}(s)$,
	encontramos que
	\begin{equation*}
		\left\lbrace
			\begin{pmatrix} 0 \\ 5 \\ 5 \end{pmatrix},
			\begin{pmatrix} 1 \\ 4 \\ 4 \end{pmatrix},
			\begin{pmatrix} 2 \\ 3 \\ 3 \end{pmatrix},
			\begin{pmatrix} 3 \\ 2 \\ 2 \end{pmatrix},
			\begin{pmatrix} 4 \\ 1 \\ 1 \end{pmatrix},
			\begin{pmatrix} 5 \\ 0 \\ 0 \end{pmatrix}
		\right\rbrace
	\end{equation*}
	son las seis soluciones del problema \eqref{formulation:multiple}. Ciertamente, todas alcanzan
	un nivel de utilidad $k^* = 5$.
\end{example}

Si el problema \eqref{formulation:multiple} es factible, por la equivalencia del teorema
\ref{th:multeq}, existe $k^* \in \Z$ que es el valor óptimo del problema
\eqref{formulation:lattice}, así que eventualmente saldremos del ciclo infito de la línea
\ref{algo:mult:infloop}. En caso de que el problema \eqref{formulation:multiple} sea infactible,
nada asegura, por el momento, que salgamos de este ciclo infinito. A continuación veremos cómo
arreglar este problema, y en el proceso seremos capaces de eliminar la restricción presupuestaria
\eqref{formulation:multiple:constraint:budget}. Por lo tanto, en esta última parte, podremos
encontrar soluciones a programas lineales enteros generales.

Sea $A \in \Z^{m \times n}$ una matriz con renglones linealmente independientes y sea $\vec{b} \in
\Z^m$ un vector. Definamos el poliedro
\begin{equation*}
	P \coloneq \braces{ \vec{x} \in \R^n \vcentcolon A\vec{x} = \vec{b}, \vec{x} \geq \vec{0} }.
\end{equation*}
Sea $\vec{q} \in \Z^n$ un vector coprimo y consideremos ambos problemas de maximización y
minimización sobre este poliedro
\begin{equation}
	\label{eq:last:bounds}
	\ell^* \coloneq \min_{\vec{x} \in P}\braces{\vec{q}^T\vec{x}},\quad
	u^* \coloneq \max_{\vec{x} \in P}\braces{\vec{q}^T\vec{x}},
\end{equation}
y definamos
\begin{equation}
	\tau \coloneq \ceil{\ell^*},\quad \eta \coloneq \floor{u^*}.
\end{equation}
Observemos de \eqref{eq:last:bounds} que siempre se cumple que $\tau \leq \eta$. Ciertamente, las
restricción $\tau \leq \vec{q}^T\vec{x} \leq \eta$ es válida para el programa lineal entero $\max_{P
\cap \Z^n} \braces{\vec{q}^T\vec{x}}$ y, por lo tanto, este problema es equivalente a 
\begin{subequations}
	\label{formulation:last}
	\begin{align}
		\max_{\vec{x} \in \Z^n} \quad
			& \vec{q}^T\vec{x}, \label{formulation:last:objective} \\
		\text{s.a.} \quad
			& \tau \leq \vec{q}^T\vec{x} \leq \eta, \\
			& A\vec{x} = \vec{b}, \label{formulation:last:constraints} \\
			& \vec{x} \geq \vec{0}, \nonumber
	\end{align}
\end{subequations}

En primer lugar, si $\eta = \infty$, entonces el problema relajado es no acotado y no tiene sentido
usar el pseudocódigo \eqref{algo:multiple} para buscar una solución de este problema. Esta estrategia
es de igual manera usada por todo \textit{solver} de código libre o comercial antes de tan siquiera
buscar una solución.

En segundo lugar, si $\tau = -\infty$ y $\eta < \infty$, entonces el problema
\eqref{formulation:last} es factible y representa exactamente el mismo problema que
\eqref{formulation:multiple}. En este caso, como lo hemos discutido, siempre saldremos del ciclo
infinito en la línea \ref{algo:mult:infloop}, por lo que el método delineado por el pseudocódigo
\eqref{algo:multiple} eventualmente terminará con una solución de este problema.

Finalmente, si $-\infty < \tau \leq \eta < \infty$ nos encontramos en la situación ideal. Esto
se debe a que podemos reemplazar el ciclo en la línea \ref{algo:mult:infloop} por algo del estilo
``\texttt{para} $k \leftarrow \eta$ \texttt{a} $\tau$ \texttt{hacer}...''. Es decir, sabemos exactamente
cuántas capas enteras debemos recorrer para que el método delineado por el pseudocódigo
\eqref{algo:multiple} termine. Observemos que, en este caso, existe la posibilidad de que $P \neq
\emptyset$ pero $P \cap \Z^n = \emptyset$. Sin realizar modificaciones grandes al pseudocódigo
\eqref{algo:multiple}, encontramos que, o bien termina con una solución $\vec{x}^*$ del problema
\eqref{formulation:last}, o bien certifica en $\eta - \tau + 1$ pasos que este problema es infactible.

Cabe mencionar que en la subsección \eqref{subsec:lp} indicamos que existen diversos algoritmos
capaces de resolver rápidamente problemas lineales del estilo \eqref{eq:last:bounds}. Así pues, la
parte de calcular los valores $\tau$ y $\eta$ puede ser considerada como una parte de
preprocesamiento. Recordemos que el método de Ramificación y Acotamiento, en el peor de los casos,
necesita resolver un número exponencial de problemas relajados de \eqref{formulation:last}. Nuestro
método, en cambio, solo necesita resolver, en el peor de los casos, dos problemas relajados.

A modo de conclusión, al autor le gustaría mencionar que futuras líneas de investigación podrían
estar concentradas en resolver el problema de la línea \ref{algo:mult:hard}. Esto se reduce a
investigar sistemas de desigualdades lineales en los enteros. Existen tres posibilidades para estas
investigaciones con respecto al vector de variables libres $\vec{s} \in \Z^{n-m-1}$:
\begin{enumerate}
	\item Decidir la existencia de este vector: si bien no podríamos obtener la solución entera
		$\vec{x}^*$, sí podríamos concluir que $k^*$ es el valor óptimo de
		\eqref{formulation:lattice} y, por el lema \ref{lemma:iso1} así como del teorema
		\ref{th:multeq}, también es el valor óptimo de \eqref{formulation:multiple}.
	\item En caso de tener existencia, determinar el número de estos vectores: además de saber que
		$k^*$ es el valor óptimo de \eqref{formulation:multiple}, también conoceríamos el número de
		soluciones que tiene este problema.
	\item En caso de tener existencia, calcular todos estos vectores: además de saber que $k^*$ es
		el óptimo de \eqref{formulation:multiple} y de conocer cuántas soluciones tiene este
		problema, conoceríamos también cuáles son esas soluciones.
\end{enumerate}

Otra posible futura línea de investigación, más aplicada pero no por ello menos interesante, es
desarrollar las consecuencias del teorema \ref{th:bbspeed}. Es una creencia del autor que los
tiempos de terminación de Ramificación y Acotamiento usando la formulación equivalente
\eqref{formulation:lattice} serán menores que usando la formulación tradicional
\eqref{formulation:multiple}. Para lograr esto, necesitaremos calcular rápida y eficientemente la
matriz $M \in \Z^{n \times (n - 1)}$ y el vector $\vec{\nu} \in \Z^n$ definidos en \eqref{eq:mat-T}
y \eqref{eq:vec-omega}, respectivamente.
