\usepackage[T1]{fontenc}

\usepackage[utf8]{inputenc}
\usepackage[spanish, es-tabla, es-nodecimaldot]{babel}
\usepackage{paperHeaderSpaces}

\usepackage{amsmath}
\usepackage{amssymb}
\usepackage{amsthm}
\usepackage{mathtools}

\usepackage{subcaption}
\usepackage{caption}

\usepackage{graphicx}
\usepackage{geometry}
% \geometry{margin=1in}

\usepackage{tikz}
\usetikzlibrary{intersections}
\usetikzlibrary{babel}

\usepackage{pgfplots}
\pgfplotsset{compat=1.18}
\usepgfplotslibrary{fillbetween}

\usepackage{hyperref}
\usepackage{cleveref}

% tables
\usepackage{booktabs, tabularx, siunitx, subcaption}

\sisetup{
  round-mode=places,
  group-separator = {\,},
  table-number-alignment = center,
  table-text-alignment = center
}

\usepackage[spanish,onelanguage,ruled,vlined,commentsnumbered,rightnl]{algorithm2e}
% \RestyleAlgo{boxed}

\allowdisplaybreaks

% theorems
\newtheorem{theorem}{Teorema}[chapter]
\newtheorem{corollary}[theorem]{Corolario}
\newtheorem{lemma}[theorem]{Lema}

\theoremstyle{definition}
\newtheorem{definition}[theorem]{Definición}

\newtheorem{example}[theorem]{Ejemplo}

\theoremstyle{remark}
\newtheorem*{observation}{Observación}

\numberwithin{theorem}{section}

% custom commands
\newcommand{\Z}{\mathbb{Z}}
\newcommand{\N}{\mathbb{N}}
\newcommand{\Q}{\mathbb{Q}}
\newcommand{\R}{\mathbb{R}}
\newcommand{\F}{\mathbb{F}}
\newcommand{\NIL}{\textnormal{\textsc{NIL}}}

% vectors and linear algebra
\newcommand{\norm}[1]{\left\lVert #1 \right\rVert}
\renewcommand{\ker}[1]{\mathop{\mathrm{ker}{\left\lbrace #1 \right\rbrace}}}
\renewcommand{\vec}[1]{\boldsymbol{#1}}
\newcommand{\inv}[1]{#1^{-1}}
\DeclarePairedDelimiter\braket{\langle}{\rangle}

% algebra
\DeclareMathOperator{\GL}{GL}
\newcommand{\glz}[2]{\GL_{#1}\left(#2\right)}
\DeclareMathOperator{\orb}{orb}
\newcommand{\uvec}[1]{\hat{\vec{#1}}}

% optimization
\DeclareMathOperator{\argmax}{arg\,max}

% convex geom
\DeclareMathOperator{\gen}{gen}
\DeclareMathOperator{\aff}{aff}
\DeclareMathOperator{\conv}{conv}

% shortcuts
\newcommand{\est}[1]{\hat{\vec{ #1 }}}
\newcommand{\optilp}[1]{#1^*_{\text{PE}}}
\newcommand{\optr}[1]{#1^*_{\text{PR}}}
\newcommand{\braces}[1]{\lbrace #1 \rbrace}
\newcommand{\paren}[1]{\left( #1 \right)}
\newcommand{\tvec}[1]{\vec{\tilde{#1}}}
\newcommand{\proy}[1]{\pi^{(#1)}}

\newcommand{\clayer}[2]{H_{#1, #2}}
\newcommand{\qlayer}[2]{\clayer{\vec{#1}}{#2\norm{\vec{#1}}^{-2}}}

\newcommand{\floor}[1]{\left\lfloor #1 \right\rfloor}
\newcommand{\ceil}[1]{\left\lceil #1 \right\rceil}

\renewcommand{\gcd}[1]{\mathop{\mathrm{mcd}{\left\lbrace #1 \right\rbrace}}}
\newcommand{\lcm}[1]{\mathop{\mathrm{mcm}{\left\lbrace #1 \right\rbrace}}}
