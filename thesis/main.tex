\documentclass[11pt, oneside, headings=twolinechapter, spanish]{scrreprt}

% packages
\usepackage{tikz}
\usepackage{subcaption}
\usepackage{caption}

\usepackage[utf8]{inputenc}
\usepackage[spanish]{babel}
\usepackage[T1]{fontenc}
\usepackage{amsmath, amssymb, amsthm}
\usepackage{mathtools}
\usepackage{graphicx}
\usepackage{hyperref}
% \usepackage{natbib}
\usepackage{cleveref}
\usepackage{geometry}
\geometry{margin=1in}

% theorems
\newtheorem{theorem}{Teorema}[chapter]
\newtheorem{corollary}[theorem]{Corolario}
\newtheorem{lemma}[theorem]{Lema}
\newtheorem{definition}[theorem]{Definición}
\newtheorem{example}[theorem]{Ejemplo}

\theoremstyle{remark}
\newtheorem*{observation}{Observación}

% custom commands
\newcommand{\Z}{\mathbb{Z}}
\newcommand{\N}{\mathbb{N}}
\newcommand{\Q}{\mathbb{Q}}
\newcommand{\R}{\mathbb{R}}

% vectors and linear algebra
\newcommand{\norm}[1]{\left\lVert #1 \right\rVert}
\renewcommand{\ker}[1]{\mathop{\mathrm{ker}{\left\lbrace #1 \right\rbrace}}}
\renewcommand{\vec}[1]{\boldsymbol{#1}}

\renewcommand{\gcd}[1]{\mathop{\mathrm{mcd}{\left\lbrace #1 \right\rbrace}}}
\newcommand{\lcm}[1]{\mathop{\mathrm{mcm}{\left\lbrace #1 \right\rbrace}}}


\includeonly{
	chapters/intro,
	chapters/theory,
	chapters/infinite,
	chapters/finite,
	chapters/multiple,
	chapters/conclusion,
	chapters/apps
}

\begin{document}

\input{title}
\input{declaration.tex}

\pagenumbering{Roman}
\chapter*{Agradecimientos}
{
  \setlength{\epigraphwidth}{0.80\textwidth}
  \epigraph{¿Cómo iba yo a saber que la acumulación de esos ``mañana'' que ni siquiera distinguía, y
  que sin notarlo ya eran ``hoy'' y ``ayer'', harían de pasar no solo el tiempo, sino mi tiempo, el
único mío?}{\textit{Josefina Vicens,} \emph{El libro vacío}}
}
\noindent
A mi hermano Cristóbal. A mis padres Javier y Yanitzin. A mis abuelos Isidoro y Gloria, Germán y
Virginia. A mis tíos y a todos mis primos. Agradezco en lo más profundo de mi corazón tener el
privilegio de llamarlos familia. Mi hogar es dondequiera que ustedes se encuentren.

A mis amigos María José Borges, Fernando Yedra, Hugo Nava, Luis Alfonso Maciel, Mauricio Pazos,
Hazel Yáñez, Micho Altamirano, Santiago Prado, José Luis Bravo, Annia Pi-Suñer, Alejandro Naranjo,
Pedro Olivares, Adrián González, Orlando Almazán, Ricardo Bravo, Irvin Ramírez, Adrián Berrón,
Mauricio Díaz, Carlos Rivera, Rodrigo Arjona, Miguel Ángel Torres y Ariadna Irena. Búsquenme si
quieren su agradecimiento personalizado, pues es buena excusa para salir y crear más memorias.

A mi asesor Andreas Wachtel por la inagotable paciencia con la que ha transformado una colección de
símbolos inconexos en una tesis coherente. A mis sinodales Edgar Possani, Edith Vargas García, y
Ezequiel Soto Sánchez por el tiempo que dedicaron a revisar este trabajo.

A todos los que he tenido la fortuna de conocer pero que nuestros caminos han sido interrumpidos.
Las olas de nuestros recuerdos compartidos me atrapan en una marea de nostalgia y melancolía.

\nomenclature[A01]{R\&A}{Ramificación y Acotamiento}
\nomenclature[A02]{FPD}{Formulación de programación dinámica}

\nomenclature[B01]{$\vec{p}$}{Vector esencialmente entero (ver definición \ref{theory:def:rational}).}
\nomenclature[B02]{$\vec{q}$}{Vector coprimo asociado a $\vec{p}$ (ver definición \ref{theory:def:rational}).}
\nomenclature[B03]{$\qlayer{q}{k}$}{$k$-ésima capa entera con parámetro $k$ entero (ver definición \ref{phase-1:def:c-layer}).}
\nomenclature[B04]{$\eta$}{Entero que parametriza la primera capa entera en satisfacer la restricción presupuestaria.}
\nomenclature[B05]{$\vec{\nu}$}{Vector entero (ver definición \eqref{eq:vec-omega}).}
\nomenclature[B06]{$M$}{Matriz entera y rectangular (ver definición \eqref{eq:mat-T}).}
\nomenclature[B07]{$\sigma$}{Símplice de dimensión $m - 1$ generado por un conjunto de $m$ vectores linealmente independientes (ver definición \ref{def:simplex}).}
\nomenclature[B08]{$\est{\sigma}$}{Baricentro del símplice $\sigma$ (ver definición \ref{def:barycenter}).}
\nomenclature[B09]{$\sigma_j$}{$j$-ésima faceta del símplice $\sigma$ (ver definición \ref{def:simplex}).}
\nomenclature[B10]{$\est{\sigma}_j$}{Baricentro de la $j$-ésima faceta $\sigma_j$ del símplice $\sigma$ (ver definición \ref{def:barycenter}).}
\nomenclature[B11]{$\est{\mu}_j$}{Vector unitario normal a $\sigma_j$ que apunta al interior relativo de $\sigma$ y que además es paralelo a $\sigma$.}
\nomenclature[B12]{$S_i$}{Región factible de un subproblema relajado de un programa lineal entero que se resolverá a través de R\&A.}

\nomenclature[C01]{$\gen$}{Espacio vectorial generado por una colección de vectores.}
\nomenclature[C02]{ker}{Espacio nulo de una transformación lineal o de su matriz asociada.}
\nomenclature[C03]{$\img$}{Imagen de una transformación lineal o de su matriz asociada.}
\nomenclature[C04]{$\norm{\cdot}$}{Norma 2 de un vector o de un operador lineal acotado.}
\nomenclature[C05]{$\lfloor \cdot \rfloor$}{Función piso.}
\nomenclature[C06]{$\lceil \cdot \rceil$}{Función techo.}
\nomenclature[C07]{$\coloneq$}{Definición dentro de una expresión matemática.}
\printnomenclature

\chapter*{Resumen}
El algoritmo de \emph{Ramificación y Acotamiento} (R\&A) es uno de los más utilizados para resolver
programas lineales enteros. Este método se basa en el famoso paradigma de ``divide y vencerás'', el
cual combina las soluciones de subproblemas más pequeños a fin de obtener una solución del problema
original. Estos problemas se estructuran en forma de árbol: el problema original genera una
colección de subproblemas, y cada subproblema genera su propia colección de subsubproblemas,
etcétera.

En esta tesis se muestra que existe una colección de programas lineales enteros con una sola
restricción para los cuales el paradigma de ``divide y vencerás'' es inadecuado. En particular, las
simetrías que exhiben estos problemas impiden que R\&A los resuelva eficientemente. Además, se
muestra la existencia de una subcolección de programas lineales enteros cuyos árboles asociados son
tales que R\&A no termina en tiempo finito. Se desarrollan dos nuevos algoritmos que resuelven de
manera más eficiente este tipo de instancias problemáticas a partir de la búsqueda de soluciones de
ecuaciones lineales diofantinas. De manera simultánea, se obtiene un método pseudopolinomial para
calcular cotas superiores del número de Frobenius en el \emph{Problema de la Moneda de Frobenius}.
Finalmente, se generalizan estos algoritmos para encontrar soluciones de programas lineales enteros
con más de una restricción.

\clearpage

\setcounter{tocdepth}{1}
\tableofcontents
\cleardoublepage
\pagenumbering{arabic}

\chapter{Introduction}

This thesis explores integer linear programming through the lens of linear Diophantine equations.

Let us consider the feasibility problem:
\[
\text{Find } x \in \Z_{\geq 0}^n \text{ such that } Ax \leq b.
\]

We analyze its structure, derive bounds, and explore algorithmic solutions.

\chapter{Aspectos Teóricos}

% TODO: escribir la idea general

\noindent
En este capítulo cimentamos las bases teóricas necesarias para resolver instancias particulares de
programas lineales enteros. En primer lugar, la sección de Prerrequisitos recopila resultados
básicos de teoría de números y de programación lineal para refrescar la memoria del lector. En
segundo lugar, la sección de Fundamentos comienza con definiciones y enunciados obtenidos de
\cite{herr}, los cuales utilizaremos para obtener resultados que, en pleno conocimiento del autor,
son originales. El problema fundamental que permitirá construir incrementalmente nuestro algoritmo
es
\begin{subequations}
	\label{theory:formulation}
	\begin{align}
		\max_{\vec{x} \in \Z^n} \quad
			& \vec{p}^T\vec{x}, \label{theory:objective} \\
		\text{s.a.} \quad
			& \vec{p}^T\vec{x} \leq u, \label{theory:constraint:budget} \\
			& \vec{x} \geq \vec{0}. \nonumber
	\end{align}
\end{subequations}

Por ello mismo, es razonable suponer que $p_i \neq 0$ para cualquier $i \in \lbrace 1, \ldots,
n \rbrace$. En la sección de Fundamentos analizaremos a profundidad este problema, cuyo punto de
culminación será el Teorema \ref{theory:th:feasibility}. Veremos que es recomendable separar en dos
partes el análisis de este problema: el caso $p_i < 0$ para alguna $i \in \lbrace 1, \ldots, n
\rbrace$; y el caso $\vec{p} \geq \vec{0}$. Los siguientes dos capítulos examinarán respectivamente
estos casos. Por el momento, cabe destacar que el segundo caso será de mayor interés y tendrá mayor
aplicabilidad en problemas reales, pues es una instancia particular del Problema de la Mochila. No
obstante, el caso $p_i < 0$ también será de utilidad para exhibir casos particulares en donde
el algoritmo de Ramificación y Acotamiento obtiene un rendimiento deficiente.

\section{Prerrequisitos}
\noindent
En los siguientes capítulos usaremos extensivamente resultados básicos de teoría de números y de
programación lineal, por lo que es provechoso recopilarlos en esta primera sección. En
particular, destaca la importancia de las ecuaciones lineales diofantinas para la construcción
de nuestro algoritmo. En esta sección el autor consideró pertinente no incluir demostraciones, pues los
enunciados son mostrados en cualquier clase de álgebra superior, programación lineal, o
investigación de operaciones, por ejemplo. La referencia principal para la parte de teoría de
números es \cite{carmen}, mientras que la de programación lineal es \cite{fabs}.

\subsection{Teoría de Números}
\label{section:number-theory}
\subsubsection{Máximo común divisor y mínimo común múltiplo}
\noindent
En primer lugar, introducimos el símbolo de relación ``$\mid$'' para indicar divisibilidad. Dados
dos enteros $a, b$, decimos que $b$ divide a $a$ (y escribimos $b \mid a$) si y solo si existe un
entero $k$ tal que $a = k \cdot b$. Así también, denotamos el conjunto de divisores de $a$ como
\begin{equation*}
	D(a) \coloneq \lbrace b \in \Z \vcentcolon b \mid a \rbrace.
\end{equation*}
Si $a$ es distinto de cero, encontramos que $D(a)$ es finito, puesto que si $b \mid a$, entonces
$|b| \leq |a|$, lo cual implica que $|D(a)| \leq 2|a|$. En caso de que $a$ sea nulo, obtenemos $D(a)
= \Z$. Observemos también que $\lbrace -1, 1 \rbrace \subseteq D(a)$ para todo entero $a$.

\begin{definition}
	\label{prerreq:def:gcd}
	Sean $a_1, \ldots, a_n$ enteros no todos iguales a cero, entonces definimos su máximo común
	divisor $d$ como el elemento maximal del conjunto $\bigcap_{i=1}^{n}D(a_i)$, y escribimos $d =
	\gcd{a_1, \ldots, a_n}$. Si $d = 1$, entonces decimos que $a_1, \ldots, a_n$ son coprimos.
\end{definition}

Puesto que $a_i \neq 0$ para alguna $i$ en la definición anterior, encontramos que el conjunto
$\bigcap_{i=1}^{n}D(a_i)$ es finito y, como también es no vacío, en efecto existe un elemento maximal.
Es decir, el máximo común divisor $d$ siempre está bien definido. Cabe mencionar que el máximo común
divisor siempre es estrictamente positivo, pues se cumple que $1 \in D(a)$ para todo entero $a$.

Otra definición equivalente para el máximo común divisor normalmente es dada de manera inductiva.
Decimos que $d$ es el máximo común divisor de dos enteros $a_1, a_2$, no ambos nulos, si se
satisface
\begin{enumerate}
	\item $d \mid a_1$ y $d \mid a_2$, y también,
	\item si $d' \mid a_1$ y $d' \mid a_2$, entonces $d' \mid d$.
\end{enumerate}
Luego, para un conjunto de enteros $a_1, a_2 \ldots a_n$, no todos nulos, definimos el máximo común
divisor entre ellos a partir de
\begin{equation*}
	\gcd{a_1, a_2, \ldots, a_{n-1}, a_n} \coloneq \gcd{a_1, \gcd{a_2, \ldots, \gcd{a_{n-1}, a_n}}}.
\end{equation*}
Sin embargo, debemos ser cuidadosos con esta manera de definir las cosas, pues puede ser el caso,
por ejemplo, que $a_{n-1} = a_n = 0$ y entonces $\gcd{a_{n-1}, a_n}$ no está bien definido. Una
manera simple de arreglar esto último es agregar la condición de que $a_n$ sea no nulo. Es por esto
último que el autor prefirió la Definición \ref{prerreq:def:gcd}. Independientemente de cómo
definamos el máximo común divisor, la manera de calcularlo siempre es a través del Algoritmo de
Euclides. A partir de este punto usaremos ambas definiciones de manera indistinta.

\begin{observation}
	No porque una colección de enteros sea coprima ($\gcd{a_1, \ldots, a_n} = 1$) se sigue que
	estos enteros sean coprimos a pares ($\gcd{a_i, a_j} = 1$ para todo $i, j$). Por ejemplo,
	los enteros 1, 3 y 3 son coprimos pero evidentemente 3 y 3 no lo son.
\end{observation}

\begin{definition}
	Decimos que $c \in \Z$ es una combinación lineal entera de un conjunto de enteros $a_1, \ldots,
	a_n$ si existen enteros $x_1, \ldots, x_n$ tales que $c = a_1x_1 + \cdots + a_nx_n$. Si $c$ es
	positivo, también decimos que esto último es una combinación lineal positiva.
\end{definition}

El siguiente teorema, a pesar de su simpleza, es central para los resultados obtenidos en esta
tesis.
\begin{theorem}
	\label{prerreq:th:bezout}
	Sea $d$ un entero y sean $a_1, \ldots, a_n$ una colección de enteros no todos iguales a cero.
	Entonces $d = \gcd{a_1, \ldots, a_n}$ si y solo si $d$ es la mínima combinación lineal entera
	positiva de $a_1, \ldots, a_n$.
\end{theorem}

\begin{example}
	Para $a_1 \coloneq 2$, $a_2 \coloneq 3$ y $a_3 \coloneq 5$ se cumple que $\gcd{a_1, a_2, a_3} =
	1$ y además $-3a_1 - a_2 + 2a_3 = 1$.
\end{example}

\begin{lemma}
	\label{prerreq:lemma:gcd}
	Si $d = \gcd{a_1, \ldots, a_n}$, entonces $\gcd{\frac{a_1}{d}, \ldots, \frac{a_n}{d}} = 1$.
\end{lemma}

Además del máximo común divisor, requeriremos al mínimo común múltiplo, empero en menor medida. Sea
$a$ un entero y denotamos el conjunto de sus múltiplos como
\begin{equation*}
	M(a) \coloneq \lbrace x \in \Z \vcentcolon a \mid x \rbrace.
\end{equation*}
Si $a$ es nulo, entonces $M(a) = \lbrace 0 \rbrace$. En caso contrario encontramos que $M(a)$ es un
conjunto infinito. Ánalogamente a la Definición \ref{prerreq:def:gcd}, definimos el mínimo común
múltiplo $m$ de una colección de enteros $a_1, \ldots, a_n \in \Z \setminus \lbrace 0 \rbrace$ como
el elemento minimal de $\N \cap \bigcap_{i=1}^{n}M(a_i)$. Escribimos $m = \lcm{a_1, \ldots, a_n}$.
Para observar que está bien definido, basta mencionar que el producto $|a_1 \cdots a_n|$ es un
elemento de la intersección y por lo tanto esta es no vacía.

\subsubsection{Ecuaciones lineales diofantinas}

\noindent
Sea $c \in \Z$ y sean $a_1, \ldots, a_n$ enteros. Una ecuación lineal diofantina es una ecuación
donde queremos encontrar enteros $x_1, \ldots, x_n$ que satisfagan
\begin{equation*}
	a_1x_1 + \cdots + a_nx_n = c.
\end{equation*}
Será de nuestro interés en las siguientes secciones resolver iterativamente este tipo de ecuaciones.
Por el momento basta mencionar que podemos enfocarnos en el caso $n = 2$ sin ninguna pérdida de
generalidad. No obstante, los resultados se mantienen para cualquier $n \in \N$. Los siguientes
enunciados abordan el problema de determinar existencia y unicidad para las ecuaciones lineales
diofantinas, así como la construcción de sus soluciones.

\begin{theorem}[Existencia]
	\label{prerreq:th:existence}
	Sean $a, b \in \Z$, no ambos cero. La ecuación $ax + by = c$ tiene solución entera si y solo si
	$\gcd{a, b} \mid c$.
\end{theorem}

Para construir el conjunto de soluciones a una ecuación lineal diofantina, encontramos primero una
solución particular.
\begin{definition}
	\label{prerreq:def:bezout}
	Sea $d \coloneq \gcd{a, b}$ y sean $x', y'$ enteros tales que $ax' + by' = d$ (c.f. Teorema
	\ref{prerreq:th:bezout}). Decimos entonces que $x', y'$ son coeficientes de Bézout asociados a
	$a, b$, respectivamente\footnote{
		Los coeficientes de Bézout se pueden calcular a través del Algoritmo Extendido de Euclides.
		Véase \url{https://en.wikipedia.org/wiki/Extended_Euclidean_algorithm}.
	}.
\end{definition}

\begin{observation}
	Los coeficientes de Bézout asociados a un par de enteros no son únicos. En efecto, si $x', y'$
	son coeficientes de Bézout de $a, b$, entonces $x' + b$, $y' - a$ también lo son:
	\begin{equation*}
		a(x' + b) + b(y' - a) = ax' + by' + ab - ab = ax' + by' = d.
	\end{equation*}
	Para fines de esta tesis basta la existencia de estos coeficientes, por lo que decimos de manera
	indistinta ``los coeficientes de Bézout'' y ``una elección de coeficientes de Bézout''.
\end{observation}

Definamos $d \coloneq \gcd{a, b}$ y supongamos que la ecuación $ax + by = c$ tiene solución.
Por el Teorema \ref{prerreq:th:existence}, se sigue que $d \mid c$, y entonces existe $c' \in \Z$
tal que $c = c' \cdot d$. Sean $x', y'$ los coeficientes de Bézout asociados a $a, b$
respectivamente. Así,
\begin{equation*}
	a(c' \cdot x') + b(c' \cdot y') = c'(ax' + by') = c'd = c,
\end{equation*}
por lo que $(c' \cdot x', c' \cdot y')$ es una solución particular de la ecuación $ax + by = c$.

\begin{theorem}[Construcción]
	\label{prerreq:th:construction}
	Sea $(x_0, y_0)$ una solución particular de la ecuación lineal diofantina $ax + by = c$.
	Entonces todas las soluciones de la ecuación están dadas por
	\begin{equation}
		\label{prerreq:eq:construction}
		\begin{cases}
			x = x_0 + \frac{b}{d}t, \\
			y = y_0 - \frac{a}{d}t,
		\end{cases}
	\end{equation}
	donde $d \coloneq \gcd{a, b}$ y $t \in \Z$.
\end{theorem}

\begin{example}
	Consideremos la ecuación lineal $2x + 3y = 5$. Los coeficientes de Bézout asociados a 2 y 3 son,
	respectivamente, -1 y 1. Luego, una solución particular para la ecuación es $(x_0, y_0) = (-5, 5)$.
	Por el Teorema anterior encontramos que todas las soluciones están dadas por
	\begin{equation*}
		\begin{cases}
			x = -5 + 3t, \\
			y = 5 - 2t,
		\end{cases}
	\end{equation*}
	donde $t \in \Z$ es una variable libre. En efecto, sustituyendo obtenemos
	\begin{equation*}
		2(-5 + 3t) + 3(5 - 2t) = -10 + 15 + 6t - 6t = 5.
	\end{equation*}
\end{example}

\subsection{Programación lineal}
\noindent
La programación lineal se encarga de resolver problemas de optimización de la forma
\begin{equation}
	\label{prim:lineal-opt}
	\max_{\vec{x}} ~\lbrace \vec{c}^T\vec{x} \colon \vec{x} \in P \rbrace,
\end{equation}
donde $P$ es un poliedro. En esta sección repasamos brevemente propiedades del poliedro $P$ al cual
llamamos región factible. Así también, indicamos dónde se encuentra el óptimo del problema y
hacemos mención rápida sobre cómo obtenerlo. Finalmente, nos enfocamos en programas lineales enteros
y, más importantemente, describimos cómo funciona el algoritmo de Ramificación y Acotamiento para
encontrar sus soluciones.

\begin{definition}
	Sea $\vec{a} \in \R^n$ un vector no nulo y sea $b \in \R$ un escalar. Llamamos hiperplano afino
	al conjunto de vectores $\vec{x} \in \R^n$ que satisfacen $\vec{a}^T\vec{x} = b$. Así también,
	llamamos semi-espacios afinos a los conjuntos de vectores $\vec{x}, \vec{y} \in \R^n$ que
	satisfacen $\vec{a}^T\vec{x} \geq b$ y $\vec{a}^T\vec{y} \leq b$.
\end{definition}

\begin{definition}
	Sea $A \in \R^{m \times n}$ una matriz con renglones linealmente independientes y $\vec{b} \in
	\R^m$ un vector. Entonces al conjunto definido por
	\begin{equation}
		\label{prerreq:def:poly}
		P \coloneq \lbrace \vec{x} \in \R^n \colon A\vec{x} \geq b \rbrace
	\end{equation}
	lo llamamos poliedro. Si, además, $P$ es acotado, entonces decimos que $P$ es un politopo.
\end{definition}

\begin{observation}
	Todo poliedro $P$ definido de esta manera representa la intersección de $m$ semi-espacios
	afinos. Esto se debe a que $A\vec{x} \geq \vec{b}$ si y solo si $\vec{a}_i^T\vec{x} \geq b_i$
	para toda $1 \leq i \leq m$ y donde $\vec{a}_i^T$ representa el $i$-ésimo renglón de la matriz
	$A$. En la Figura \ref{fig:hyp} se muestra visualmente esta relación entre hiperplanos afinos y
	poliedros.
\end{observation}

\begin{figure}[ht]
	\centering
	\begin{minipage}{0.45\textwidth}
		\centering
		\begin{tikzpicture}[scale=1.2]
			% hyperplane
			\draw[ultra thick, black] (-2,1) -- (2,-2) node[above right] {$\vec{a}^T\vec{x} = b$};
			% labels
			\node at (-1.2,-1.2) {${\vec{a}^T \vec{x} \leq b}$};
			%shading
			\fill[gray!15, domain=-2:2, variable=\x]
				(-2,1) -- plot ({\x}, {-0.75*\x - 0.5}) -- (2,1) -- cycle;
			\draw[->, thick] (1,-1.25) -- (1.5,-0.583) node[right] {$\vec{a}$};
			\node at (0.8,0.5) {${\vec{a}^T \vec{x} \geq b}$};
		\end{tikzpicture}
	\end{minipage}
	\begin{minipage}{0.45\textwidth}
		\centering
		\begin{tikzpicture}[scale=2, >=stealth]
			% coordinates
			\coordinate (A) at (0,0);
			\coordinate (B) at (1.5,0);
			\coordinate (C) at (1.8,1);
			\coordinate (D) at (0.75,1.7);
			\coordinate (E) at (-0.3,1);
			% pentagon
			\filldraw[fill=gray!15, thick]
			(A) -- (B) -- (C) -- (D) -- (E) -- cycle;
		  % normal vectors
			% AB
			\draw[->, black, thick] (0.75,0) -- (0.75,0.46) node[below right] {$\vec{a}_1$};
			% BC
			\draw[->, black, thick] (1.65,0.5) -- (1.34,0.59) node[above right] {$\vec{a}_2$};
			% CD
			\draw[->, black, thick] (1.275,1.35) -- (1.06,1.027) node[above left] {$\vec{a}_3$};
			% DE
			\draw[->, black, thick] (0.225,1.35) -- (0.44,1.027) node[below left] {$\vec{a}_4$};
			% EA
			\draw[->, black, thick] (-0.15,0.5) -- (0.157,0.592) node[below right] {$\vec{a}_5$};
			% polyhedron label
			\node at (0.75, 0.75) {$P$};
		\end{tikzpicture}
	\end{minipage}
	\caption{\textit{Izquierda:} Un hiperplano afino $\braces{\vec{x} \colon \vec{a}^T\vec{x} = b}$
	junto con los dos semi-espacios que induce. \textit{Derecha:} Todo poliedro $P$ es la intersección
	de un conjunto finito de semi-espacios inducidos por un hiperplano afino.}
	\label{fig:hyp}
\end{figure}

\begin{definition}
	Sea $P$ un poliedro. Decimos que el vector $\vec{x} \in P$ es un vértice de $P$ si existe
	$\vec{c} \in \R^n$ de manera que $\vec{c}^T\vec{x} < \vec{c}^T\vec{y}$ para todo $\vec{y} \in P
	\setminus \lbrace \vec{x} \rbrace$.
\end{definition}
En términos gráficos, decimos que $\vec{x}$ es un vértice si se satisfacen dos condiciones: en
primer lugar, existe un hiperplano afino que pasa por $\vec{x}$ y uno de sus semi-espacios inducidos
contiene completamente al poliedro $P$; en segundo lugar, ningún otro punto de $P$ se encuentra
sobre este hiperplano.

\begin{definition}
Sea $P$ un poliedro y sea $\vec{c} \in \R^n$ un vector. Todo problema de optimización de la forma
\eqref{prim:lineal-opt} entra en una de las siguientes tres categorías:
\begin{enumerate}
	\item El valor óptimo no existe. Esto ocurre porque ningún vector $\vec{x} \in \R^n$ satisface
		el sistema de desigualdades $A\vec{x} \geq \vec{b}$. Es decir, la región factible es vacía.
	\item El valor óptimo existe y es infinito. Esto ocurre porque el poliedro $P$ no es acotado y
		somos capaces de encontrar una sucesión de vectores $\lbrace \vec{x}_k \rbrace_{k \in \N}$
		en el poliedro $P$ que satisface $\vec{c}^T\vec{x}_{k+1} > \vec{c}^T\vec{x}_k$ para todo $k \in \N$.
	\item El valor óptimo existe y es finito. Este caso es la negación de los dos casos anteriores,
		pero cabe recalcar que esto no significa que el poliedro $P$ es acotado.
\end{enumerate}
En el primer caso decimos que \textbf{el problema es infactible}, mientras que en los últimos dos
decimos que \textbf{el problema es factible}. También diremos comúnmente del segundo caso que
\textbf{el problema es no acotado}.
\end{definition}

Es posible mostrar que todo poliedro $P \coloneq \lbrace \vec{x} \in \R^n \colon A\vec{x} \geq
\vec{b} \rbrace$ puede ser transformado a la forma estándar
\begin{equation*}
	\lbrace \left( \vec{x}^+, \vec{x}^-, \vec{s} \right) \in \R^{n + n + m} \colon A(\vec{x}^+ -
\vec{x}^-) - \vec{s} = \vec{b}, \left(\vec{x}^+, \vec{x}^-, \vec{s}\right) \geq \vec{0}\rbrace,
\end{equation*}
de manera que todo problema de optimización de la forma \eqref{prim:lineal-opt} puede ser escrito
sin pérdida de generalidad como
\begin{subequations}
	\label{prerreq:formulation}
	\begin{align}
		\max_{\vec{x} \in \R^n} \quad
			& \vec{c}^T\vec{x}, \label{prerreq:formulation:objective} \\
		\text{s.a.} \quad
			& A\vec{x} = \vec{b}, \label{prerreq:formulation:constraints} \\
			& \vec{x} \geq \vec{0} \nonumber,
	\end{align}
\end{subequations}
donde ``s.a.'' es una abreviación de ``sujeto a''. De ahora en adelante, nuestro análisis se
concentrará exclusivamente en problemas lineales de este tipo. Es decir, supondremos, sin pérdida de
generalidad, que todo problema lineal se encuentra en esta forma estándar.

\begin{theorem}
	\label{prerreq:th:linear-sol}
	Sea $P$ un poliedro que tiene al menos un vértice, consideremos el problema
	\eqref{prerreq:formulation}, y supongamos que el valor óptimo $z^*$ existe y es finito. Entonces
	el conjunto de soluciones óptimas contiene al menos un vértice de $P$.
\end{theorem}

Este Teorema fundamental constituye el primer paso para la construcción de varios algoritmos que
encuentran soluciones del problema \eqref{prerreq:formulation}. Ciertamente el más famoso de todos
es el algoritmo simplex, el cual ``salta'' de vértice en vértice hasta llegar a uno con valor
óptimo. Otros, más modernos y conocidos como métodos de puntos interiores, comienzan en el interior
del poliedro $P$ y son ``atraídos'' como imanes a uno de los vértices con valor óptimo. No es el
objetivo de esta tesis exponer la maquinaria matemática detrás de estos algoritmos\footnote{
	Sin embargo, la literatura para explicar estos métodos es harto abundante. Véase, por ejemplo,
	\cite{nocedal}.
}.

Ahora describimos brevemente los programas lineales enteros y pasamos a explicar el método de
Ramificación y Acotamiento. Por ello, lo que se encuentra a continuación supone que contamos con un
algoritmo para resolver problemas del tipo \eqref{prerreq:formulation}.
\begin{definition}
	Sea $A \in \R^{m \times n}$ una matriz con renglones linealmente independientes y sea $\vec{b}
	\in \R^m$ un vector. Al problema de optimización lineal \eqref{prerreq:formulation} lo llamamos
	problema relajado del programa lineal entero
	\begin{subequations}
		\label{prerreq:formulation:ilp}
		\begin{align}
			\max_{\vec{x} \in \Z^n} \quad
				& \vec{c}^T\vec{x}, \label{prerreq:formulation:objective:ilp} \\
			\text{s.a.} \quad
				& A\vec{x} = \vec{b}, \label{prerreq:formulation:constraints:ilp} \\
				& \vec{x} \geq \vec{0} \nonumber.
		\end{align}
	\end{subequations}
\end{definition}
Resalta el hecho de que la formulación de un programa lineal entero es idéntico a su formulación
relajada, solamente agregamos la restricción de que nuestro vector solución $\vec{x}^*$
sea entero. Es decir, lo único que cambia es la región de factibilidad. De hecho, si
definimos el poliedro
\begin{equation*}
	P \coloneq \lbrace \vec{x} \in \R^n \colon A\vec{x} = \vec{b}, \vec{x} \geq \vec{0} \rbrace,
\end{equation*}
entonces tenemos que $P \cap \Z^n$ corresponde a la región factible de
\eqref{prerreq:formulation:ilp}, mientras que $P$ corresponde a la región factible de su problema
relajado.

A partir de lo anterior, deducimos inmediatamente que la solución óptima $\optilp{z}$ de un programa
entero es una cota inferior del óptimo $z^*$ de su problema relajado, pues ambos son problemas de
maximización y es cierto que $P \cap \Z^n \subseteq P$. De aquí se sigue entonces que si $z^* =
\optilp{z}$, entonces la solución óptima $\vec{x}^*$ del problema relajado también es la solución
óptima del programa lineal entero.

Para resolver problemas lineales enteros más generales, comúnmente se utiliza el algoritmo de
Ramificación y Acotamiento. Este método consiste en generar un árbol binario donde cada nodo
representa un subproblema lineal a resolver. En la raíz del árbol resolvemos el problema relajado
\eqref{prerreq:formulation} y, si la solución óptima $\vec{x}^* \in \R^n$ no es entera, entonces
para alguna entrada $x_i^*$ no entera agregamos la restricción $x_i \leq \lfloor x_i^* \rfloor$ para
crear un subproblema, y también añadimos la restricción $x_i \geq \lceil x_i^* \rceil$ para crear
otro subproblema. Este procedimiento se realiza de manera recursiva.

Observemos que, si decidimos recorrer todos los nodos del árbol binario, entonces tendremos que
resolver al menos $2^n$ subproblemas, donde $n$ es la dimensión del problema lineal. Por esta razón,
el algoritmo cuenta con políticas para deshacerse de subárboles que nunca proveerán la solución
óptima. El autor considera que es mejor ilustrar estas políticas a partir de un ejemplo. El
Algoritmo \ref{algo:bb} en el Apéndice \ref{app:bb} presenta una versión rudimentaria del método de
Ramificación y Acotamiento.

\begin{example}[\cite{fabs}]
	\label{ex:ilp}
	Consideremos el programa lineal entero
	\begin{align*}
		\max_{\vec{x} \in \Z^2} \quad
			& 4x_1 - x_2, \\
			\text{s.a.} \quad
			& 7x_1 - 2x_2 \leq 14, \\
			& x_2 \leq 3, \\
			& x_1, x_2 \geq 0.
	\end{align*}
	La región factible de este problema se muestra en la Figura \ref{fig:feas}. La solución al problema
	relajado, cuya región factible denotamos por $S_0$, está dada por $\vec{x}^0 \coloneq (20/7,
	3)^T$. Como $x_1^0 = 20/7$ no es entero, generamos dos nuevos subproblemas con regiones
	factibles
	\begin{align*}
		S_{00} &\coloneq S_0 \cup \braces{ x_1 \leq \floor{20/7} = 2}, \\
		S_{01} &\coloneq S_0 \cup \braces{ x_1 \geq \ceil{20/7} = 3}.
	\end{align*}
	De la Figura \ref{fig:feas}, observamos que $S_{01}$ es vacío y por lo tanto de este problema no
	podemos generar otros subproblemas. En este caso, decimos que \textbf{podamos $S_{01}$ por
	infactibilidad}.

	Ahora bien, la solución al problema $S_{00}$ está dada por $\vec{x}^1 \coloneq (2, 1/2)^T$.
	Encontramos que $x_2^1$ = 1/2 no es entero y por lo tanto generamos dos nuevos subproblemas:
	\begin{align*}
		S_{000} &\coloneq S_{00} \cup \braces{ x_2 \leq \floor{1/2} = 0}, \\
		S_{001} &\coloneq S_{00} \cup \braces{ x_2 \geq \ceil{1/2} = 1}.
	\end{align*}
	Observemos que la solución $\vec{x}^2$ de $S_{001}$ es $(2, 1)^T$, la cual es entera y tiene valor
	objetivo $z_2^* \coloneq 7$. No generamos otros subproblemas a partir de este problema porque
	sus regiones factibles estarán contenidas en $S_{001}$ y por lo tanto sus valores objetivos
	serán menores o iguales al de $S_{001}$. Así pues, decimos que \textbf{podamos $S_{001}$ por
	integralidad}.

	La solución de $S_{000}$, en cambio, es $\vec{x}^3 \coloneq (3/2, 0)^T$ y tendríamos que ramificar
	de nuevo en otros dos subproblemas. No obstante, observemos que el valor objetivo de este
	subproblema es $z_3^* \coloneq 6$, el cual es menor que $z_2^* = 7$. Como la región factible de
	cualquier subproblema generado a partir de este último problema estára contenido en $S_{000}$,
	se sigue que su valor objetivo será menor o igual al de $S_{000}$ Decimos entonces que
	\textbf{podamos $S_{000}$ por cota}.

	Como hemos agotado todos los subproblemas que podríamos generar, entonces concluimos que la
	solución óptima de este programa  lineal entero es $\vec{x}^2 = (2, 1)^T$ y tiene valor objetivo
	$z_2^* = 7$.
\end{example}
\begin{figure}
	\centering
	\begin{tikzpicture}[scale=1.2]
		\begin{axis}[
			axis lines=left,
			xmin=-0.5, xmax=3.5,
			ymin=-0.5, ymax=3.5,
			grid=both,
			xlabel=$x_1$,
			ylabel=$x_2$
			]
			\addplot+ [thick,color=black,fill=gray!15,mark=none] table {
				0 0
				1.5 0
				2.2 0.7
				2.85 3
				0 3
				0 3
				0 0
			};
			\addplot+ [only marks,,mark=*,mark options={scale=1, color=black, fill=black}] table {
				0 0
				0 1
				0 2
				0 3
				1 0
				1 1
				1 2
				1 3
				2 1
				2 2
				2 3
			};
		\end{axis}
	\end{tikzpicture}
	\caption{Los puntos negros forman la región factible del programa lineal entero del Ejemplo
	\ref{ex:ilp}, mientras que la región sombreada es la región factible de su problema relajado.}
	\label{fig:feas}
\end{figure}

\section{Fundamentos}
\noindent
Esta sección constituye el primer paso para la construcción de nuestro algoritmo. Se divide en dos
partes. Primeramente damos a conocer las definiciones y enunciados provistos por \cite{herr}, al
mismo tiempo que hacemos un par de observaciones. Esta primera parte puede darse por concluida una
vez citado el Teorema \ref{phase-1:th:cover}. Así también, es importante aclarar que el autor
tradujo libremente algunos términos a falta de encontrar fuentes en español que hicieran uso de
ellos. A saber, el autor decidió nombrar ``vectores esencialmente enteros'' a los
\textit{projectively rational vectors} y ``capas enteras'' a los \textit{c-layers} en las
Definiciones \ref{theory:def:rational} y \ref{phase-1:def:c-layer}, respectivamente.

En la segunda parte de esta sección comenzamos con nuestro análisis del problema
(\ref{theory:formulation}). La razón de considerarlo fundamental para esta tesis fue mencionado en
el capítulo de Motivación, pero lo repetimos una vez más: en esta clase de problemas el vector es
ortogonal a la única restricción, y esto implica que el problema relajado tenga una infinidad de
soluciones. Hemos observado que, en presencia de este fenómeno, el algoritmo de Ramificación y
Acotamiento no divide la región factible de manera óptima. Por ello investigamos formas alternativas
para atacar este problema antes de hacer la separación de casos $p_i \leq 0$ para alguna $i \in
\lbrace 1, \ldots, n \rbrace$ o $\vec{p} > \vec{0}$.
\begin{definition}
	\label{theory:def:rational}
	Decimos que un vector $\vec{v} \in \R^n \setminus \lbrace \vec{0} \rbrace$ es esencialmente
	entero si existe un vector $\vec{w} \in \Z^n$ y un escalar $m \in \R \setminus \lbrace 0
	\rbrace$ tal que $\vec{v} = m\vec{w}$. Además, decimos que $\vec{w}$ es el múltiplo coprimo de
	$\vec{v}$ si sus entradas son coprimas (c.f. Definición \ref{prerreq:def:gcd}) y si su primera
	entrada no nula $v_i$ también es positiva.
\end{definition}
En otras palabras, decimos que $\vec{v}$ es esencialmente entero si es un múltiplo real de un vector
entero.
\begin{example}
	El vector $\left(-\sqrt{2}, 1/\sqrt{2}\right)^T = \sqrt{2}(-1, 1/2)^T$ es esencialmente entero
	y $(2, -1)^T$ es su múltiplo coprimo. Contrariamente, el vector $(\sqrt{2}, \sqrt{3})^T$ no es
	esencialmente entero.
\end{example}
\begin{observation}
	Todo vector $\vec{v}$ cuyas entradas son racionales ($\vec{v} \in \Q^n$) es esencialmente
	entero. En efecto, $v_i = \frac{p_i}{q_i}$ para algunos enteros $p_i$ y $q_i$ con $q_i$
	distinto de cero. Si definimos $m \coloneq \lcm{q_1, \ldots, q_n} \neq 0$ y $\vec{w} \coloneq
	m\vec{v}$, se sigue que $\vec{v} = \frac{1}{m}\vec{w}$ y también $\vec{w} \in \Z^n$.
\end{observation}
\begin{observation}
	\label{obs:coprime-unique}
	Todo vector $\vec{v}$ esencialmente entero tiene a lo más dos vectores coprimos asociados. Sean
	$m \in \R$ y $\vec{w} \in \Z^n$ tales que $\vec{v} = m\vec{w}$. Entonces
	\begin{equation*}
		\pm \frac{1}{\gcd{w_1, \ldots, w_n}}\vec{w}
	\end{equation*}
	son dos vectores cuyas entradas son coprimas, de acuerdo al Lema \ref{prerreq:lemma:gcd}. Como
	la primera entrada no nula $w_i$ también debe ser positiva, se sigue que solo uno de estos dos
	vectores es el múltiplo coprimo de $\vec{v}$. Así, el múltiplo coprimo de un vector
	esencialmente entero es único.
\end{observation}

Porque todo número representable en cualquier sistema de aritmética finita es necesariamente
racional, decidimos enfocar nuestro análisis en vectores esencialmente enteros. Desde el punto de
vista puramente teórico, esta condición reduce drásticamente el tipo de programas lineales que
podemos resolver. No obstante, esta clase de vectores es un poco más general que los considerados en
otros textos de programación lineal, por ejemplo, \cite{martello} y \cite{alex} toman en cuenta
vectores puramente racionales. En \cite{herr} se revelan propiedades de los vectores esencialmente
enteros que reproducimos aquí y que nos permitirán plantear ecuaciones lineales diofantinas cuyas
soluciones otorgan candidatos para puntos óptimos de un problema lineal.

\begin{definition}
	\label{phase-1:def:c-layer}
	Sea $\vec{v} \in \R^n$ un vector esencialmente entero y sea $t \in \R$ un escalar. Decimos que
	su hiperplano afino asociado
	\begin{equation}
		\label{phase-1:def:affine-hyperplane}
		H_{\vec{v}, t} \coloneq \ker{\vec{x} \mapsto \vec{v}^T\vec{x}} + t\vec{v}
		= \lbrace \vec{v}^{\perp} + t\vec{v} \vcentcolon \vec{v}^T\vec{v}^{\perp} = 0 \rbrace
	\end{equation}
	es una capa entera si contiene al menos un punto entero.
\end{definition}
Observemos que todo hiperplano afino $H_{\vec{v}, t}$ es invariante ante reescalamientos en
$\vec{v}$. Es decir, si $r \in \R \setminus \lbrace 0 \rbrace$ es un escalar, entonces $H_{\vec{v},
t} = H_{r\vec{v}, t/r}$. En particular, el conjunto de hiperplanos afinos asociados a un vector
$\vec{v}$ esencialmente entero es igual al conjunto de hiperplanos afinos asociados a su múltiplo
coprimo $\vec{w}$. Ahora bien, cualquier vector coprimo induce una familia de capas enteras y,
sorprendentemente, esa familia forma una cobertura de $\Z^n$, como lo indica el Teorema
\ref{phase-1:th:cover}.

% \begin{figure}
% 	\centering
% 	\begin{tikzpicture}[scale=1.0, xscale=1, yscale=0.7]
% 		\centering
% 		\draw (0,0) -- (5,0) node[right] {\(x\)};
% 		\draw (0,0) -- (0,7) node[above] {\(y\)};
% 		% Add tick marks (optional)
% 
% 		\foreach \x in {1,2,3,4}
% 			\draw (\x,0.1) -- (\x,-0.1) node[below] {\x};
% 
% 		\foreach \y in {1,2,3,4,5,6}
% 			\draw (0.1,\y) -- (-0.1,\y) node[left] {\y};
% 
% 		\draw (0,0.1) -- (0, -0.1) node[below left] {0};
% 		\draw (0.1,0) -- (-0.1, 0) node[below left] {};
% 
% 		\draw[very thin, gray!30] (0,0) grid (5,7);
% 
% 		\draw[thick, black, domain=0:3.5] plot (\x, {7 - 2*\x}) node[right] {};
% 		\draw[thick, gray, domain=0:5.0] plot (\x, {9/sqrt(2) - sqrt(1.5)*\x}) node[right] {};
% 
% 		\filldraw[black] (3,1) circle (2pt) node[below right] {};
% 		\filldraw[black] (2,3) circle (2pt) node[above right] {};
% 		\filldraw[black] (1,5) circle (2pt) node[above left] {};
% 	\end{tikzpicture}
% 	\caption{Representación de una capa entera (en negro) junto a un hiperplano afino que no es capa
% 	entera (en gris). La capa entera tiene como parámetros $\vec{v} = (2, 1)^T$ y $t = 1.4$,
% 	mientras que los del hiperplano afino son $\vec{v} = (\sqrt{3}, \sqrt{2})^T$ y $t = 1.4$.}
% 	\label{phase-1:fig:c-layer}
% \end{figure}

\begin{lemma}
	\label{phase-1:lemma:layer}
	Sean $\vec{v}, \vec{x} \in \R^n$ con $\vec{v}$ distinto de cero. Entonces $\vec{x} \in
	H_{\vec{v}, t_{\vec{x}}}$, donde $t_{\vec{x}} \coloneq \frac{\vec{v}^T\vec{x}}{\norm{\vec{v}}^2}$.
\end{lemma}

\begin{theorem}
	\label{phase-1:th:cover}
	Sea $\vec{v} \in \R^n$ un vector esencialmente entero y sea $\vec{w}$ su múltiplo coprimo.
	Entonces la familia de capas enteras $\left\lbrace H_{\vec{w}, k\norm{\vec{w}}^{-2}} \vcentcolon k
			\in \Z \right\rbrace$ cubre a $\Z^n$.
\end{theorem}

Pasemos a considerar el programa lineal (\ref{theory:formulation}) donde $\vec{p}$ es un vector
esencialmente entero y $\vec{q}$ es su múltiplo coprimo. Comúnmente a la función objetivo
(\ref{theory:objective}) le daremos el nombre de utilidad y a la restricción
(\ref{theory:constraint:budget}) la llamaremos restricción presupuestaria, así como presupuesto al
lado derecho de esta restricción.
\begin{observation}
	Debido a la restricción presupuestaria, encontramos que el politopo está acotado por arriba. Así
	pues, el problema o bien es infactible, o bien tiene una utilidad finita.
\end{observation}

Cada escalar $t \in \R$ induce un hiperplano afino $H_{\vec{p}, t}$ donde se cumple que todo punto
$\vec{x} \in H_{\vec{p}, t}$ tiene un mismo nivel de utilidad. Como observamos previamente,
\begin{equation*}
	\left \lbrace H_{\vec{p}, t} \vcentcolon t \in \R \right\rbrace
	=
	\left \lbrace H_{\vec{q}, t} \vcentcolon t \in \R \right\rbrace.
\end{equation*}
A causa del Teorema \ref{phase-1:th:cover}, somos capaces de caracterizar todos los puntos enteros a
partir de $\vec{q}$. Aún más, obtenemos una enumeración de las capas enteras que cubren $\Z^n$, lo
cual nos permite determinar si la $k$-ésima capa entera contiene puntos factibles para el problema.

\begin{lemma}
	\label{theory:lemma:utility}
	Sea $\vec{x} \in H_{\vec{q}, k\norm{\vec{q}}^2}$, entonces su nivel de utilidad
	$\vec{q}^T\vec{x}$ es $k$.
\end{lemma}
\begin{proof}
	Sea $\vec{x} \in H_{\vec{q}, k\norm{\vec{q}}^{-2}}$, entonces tenemos
	\begin{equation*}
		\vec{x} = \vec{q}^{\perp} + k\norm{\vec{q}}^{-2}\vec{q},
	\end{equation*}
	donde $\vec{q}^{\perp}$ es un vector ortogonal a $\vec{q}$. Por lo tanto,
	\begin{equation*}
		\vec{q}^T\vec{x} = \vec{q}^T\vec{q}^{\perp} + k\norm{\vec{q}}^{-2}\vec{q}^T\vec{q}
		= 0 + k \norm{\vec{q}}^{-2} \norm{\vec{q}}^{2} = k.
	\end{equation*}
\end{proof}

Consideremos el vector esencialmente entero $\vec{p}$ y su múltiplo coprimo $\vec{q}$, entonces
existe un escalar $m \in \R \setminus \lbrace 0 \rbrace$ tal que $\vec{p} = m\vec{q}$. Si la
restricción \eqref{theory:constraint:budget} se cumple, es decir $\vec{p}^T\vec{x} \leq u$, también
se cumple que $\vec{q}^T\vec{x} \leq u/m$ si $m$ es positivo, o bien que $\vec{q}^T\vec{x} \geq u/m$
si $m$ es negativo.

La gran mayoría de resultados que obtendremos dependerán de un entero que denotamos como $\eta$, el
cual depende de $m$ y por lo tanto del signo que este tenga. Para evitar ser repetitivos o dividir
los resultados innecesariamente en casos, supondremos de ahora en adelante que $m$ es positivo. Esto
equivale a decir que la primera entrada no nula $p_i$ es positiva, pues se debe cumplir que $q_i$
sea positivo. Basta mencionar que la gran mayoría de desigualdades se invierten en caso de que $m$
sea negativo, y también que usamos la función techo en vez de la función piso.

Para respetar la restricción presupuestaria, podemos encontrar el entero $\eta$ más grande que
satisfaga $\vec{q}^T\vec{x} \leq u/m$ para todo $\vec{x} \in H_{\vec{q}, \eta\norm{\vec{q}}^{-2}}$.
Diremos que $\eta$ es el primer entero que satisface la restricción presupuestaria, o bien que
$H_{\vec{q}, \eta\norm{\vec{q}}^{-2}}$ es la primera capa entera que satisface el presupuesto.
\begin{lemma}
	\label{phase-1:lemma:eta}
	Sea $\vec{p} \in \R^n$ un vector esencialmente entero y sea $\vec{q}$ su múltiplo coprimo, de
	manera que $\vec{p} = m\vec{q}$ para algún escalar $m > 0$. Entonces la primera capa
	entera $H_{\vec{q}, \eta \norm{\vec{q}}^{-2}}$ que satisface el presupuesto está parametrizada
	por $\eta \coloneq \lfloor u/m \rfloor$.
\end{lemma}
\begin{proof}
	Sea $\vec{x}$ tal que $\vec{p}^T\vec{x} \leq u$. Entonces buscamos el mayor entero $\eta$ que
	satisfaga $\vec{q}^T\vec{x} \leq u/m$ para todo $\vec{x} \in H_{\vec{q},
	\eta\norm{\vec{q}}^{-2}}$. Por el Lema \ref{phase-1:lemma:layer} sabemos que
	\begin{equation*}
		\eta\norm{\vec{q}}^{-2} = \frac{\vec{q}^T\vec{x}}{\norm{\vec{q}}^2} \leq
		\frac{u/m}{\norm{\vec{q}}^2},
	\end{equation*}
	de donde se sigue inmediatamente que $\eta = \lfloor u/m \rfloor$.
\end{proof}

Encontramos que las capas enteras que satisfacen el presupuesto son parametrizadas por $k \in
\lbrace \eta, \eta - 1, \ldots \rbrace$ y, debido al Lema \ref{theory:lemma:utility}, se cumple
inmediatamente que $\vec{q}^T\vec{x} = k$. Deducimos que si la $\eta$-ésima capa entera contiene
puntos no negativos, entonces las soluciones se encuentran en esa capa. En caso contrario,
descendemos a la $(\eta - 1)$-ésima capa entera y buscamos puntos enteros no negativos, etcétera.

\begin{theorem}
	\label{theory:th:infeasibility}
	Sea $\vec{p} \in \R^n \setminus \lbrace \vec{0} \rbrace $ un vector esencialmente entero y sea
	$\vec{q}$ su múltiplo coprimo. Entonces el problema (\ref{theory:formulation}) es infactible si
	y solo si $\vec{q} \geq \vec{0}$ y $u < 0$.
\end{theorem}
\begin{proof}
	Supongamos que $\vec{q} \geq \vec{0}$ y $u < 0$. Si $\vec{x} \in \Z_{\geq \vec{0}}^n$
	entonces $\vec{q}^T\vec{x} \geq 0 > u$ y por lo tanto $\vec{x}$ no es factible. Luego,
	\begin{equation*}
		\Z_{\geq \vec{0}}^{n} \cap \lbrace \vec{x} \vcentcolon \vec{q}^T\vec{x} 
		\leq u \rbrace = \emptyset,
	\end{equation*}
	y el problema no es factible. Mostramos la otra implicación por contraposición. Si $u
	\geq 0$ observamos que $\vec{0} \in \Z^n$ es factible. Se debe cumplir $u < 0$. Similarmente, si
	$q_i < 0$ para algún $i \in \lbrace 1, \ldots, n \rbrace$, encontramos que $\lceil u/q_i
	\rceil\vec{e}_i \in \Z^n$ es factible:
	\begin{equation*}
		\vec{q}^T\left\lceil \frac{u}{q_i} \right\rceil\vec{e}_i
		= q_i \left\lceil \frac{u}{q_i} \right\rceil
		\leq q_i \frac{u}{q_i} = u,
	\end{equation*}
	además, como $u < 0$, concluimos que $\lceil u/q_i \rceil\vec{e}_i$ es no negativo.
\end{proof}

Debido al Teorema anterior, somos capaces de determinar inmediatamente si el problema
\eqref{theory:formulation} es infactible, por lo que supondremos de ahora en adelante que es
factible. El siguiente Teorema muestra que nuestro análisis para resolver el problema anterior
deberá dividirse en dos casos.
\begin{theorem}
	\label{theory:th:feasibility}
	Sea $\vec{p} \in \R^n \setminus \lbrace \vec{0} \rbrace $ un vector esencialmente entero y sea
	$\vec{q}$ su múltiplo coprimo. Supongamos que el problema (\ref{theory:formulation}) es factible
	y tomemos $\eta$ del Lema \ref{phase-1:lemma:eta}. Entonces se satisface lo siguiente:
	\begin{enumerate}
		\item Si $q_i \leq 0$ para algún $i \in \lbrace 1, \ldots, n \rbrace$, entonces la $\eta$-ésima
			capa entera contiene un número infinito de puntos factibles.
		\item Si $\vec{q} > \vec{0}$, enconces para todo $k \in \lbrace \eta, \eta - 1, \ldots, 0
			\rbrace$, la $k$-ésima capa entera contiene un número finito de puntos factibles.
	\end{enumerate}
\end{theorem}
\begin{proof} \hfill
	\begin{enumerate}
		\item
			En la siguiente sección mostraremos que, como $\vec{q}$ es un vector cuyas entradas son
			coprimas, entonces existe un punto entero $\vec{x}$ que satisface la ecuación lineal
			diofantina $\vec{q}^T\vec{x} = \eta$. Por el momento, confiemos que esto es verdadero.
			Como no tenemos asegurada la no negatividad de $\vec{x}$, construiremos un vector entero
			$\vec{x}^+$ que sí satisface la restricción de no negatividad y también la restricción
			presupuestaria $\vec{q}^T\vec{x}^+ = \eta$, de manera que $\vec{x}^+$ sí será factible.

			Definamos los siguientes conjuntos de índices
			\begin{equation*}
				I^+ \coloneq \lbrace i \vcentcolon q_i > 0 \rbrace,
				\quad I^\circ \coloneq \lbrace \ell \vcentcolon q_\ell = 0 \rbrace.
				\quad I^- \coloneq \lbrace j \vcentcolon q_j < 0 \rbrace.
			\end{equation*}
			Podemos suponer sin pérdida de generalidad que $I^\circ$ es vacío. En efecto, si $x_k
			< 0$ para algún $k \in I^\circ$, esa entrada no sería factible, pero fácilmente
			podríamos definir $x_k^+ = 0$ para hacerla factible.

			Entonces, ambos conjuntos $I^+$ e $I^-$ forman una partición de $\lbrace 1, \ldots,
			n\rbrace$. Podemos escoger escalares positivos $c_1, \ldots, c_n$ que satisfagan
			simultáneamente
			\begin{align}
				x_k + \sum_{i \in I^+}q_ic_i &\geq 0, \quad \forall k \in I^-,
				\label{theory:pf:1} \\
				x_k - \sum_{j \in I^-}q_jc_k &\geq 0, \quad \forall k \in I^+.
				\label{theory:pf:2}
			\end{align}
			Definamos el vector $\vec{x}^+ \in \Z^n$ de manera que
			\begin{equation*}
				x^+_k \coloneq \begin{cases}
					x_k + \sum_{i \in I^+}q_ic_i, \quad k \in I^-, \\
					x_k - \sum_{j \in I^-}q_jc_k, \quad k \in I^+.
				\end{cases}
			\end{equation*}
			Se verifica que $\vec{x}^+$ es no negativo y, además,
			\begin{align*}
				\vec{q}^T\vec{x}^+
				&= \vec{q}^T\vec{x}
				+ \sum_{k \in I^-}\sum_{i \in I^+}q_kq_ic_i
				- \sum_{k \in I^+}\sum_{j \in I^-}q_kq_jc_k \\
				&= \eta
				+ \sum_{j \in I^-}\sum_{i \in I^+}q_jq_ic_i
				- \sum_{i \in I^+}\sum_{j \in I^-}q_iq_jc_i \\
				&= \eta.
			\end{align*}
			Así pues, tenemos existencia de un punto factible. Para concluir que hay un número
			infinito de puntos factibles, basta observar que si la elección de coeficientes $c_1,
			\ldots, c_n$ satisface ambas desigualdades (\ref{theory:pf:1}) y (\ref{theory:pf:2}),
			entonces cualquier múltiplo positivo de estos coeficientes también las satisface.
		\item Se sigue que $u \geq 0$. Definamos
			\begin{equation}
				\label{theory:pf:p_k}
				P_k \coloneq H_{\vec{q}, k\norm{\vec{q}}^{-2}} \cap \Z_{\geq \vec{0}}^n
				= \left\lbrace \vec{x} \in \Z^n \vcentcolon \vec{q}^T\vec{x} = k,
					\vec{x} \geq \vec{0} \right\rbrace.
			\end{equation}
			Observemos que $P_k = \emptyset$ para todo $k$ negativo, pues $\vec{q} > \vec{0}$ y por
			lo tanto $\vec{q}^T\vec{x} \geq 0$ para cualquier $\vec{x}$ no negativo. Esto implica que
			ningún punto sobre capas enteras con parámetros negativos es factible.

			Sea $k \in \lbrace \eta, \eta - 1, \ldots, 0 \rbrace$. La capa entera $H_{\vec{q},
			k\norm{\vec{q}}^{-2}}$ interseca los ejes positivos en $\frac{k}{q_i}\vec{e}_i$.
			Definamos $\ell_i \coloneq \lceil k/q_i \rceil$. No es difícil ver que $P_k$ está
			contenido en el prisma cuyas aristas son $[0, \ell_i]$ y, por lo tanto,
			\begin{equation*}
				P_k \subseteq \prod_{i = 1}^{n} [0, \ell_i] \cap \Z^n = \prod_{i = 1}^{n}
				\left( [0, \ell_i] \cap \Z \right).
			\end{equation*}
			Pero $\left| [0, \ell_i] \cap \Z \right| = \ell_i + 1$. Así,
			\begin{equation*}
				|P_k| \leq \prod_{i = 1}^{n} (\ell_i + 1) < \infty.
			\end{equation*}
			Entonces la $k$-ésima capa entera contiene un número finito de puntos factibles.
	\end{enumerate}
\end{proof}

Así pues, suponiendo que el problema (\ref{theory:formulation}) tiene solución, el Teorema
\ref{theory:th:feasibility} nos sugiere dividir nuestro análisis en dos casos: uno donde donde una
entrada $p_i$ es no positiva y por lo tanto existe una infinidad de soluciones en la $\eta$-ésima
capa entera; y uno donde $\vec{p}$ es estrictamente positivo, lo que implica la finitud de puntos
factibles. Ciertamente el segundo caso es el más interesante, pues de alguna manera conocemos
automáticamente el óptimo de los problemas que recaen en el primer caso. Efectivamente esta es una
de las razones por las que el autor decidió ordenar de tal manera los casos: porque en el primero
sabemos exactamente dónde buscar la solución. Aún así, a pesar de encontrarnos con esta primera
división, existen muchos elementos en común que comparten ambos casos.

Antes de atacar los casos anteriores, primero debemos mostrar que la ecuación lineal diofantina
$\vec{q}^T\vec{x} = k$ tiene soluciones enteras para toda $k$ entera siempre que las entradas de
$\vec{q}$ sean coprimas\footnote{
	Recordemos que supusimos que esto era cierto para demostrar una parte del Teorema
	\ref{theory:th:feasibility}. Además, la construcción de estas soluciones enteras proveerá
	herramientas útiles para cuando decidamos agregar más restricciones.
}. La siguiente sección se encarga de mostrar la existencia de tales
soluciones enteras y, más tarde, nos enfocaremos en cómo obtener soluciones no negativas a partir de
ellas.

\subsection{Una ecuación lineal diofantina}
\noindent
De acuerdo al Teorema \ref{theory:th:feasibility}, las soluciones del problema
(\ref{theory:formulation}) se encuentran en una capa entera $H_{\vec{q}, k\norm{\vec{q}}^{-2}}$.
Así, los puntos $\vec{x} \in \Z^n$ que se encuentran sobre esa capa satisfacen la ecuación lineal
diofantina
\begin{equation}
	\label{eq:dioph}
	\vec{q}^T\vec{x} = q_1x_1 + q_2x_2 + \cdots + q_nx_n = k.
\end{equation}
Como $\vec{q} \neq \vec{0}$, podemos suponer, por el momento, que $q_n \neq 0$. En la sección
\ref{section:number-theory} de Teoría de Números mostramos bajo qué condiciones existen soluciones a
este tipo de ecuaciones y también cómo construirlas cuando solamente tenemos dos incógnitas.
Partimos de la observación que podemos resolver recursivamente esta ecuación. Definamos, por
conveniencia, $g_1 \coloneq \gcd{q_1, \ldots, q_n}$ y también $\omega_1 \coloneq k$. Como $\vec{q}$
es un vector coprimo, sabemos que $g_1 = 1$. Además, definamos
\begin{equation*}
	\omega_2 \coloneq \frac{q_2}{g_2 \cdot g_1}x_2 + \cdots + \frac{q_n}{g_2 \cdot
	g_1}x_n,
\end{equation*}
donde $g_2 \coloneq \gcd{q_2/g_1, \ldots, q_n/g_1}$. Como $q_n \neq 0$, tenemos que $g_2$ está bien
definido y además es positivo. Así, la ecuación (\ref{eq:dioph}) es equivalente a
\begin{equation}
	\label{eq:dioph:first-step}
	\frac{q_1}{g_1}x_1 + g_2\omega_2 = \omega_1.
\end{equation}
Observemos que
\begin{equation*}
	\gcd{\frac{q_1}{g_1}, g_2}
	= \gcd{\frac{q_1}{g_1}, \gcd{\frac{q_2}{g_1}, \ldots, \frac{q_n}{g_1}}}
	= \gcd{\frac{q_1}{g_1}, \frac{q_2}{g_1}, \ldots, \frac{q_n}{g_1}} = 1.
\end{equation*}
Por el Teorema \ref{prerreq:th:existence}, existen soluciones enteras para todo $\omega_1 \in \Z$.
Como $q_1/g_1$ y $g_2$ son coprimos, encontramos que sus coeficientes de Bézout asociados (c.f.
Definición \ref{prerreq:def:bezout}) $x_1', \omega_2'$ son soluciones particulares de la ecuación
\begin{equation*}
	\frac{q_1}{g_1}x_1 + g_2\omega_2 = 1.
\end{equation*}
Deducimos del Teorema \ref{prerreq:th:construction} que las soluciones de la ecuación
(\ref{eq:dioph:first-step}) están dadas por
\begin{equation}
	\label{dummy:eq:first-step}
	\begin{cases}
		x_1 = \omega_1x_1' + g_2t_1, \\
		\omega_2 = \omega_1\omega_2' - \frac{q_1}{g_1}t_1,
	\end{cases}
\end{equation}
donde $t_1 \in \Z$ es una variable libre.

\begin{observation}
	Los coeficientes de Bézout $x_1'$ y $\omega_2'$ dependen exclusivamente de $\vec{q}$ y no del
	punto $\vec{x}$. En efecto, $x_1'$ está asociado a $q_1/g_1$ y $\omega_2'$ está asociado a
	$g_2$. Pero ambos $g_1$ y $g_2$ son el máximo común divisor de $q_1, \ldots q_n$ y
	$q_1/g_1, \ldots, q_n/g_1$, respectivamente. 
\end{observation}

Para el siguiente paso de la recursión escogemos cualquier $t_1 \in \Z$ para fijar $\omega_2$  y
resolvemos la ecuación
\begin{equation}
	\label{eq:dioph:second-step}
	\frac{q_2}{g_2 \cdot g_1}x_2 +
	\frac{q_3}{g_2 \cdot g_1}x_3 +
	\cdots +
	\frac{q_n}{g_2 \cdot g_1}x_n
	= \omega_2.
\end{equation}
Como $g_2 = \gcd{q_2/g_1, \ldots, q_n/g_1}$, sabemos del Lema \ref{prerreq:lemma:gcd}
que
\begin{equation*}
	\gcd{\frac{q_2}{g_2 \cdot g_1}, \ldots, \frac{q_n}{g_2 \cdot g_1}} = 1.
\end{equation*}
En el mismo espíritu que el primer paso de la recursión, definimos
\begin{equation*}
	\omega_3 \coloneq \frac{q_3}{g_3 \cdot g_2 \cdot g_1}x_3 + \cdots + \frac{q_n}{g_3
	\cdot g_2 \cdot g_1}x_n,
\end{equation*}
donde
\begin{equation*}
	g_3 \coloneq  \gcd{\frac{q_3}{g_2 \cdot g_1}, \ldots, \frac{q_n}{g_2 \cdot g_1}}.
\end{equation*}
Nuevamente, como $q_n$ es no nulo, $g_3$ está bien definido y además es positivo. Por lo que la
ecuación (\ref{eq:dioph:second-step}) es equivalente a
\begin{equation}
	\label{eq:dioph:second-step:short}
	\frac{q_2}{g_2 \cdot g_1}x_2 + g_3\omega_3 = \omega_2.
\end{equation}
También se cumple que
\begin{equation*}
	\gcd{\frac{q_2}{g_2 \cdot g_1}, g_3} = 1,
\end{equation*}
y entonces (\ref{eq:dioph:second-step:short}) tiene una infinidad de soluciones para todo $\omega_2 \in
\Z$, las cuales están dadas por
\begin{equation*}
	\begin{cases}
		x_2 = \omega_2x_2' + g_3t_2, \\
		\omega_3 = \omega_2\omega_3' - \frac{q_2}{g_2 \cdot g_1}t_2,
	\end{cases}
\end{equation*}
donde $t_2 \in \Z$ es una variable libre, y $x_2', \omega_3'$ son los coeficientes de Bézout
asociados a $\frac{q_2}{g_2 \cdot g_2}$ y $g_3$, respectivamente.

De manera general, para $i \in \lbrace 1, \ldots, n - 2 \rbrace$, el $i$-ésimo paso de la recursión
provee la ecuación
\begin{equation}
	\label{dummy:eq:ith-equation}
	\frac{q_i}{\prod_{j=1}^{i}g_j}x_i
	+ \frac{q_{i+1}}{\prod_{j=1}^{i}g_j}x_{i+1}
	+ \cdots
	+ \frac{q_{n}}{\prod_{j=1}^{i}g_j}x_n
	= \omega_i,
\end{equation}
donde
\begin{equation*}
	g_i \coloneq \gcd{\frac{q_i}{\prod_{j=1}^{i-1}g_j}, \ldots, \frac{q_n}{\prod_{j=1}^{i-1}g_j}},
\end{equation*}
por el Lema \ref{prerreq:lemma:gcd} se sigue que
\begin{equation}
	\label{dummy:coprime}
	\gcd{\frac{q_i}{\prod_{j=1}^{i}g_j}, \ldots, \frac{q_n}{\prod_{j=1}^{i}g_j}} = 1.
\end{equation}
Ahora bien, definamos
\begin{equation}
	\label{dummy:next-g}
	g_{i + 1} \coloneq \gcd{
		\frac{q_{i+1}}{\prod_{j=1}^{i}g_j},
		\ldots,
		\frac{q_{n}}{\prod_{j=1}^{i}g_j},
	}.
\end{equation}
Como $q_n$ es no nulo, se sigue que $g_{i + 1}$ está bien definido y es positivo. Definamos,
también,
\begin{equation*}
	\omega_{i+1} =
	\frac{q_{i+1}}{\prod_{j=1}^{i + 1}g_j}x_{i+1}
	+ \cdots +
	\frac{q_{n}}{\prod_{j=1}^{i + 1}g_j}x_{n},
\end{equation*}
de manera que la ecuación \eqref{dummy:eq:ith-equation} es equivalente a
\begin{equation}
	\label{dummy:eq:simplified}
	\frac{q_i}{\prod_{j=1}^{i}g_j}x_i + g_{i+1}\omega_{i+1} = \omega_i.
\end{equation}
A partir de \eqref{dummy:coprime} y de \eqref{dummy:next-g}, encontramos que
\begin{align*}
	\gcd{
		\frac{q_i}{\prod_{j=1}^{i}g_j},
		g_{i+1}
	}
	&=
	\gcd{
		\frac{q_i}{\prod_{j=1}^{i}g_j},
		\frac{q_{i+1}}{\prod_{j=1}^{i}g_j},
		\ldots,
		\frac{q_n}{\prod_{j=1}^{i}g_j}
	} = 1,
\end{align*}
y del Teorema \ref{prerreq:th:existence} se sigue que la ecuación \eqref{dummy:eq:simplified}
tiene soluciones enteras para todo $\omega_i \in \Z$. Por el Teorema \ref{prerreq:th:construction},
las soluciones enteras de \eqref{dummy:eq:simplified} están dadas por
\begin{equation}
	\label{eq:recurrence}
	\begin{cases}
		x_i = \omega_ix_i' + g_{i + 1}t_i, \\
		\omega_{i + 1} = \omega_i\omega_{i + 1}' - \frac{q_i}{\prod_{j=1}^{i}g_j}t_i,
	\end{cases}
\end{equation}
donde $t_i \in \Z$ es la $i$-ésima variable libre. Es valioso mencionar, otra vez, que los
coeficientes de Bézout $x_i', \omega_{i+1}'$ dependen exclusivamente de $\vec{q}$ a través de sus
entradas $q_i$ y de los máximos común divisores entre ellas. En efecto, por el Teorema
\ref{prerreq:th:bezout}, estos coeficientes son soluciones particulares de la ecuación
\begin{equation}
	\label{dummy:eq:bez-eq}
	\frac{q_i}{\prod_{j=1}^{i}g_j}x_i' + g_{i+1}\omega_{i+1}' = 1.
\end{equation}

Finalmente, en el último paso de la recursión obtenemos la ecuación lineal diofantina
\begin{equation}
	\label{eq:last-equation}
	\frac{q_{n-1}}{\prod_{j=1}^{n-1}g_j}x_{n-1} +
	\frac{q_{n}}{\prod_{j=1}^{n-1}g_j}x_n
	= \omega_{n-1}.
\end{equation}
Por construcción, los coeficientes de $x_{n - 1}$ y $x_n$ son coprimos. A causa del Teorema
\ref{prerreq:th:construction} las soluciones enteras están dadas por
\begin{equation}
	\label{eq:last-solution}
	\begin{cases}
		x_{n-1} = \omega_{n-1}x_{n-1}' + \frac{q_n}{\prod_{j=1}^{n-1}g_j}t_{n-1}, \\
		x_n = \omega_{n-1}x_n' - \frac{q_{n-1}}{\prod_{j=1}^{n-1}g_j}t_{n-1},
	\end{cases}
\end{equation}
donde $x_{n-1}', x_n'$ son los coeficientes de Bézout asociados a
$\frac{q_n}{\prod_{j=1}^{n-1}g_j}$ y $\frac{q_{n-1}}{\prod_{j=1}^{n-1}g_j}$,
respectivamente, por lo que satisfacen
\begin{equation}
	\label{eq:last-equation-bez}
	\frac{q_{n-1}}{\prod_{j=1}^{n-1}g_j}x_{n-1}' +
	\frac{q_{n}}{\prod_{j=1}^{n-1}g_j}x_n'
	= 1.
\end{equation}

Hasta este punto, hemos demostrado que la ecuación lineal diofantina $\vec{q}^T\vec{x} = k$ tiene
soluciones enteras para todo $k \in \Z$ siempre que las entradas de $\vec{q}$ sean coprimas. En
realidad, hemos mostrado también la existencia de una infinidad de soluciones enteras, pues cada
elección distinta de $t_i \in \Z$ para cualquier $i \in \lbrace 1, \ldots, n - 1\rbrace$ proveerá una
solución distinta. Por lo tanto, hemos saldado nuestra cuenta pendiente con respecto a una parte de
la demostración en el Teorema \ref{theory:th:feasibility}.

Con respecto a la restricción de no negatividad $\vec{x} \geq 0$ en el problema
(\ref{theory:formulation}), podemos acotar nuestra elección de variables libres $t_i \in \Z$ a
partir de (\ref{eq:recurrence}). De la primera igualdad encontramos que necesariamente se debe
satisfacer
\begin{equation}
	\label{eq:param-lb}
	t_i \geq \left\lceil -\frac{\omega_ix_i'}{g_{i + 1}} \right\rceil,
\end{equation}
para $i \in \lbrace 1, \ldots, n - 2\rbrace$. Para determinar intervalos de no negatividad de
$x_{n-1}$ y $x_n$, observamos de (\ref{eq:last-solution}) que dependemos de los signos
de $q_{n-1}$ y de $q_n$. Mucho tendremos que decir en los siguientes dos capítulos sobre
cómo acotar mejor $t_1, \ldots, t_{n-1}$ para asegurar la no negatividad de $\vec{x}$. Así pues,
relegamos la discusión en los siguientes dos capítulos cuando analicemos separadamente el caso
infinito y el caso finito del Teorema \ref{theory:th:feasibility}.

Ahora bien, hemos encontrado una relación entre el vector de soluciones $\vec{x} \in \Z^n$ y el
vector de variables libres $\vec{t} \in \Z^{n-1}$. Hemos manejado esta relación de manera recursiva
a través de (\ref{eq:recurrence}). Resultará conveniente encontrar una forma cerrada a la relación
de recurrencia inducida. Para ello, recordemos que el vector $\vec{x}$ se encuentra sobre la capa entera
$H_{\vec{q}, k\norm{\vec{q}}^{-2}}$ y por lo tanto satisface (\ref{eq:dioph}). Recordemos que
habíamos definido, por construcción, $\omega_1 \coloneq k$. Combinando esto último con la última
igualdad de \eqref{eq:recurrence}, llegamos a
\begin{equation}
	\label{eq:omega-recurrence}
	\begin{cases}
		\omega_1 = k, \\
		\omega_{i + 1} = \omega_i \cdot \omega_{i + 1}' - \frac{q_i}{\prod_{\ell=1}^{i}g_\ell} \cdot t_i.
	\end{cases}
\end{equation}
\begin{lemma}
	La forma cerrada de la relación de recurrencia (\ref{eq:omega-recurrence}) está dada por
	\begin{equation}
		\label{eq:omega-formula}
		\omega_i =
		k \cdot \prod_{j=2}^{i} \omega_j'
		- \sum_{j=1}^{i - 1}\frac{q_j}{\prod_{\ell=1}^{j}g_\ell}
		\cdot \prod_{\ell=j+2}^{i}\omega_\ell' \cdot t_j,
	\end{equation}
	donde, por conveniencia, le asignamos el valor de 0 a la suma vacía y el valor de 1 al producto
	vacío.
\end{lemma}
\begin{proof}
	Lo demostramos inductivamente. Observemos que
	\begin{equation*}
		\omega_1 =
		k \cdot \prod_{j=2}^{1} \omega_j'
		- \sum_{j=1}^{0}\frac{q_j}{\prod_{\ell=1}^{j}g_\ell}
		\cdot \prod_{\ell=j+2}^{1}\omega_\ell' \cdot t_j
		= k,
	\end{equation*}
	debido a que definimos el producto vacío como 1 y la suma vacía como 0. Supongamos
	inductivamente que (\ref{eq:omega-formula}) se satisface para alguna $i \in \N$. Entonces,
	tenemos
	\begin{align*}
		&k \cdot \prod_{j=2}^{i + 1} \omega_j'
		- \sum_{j=1}^{i}\frac{q_j}{\prod_{\ell=1}^{j}g_\ell}
		\cdot \prod_{\ell=j+2}^{i + 1}\omega_\ell' \cdot t_j \\
		&=
		k \cdot \prod_{j=2}^{i} \omega_j' \cdot \omega_{i+1}'
		- \sum_{j=1}^{i - 1}\frac{q_j}{\prod_{\ell=1}^{j}g_\ell}
		\cdot \prod_{\ell=j+2}^{i}\omega_\ell' \cdot t_j \cdot \omega_{i + 1}'
		- \frac{q_i}{\prod_{\ell = 1}^{i}g_\ell}
		\cdot \prod_{\ell = i + 2}^{i + 1}\omega_\ell' \cdot t_i \\
		&= 
		\left( k \cdot \prod_{j=2}^{i} \omega_j'
		- \sum_{j=1}^{i - 1}\frac{q_j}{\prod_{\ell=1}^{j}g_\ell}
		\cdot \prod_{\ell=j+2}^{i}\omega_\ell' \cdot t_j \right) \omega_{i+1}'
		- \frac{q_i}{\prod_{\ell = 1}^{i}g_\ell} \cdot t_i  \\
		&= \omega_i \cdot \omega_{i + 1}' - \frac{q_i}{\prod_{\ell = 1}^{i}g_\ell} \cdot t_i \\
		&= \omega_{i+1}.
	\end{align*}
	Por el principio de inducción se sigue que (\ref{eq:omega-formula}) satisface
	(\ref{eq:omega-recurrence}) para todo $i \in \N$. Así, esta fórmula es la forma cerrada de la
	relación de recurrencia propuesta.
\end{proof}

Ahora que encontramos una forma cerrada a la relación de recurrencia \eqref{eq:omega-recurrence},
somos capaces de determinar una relación lineal entre $\vec{x} \in \Z^n$ y $\vec{t} \in \Z^{n-1}$.
Definamos, por conveniencia, los coeficientes $m_{ij} \in \mathbb{Z}$ con $i > j$ como
\begin{equation}
	\label{phase-2:eq:coeffs}
	m_{ij} \coloneq \frac{q_j}{\prod_{\ell = 1}^{j}g_\ell} \cdot \prod_{\ell = j +
	2}^{i}\omega_\ell'.
\end{equation}
Sustituyendo en la forma cerrada \eqref{eq:omega-formula}, obtenemos la fórmula simplificada
\begin{equation}
	\label{eq:omega-formula-simplified}
	\omega_i =
	k \cdot \prod_{j=2}^{i} \omega_j'
	- \sum_{j=1}^{i - 1}m_{ij}t_j,
\end{equation}
Así pues, juntando esto último con \eqref{eq:recurrence}, obtenemos para $i \in \{1, \ldots, n -
2\}$, 
\begin{align}
	x_i &= \omega_i \cdot x_i' + g_{i + 1}t_i \nonumber \\
		&= k \cdot \prod_{j=2}^{i}\omega_j' \cdot x_i' - \sum_{j=1}^{i - 1}m_{ij}x_i'
		t_j + g_{i + 1}t_i \label{eq:x:i}.
\end{align}
Similarmente, usando \eqref{eq:omega-formula-simplified} y sustituyendo en \eqref{eq:last-solution},
\begin{subequations}
	\label{eq:x:last}
	\begin{align}
		x_{n-1} &= k \cdot \prod_{j=2}^{n-1} \omega_j' \cdot x_{n-1}' - \sum_{j=1}^{n-2}
		m_{n-1,j}x_{n-1}' t_j + \frac{q_n}{\prod_{j=1}^{n-2}g_j} t_{n-1}, \\
		x_{n} &= k \cdot \prod_{j=2}^{n-1} \omega_j' \cdot x_{n}' - \sum_{j=1}^{n-2}
		m_{n-1,j}x_{n}' t_j - \frac{q_{n - 1}}{\prod_{j=1}^{n-2}g_j} t_{n-1}.
	\end{align}
\end{subequations}

Con este trabajo anterior, ya podemos establecer una relación lineal entre $\vec{t} \in \Z^{n-1}$ y
$\vec{x} \in \Z^n$. Definimos $\vec{\omega} \in \Z^n$ a partir de
\begin{equation}
	\label{eq:vec-omega}
	\omega_i \coloneq x_i' \cdot \prod_{j = 2}^{\min{\lbrace i, n - 1 \rbrace}}\omega_j'.
\end{equation}
También definimos la matriz $M \in \Z^{n \times (n - 1)}$ a través de
\begin{equation}
	\label{eq:mat-T}
	M_{ij} \coloneq \begin{cases}
		-m_{ij}x_i', &\quad j < i, \\
		g_{i + 1},  &\quad i = j < n - 1, \\
		\frac{q_n}{\prod_{k=1}^{n-1}g_k}, &\quad i = j = n - 1, \\
		-\frac{q_{n-1}}{\prod_{k=1}^{n-1}g_k}, &\quad i = n, j = n - 1, \\
		0, &\quad \text{e.o.c.}
	\end{cases}
\end{equation}
De (\ref{eq:x:i}) y (\ref{eq:x:last}) encontramos que
\begin{equation}
	\label{eq:transf}
	\vec{x} = k\vec{\omega} + M\vec{t}.
\end{equation}

En una observación pasada mencionamos que los coeficientes de Bézout $\omega_i', x_i'$ están
asociados a términos exclusivamente dependientes de $\vec{q}$, por lo que no dependen de la elección
$\vec{x} \in \Z$. De esta manera, $\vec{\omega}$ depende exclusivamente de $\vec{q}$. El mismo
razonamiento aplica para la matriz $M$. Entonces, como $\vec{q}$ es fijo, se sigue que
$\vec{\omega}$ y $M$ lo son también.

\begin{lemma} \label{lemma:iso1}
	Sea $\vec{q} \in \Z^{n}$ un vector no nulo y cuyas entradas son coprimas. Entonces el vector
	$\vec{\omega} \in \Z^n$ definido en \eqref{eq:vec-omega} satisface $\vec{q}^T\vec{\omega} = 1$.
\end{lemma}
\begin{proof}
	Primero mostramos por inducción hacia atrás que se cumple
	\begin{equation}
		\label{eq:omega-induction} \sum_{j=i}^{n}q_j\omega_j =
		\prod_{j=2}^{i}\omega_j' \cdot \prod_{j=1}^{i}g_j,
	\end{equation}
	para todo $i \in \lbrace 1, \ldots, n - 1\rbrace$. Empezamos con el caso base $i = n - 1$. De
	\eqref{eq:vec-omega}, encontramos que
	\begin{equation}
		\label{eq:omega-base-case}
		q_{n-1}\omega_{n-1} + q_n\omega_n =
		\prod_{j=2}^{n-1}\omega_j' \cdot \left(q_{n-1}x_{n-1}' + q_nx_n'\right).
	\end{equation}
	Recordemos que $x_{n-1}'$ y $x_n'$ son coeficientes de Bézout asociados a los coeficientes del
	lado izquierdo de \eqref{eq:last-equation}, los cuales son coprimos. Entonces se cumple, por el
	Teorema \ref{prerreq:th:bezout},
	\begin{equation*}
		\frac{q_{n-1}}{\prod_{j=1}^{n-1}g_j}x_{n-1}' +
		\frac{q_n}{\prod_{j=1}^{n-1}g_j}x_n' = 1,
	\end{equation*}
	o, equivalentemente,
	\begin{equation*}
		q_{n-1}x_{n-1}' + q_nx_n' = \prod_{j=1}^{n-1}g_j.
	\end{equation*}
	Sustituyendo en (\ref{eq:omega-base-case}), obtenemos la base de la inducción, i.e.,
	\begin{equation*}
		q_{n-1}\omega_{n-1} + q_n\omega_n  =
		\prod_{j=2}^{n-1}\omega_j' \cdot \prod_{j=1}^{n-1}g_j.
	\end{equation*}
	Supongamos inductivamente que (\ref{eq:omega-induction}) se satisface para alguna $2 \leq i \leq
	n - 1$. Reduciendo $i$, ocupando \eqref{eq:vec-omega} y usando la hipótesis inductiva,
	obtenemos
	\begin{align*}
		\sum_{j=i-1}^{n}q_j\omega_j
		&= q_{i-1}\omega_{i-1} + \sum_{j=i}^{n}q_j\omega_j \\
		&= \prod_{j=2}^{i-1}\omega_j' \cdot q_{i-1}x_{i-1}' + \prod_{j=2}^{i}\omega_j' \cdot
		\prod_{j=1}^{i}g_j \\
		&= \prod_{j=2}^{i-1}\omega_j' \cdot \left( q_{i-1}x_{i-1}' + \omega_i'
			\prod_{j=1}^{i}g_j \right).
	\end{align*}
	Nuevamente, $x_{i-1}'$ y $\omega_i'$ son coeficientes de Bézout asociados, respectivamente, a
	$\frac{q_{i-1}}{\prod_{j=1}^{i-1}g_j}$ y $g_i$, los cuales son coprimos. De esta manera
	satisfacen \eqref{dummy:eq:bez-eq} pero sustituyendo $i$ por $i - 1$. Es decir, se satisface
	\begin{equation*}
		\frac{q_{i-1}}{\prod_{j=1}^{i-1}g_j}x_{i-1}' +
		g_i \omega_i' = 1,
	\end{equation*}
	o, equivalentemente,
	\begin{equation*}
		q_{i-1}x_{i-1}' + \omega_i'\prod_{j=1}^{i}g_j = \prod_{j=1}^{i-1}g_j.
	\end{equation*}
	Sustituyendo, obtenemos el resultado (\ref{eq:omega-induction}) para $i - 1$. Así, por inducción
	hacía atrás, (\ref{eq:omega-induction}) se cumple para todo $i \in \lbrace 1, \ldots, n - 1
	\rbrace$. Finalmente, para demostrar este lema, observamos que
	\begin{equation*}
		\vec{q}^T\vec{\omega} = \sum_{j=1}^{n}q_j\omega_j = \prod_{j=2}^{1}\omega_j'
		\cdot \prod_{j=1}^{1}g_j = g_1 = 1.
	\end{equation*}
	El primer producto es uno por ser el producto vacío. Recordemos también que $g_1$ es el máximo
	común divisor de $q_1, \ldots, q_n$, los cuales son coprimos, y entonces $g_1 = 1$.
\end{proof}
\begin{lemma}
	\label{lemma:iso2}
	El vector $\vec{q}$ genera $\ker{M^T}$ si $q_n \neq 0$.
\end{lemma}
\begin{proof}
	La matriz $M$ es triangular inferior cuya diagonal principal es distinta de cero. En efecto,
	para todo $i \in \lbrace 1, \ldots, n - 2\rbrace$, tenemos
	\begin{equation*}
		M_{ii} = g_{i + 1} = \gcd{\frac{q_i}{\prod_{j=1}^{i}g_j}, \ldots,
		\frac{q_n}{\prod_{j=1}^{i}g_j}}.
	\end{equation*}
	Pero el máximo común divisor entre cualesquiera enteros siempre es positivo. También tenemos
	\begin{equation*}
		M_{n-1, n-1} = \frac{q_n}{\prod_{j=1}^{n-1}g_j} \neq 0.
	\end{equation*}
	Se sigue que las columnas de $M$ son linealmente
	independientes, y entonces su imagen tiene dimensión $n - 1$. Por lo tanto, $M^T$ tiene $n - 1$
	renglones linealmente independientes. Se sigue por el Teorema de la Dimensión que $\dim
	\ker{M^T} = 1$, así que basta mostrar que $\vec{q} \in \ker{M^T}$.

	Sea $\vec{x} \in \Z^n$. Por el Teorema \ref{phase-1:th:cover}, existe una capa entera
	$H_{\vec{q}, k\norm{\vec{q}}^{-2}}$ que contiene a $\vec{x}$. Así, por el Lema
	\ref{theory:lemma:utility}, $\vec{x}$ satisface la
	ecuación lineal diofantina $\vec{q}^T\vec{x} = k$. Por construcción, existe $\vec{t} \in
	\Z^{n-1}$ tal que $\vec{x} = k\vec{\omega} + M\vec{t}$. Luego, por el Lema \ref{lemma:iso1},
	tenemos
	\begin{equation*}
		k = \vec{q}^T\vec{x} = k \vec{q}^T\vec{\omega} + \vec{q}^TM\vec{t} = k +
		(\vec{q}^TM)\vec{t}.
	\end{equation*}
	De donde obtenemos $(\vec{q}^TM)\vec{t} = 0$. Pero $\vec{x}$ fue arbitrario, así que también lo
	fue $\vec{t}$. Entonces se debe cumplir $\vec{q}^TM = \vec{0}^T$, lo que implica que $\vec{q} \in
	\ker{M^T}$.
\end{proof}

La gran mayoría de nuestra argumentación para demostrar los resultados ha sido fundamentada a través
de las capas enteras $H_{\vec{q}, k\norm{\vec{q}}^{-2}}$, así como por el Teorema
\ref{phase-1:th:cover}. Sin embargo, estas capas enteras contienen puntos que, en el contexto de
programación lineal entera, no son de interés, a saber, contienen puntos no enteros. Nos gustaría
concentrarnos exclusivamente en estos puntos enteros, al mismo tiempo que buscamos caracterizarlos
por medio de $\vec{q}$. La siguiente Definición hará que logremos este primer objetivo de enfocarnos
exclusivamente en los puntos enteros, mientras que el Teorema \ref{th:lattice} permitirá que los
caractericemos a partir de $\vec{q}$.

\begin{definition}[\cite{alex}]
	Decimos que un subconjunto $\Lambda$ de $\R^n$ es un grupo aditivo si
	\begin{enumerate}
		\item $\vec{0} \in \Lambda$, y
		\item si $\vec{x}, \vec{y} \in \Lambda$, entonces $\vec{x} + \vec{y} \in \Lambda$, y también
			$-\vec{x} \in \Lambda$.
	\end{enumerate}
	Además, decimos que $\Lambda$ es una red si existen vectores $\vec{v}_1, \ldots, \vec{v}_n$
	linealmente independientes tales que
	\begin{equation*}
		\Lambda = \lbrace \lambda_1\vec{v}_1 + \cdots + \lambda_n\vec{v}_n \vcentcolon \lambda_i \in
		\Z \rbrace.
	\end{equation*}
	A los vectores $\vec{v}_1, \ldots, \vec{v}_n$ los llamamos la base de la red $\Lambda$.
\end{definition}

\begin{example}
	No es difícil ver que $\Z^n$ es un grupo aditivo. Si consideramos los vectores canónicos
	$\vec{e}_1, \ldots, \vec{e}_n$, entonces encontramos que son linealmente independientes, pero
	también se cumple
	\begin{equation*}
		\Z^n = \lbrace \lambda_1\vec{e}_1 + \cdots + \lambda_n\vec{e}_n \vcentcolon \lambda_i \in
		\Z \rbrace.
	\end{equation*}
	De esta, manera $\Z^n$ es una red que tiene como base canónica a los vectores $\vec{e}_1,
	\ldots, \vec{e}_n$.
\end{example}

\begin{theorem}
	\label{th:lattice}
	Supongamos que $q_n \neq 0$. Entonces $\vec{\omega}$ y las columnas de $M$ forman una base
	de la red $\Z^n$.
\end{theorem}
\begin{proof}
	En el Lema \ref{lemma:iso2} mostramos que las columnas de $M$ son linealmente independientes.
	Mostramos por contradicción que $\vec{\omega}$ es linealmente independiente de las columnas de
	$M$, así que supongamos que no lo es, por lo que existen escalares $\lambda_1, \ldots,
	\lambda_{n-1}$ tales que
	\begin{equation*}
		\vec{\omega} = \lambda_1 \vec{m}_1 + \cdots + \lambda_{n-1} \vec{m}_{n-1},
	\end{equation*}
	donde $\vec{m}_1, \ldots, \vec{m}_{n-1}$ son las columnas de $M$. De los Lemas \ref{lemma:iso1}
	y \ref{lemma:iso2} obtenemos
	\begin{equation*}
		1 = \vec{q}^T\vec{\omega} = \lambda_1 \vec{q}^T\vec{m}_1 + \cdots + \lambda_{n-1}
		\vec{q}^T\vec{m}_{n-1} = 0,
	\end{equation*}
	lo cual es una contradicción. Se sigue que $\lbrace \vec{\omega}, \vec{m}_1, \ldots,
	\vec{m}_{n-1}\rbrace$ es un conjunto de vectores linealmente independiente.

	Ahora bien, sea $\vec{x} \in \Z^n$, por el Teorema \ref{phase-1:th:cover}, sabemos que se
	encuentra sobre una capa entera, y entonces satisface la ecuación lineal diofantina
	$\vec{q}^T\vec{x} = k$ para alguna $k \in \Z$. Por construcción al inicio de esta sección junto
	con la exhaustividad del Teorema \ref{prerreq:th:construction}, existe un vector $\vec{t} \in
	\Z^{n-1}$ tal que
	\begin{equation*}
		\vec{x} = k\vec{\omega} + M\vec{t} = k\vec{\omega} + t_1\vec{m}_1 + \cdots +
		t_{n-1}\vec{m}_{n-1}.
	\end{equation*}
	Como $\vec{x}$ fue arbitrario, se sigue que
	\begin{equation*}
		\Z^n = \lbrace
		k\vec{\omega} + t_1\vec{m}_1 + \cdots + t_{n-1}\vec{m}_{n-1}
		\vcentcolon k, t_1, \ldots, t_{n-1} \in \Z
		\rbrace.
	\end{equation*}
	De esta manera, se cumple que $\lbrace \vec{\omega}, \vec{m}_1, \ldots, \vec{m}_{n-1}\rbrace$ es
	una base de $\Z^n$.
\end{proof}

\begin{corollary}
	Supongamos que $q_n \neq 0$ y consideremos $\vec{\omega} \in \Z^n$ (definido en
	\eqref{eq:vec-omega}) y las columnas $\vec{m}_1, \ldots, \vec{m}_{n-1} \in \Z^n$ de la matriz $M
	\in \Z^{n \times (n-1)}$ (definida en \eqref{eq:mat-T}), entonces la matriz
	\begin{equation*}
		[ \vec{\omega} \mid \vec{m}_1 \mid \cdots \mid \vec{m}_{n-1} ] \in \Z^{n \times n}
	\end{equation*}
	es unimodular, es decir, su determinante es $\pm 1$.
\end{corollary}

Geométricamente, a partir de $\vec{q}$ descomponemos la red $\Z^n$ como una suma directa de dos
subredes isomorfas a $\Z$ y $\Z^{n-1}$, cuyas bases están dadas por $\vec{\omega}$ y las columnas de
$M$, respectivamente. El vector $\vec{\omega}$ es una solución particular de la ecuación no
homogénea $\vec{q}^T\vec{\omega} = 1$, mientras que las columnas de $M$ forman una base del conjunto
de soluciones de la ecuación homogénea $\vec{q}^T\vec{m} = 0$. Tenemos entonces que si $q_n \neq 0$,
el vector $\vec{q}$ induce una descomposición de $\Z^n$. Ciertamente, esta idea de descomponer el
espacio vectorial completo a partir de soluciones particulares y homogéneas no es novedosa.

Hasta este punto hemos supuesto que $q_n \neq 0$. Ciertamente si este no es el caso podemos
permutar las entradas de $\vec{q}$ de manera que el vector permutado $\tilde{\vec{q}}$ cumpla el
supuesto. Ahora bien, podemos preguntarnos cómo se relacionan las imágenes de las matrices $M$ y
$\tilde{M}$ de estos dos vectores. No obstante, si $q_n = 0$, puede ser el caso que la matriz
$M$ no esté bien definida\footnote{
	Por ejemplo, si $q_n = q_{n-1} = 0$, encontramos que
	\begin{equation*}
		g_{n-1} \coloneq \gcd{\frac{q_{n-1}}{\prod_{j=1}^{n-2}g_j},
		\frac{q_n}{\prod_{j=1}^{n-2}g_j}} = \gcd{0, 0}.
	\end{equation*}
	Pero el máximo común divisor de dos números no está bien definido si ambos son cero. Esto
	implica que la entrada $M_{n-2, n-2} \coloneq g_{n-1}$ no está bien definida.
}. Para responder la pregunta requerimos de un supuesto más fuerte.
\begin{corollary}
	\label{cor:iso3}
	Sea $\vec{q}$ un vector coprimo y sea $\tilde{\vec{q}}$ el vector coprimo resultante de haber
	permutado las entradas de $\vec{q}$. Supongamos además que $q_n, \tilde{q}_n \neq 0$.
	Entonces
	\begin{equation*}
		\ker{M^T} \cong \ker{\tilde{M}^T}.
	\end{equation*}
\end{corollary}
\begin{proof}
	Existe una matriz de permutación $P \in \Z^{n \times n}$ tal que $\tilde{\vec{q}} = P\vec{q}$.
	Por el Lema \ref{lemma:iso2} sabemos que
	\begin{equation*}
		\ker{\tilde{M}^T} = \langle \vec{\tilde{\vec{q}}} \rangle = \langle P\vec{q} \rangle,
	\end{equation*}
	pero también $\langle \vec{q} \rangle = \ker{M^T}$. Como $P$ es una matriz invertible, se sigue
	que $\langle P\vec{q} \rangle \cong \langle \vec{q} \rangle$ y obtenemos nuestro resultado.
\end{proof}
\begin{observation}
	No es cierto que $\tilde{M} = PM$ si $\tilde{\vec{q}} = P\vec{q}$. Consideremos el vector
	$\vec{q} \coloneq (1, 1, -2)^T$ y la permutación
	\begin{equation*}
		P \coloneq \begin{pmatrix}
			1 & 0 & 0 \\
			0 & 0 & 1 \\
			0 & 1 & 0 \\
		\end{pmatrix},
	\end{equation*}
	de donde obtenemos $\tilde{\vec{q}} = (1, -2, 1)^T$. Observemos que
	\begin{equation*}
		M = \begin{pmatrix}
			1 & 0 \\
			1 & -2 \\
			1 & -1 \\
		\end{pmatrix}, \quad
		\tilde{M} = \begin{pmatrix}
			1 & 0 \\
			1 & 1 \\
			1 & 2 \\
		\end{pmatrix}.
	\end{equation*}
	Sí se cumple que
	\begin{equation*}
		\ker{\tilde{M}^T} = \langle \tilde{\vec{q}} \rangle = \langle P\vec{q} \rangle
		\cong \langle \vec{q} \rangle = \ker{M^T},
	\end{equation*}
	pero
	\begin{equation*}
		PM = \begin{pmatrix}
			1 & 0 \\
			1 & -1 \\
			1 & -2 \\
		\end{pmatrix}
		\neq \tilde{M}.
	\end{equation*}
\end{observation}

Extendamos más la idea anterior y denotemos por $\glz{n}{\Z}$ el grupo de permutaciones de $\Z^n$.
Es decir,
\begin{equation*}
	\glz{n}{\Z} \coloneq \lbrace P \in \Z^{n \times n} ~\text{es matriz de
	permutación} \rbrace.
\end{equation*}
También definamos el grupo de permutaciones en $n$ letras como
\begin{equation*}
	S_n \coloneq \lbrace \sigma \colon \lbrace 1, \ldots, n \rbrace \to \lbrace
	1, \ldots, n \rbrace ~\text{es función biyectiva} \rbrace.
\end{equation*}
Entonces $S_n$ actúa naturalmente sobre la red $\Z^n$. En efecto, consideremos el homomorfismo
\begin{align*}
	\varphi &\colon S_n \to \glz{n}{\Z},\\
	\sigma &\mapsto (\Z^n \mapsto \Z^n),
\end{align*}
a partir de la extensión lineal de $\varphi(\sigma)(\vec{e}_i) = \vec{e}_{\sigma(i)}$. Escribimos
$\sigma.\vec{e}_i = \vec{e}_{\sigma(i)}$ para tener una notación más clara.
\begin{example}
	Sea $\sigma \in S_4$ definida por $\sigma \coloneq (12)(34)$. La permutación $\sigma$ actúa
	sobre la base canónica como
	\begin{align*}
		\sigma.\vec{e}_1 &= \vec{e}_2, \quad \sigma.\vec{e}_2 = \vec{e}_1, \\
		\sigma.\vec{e}_3 &= \vec{e}_4, \quad \sigma.\vec{e}_4 = \vec{e}_3.
	\end{align*}
	Por lo que $\sigma$ es realizada como
	\begin{equation*}
		\sigma \mapsto [\vec{e}_2, \vec{e}_1, \vec{e}_4, \vec{e}_3]
		= \begin{pmatrix}
			0 & 1 & 0 & 0 \\
			1 & 0 & 0 & 0 \\
			0 & 0 & 0 & 1 \\
			0 & 0 & 1 & 0
		\end{pmatrix}.
	\end{equation*}
\end{example}

De manera informal, podemos fortalecer el Corolario \ref{cor:iso3} con el siguiente argumento. Si
$q_i = 0$ para alguna $i \in \lbrace 1, \ldots, n \rbrace$, podemos proyectar $\vec{q}$ sobre
la subred $\Z^{n-1}$ de $\Z$. Repetimos este proceso hasta que todas las entradas de $\vec{q}$ sean
distintas de cero. Y es sobre esta subred que podemos considerar cualquier permutación en las
entradas del vector $\vec{q}$ proyectado.
\begin{definition}
	Sean $\vec{q}, \tilde{\vec{q}} \in \Z^n$ dos vectores coprimos cuyas entradas son todas
	distintas de cero. Entonces decimos que $\vec{q}$ y $\tilde{\vec{q}}$ son equivalentes si y solo
	si existe $\sigma \in S_n$ tal que $\tilde{\vec{q}} = \sigma.\vec{q}$. En este caso escribimos
	$\vec{q} \sim \tilde{\vec{q}}$.
\end{definition}
Como $\varphi$ es un homomorfismo, es posible mostrar que $\sim$ es una relación de equivalencia
sobre el conjunto de vectores coprimos cuyas entradas son distintas de cero. De esta manera, sabemos
del Corolario \ref{cor:iso3} que si $\vec{q} \sim \tilde{\vec{q}}$, entonces ambos vectores
descomponen la red $\Z^n$ de la misma forma. Es decir, existe un isomorfismo tal que $(\vec{\omega},
M) \mapsto (\tilde{\vec{\omega}}, \tilde{M})$. Podemos entonces empezar a hablar de una
clasificación de programas lineales a partir de las clases de equivalencia de $\vec{q}$. Esto, no
obstante, se encuentra fuera del propósito de la tesis.

En conclusión, somos completamente capaces de caracterizar los puntos enteros sobre la $k$-ésima
capa entera. O lo que es lo mismo, podemos resolver ecuaciones lineales diofantinas con $n$
incógnitas. Estas ecuaciones son inducidas por el vector coprimo $\vec{q}$. Hemos analizado también
como es que $\vec{q}$ descompone el espacio $\Z^n$ a través del vector $\vec{\omega}$ y de la matriz
$M$. Recordemos que $\vec{\omega}$ representa el conjunto de soluciones particulares a estas ecuaciones
lineales, mientras que las columnas de $M$ representan el conjunto de soluciones homogéneas. En la
siguiente sección observamos cómo esta descomposición permite que desacoplemos un programa lineal
entero en dos partes: una de maximización y otra de factibilidad.

\subsection{Múltiples restricciones}
\noindent
En esta sección hacemos un análisis extensivo sobre lo resulta de agregar más restricciones al
problema (\ref{theory:formulation}). Sea $\vec{p} \in \R^n$ esencialmente entero y consideremos su
múltiplo coprimo $\vec{q} \in \Z^n$. Sea $A \in \Q^{m \times n}$ una matriz racional con renglones
linealmente independientes y sea $\vec{b} \in \Q^m$ un vector. Consideremos el problema
\begin{subequations}
	\label{formulation:multiple}
	\begin{align}
		\max_{\vec{x} \in \Z^n} \quad
			& \vec{q}^T\vec{x}, \label{formulation:multiple:objective} \\
		\text{s.a.} \quad
			& \vec{q}^T\vec{x} \leq u, \label{formulation:multiple:constraint:budget} \\
			& A\vec{x} = \vec{b}, \label{formulation:multiple:constraints} \\
			& \vec{x} \geq \vec{0}. \nonumber
	\end{align}
\end{subequations}
Ciertamente, la solución no se encuentra necesariamente en la $\eta$-ésima capa entera. Por ejemplo,
si dejamos que $A \coloneq \vec{q}^T$ y $b \coloneq u - m$, la solución se encontrará en la
$\xi$-ésima capa entera, donde
\begin{equation*}
	\xi \coloneq \left\lfloor \frac{u}{m} - 1 \right\rfloor < \eta.
\end{equation*}
No obstante, si el problema (\ref{formulation:multiple}) es factible, sabemos que la solución se
encontrará en alguna capa entera con parámetro $k \in \lbrace \eta, \eta - 1, \ldots \rbrace$, pues
todavía contamos con una restricción presupuestaria que se debe satisfacer.
\begin{observation}
	Recordemos del Teorema \ref{theory:th:feasibility} que, si tenemos solamente la restricción
	presupuestaria, entonces la utilidad máxima es $\eta$ si $q_i < 0$ para alguna $i \in
	\lbrace 2, \ldots, n - 1\rbrace$. Al igual que en el caso finito, ahora no somos capaces de
	saber inmediatamente en qué capa entera se encuentra nuestra solución.
\end{observation}

Ahora bien, en el contexto del problema (\ref{formulation:multiple}), el parámetro $k \in \Z$ se
encarga de maximizar la utilidad (\ref{formulation:multiple:objective}), así como de respetar el
presupuesto (\ref{formulation:multiple:constraint:budget}) a través de $k \leq \eta$. Similarmente,
el vector $\vec{t} \in \Z^{n-1}$ se encarga de respetar las otras restricciones
(\ref{formulation:multiple:constraints}).
\begin{theorem}
	El problema (\ref{formulation:multiple}) es equivalente al problema de maximización
	\begin{subequations}
		\label{formulation:lattice}
		\begin{align}
			\max_{k \in \Z, \vec{t} \in \Z^{n-1}}
				& k, \\
			\text{s.a.} \quad
				& k \leq \eta, \label{lattice:c-layer} \\
				& AM\vec{t} = kA\vec{\omega} - \vec{b}, \label{lattice:constraints} \\
				& M\vec{t} \geq -k\vec{\omega}.
		\end{align}
	\end{subequations}
\end{theorem}
\begin{proof}
	Por el Teorema \ref{th:lattice}, sabemos que la transformación lineal
	\begin{align*}
		(k, \vec{t}) &\mapsto \vec{x} \coloneq k\vec{\omega} + M\vec{t}
	\end{align*}
	es un isomorfismo entre las redes $\Z \oplus \Z^{n - 1}$ y $\Z^n$. Así, tenemos
	\begin{align*}
		A\vec{x} = \vec{b} &\iff AM\vec{t} = \vec{b} - kA\vec{\omega}, \\
		\vec{x} \geq \vec{0} &\iff M\vec{t} \geq -k\vec{\omega},
	\end{align*}
	y por lo tanto basta mostrar que si un vector es factible para un problema, entonces satisface
	la correspondiente restricción presupuestaria del otro problema. Para ello, es de utilidad
	recordar que $\eta$ parametriza la primera capa entera que satisface el presupuesto.

	Sea $\vec{x} \in \Z^n$ un vector factible de (\ref{formulation:multiple}) Como $\vec{x}$ es
	entero, entonces se debe cumplir $\vec{q}^T\vec{x} \leq \eta$. Ahora bien, existe $(k, \vec{t})
	\in \Z^n$ que satisface $\vec{x} = k\vec{\omega} + M\vec{t}$. Por el Lema \ref{lemma:iso1} y el
	Corolario \ref{lemma:iso2} encontramos que
	\begin{equation*}
		k = \vec{q}^T\vec{x} \leq \eta,
	\end{equation*}
	y entonces $(k, \vec{t})$ es factible. Como $\vec{x}$ fue arbitrario, se sigue que la solución
	del problema (\ref{formulation:multiple}) es una cota inferior del problema
	(\ref{formulation:lattice}). La demostración de que la solución de (\ref{formulation:lattice})
	es una cota inferior de (\ref{formulation:multiple}) es análoga.

	Finalmente, supongamos que $(k, \vec{t}) \in \Z^n$ es solución de (\ref{formulation:lattice}).
	Si existe $\hat{\vec{x}}$ factible para (\ref{formulation:multiple}) con utilidad
	$\vec{q}^T\hat{\vec{x}} = \hat{k}$ estrictamente mayor, entonces consideramos $(\hat{k},
	\hat{\vec{t}})$ tal que $\hat{\vec{x}} = \hat{k}\vec{\omega} + M\hat{\vec{t}}$. Este vector
	también es factible con utilidad $k < \hat{k} \leq \eta$, y entonces $(k, \vec{t})$ no era la
	solución de (\ref{formulation:lattice}). Obtenemos una contradicción.
\end{proof}

\begin{observation}
	El vector objetivo todavía es ortogonal a la restricción presupuestaria. No obstante, es más
	fácil de manejar en caso de usar cortes como en Ramificación y Acotamiento. Si $k^*$ no es
	entero en la solución al problema relajado, la única manera de ramificar es con el nuevo corte
	$k \leq \lfloor k^* \rfloor$, pues el otro corte $k \geq \lceil k^* \rceil$ generará un
	subproblema infactible. Evidentemente, en la sección de análisis de resultados haremos
	comparaciones de tiempo en los tiempos de terminación entre esta formulación y la original.
\end{observation}

La formulación del problema equivalente en el Teorema anterior resulta ser más interesante. Podemos
desacoplar esta nueva formulación de manera que obtengamos un problema de maximización y otro de
factibilidad. Supongamos, sin pérdida de generalidad, que las entradas de $A$ y $\vec{b}$ son
enteras. Como los renglones de $A$ son linealmente independientes, de \cite{alex} sabemos que tiene
una única factorización de Hermite. Es decir, existe una matriz $U \in \Z^{n \times n}$ unimodular
que satisface $AU = [H, \vec{0}]$, donde $H \in \Z^{m \times m}$ es triangular inferior y no
singular.

Consideremos el subproblema de maximización
\begin{subequations}
	\label{subformulation:lattice}
	\begin{align}
		\max_{k \in \Z}
			& ~ k, \\
		\text{s.a.} \quad
		k &\leq \eta, \\
			A\tilde{\vec{y}} &= kA\vec{\omega} - \vec{b},
	\end{align}
\end{subequations}
donde 
\begin{equation*}
	\tilde{\vec{y}} \coloneq U \begin{pmatrix} \tilde{\vec{y}}_m \\ \tilde{\vec{y}}_{n-m} \end{pmatrix}
	= U_m\tilde{\vec{y}}_m + U_{n-m}\tilde{\vec{y}}_{n-m} \in \Z^n,
\end{equation*}
con $\tilde{\vec{y}}_m \in \Z^m$ y $\tilde{\vec{y}}_{n-m} \in \Z^{n-m}$. Así también, $U_m$ y
$U_{n-m}$ denotan las primeras $m$ columnas y últimas $n - m$ columnas de $U$, respectivamente.
Observemos que para toda $k \in \Z$ se cumple
\begin{equation}
	AU \begin{pmatrix} \inv{H}\left(kA\vec{\omega} - \vec{b}\right) \\ \tilde{\vec{y}}_{n-m} \end{pmatrix}
	=
	[H, \vec{0}] \begin{pmatrix} \inv{H}\left(kA\vec{\omega} - \vec{b}\right) \\ \tilde{\vec{y}}_{n-m} \end{pmatrix}
	= kA\vec{\omega} - \vec{b},
\end{equation}
lo cual sugiere definir $\tilde{\vec{y}}_m \coloneq \inv{H}(\vec{b} - kA\vec{w})$. No obstante,
también debemos asegurarnos que este vector sea entero. Observemos que $\tilde{\vec{y}}_{n-m}$ queda
libre, así que en realidad este subproblema tiene dimensión $m + 1$. Definimos el conjunto de
factibilidad
\begin{equation}
	\label{eq:feas-set}
	F \coloneq \lbrace k \in \Z \vcentcolon \inv{H}\left(kA\vec{\omega} - \vec{b}\right) \in \Z^m \rbrace
	\cap \lbrace k \in \Z \vcentcolon k \leq \eta \rbrace.
\end{equation}
\begin{observation}
	Para que $F$ sea no vacío, debe existir $k \in \Z$ tal que $\det(H) \mid (k\vec{a}_j^T
	\vec{\omega} - b_j)$ para todo $j \in \lbrace 1, \ldots, m \rbrace$, donde $\vec{a}^T_j$
	denota el $j$-ésimo vector renglón de $A$. Es decir, una condición suficiente y necesaria para
	la no vacuidad de $F$ es
	\begin{equation*}
		\det(H) \mid \gcd{k\vec{a}_1^T\vec{\omega} - b_1, \ldots, k\vec{a}_m^T\vec{\omega} - b_m}.
	\end{equation*}
	Ahora bien, $H$ es triangular inferior e invertible, por lo que $\det(H) \neq 0$ es el producto
	de los elementos $h_1, \ldots, h_m$ en su diagonal. Entonces $h_j \mid \det(H)$ para todo $j \in
	\lbrace 1, \ldots m \rbrace$ y una condición necesaria para la no vacuidad de $F$ es
	\begin{equation*}
		\lcm{h_1, \ldots, h_m} \mid \gcd{k\vec{a}_1^T\vec{\omega} - b_1, \ldots, k\vec{a}_m^T\vec{\omega} - b_m}.
	\end{equation*}
\end{observation}

Si $F$ es vacío, deducimos que este subproblema es infactible y por lo tanto
(\ref{formulation:lattice}) también lo es. Supongamos, pues, que $F \neq \emptyset$. No es difícil
observar que $F$ tiene un elemento maximal $k^*$ y que este elemento es la solución al subproblema
(\ref{subformulation:lattice}). Luego, dada esta solución $k^* \in \Z$, buscamos resolver el
subproblema de factibilidad
\begin{subequations}
	\label{subformulation:feasibility}
	\begin{align}
		M\vec{t} &= \tilde{\vec{y}}, \\
		M\vec{t} &\geq -k^*\vec{\omega}.
	\end{align}
\end{subequations}
Observemos que tenemos un sistema de $n$ ecuaciones lineales con $2n - m - 1$ incógnitas, por lo que
tendremos que lidiar con $n - m - 1$ parámetros libres:
\begin{align}
	\label{eq:feasibility-eqs}
	M\vec{t} = \tilde{\vec{y}} = U_m\tilde{\vec{y}}_m + U_{n-m}\tilde{\vec{y}}_{n-m}
   \iff [M, -U_{n-m}] \begin{pmatrix} \vec{t} \\ \tilde{\vec{y}}_{n-m} \end{pmatrix} = U_m\tilde{\vec{y}}_m.
\end{align}
Si consideramos ahora la forma normal de Smith de esta matriz por bloques, obtenemos dos matrices
unimodulares $S \in \Z^{n \times n}$ y $T \in \Z^{(2n - m - 1) \times (2n - m -1)}$ que satisfacen
\begin{equation*}
	S[M, -U_{n-m}]T = D \in \Z^{n \times (2n - m - 1)},
\end{equation*}
donde $D$ es una matriz diagonal cuyas $n$ primeras entradas son distintas de cero y las restantes
$n - m - 1$ son cero. Si multiplicamos $S$ por la izquierda en ambos lados de la ecuación
(\ref{eq:feasibility-eqs}), tenemos
\begin{equation*}
	D\inv{T}\begin{pmatrix} \vec{t} \\ \tilde{\vec{y}}_{n-m} \end{pmatrix}
	= SU_m\tilde{\vec{y}}_{m}.
\end{equation*}
Si $d_i$ no divide a $(SU_m\tilde{\vec{y}}_{m})_i$ para alguna $i \in \lbrace 1, \ldots, n \rbrace$,
encontramos que la primera ecuación del subproblema (\ref{subformulation:feasibility}) no tiene
solución en los enteros, lo que implica que la elección de $k^*$ fue la incorrecta para asegurar
soluciones enteras a este subproblema. De ser este el caso, redefinimos $F \leftarrow F \setminus
\lbrace k^* \rbrace$. Si $F$ ahora es vacío, entonces (\ref{formulation:lattice}) es
infactible, de caso contrario escogemos el nuevo elemento de maximal de $F$ y repetimos el proceso.

Supongamos, pues que $d_i \mid (SU_m\tilde{\vec{y}}_{m})_i$ para todo $i \in \lbrace 1, \ldots,
n\rbrace$, por lo que obtenemos $n$ soluciones enteras $\vec{r} \in \Z^n$ y $n - m - 1$ variables
libres $\vec{s} \in \Z^{n-m-1}$:
\begin{equation*}
	\inv{T}\begin{pmatrix} \vec{t} \\ \tilde{\vec{y}}_{n-m} \end{pmatrix}
	=
	\begin{pmatrix} \vec{r} \\ \vec{s} \end{pmatrix}.
\end{equation*}
Por lo tanto, nuestro vector $\vec{t}$ es una función lineal de $\vec{s}$, es decir, $\vec{t} =
\vec{t}(\vec{s})$. Hasta este punto el proceso no ha sido complicado, pues nos hemos encargado de
resolver sistemas de ecuaciones lineales diofantinas. En términos del problema original
(\ref{formulation:multiple}), hemos encontrado los vectores $\vec{x}(\vec{s}) \coloneq
k^*\vec{\omega} + M\vec{t}(\vec{s})$ que maximizan la utilidad y que satisfacen todas las
restricciones excepto, posiblemente, las de no negatividad.

La dificultad entra en juego cuando queremos determinar el vector de variables libres $\vec{s} \in
\Z^{n-m-1}$ que hagan que $\vec{t}(\vec{s})$ satisfaga la desigualdad en el subproblema
(\ref{subformulation:feasibility}). Debilitando más esta condición, nos gustaría determinar si el
conjunto
\begin{equation*}
	\lbrace \vec{s} \in \Z^{n-m-1} \vcentcolon M\vec{t}(\vec{s}) \geq -k^*\vec{\omega} \rbrace
\end{equation*}
es vacío o no. En esta versión debilitada no nos interesa saber qué elementos contiene o tan
siquiera cuántos elementos contiene. Es sabido que los programas enteros tales como
(\ref{formulation:multiple}) o (\ref{formulation:lattice}) son problemas difíciles de resolver, en
el sentido de que no es conocido si se pueden resolver en tiempo polinomial. A lo largo de este
capítulo, no obstante, hemos resuelto todos los problemas en tiempo polinomial\footnote{En
	\cite{alex} se muestra que calcular el máximo común divisor, resolver ecuaciones lineales
	diofantinas, y calcular las factorizaciones tanto de Hermite como de Smith son operaciones
	acotadas por tiempo polinomial.}.
La única deducción posible, entonces, es que el problema de determinar las variables $\vec{s}$, o
bien de determinar cuántas hay, o bien de determinar su existencia, son todos problemas difíciles de
resolver.

A pesar de lo anterior, hay dos casos donde la dificultad se reduce drásticamente. El caso menos
interesante es cuando $m = n - 1$, de manera que no hay parámetros libres. Esto se debe a que el
politopo factible resultante es un semirrayo o un segmento de línea. Al momento de escoger la
$k^*$-ésima capa entera, estamos agregando la ecuación $k^* = k$, con lo que obtenemos un sistema
lineal entero de $n$ ecuaciones con $n$ incógnitas, y entonces la solución es única. Basta entonces
verificar que este único vector $\vec{t}$ satisface la desigualdad en el subproblema
(\ref{subformulation:feasibility}). El caso un poco más interesante se obtiene cuando $m = n - 2$.
De esta manera obtenemos un solo parámetro, con lo que podemos determinar rápidamente la existencia
o inexistencia de un intervalo de factibilidad.

\begin{example}
	\label{ex:two-var}
	Consideremos el problema con $n = 2$ variables y $m = 1$ restricciones
	\begin{align*}
		\max
			~& x - y, \\
		\text{s.a.} \quad
			& x - y \leq 12, \\
			& 3x + 5y = 25, \\
			& x, y \geq 0.
	\end{align*}
	En este caso tenemos $A = (3, 5), \vec{b} = 25$, y también $\vec{q} = (1, -1)^T$, al igual que
	$\eta = 12$. De (\ref{eq:vec-omega}) y (\ref{eq:mat-T}) obtenemos
	\begin{equation*}
		\vec{\omega} = \begin{pmatrix} 1 \\ 0 \end{pmatrix},
		M = \begin{pmatrix} -1 \\ -1 \end{pmatrix}.
	\end{equation*}
	De la forma normal de Hermite de $A$ tenemos
	\begin{equation*}
		H = 1, U = \begin{pmatrix} 2 & -5 \\ -1 & 3 \end{pmatrix},
	\end{equation*}
	y de la forma normal de Smith de $[M, -U_m]$,
	\begin{equation*}
		S = \begin{pmatrix} -1 & 0 \\ 1 & -1 \end{pmatrix},
		D = \begin{pmatrix} 1 & 0 \\ 0 & 8 \end{pmatrix},
		T = \begin{pmatrix} 1 & 5 \\ 0 & 1 \end{pmatrix}. 
	\end{equation*}

	Como $H = 1$, se sigue que $\inv{H} (\vec{b} - kA\vec{\omega}) = 25 - 3k$ es entero para todo $k
	\in \Z$. Así, el conjunto factible $F$ (c.f. \ref{eq:feas-set}) está dado por
	\begin{equation*}
		F = \Z \cap \lbrace k \in \Z \vcentcolon k \leq 12 \rbrace
		= \lbrace k \in \Z \vcentcolon k \leq \eta = 12 \rbrace.
	\end{equation*}
	Entonces escogemos $k^* = 12$ por ser el elemento maximal de $F$. Así, encontramos
	\begin{equation*}
		SU_m\tilde{\vec{y}}_m = SU_m \left(\inv{H} (\vec{b} - k^*A\vec{\omega})\right)
		= \begin{pmatrix} 22 \\ 33 \end{pmatrix}
	\end{equation*}
	Observemos que la segunda entrada de $SU_m\tilde{\vec{y}}_m$ no es divisible por $D_{22} = 8$.
	Así, el subproblema (\ref{subformulation:feasibility}) no es factible para la elección de $k^*$
	previa. Escogemos el segundo elemento de $F$ más grande, con lo que tenemos $k^* \leftarrow 11$.
	En este caso obtenemos $SU_m\tilde{\vec{y}}_m = (-16, -24)^T$, por lo que sí hay soluciones
	enteras. Luego, se debe satisfacer,
	\begin{equation*}
		\inv{T} \begin{pmatrix} \vec{t} \\ \tilde{\vec{y}}_{n-m} \end{pmatrix} =
		\begin{pmatrix} 16 \\ 24 \end{pmatrix},
	\end{equation*}
	de donde se sigue que $(\vec{t}, \tilde{\vec{y}}_{n-m}) = (1, 3)$. Verificamos factibilidad:
	\begin{equation*}
		M\vec{t} + k^*\vec{\omega}
		= 1 \begin{pmatrix} -1 \\ -1 \end{pmatrix} + 11 \begin{pmatrix} 1 \\ 0 \end{pmatrix}
		= \begin{pmatrix} 10 \\ -1 \end{pmatrix} \not \geq \vec{0}.
	\end{equation*}
	Ahora la elección de $k^*$ dio un punto entero pero con una entrada negativa. Seguimos este
	procedimiento hasta llegar a $k^* \leftarrow 3$. En este caso obtenemos $(\vec{t},
	\tilde{\vec{y}}_{n-m}) = (-2, -6)^T$, de donde
	\begin{equation*}
		M\vec{t} + k^*\vec{\omega}
		= -2 \begin{pmatrix} -1 \\ -1 \end{pmatrix} + 3 \begin{pmatrix} 1 \\ 0 \end{pmatrix}
		= \begin{pmatrix} 5 \\ 2 \end{pmatrix} \geq \vec{0}.
	\end{equation*}
	Concluimos diciendo que $(k^*, \vec{t}) \coloneq (3, -2)$ es el óptimo del programa
	(\ref{formulation:lattice}) y entonces $(x, y) = (5, 2)$ es el óptimo de
	(\ref{formulation:multiple}).
\end{example}
\begin{example}
	Ahora consideremos el problema con $n = 3$ variables y $m = 1$ restricciones
	\begin{align*}
		\max
			~& x - y + 2z, \\
		\text{s.a.} \quad
			& x - y  + 2z \leq 10 \\
			& 3x + 4y - z = 15 \\
			& x, y, z \geq 0.
	\end{align*}
	En este caso tenemos $A = (3, 4, -1), \vec{b} = 15$, y también $\vec{q} = (1, -1, 2)^T$, al igual que
	$\eta = 10$. De (\ref{eq:vec-omega}) y (\ref{eq:mat-T}) obtenemos
	\begin{equation*}
		\vec{\omega} = \begin{pmatrix} 1 \\ 0 \\ 0 \end{pmatrix},
		M = \begin{pmatrix} 1 & 0 \\ -1 & 2 \\ -1 & 1 \end{pmatrix}.
	\end{equation*}
	De la forma normal de Hermite de $A$ tenemos
	\begin{equation*}
		H = 1, U = \begin{pmatrix} 0 & 0 & 1 \\ 0 & 1 & 0 \\ -1 & 4 & 3 \end{pmatrix},
	\end{equation*}
	y de la forma normal de Smith de $[M, -U_m]$,
	\begin{equation*}
		S = \begin{pmatrix}
			1 & 0 & 0 \\
			-1 & -1 & 0 \\
			3 & 4 & -1
		\end{pmatrix},
		D = \begin{pmatrix}
			1 & 0 & 0 & 0 \\
			0 & 1 & 0 & 0 \\
			0 & 0 & 7 & 0
		\end{pmatrix},
		T = \begin{pmatrix}
			1 & 0 & 0 & 1 \\
			0 & 0 & 1 & 0 \\
			0 & 1 & 2 & -1 \\
			0 & 0 & 0 & 1
		\end{pmatrix}.
	\end{equation*}
	Nuevamente, observemos que $H = 1$ y por lo tanto $F = \lbrace k \in \Z \vcentcolon k \leq 10
	\rbrace$. Seguimos exactamente el mismo procedimiento que en el Ejemplo \ref{ex:two-var} hasta
	llegar a $k^* \leftarrow 5$. Encontramos que se satisface
	\begin{equation*}
		\inv{T} \begin{pmatrix} \vec{t} \\ \tilde{\vec{y}}_{n-m} \end{pmatrix}
		=
		\begin{pmatrix} 0 \\ 0 \\ 0 \\ s \end{pmatrix}
		\implies
		\begin{pmatrix} \vec{t} \\ \tilde{\vec{y}}_{n-m} \end{pmatrix}
		=
		s \begin{pmatrix} 1 \\ 0 \\ -1 \\ 1 \end{pmatrix},
	\end{equation*}
	donde $s \in \Z$ es la única variable libre. En este caso podemos determinar rápidamente un
	intervalo de existencia: tenemos $M\vec{t} \geq -k^*\vec{\omega}$ si y solo si
	\begin{equation*}
		s\begin{pmatrix} 1 \\ 0 \\ -1 \end{pmatrix} \geq
		\begin{pmatrix} -5 \\ 0 \\ 0 \end{pmatrix},
	\end{equation*}
	de donde se sigue inmediatamente que $s \in \lbrace -5, -4, \ldots, 0 \rbrace$. Sustituyendo en
	$\vec{t}$ y transformando a $\vec{x}$, encontramos que
	\begin{equation*}
		\left\lbrace
			\begin{pmatrix} 0 \\ 5 \\ 5 \end{pmatrix},
			\begin{pmatrix} 1 \\ 4 \\ 4 \end{pmatrix},
			\begin{pmatrix} 2 \\ 3 \\ 3 \end{pmatrix},
			\begin{pmatrix} 3 \\ 2 \\ 2 \end{pmatrix},
			\begin{pmatrix} 4 \\ 1 \\ 1 \end{pmatrix},
			\begin{pmatrix} 5 \\ 0 \\ 0 \end{pmatrix}
		\right\rbrace
	\end{equation*}
	son las seis soluciones del problema. Todas alcanzan un nivel de utilidad $k^* = 5$.
\end{example}

Si el programa (\ref{formulation:multiple}) es factible, entonces el programa
(\ref{formulation:lattice}) también lo es. A partir de nuestro procedimiento, eventualmente
encontraremos un par $(k^*, \vec{t}^*)$ que resuelva tanto el subproblema de maximización
(\ref{subformulation:lattice}) como el de factibilidad (\ref{subformulation:feasibility}).

Ahora bien, son dos las maneras en las que nuestro problema sea infactible. Puede que nuestro
conjunto de factibilidad $F$ sea vacío y por lo tanto el sistema de ecuaciones lineales
(\ref{formulation:multiple:constraints}) sea inconsistente. O bien, puede ser que $F$ tenga
cardinalidad infinita pero para ninguno de sus elementos se satisfaga el subproblema de
factibilidad.

% TODO: mostrar una imagen.
Esto último puede ocurrir cuando el sistema de ecuaciones siempre tiene solución pero todas ellas
son negativas. En efecto, si en el Ejemplo \ref{ex:two-var} reemplazamos el lado derecho de la
igualdad $\vec{b} = 25$ por $\vec{b} = -4$, nos encontramos en aquella situación.

En conclusión, para asegurar terminación en tiempo finito, cualquier algoritmo basado en este método debe
asegurarse primero que el conjunto de factibilidad $F$ tiene un número finito de puntos. Este caso
lo estudiamos en la siguiente sección.

\subsection{Eliminando la restricción presupuestaria}
\noindent
Consideremos ahora el problema
\begin{subequations}
	\label{formulation:last}
	\begin{align}
		\max_{\vec{x} \in \Z^n} \quad
			& \vec{q}^T\vec{x}, \label{formulation:last:objective} \\
		\text{s.a.} \quad
			& A\vec{x} = \vec{b}, \label{formulation:last:constraints} \\
			& \vec{x} \geq \vec{0}, \nonumber
	\end{align}
\end{subequations}
Evidentemente, si su programa relajado tiene un valor objetivo $u^*$ finito, podemos agregar la
restricción presupuestaria $\vec{q}^T\vec{x} \leq u^*$ a este problema de manera válida. Entonces
podemos suponer sin pérdida de generalidad que este programa es equivalente a
(\ref{formulation:multiple}) siempre que su valor objetivo sea finito. Consecuentemente, podemos
utilizar las herramientas desarrolladas en la sección pasada para resolver este problema entero.

Es más, supongamos que el politopo asociado al problema relajado es acotado y no vacío. Entonces
tanto el problema de maximización como de minimización tienen valores objetivos finitos. Llamemos a
estos valores $\ell^*$ y $u^*$, respectivamente. Ahora la restricción
\begin{equation*}
	\ell^* \leq \vec{q}^T\vec{x} \leq u^*
\end{equation*}
es válida para el problema (\ref{formulation:last}). De la misma manera que $\eta$ parametriza la
primera capa entera que satisface el presupuesto, podemos definir análogamente la última capa que
satisface el presupuesto. Usando el mismo razonamiento que en el Lema \ref{phase-1:lemma:eta},
encontramos que esta capa está parametrizada por $\tau \coloneq \lceil \ell^*/m \rceil$ si $m$ es
positiva. Así pues, al definir nuestro conjunto de factibilidad $F$ como
\begin{equation*}
	F \coloneq \lbrace k \in \Z \vcentcolon \inv{H}\left(kA\vec{\omega} - \vec{b}\right) \in \Z^m \rbrace
	\cap \lbrace k \in \Z \vcentcolon \tau \leq k \leq \eta \rbrace,
\end{equation*}
podemos replicar las mismas técnicas que en la sección pasada. Pero además, $F$ es un conjunto
finito y por lo tanto tenemos terminación en tiempo finito para este caso. Es decir, cualquier
algoritmo basado en los métodos desarrollados en la sección pasada podrá decidir en tiempo finito si
el problema es factible o no. En caso de que sí lo sea entonces terminará con la solución óptima.

Existen varios algoritmos para resolver el problema relajado de (\ref{formulation:last}) en su
versión general. Es cierto que el método del simplex es el más utilizado, a pesar de tener una
complejidad algorítmica no acotada polinomialmente. También es cierto que existen métodos
polinomiales para resolver este problema, tales como el método elipsoidal o el algoritmo de
Karmarkar. Pero más interesante es el hecho de que ya existen cotas superiores para ciertas
instancias de estos problemas, por ejemplo, en el caso del Problema de la Mochila, \cite{martello}
provee una cota superior razonable, y ciertamente el valor de 0 es una cota inferior justa. Mucho
hablaremos de este problema en el Capítulo 3. No obstante, el autor considera prudente dedicar el
siguiente capítulo para el caso infinito.

\chapter{El caso infinito}
\label{chap:inf}

% TODO: resumen del capítulo

\noindent
Sea $\vec{p} \in \R^n \setminus \braces{\vec{0}}$ un vector esencialmente entero y recordemos de la
definición \ref{theory:def:rational} que tiene un único múltiplo coprimo $\vec{q} \in \Z^n$. Es
decir, existe un único escalar $m \in \R$ que satisface tres cosas: $\vec{p} = m\vec{q}$, las
entradas $q_1, \ldots, q_n$ son coprimas, y la primera entrada no nula $q_i$ es positiva. Al igual
que en el capítulo anterior, supondremos que $m$ es positivo. Equivalentemente, supondremos que la
primera entrada no nula $p_i$ es positiva\footnote{ El autor hace recordar que esta es una cuestión
	puramente de comodidad y no hay pérdida de generalidad. Cuando $m$ es negativo, los resultados
se mantienen pero es necesario voltear las desigualdades y cambiar las funciones piso por las
funciones techo, lo cual añadiría un número innecesario de casos a analizar.}.

Retomemos el entero $\eta \in \Z$ del lema \ref{phase-1:lemma:eta} que parametriza la primera capa
entera que satisface el presupuesto \eqref{theory:constraint:budget}. A causa del teorema
\ref{theory:th:feasibility} sabemos que si $q_i \leq 0$ para alguna $i \in \braces{1, \ldots, n}$,
entonces la $\eta$-ésima capa entera contiene un número infinito de puntos factibles. A partir de
esto último, somos capaces de resolver el automáticamente el problema de decisión de determinar si
un escalar $u^* \in \R$ es el valor óptimo del programa (\ref{theory:formulation}).
\begin{corollary}
	\label{cor:inf:obj}
	Supongamos que $q_i \leq 0$ para algún $i \in \braces{1, \ldots, n}$. Entonces el valor óptimo
	del programa lineal entero (\ref{theory:formulation}) es $m\eta$. Además, si $m$ es positivo,
	tenemos que $\eta$ es el múltiplo de $m$ más grande que satisface $m\eta \leq u$, donde $u$ es
	el lado derecho de la restricción presupuestaria \eqref{theory:constraint:budget}.
\end{corollary}
\begin{proof}
	Por el teorema \ref{theory:th:feasibility} sabemos que existen una infinidad de soluciones en la
	$\eta$-ésima capa entera, así que sea $\vec{x}^*$ una de ellas. Entonces $\vec{q}^T\vec{x}^* =
	\eta$, pero $\vec{p} = m\vec{q}$ por la definición \ref{theory:def:rational}, por lo que
	obtenemos $\vec{p}^T\vec{x}^* = m\vec{q}^T\vec{x}^* = m\eta$.

	Ahora bien, si $m$ es positivo, por el lema \ref{phase-1:lemma:eta} tenemos que
	$\eta = \lfloor u/m \rfloor$
	. Supongamos que $\xi \in \Z$ satisface $m\xi \leq u$ y también $\lfloor
	u/m \rfloor < \xi$. Luego,
	\begin{equation*}
		m\left\lfloor \frac{u}{m} \right\rfloor < m\xi \leq u
		\implies \left\lfloor \frac{u}{m} \right\rfloor < \xi \leq \frac{u}{m},
	\end{equation*}
	pero esto contradice las propiedades de la función piso.
\end{proof}

\begin{observation}
	Para ilustrar la conveniencia de restringir $m$ a que sea positivo, consideremos el caso cuando
	$m < 0$. De una manera similar a la del lema \ref{phase-1:lemma:eta}, podemos demostrar que
	$\eta \coloneq \lceil u/m \rceil$ parametriza también la primera capa entera que satisface el
	presupuesto, pues ahora tenemos de la restricción \eqref{theory:constraint:budget} que
	$\vec{p}^T\vec{x} \leq u$ si y solo si $\vec{q}^T\vec{x} \geq u/m$. Se sigue cumpliendo que el
	valor óptimo del problema \eqref{theory:formulation} es $m\eta$. Sin embargo, $\eta$ ahora es el
	múltiplo más chico de $m$ que satisface $m\eta \geq u$.
\end{observation}

Una vez resuelto el problema de decisión, podemos preguntarnos concretamente cómo obtener el punto
óptimo. Por el teorema \ref{theory:th:feasibility} sabemos que debemos resolver la ecuación lineal
diofantina $\vec{q}^T\vec{x} = \eta$. Del teorema \ref{th:lattice} sabemos que si $q_n \neq 0$,
entonces existen $k \in \Z$ y $\vec{t} \in \Z^{n-1}$ tales que
\begin{equation*}
	\vec{x} = k\vec{\nu} + M\vec{t},
\end{equation*}
donde $\vec{\nu}$ y $M$ están definidas por \eqref{eq:vec-omega} y \eqref{eq:mat-T},
respectivamente. De los lemas \ref{lemma:iso1} y \ref{lemma:iso2} sabemos que
\begin{equation*}
	\vec{q}^T\left(\eta\vec{\nu} + M\vec{t}\right) = \eta\vec{q}^T\vec{\nu} + \vec{q}^TM\vec{t} = \eta
\end{equation*}
para todo $\vec{t} \in \Z^{n-1}$. Así pues, debe ser el caso que $k = \eta$ y debemos encontrar
condiciones suficientes en $\vec{t}$ para asegurar la no-negatividad de $\vec{x}$. En primer lugar,
sabemos que para todo $i \in \braces{1, \ldots, n - 2}$, la entrada $t_i$ debe satisfacer
\eqref{eq:param-lb}. En segundo lugar, recuperamos de \eqref{eq:last-solution} que las últimas dos
soluciones de la ecuación $\vec{q}^T\vec{x} = \eta$ están dadas por
\begin{equation*}
	\begin{cases}
		x_{n-1} = \omega_{n-1}x_{n-1}' + \frac{q_n}{\prod_{j=1}^{n-1}g_j}t_{n-1}, \\
		x_n = \omega_{n-1}x_n' - \frac{q_{n-1}}{\prod_{j=1}^{n-1}g_j}t_{n-1},
	\end{cases}
\end{equation*}
donde los enteros $g_i$ están definidos por \eqref{dummy:next-g} con $g_1 = 1$, $\omega_{n-1}$ está
definida a través de la relación de recurrencia \eqref{eq:omega-recurrence} con condición inicial
$\omega_1 = \eta$ (o bien a partir del lema \ref{eq:omega-formula} con $k = \eta$), y $x_{n-1}',
x_n'$ son coeficientes de Bézout que satisfacen \eqref{eq:last-equation-bez}.

En lo que se encuentra a continuación supondremos que ninguna entrada $q_i$ es nula. Esto no
constituye problema alguno debido a un razonamiento similar al de la demostración del teorema
\ref{theory:th:feasibility}. Definimos
\begin{equation*}
	I^\circ \coloneq \braces{i \colon q_i = 0},
\end{equation*}
y también definimos el vector $\tvec{q}$ cuyas entradas son las entradas no nulas de
$\vec{q}$. A partir de lo que sigue vamos a determinar un vector entero no nulo $\tvec{x}$
que satisfaga $\tvec{q}^T\tvec{x} = \eta$. Luego, encontramos que el vector $\vec{x}$
dado por
\begin{equation*}
	x_i \coloneq
	\begin{cases}
		\tilde{x}_i, & i \not \in I^\circ, \\
		0, & i \in I^\circ,
	\end{cases}
\end{equation*}
es entero, no negativo, y también satisface $\vec{q}^T\vec{x} = \eta$. Así pues, la suposición de
que $q_i \neq 0$ para todo $i \in \braces{1, \ldots, n}$ toma lugar sin pérdida de generalidad.

Para que se satisfagan las condiciones de no negatividad de $x_{n-1}$ y de $x_n$, encontramos que
$t_{n-1} \in \Z$ debe cumplir ciertas desigualdades según los signos de $q_{n-1}$ y de $q_n$.
Definamos, por conveniencia,
\begin{equation}
	\label{eq:lr-bounds}
	b_1 \coloneq -\frac{\omega_{n-1}x_{n-1}'}{q_n} \cdot \prod_{j=1}^{n-1}g_j, \quad
	b_2 \coloneq \frac{\omega_{n-1}x_{n}'}{q_{n-1}} \cdot \prod_{j=1}^{n-1}g_j.
\end{equation}
Entonces, para asegurar la no-negatividad de $x_{n - 1}$ y de $x_n$, debe ser el caso que
\begin{equation}
	\label{eq:feasible-param}
	t_{n-1} \in 
	\begin{cases}
		\big[ \lceil \max\lbrace b_1 ,  b_2 \rbrace \rceil, \infty \big), &  q_{n-1} < 0 < q_n, \\[0.5em]
		\big( -\infty, \lfloor \min\lbrace b_1, b_2\rbrace \rfloor \big], &  q_n < 0 < q_{n-1}, \\[0.5em]
		\big[ \lceil b_2 \rceil, \lfloor b_1 \rfloor \big], &  q_{n-1}, q_n < 0, \\[0.5em]
		\big[ \lceil b_1 \rceil, \lfloor b_2 \rfloor \big], &  0 < q_{n-1}, q_n.
	\end{cases}
\end{equation}

Podemos emplear la misma estrategia de permutar las entradas de $q_i$ de manera que colapsemos estos
cuatro casos distintos en uno solo. Como estamos en el caso infinito del teorema
\ref{theory:th:feasibility}, naturalmente supondremos que $q_i < 0$ para alguna $i \in \braces{1,
\ldots n}$. Así pues, podemos permutar esta $i$-ésima entrada de $\vec{q}$ con $q_{n-1}$, con lo que
obtenemos $q_{n-1} < 0$. Luego, como $\vec{q}$ es el múltiplo coprimo de $\vec{p}$ y ninguna entrada
de $\vec{q}$ es nula, se sigue de la definición \ref{theory:def:rational} que $q_1 > 0$. Así pues,
podemos permutar la primera y última entrada de $\vec{q}$, de donde se sigue que $q_n > 0$.
Juntándolo todo, obtenemos $q_{n-1} < 0 < q_n$. De esta manera, para asegurar la no negatividad de
$x_{n-1}$ y $x_n$, basta con que se satisfaga el primer caso:
\begin{equation}
	\label{eq:feasible-param:collapsed}
	t_{n-1} \geq \ceil{\max\lbrace b_1, b_2 \rbrace}.
\end{equation}

\begin{lemma}
	\label{lemma:t-existence}
	Sea $\vec{p} \in \R$ un vector cuyas entradas son todas distintas de cero, y sea $\vec{q} \in
	\Z^n$ su múltiplo coprimo. Entonces existe un vector $\vec{t} \in \Z^{n-1}$ que satisface ambos
	\eqref{eq:param-lb} y \eqref{eq:feasible-param}.
\end{lemma}
\begin{proof}
	Por la discusión anterior, podemos suponer sin pérdida de generalidad que $q_{n-1} < 0 < q_n$,
	así que basta mostrar la existencia de  $\vec{t} \in \Z^{n-1}$ que satisfaga \eqref{eq:param-lb} y
	\eqref{eq:feasible-param:collapsed}. Si definimos
	\begin{equation*}
		t_i \coloneq \begin{cases}
			\ceil{-\frac{\omega_ix_i'}{g_{i + 1}}}, & i < n - 1, \\[0.5em]
			\ceil{\max\lbrace b_1, b_2 \rbrace}, & i = n - 1,
		\end{cases}
	\end{equation*}
	entonces se verifica automáticamente que estas condiciones se satisfacen.
\end{proof}

% \begin{lemma}
% 	\label{lemma:t-existence}
% 	Sea $\vec{p} \in \R$ un vector cuyas entradas son todas distintas de cero, y sea $\vec{q} \in
% 	\Z^n$ su múltiplo coprimo. Entonces existe un vector $\vec{t} \in \Z^{n-1}$ que satisface ambos
% 	\eqref{eq:param-lb} y \eqref{eq:feasible-param}.
% \end{lemma}
% \begin{proof}
% 	Tenemos cuatro casos, pero observemos que los dos en donde $q_{n - 1}$ y $q_n$
% 	tienen signo distinto no son difíciles: si $q_{n - 1} <0 < q_n$, entonces el vector
% 	$\vec{t} \in \Z^{n-1}$ dado por
% 	\begin{equation*}
% 		t_i \coloneq \begin{cases}
% 			\left\lceil -\frac{\omega_ix_i'}{g_{i + 1}} \right\rceil, & i < n - 1, \\[0.5em]
% 			\lceil \max\lbrace b_1, b_2 \rbrace \rceil, & i = n - 1,
% 		\end{cases}
% 	\end{equation*}
% 	satisface ambos (\ref{eq:param-lb}) y (\ref{eq:feasible-param}). El caso $q_n < 0 <
% 	q_{n - 1}$ es completamente similar.
% 
% 	Consideremos el caso $q_{n - 1}, q_n < 0$. Como ninguna entrada de $\vec{p}$ es nula y el vector
% 	$\vec{q}$ es su múltiplo coprimo, se sigue de la definición \ref{theory:def:rational} que $q_1 >
% 	0$. Podemos entonces permutar las entradas $q_1$ y $q_{n}$ para regresar al primer caso.
% 
% 	Finalmente, consideremos el caso $0 < q_{n - 1}, q_n$. Queremos encontrar condiciones
% 	suficientes para asegurar que el intervalo $[\ceil{b_1}, \floor{b_2}]$ esté bien definido, es
% 	decir, para asegurar que $\floor{b_2} - \ceil{b_1} \geq 0$. Podemos suponer sin pérdida de
% 	generalidad que $q_{n - 2} < 0$. En efecto, como $q_i < 0$ para alguna $i \in \braces{1, \ldots,
% 	n - 2}$, somos capaces permutar las entradas $i$ y $n - 2$ de $\vec{q}$. Observemos que
% 	\begin{align*}
% 		b_2 - 1 &\leq \lfloor b_2 \rfloor \leq b_2, \\
% 		b_1 &\leq \lceil b_1 \rceil \leq b_1 + 1.
% 	\end{align*}
% 	De donde obtenemos
% 	\begin{equation*}
% 		b_2 - b_1 - 2 \leq \lfloor b_2 \rfloor - \lceil b_1 \rceil \leq b_2 - b_1.
% 	\end{equation*}
% 	Así pues, para que el intervalo $[\lceil b_1 \rceil, \lfloor b_2 \rfloor]$ esté bien definido,
% 	es suficiente con mostrar que existe un escalar $\omega_{n - 1}$ que satisfaga $b_2 - b_1 \geq
% 	2$. De \eqref{eq:lr-bounds} tenemos
% 	\begin{equation}
% 		\label{proof:b-sub}
% 		b_2 - b_1 = \omega_{n - 1}\prod_{j = 1}^{n-1}g_j \cdot
% 			\left(\frac{x_{n-1}'}{q_n} + \frac{x_n'}{q_{n - 1}}\right)
% 	\end{equation}
% 	Pero $x_{n-1}'$ y $x_n'$ satisfacen \eqref{eq:last-equation-bez}, y entonces
% 	\begin{equation*}
% 		\frac{x_{n-1}'}{q_n} + \frac{x_n'}{q_{n - 1}} = \frac{1}{q_{n-1}q_n} \cdot \prod_{j =
% 		1}^{n-1}g_j.
% 	\end{equation*}
% 	Sustituyendo en (\ref{proof:b-sub}),
% 	\begin{equation*}
% 		b_2 - b_1 = \frac{\omega_{n-1}}{q_{n-1}q_n} \cdot \prod_{j=1}^{n-1}g_j^2,
% 	\end{equation*}
% 	por lo que $b_2 - b_1 \geq 2$ si y solo si
% 	\begin{equation}
% 		\label{proof:omega-sub}
% 		\omega_{n-1} \geq 2\frac{q_{n-1}q_n}{\prod_{j=1}^{n-1}g_j^2}.
% 	\end{equation}
% 	De (\ref{eq:recurrence}) sabemos que
% 	\begin{equation*}
% 		\omega_{n-1} = \omega_{n-2}\omega_{n-1}' -
% 		\frac{q_{n-2}}{\prod_{j=1}^{n-2}g_j}t_{n-2}.
% 	\end{equation*}
% 	Sustituyendo en (\ref{proof:omega-sub}), usando el hecho de que $q_{n-2} < 0$ y despejando
% 	$t_{n-2}$, encontramos que $\floor{b_2} - \ceil{b_1} \geq 0$ si
% 	\begin{equation*}
% 		t_{n-2} \geq \frac{\omega_{n-2}\omega_{n-1}'}{q_{n-2}}\cdot\prod_{j=1}^{n-2}g_j
% 		- 2\frac{q_{n-1}q_n}{q_{n-2}g_{n-1}^2}\cdot
% 		\prod_{j=1}^{n-2}g_j^{-1}
% 	\end{equation*}
% 	Llamemos $c$ al lado derecho de esta desigualdad. Así pues, definimos el vector
% 	$\vec{t} \in \Z^{n-1}$ de manera que
% 	\begin{equation*}
% 		t_i \coloneq \begin{cases}
% 			\left\lceil -\frac{\omega_ix_i'}{q_i} \right\rceil, & i < n - 2, \\[0.5em]
% 			\left\lceil \max\left\lbrace -\frac{\omega_ix_i'}{q_i}, c \right\rbrace
% 			\right\rceil, & i = n -2, \\[0.5em]
% 			\lceil b_1 \rceil, & i = n - 1.
% 		\end{cases}
% 	\end{equation*}
% 	Se verifica que $\vec{t}$ satisface ambos (\ref{eq:param-lb}) y (\ref{eq:feasible-param}).
% \end{proof}

En síntesis, por el lema \ref{lemma:t-existence} sabemos que existe un vector
$\vec{t} \in \Z^{n-1}$ que satisface ambos (\ref{eq:param-lb}) y (\ref{eq:feasible-param}).
Al definir $\vec{x}^* \coloneq \eta\vec{\nu} + M\vec{t}$, encontramos que $\vec{x}^*$ es entero y
no negativo, y además por los lemas \ref{lemma:iso1} y \ref{lemma:iso2} encontramos que
\begin{equation*}
	\vec{q}^T\vec{x}^* = \eta\vec{q}^T\vec{\nu} + \vec{q}^TM\vec{t} = \eta.
\end{equation*}
Por el teorema \ref{theory:th:feasibility}, se sigue que $\vec{x}^*$ es la solución al problema
\eqref{theory:formulation}.

En la práctica es mejor usar la relación de recurrencia (\ref{eq:recurrence}) y ``construir'' las
entradas $x_i$ al mismo tiempo que definimos $t_i$ de manera que satisfaga (\ref{eq:param-lb}) y
(\ref{eq:feasible-param:collapsed}). Si procedemos de esta forma no tenemos que encontrar primero
$\vec{\nu}$ y $M$, determinar $\vec{t}$ y luego recuperar $\vec{x}$. El Algoritmo \ref{algo:inf}
muestra este procedimiento constructivo.

\begin{algorithm}[ht]
	\LinesNumbered
	\SetKwProg{Fn}{Fn}{\string:}{}
	\SetKwFunction{Bezout}{Bezout}
	\SetKwFunction{NonNegativeIntSol}{NonNegativeIntSolInf}
		\KwData{\\
			$\vec{q} \in \Z^n$ coprimo tal que $q_i \neq 0$ para todo $i \in \braces{1, \ldots, n}$ y $q_{n-1} < 0 < q_n$. \\
			$\eta \in \Z_{\geq 0}$.
			}
		\KwResult{\\
			$\vec{x} \in \Z^n_{\geq \vec{0}}$ tal que $\vec{q}^T\vec{x} = \eta$.
		}
		\Begin{
			$\vec{x} \leftarrow \vec{0}$\;
			$\omega_1 \leftarrow \eta$\;
			\For{$i \leftarrow 1$ \KwTo $n - 2$}{
				$g_{i+1} \leftarrow \gcd{q_{i+1}, \ldots, q_n}$\; \label{alg:def:inf:g}
				$x_i', \omega_{i+1}' \leftarrow$ \Bezout{$q_i$, $g_{i+1}$}\; \label{alg:def:inf:bez}
				$t_i \leftarrow \ceil{-\omega_i x_i' / g_{i+1}}$\; \label{alg:def:inf:t}
				$x_i \leftarrow \omega_i x_i' + g_{i+1}t_i$\; \label{alg:def:inf:x}
				$\omega_{i+1} \leftarrow \omega_i \omega_{i+1}' - q_i t_i$\; \label{alg:def:inf:w}
				\For{$j \leftarrow i$ \KwTo $n - 1$}{
					$q_{j+1} \leftarrow q_{j+1}/g_{i+1}$\; \label{alg:def:inf:q}
				}
			}

			$x_{n-1}', x_n' \leftarrow$ \Bezout{$q_{n-1}$, $q_n$}\; \label{alg:def:inf:lastw}
			$b_1 \leftarrow -\omega_{n-1} x_{n-1}' / q_n$\;
			$b_2 \leftarrow \omega_{n-1} x_n' / q_{n-1}$\;
			$t_{n-1} \leftarrow \ceil{\max\lbrace b_1, b_2\rbrace}$\;
			$x_{n-1} \leftarrow \omega_{n-1}x_{n-1}' + q_nt_{n-1}$\;
			$x_{n} \leftarrow \omega_{n-1}x_{n}' - q_{n-1}t_{n-1}$\;

			\Return{$\vec{x}$}\;
		}
	\caption{\texttt{NonNegativeIntSolInf}}
	\label{algo:inf}
\end{algorithm}

En el Algoritmo \ref{algo:inf} supusimos la existencia de una subrutina \texttt{Bezout} que, como su
nombre lo indica, calcula los coeficientes de Bézout entre dos enteros. Es la creencia del autor
que no es necesario escribir la subrutina en esta tesis, pero reitera, así como lo hizo en la Sección
\ref{section:number-theory}, que estos coeficientes se pueden calcular por medio del Algoritmo
Extendido de Euclides.

\begin{lemma}
	\label{lemma:alg:inf:correct}
	El Algoritmo \ref{algo:inf} es correcto.
\end{lemma}
\begin{proof}
	Basta observar que el algoritmo sigue la construcción recursiva de la Sección
	\ref{subsec:dioph-eq}, donde escogemos las variables libres $t_i$ como lo indica la demostración
	del lema \ref{lemma:t-existence} para asegurar que $\vec{x}$ sea no negativo. El único punto de
	aclaración lo hacemos con respecto a las redefiniciones en la línea \ref{alg:def:inf:q}.

	Sea $\vec{q}'$ una copia del vector $\vec{q}$ antes de realizar cualquier modificación. No es
	difícil ver, por medio de inducción y recordando $g_1 \coloneq \gcd{q_1', \ldots, q_n'} = 1$,
	que
	\begin{equation*}
		q_i = \frac{q_i'}{\prod_{j=1}^{\min\lbrace i, n - 1\rbrace}g_j},
	\end{equation*}
	para todo $i \in \braces{1, \ldots, n}$. Luego, las definiciones en las líneas
	\eqref{alg:def:inf:g}, \eqref{alg:def:inf:bez} y \eqref{alg:def:inf:lastw} son consistentes con
	la construcción recursiva de la Sección \ref{subsec:dioph-eq}. Juntando esto con el lema
	\ref{lemma:t-existence} encontramos que $\vec{x}$ es no negativo y satisface la ecuación lineal
	diofantinca $\vec{q}^T\vec{x} = \eta$.

	% Observemos que en este algoritmo
	% seguimos la construcción recursiva de la Sección \ref{subsec:dioph-eq}. Mostramos esto
	% inductivamente. Denotemos por $\vec{q}'$ al vector $\vec{q}$ antes de ser modificado por el
	% algoritmo.

	% % Sin embargo, parece que cambiamos la definición del máximo común divisor $g_{i+1}$ en la línea
	% % \eqref{alg:def:inf:g} cuando la comparamos con \eqref{dummy:next-g}. Primero mostremos por
	% % inducción que ambas definiciones son equivalentes.

	% Cuando $i = 1$, tenemos $\vec{q} = \vec{q}'$. De la línea \ref{alg:def:inf:g}, obtenemos $g_2
	% = \gcd{q_2, q_3, \ldots, q_n}$, pero recordemos que habíamos definido por conveniencia $g_1
	% \coloneq \gcd{q_1', \ldots, q_n'} = 1$. Así pues, tenemos
	% \begin{equation*}
	% 	g_2 = \gcd{q_2, \ldots, q_n} = \gcd{q_2'/g_1, \ldots, q_n'/g_1},
	% \end{equation*}
	% pero esto es igual a la definición en \eqref{dummy:next-g}. De la línea \ref{alg:def:inf:bez}
	% tenemos que $x_1'$ y $\omega_2'$ son los coeficientes de Bézout de $q_1 = q_1'$ y de $g_2$,
	% respectivamente. De la línea \ref{alg:def:inf:w} tenemos
	% \begin{equation*}
	% 	\omega_2 = \omega_1\omega_2' - q_1t_1 = \omega_1\omega_2' - \frac{q_1'}{g_1}t_1,
	% \end{equation*}
	% por lo que $x_1$ y $\omega_2$ están definidos como \eqref{dummy:eq:first-step} y por lo tanto
	% satisfacen \eqref{eq:dioph:first-step}. Además, $t_1$ satisface \eqref{eq:param-lb}, de donde
	% encontramos que $x_1 \geq 0$.

	% De las redefiniciones en la línea \ref{alg:def:inf:q} obtenemos, para todo $j \in \braces{2,
	% \ldots, n - 1}$,
	% \begin{equation*}
	% 	q_j \leftarrow \frac{q_j}{g_2} = \frac{q_j'}{g_1 \cdot g_2}
	% \end{equation*}

	% Supongamos inductivamente que el algoritmo sigue la construcción recursiva de la Sección
	% \ref{subsec:dioph-eq} para alguna $i$ tal que $i - 1 < n - 2$ y que $x_{i-1} \geq 0$. De la
	% línea \ref{alg:def:inf:q} tenemos entonces
	% \begin{equation*}
	% 	q_j \leftarrow \frac{q_j}{g_i} = \frac{q_j'}{g_i \cdot \prod_{j=1}^{i-1} g_j}
	% 	= \frac{q_j'}{\prod_{j=1}^{i}g_j},
	% \end{equation*}
	% para toda $j \in \braces{i, \ldots, n - 1}$. De la línea \ref{alg:def:inf:g} se sigue que
	% \begin{equation*}
	% 	g_{i+1} = \gcd{q_{i+1}, \ldots, q_n}
	% 	=
	% 	\gcd{
	% 		\frac{q_{i+1}'}{\prod_{j=1}^{i}g_j},
	% 		\ldots,
	% 		\frac{q_{n}'}{\prod_{j=1}^{i}g_j}
	% 	},
	% \end{equation*}
	% lo cual es equivalente a \eqref{dummy:next-g}. De la línea \ref{alg:def:inf:bez} tenemos que
	% $x_{i}', \omega_{i + 1}'$ son los coeficientes de Bézout de $q_i =
	% \frac{q_i'}{\prod_{j=1}^ig_j}$ y de $g_{i+1}$, respectivamente. De la línea
	% \eqref{alg:def:inf:w} también tenemos
	% \begin{equation*}
	% 	\omega_{i+1} = \omega_i\omega_{i+1}' - q_it_i = \omega_i\omega_{i+1} -
	% 	\frac{q_i'}{\prod_{j=1}^{i}g_j}t_i,
	% \end{equation*}
	% por lo que $x_i$ y $\omega_{i+1}$ están definidos como \eqref{eq:recurrence} y por lo tanto
	% satisfacen \eqref{dummy:eq:simplified}. Observamos en la línea \ref{alg:def:inf:t} que $t_i$
	% satisface \eqref{eq:param-lb} y por lo tanto $x_i \geq 0$. Así pues, el algoritmo sigue la
	% construcción recursiva de la Sección \ref{subsec:dioph-eq} para todo $i \in \braces{1, \ldots, n
	% - 2}$ y también $x_i$ es no negativo.
	% Finalmente, 
\end{proof}

El Algoritmo \ref{algo:inf:ext} extiende el Algoritmo \ref{algo:inf}. Solamente construimos un
vector $\tvec{q}$ a partir del vector coprimo $\vec{q}$ de manera que se satisfagan las
hipótesis del Algoritmo \ref{algo:inf}. Esta construcción sigue la misma lógica con la que
justificamos los supuestos $q_i \neq 0$ y $q_{n-1} < 0 < q_n$ antes de presentar el lema
\ref{lemma:t-existence}.

Al igual que en el algoritmo anterior, suponemos la existencia de las subrutinas \texttt{length} y
\texttt{switch}, las cuales determinan la dimensión de un vector $\vec{q}$ y permutan sus entradas,
respectivamente. Ambas subrutinas son estándar en la literatura y por lo tanto diremos que son
correctos sin proveer alguna demostración. Así también, la subrutina \texttt{NonNegativeIntSol} es
el Algoritmo \ref{algo:inf}, el cual es correcto a causa del lema \ref{lemma:alg:inf:correct}.

\begin{algorithm}[ht]
	\LinesNumbered
	\SetKwProg{Fn}{Fn}{\string:}{}
	\SetKwFunction{switch}{switch}
	\SetKwFunction{NonNegativeIntSol}{NonNegativeIntSolInf}
	\SetKwFunction{FindNegEntry}{FindNegEntry}
	\SetKwFunction{length}{length}
	\SetKwFunction{Dioph}{Dioph}
		\KwData{\\
			$\vec{q} \in \Z^n$ coprimo tal que $q_i < 0$ para alguna $i \in \braces{1, \ldots, n}$. \\
			$\eta \in \Z_{\geq 0}$.
			}
		\KwResult{\\
			$\vec{x} \in \Z^n_{\geq \vec{0}}$ tal que $\vec{q}^T\vec{x} = \eta$.
		}
		$\vec{x} \leftarrow \vec{0}$\;
		$\vec{\sigma} \leftarrow \left(i \colon q_i \neq 0\right)$\;
		$\tvec{q} \leftarrow \left( q_i \colon q_i \neq 0 \right)$\;
		\label{alg:def:inf:tilde-q}

		$m \leftarrow$ \length{$\tvec{q}$}\; \label{alg:subr:length}
		\switch{$\tvec{q}$, $1$, $m$}\; \label{alg:subr:switch1}

		\For{$i \leftarrow 1$ \KwTo $m - 1$}{ \label{alg:inf:loop}
			\If{$\tilde{q}_i < 0$}{
				$j \leftarrow i$\;
				ir al paso \ref{alg:subr:switch3}\;
			}
		}
		\switch{$\tvec{q}$, $j$, $m - 1$}\; \label{alg:subr:switch3}
		$\tvec{x} \leftarrow$ \NonNegativeIntSol{$\tvec{q}$, $\eta$}\;
		\switch{$\tvec{x}$, $j$, $m - 1$}\; \label{alg:subr:switch2}
		\switch{$\tvec{x}$, $1$, $m$}\;

		\For{$i \leftarrow 1$ \KwTo $m$}{ \label{alg:inf:loop2}
			$x_{\sigma_i} \leftarrow \tilde{x}_i$\;
		}
		\Return{$\vec{x}$}
	\caption{\texttt{Dioph}}
	\label{algo:inf:ext}
\end{algorithm}

\begin{theorem}
	\label{th:alg:inf}
	El Algoritmo \ref{algo:inf:ext} es correcto.
\end{theorem}
\begin{proof}
	Primero mostramos que el vector $\tvec{q}$ satisface las hipótesis del Algoritmo
	\ref{algo:inf}. Por definición, en la línea \ref{alg:def:inf:tilde-q}, tenemos que ninguna
	entrada de $\tvec{q}$ es nula.

	Recordemos de la definición \ref{theory:def:rational} que, como $\vec{q}$ es el vector coprimo
	de un vector esencialmente entero $\vec{p}$, su primera entrada no nula es positiva. Así, es
	cierto que $\tilde{q}_1 > 0$. A partir de la permutación en la línea \ref{alg:subr:switch1}
	encontramos que $\tilde{q}_m > 0$.

	Del ciclo en la línea \ref{alg:inf:loop} recuperamos un índice $j$ tal que $\tilde{q}_j < 0$ y
	lo permutamos con la $(m - 1)$-ésima entrada de $\tvec{q}$ en la línea
	\eqref{alg:subr:switch3}, de manera que obtenemos $\tilde{q}_{m-1} < 0$.

	Con los tres puntos anteriores, encontramos que el vector $\tvec{q}$ satisface las
	hipótesis del Algoritmo \ref{algo:inf} y por lo tanto el vector $\tvec{x}$ es no negativo
	y satisface la ecuación lineal diofantina $\tvec{q}^T\tvec{x} = \eta$, debido al
	lema \ref{lemma:alg:inf:correct}.

	Las siguientes dos líneas se encargan de invertir las permutaciones hechas previamente.
	Finalmente, en el ciclo \eqref{alg:inf:loop2} insertamos en $\vec{x}$ las entradas $i$ de
	$\tvec{x}$ donde $q_{\sigma_i} \neq 0$. En otro caso tenemos $x_i = 0$. Así pues, el
	vector $\vec{x}$ es no negativo y también tenemos
	\begin{equation*}
		\vec{q}^T\vec{x} = \sum_{i = 1}^{n}q_ix_i
		= \sum_{i = 1}^{m}q_{\sigma_i}x_{\sigma_i}
		= \sum_{i = 1}^{m}\tilde{q}_i\tilde{x}_i
		= \eta,
	\end{equation*}
	por lo que concluimos que el Algoritmo \ref{algo:inf:ext} es correcto.
\end{proof}

\begin{theorem}
	\label{infinite:th:complexity}
	Sea $\vec{p} \in \R^n$ un vector esencialmente entero tal que su múltiplo coprimo $\vec{q} \in
	\Z^n$ tiene una entrada negativa. Entonces el problema \eqref{theory:formulation} se puede
	resolver a través de encontrar la solución de una ecuación lineal diofantina en $n$ incógnitas.
\end{theorem}
\begin{proof}
	Como $\vec{q}$ es el múltiplo coprimo de $\vec{p}$, existe un escalar $m \in \R$ tal que
	$\vec{p} = m\vec{q}$. Supongamos, sin pérdida de generalidad, que $m$ es positivo. Recuperemos
	$\eta$ del lema \ref{phase-1:lemma:eta}. Por hipótesis, una entrada de $\vec{q}$ es negativa, y
	entonces este vector satisface las condiciones del Algoritmo \ref{algo:inf:ext}. Por el teorema
	\ref{th:alg:inf} podemos encontrar, a partir de resolver solo una ecuación lineal diofantina, un
	vector entero no negativo $\vec{x}$ que satisface $\vec{q}^T\vec{x} = \eta$. Observemos que
	\begin{equation*}
		\vec{p}^T\vec{x} = m\vec{q}^T\vec{x} = m\eta.
	\end{equation*}
	Por el corolario \ref{cor:inf:obj} concluimos que $\vec{x}$ no solo es factible para el problema
	\eqref{theory:formulation}, sino que también es un punto óptimo.
\end{proof}

A partir del teorema anterior, podemos discutir informalmente sobre la complejidad algorítmica del
problema \eqref{theory:formulation} en el caso especial que una entrada $q_i$ sea negativa. Sea
$T(\vec{v})$ el número de pasos necesarios para calcular el máximo común divisor de las entradas de
un vector entero $\vec{v}$. Como los coeficientes de Bézout entre dos enteros $a, b$ se determinan a
partir del Algortimo Extendido de Euclides, sabemos que el número de pasos para calcularlos es un
múltiplo entero de $T(a, b)$. 

\section{Análisis de resultados}
\noindent
Una consecuencia del teorema \ref{infinite:th:complexity} es que la complejidad algoritmítica del
problema (\ref{theory:formulation}) es lineal en la dimensión $n$ siempre y cuando $q_i < 0$
para alguna $i \in \lbrace 2, \ldots, n\rbrace$. En esta sección describimos un algoritmo cuyo
tiempo de terminación es $\mathcal{O}(n)$. A través de los resultados obtenidos previamente, somos
capaces de mostrar que nuestro algoritmo es correcto. Finalmente, implementamos nuestro algoritmo en
el lenguaje de programación Python y comparamos sus tiempos de terminación con los de la
implementación de Ramificación y Acotamiento en la librería PuLP. 

\chapter{El caso finito}
\noindent
Nuevamente inspirados por el Teorema \ref{theory:th:feasibility}, en este capítulo analizamos el
caso en el que el vector coprimo $\vec{q}$ tiene entradas estrictamente positivas. De esta manera,
el problema \eqref{theory:formulation} deviene una instancia particular del famoso Problema de la
Mochila:
\begin{subequations}
	\label{knapsack-formulation}
	\begin{align}
		\max_{\vec{x} \in \Z^n} \quad
			& \vec{u}^T\vec{x}, \\
		\text{s.a.} \quad
			& \vec{w}^T\vec{x} \leq c, \\
			& \vec{x} \geq \vec{0},
	\end{align}
\end{subequations}
donde los vectores positivos $\vec{u}, \vec{w} \in \Z^n$ son conocidos como vector de útiles y
vector de pesos, respectivamente. Puesto que no acotamos $\vec{x}$, el problema recibe el nombre de
Problema de la Mochila no Acotado. Pero también como $\vec{u} = \vec{w}$, el problema puede
ser considerado como un Problema de la Suma de Conjuntos no Acotado.

En la primera sección realizamos un análisis de capas enteras a fin de obtener un resultado análogo
al Teorema \ref{infinite:th:complexity}. En concreto, el Teorema \ref{th:intnonneg2} enuncia que
para un presupuesto $u$ suficientemente grande en el problema \eqref{theory:formulation}, la
búsqueda de una solución se reduce a resolver solamente una ecuación lineal diofantina.

El resultado anterior, si bien interesante, es de existencia y no muestra cómo obtener las
soluciones enteras no negativas de ecuaciones lineales diofantinas. De manera similar a como lo
hicimos en el capítulo anterior, la segunda sección se encarga de presentar tal construcción de
soluciones a partir de los Algoritmos \ref{algo:fin:helper} y \ref{algo:fin:dioph}.

Finalmente, en la tercera y última sección de este capítulo, realizamos algunos experimentos
numéricos que comparan la eficacia de nuestros algoritmos recién desarrollados con la de
Ramificación y Acotamiento, así como de una formulación alternativa de programación dinámica.

\section{Análisis de capas enteras}
\noindent
De acuerdo al segundo caso del Teorema \ref{theory:th:feasibility}, el número de puntos enteros no
negativos sobre la $k$-ésima capa entera es finito y, por lo tanto, puede ser cero. Sea $k \in
\braces{\eta, \ldots, 0}$. Sabemos de la Sección \ref{subsec:dioph-eq} que deseamos resolver la
ecuación lineal diofantina \eqref{eq:dioph}, por lo que implementamos la misma estrategia para
plantear una formulación recursiva. 

Debido al supuesto $\vec{q} > \vec{0}$, observemos de \eqref{dummy:eq:ith-equation} que podemos
agregar la condición $\omega_{i} \geq 0$. En efecto, buscamos que $\vec{x}$ sea no negativo y
recordemos que $g_i$ es un máximo común divisor (ver \eqref{dummy:eq:ith-g}), por lo que es
estrictamente positivo. Juntando esto con el supuesto $\vec{q} > \vec{0}$, encontramos que
$\omega_i$ es no negativo para toda $i \in \braces{1, \ldots, n - 1}$. Así pues, despejando $t_i$ de
\eqref{eq:recurrence} obtenemos los intervalos de factibilidad
\begin{equation}
	\label{phase-1:finite:eq:param-bounds}
	\left\lceil -\frac{\omega_ix_i'}{g_{i+1}} \right\rceil
	\leq
	t_i
	\leq
	\left\lfloor \frac{\omega_i\omega_{i+1}'}{q_i} \prod_{j=1}^{i}g_j \right\rceil,
\end{equation}
para todo $i \in \lbrace 1, \ldots, n - 2\rbrace$. Luego, como $0 < q_{n - 1}, q_n$, se sigue de
\eqref{eq:last-solution} que
\begin{equation}
	\label{phase-1:finite:eq:param-bounds-last}
	\left\lceil -\frac{\omega_{n-1}x_{n-1}'}{q_n} \cdot \prod_{j=1}^{n-2}g_j \right\rceil
	\leq
	t_{n - 1}
	\leq
	\left\lfloor \frac{\omega_{n-1}x_{n}'}{q_{n-1}} \cdot \prod_{j=1}^{n-2}g_j \right\rfloor.
\end{equation}

Consecuentemente, el número de elecciones que podemos realizar para el vector de variables libres
$\vec{t} \in \Z^{n-1}$ es, como lo confirma el Teorema \ref{theory:th:feasibility}, finito. Si
determinamos que no existe tal punto en la $k$-ésima capa entera, descendemos a la $(k -1)$-ésima
capa entera y continuamos con nuestra búsqueda.

% Observemos también que una elección de $t_i$ modifica $\omega_{i+1}$ y por lo tanto también afecta
% el intervalo de factibilidad de $t_{i+1}$. Siguiendo con este razonamiento, encontramos que una
% elección de $t_i$ afecta a su vez los intervalos de factibilidad de $t_{i+1}, \ldots, t_{n-1}$. En
% la Sección \ref{subsec:complex} discutiremos cómo es que esta ``cadena'' acota la complejidad
% algorítmica del Algoritmo \ref{algo:fin:dioph}.

Ahora bien, en esta primera parte de la sección nos encargamos de calcular una cota superior para el
número de capas enteras que debemos analizar de manera que garanticemos la existencia de un punto
entero no negativo sobre una de estas capas enteras.

% En la primera parte de esta sección determinamos una cota superior para el número de capas enteras
% que visitamos y analizamos el comportamiento a medida que el presupuesto $u$ aumenta. En la segunda
% parte de esta sección mostramos que si el presupuesto $u$ es suficientemente grande, entonces la
% solución de (\ref{theory:formulation}) sí se encuentra sobre la $\eta$-ésima capa entera. Este
% resultado es análogo al caso infinito del Teorema \ref{theory:th:feasibility}. Finalmente, en la
% tercera parte de esta sección discutimos brevemente sobre la complejidad algorítmica de encontrar
% la solución.

\begin{lemma}
	\label{lemma:tau}
	Sea $\vec{p} \in \R^n$ un vector esencialmente entero y sea $\vec{q} \in \Z^n$ su múltiplo
	coprimo, por lo que existe $m \in \R$ tal que $\vec{p} = m\vec{q}$. Supongamos que $m > 0$ y que
	$\vec{q} > \vec{0}$. Sea $q^* \coloneq \max\lbrace q_1, \ldots, q_n \rbrace$, y sea
	\begin{equation}
		\label{eq:tau}
		\tau \coloneq \left\lfloor \left\lfloor \frac{u}{q^*} \right\rfloor
			\frac{q^*}{m} \right\rfloor,
	\end{equation}
	donde $u$ es el lado derecho de \eqref{theory:constraint:budget}. Entonces la solución del
	problema \eqref{theory:formulation}, de ser factible, se encuentra en una capa entera
	parametrizada por $k \in \lbrace \eta, \eta - 1, \ldots, \tau \rbrace$, donde recuperamos $\eta$
	del Lema \ref{phase-1:lemma:eta}.
\end{lemma}
\begin{proof}
	Definamos $i^* \coloneq \argmax\braces{q_1, \ldots, q_n}$ y consideremos el vector
	\begin{equation*}
		\vec{v} \coloneq \left\lfloor \frac{u}{q^*} \right\rfloor \vec{e}_{i^*}.
	\end{equation*}
	Por hipótesis tenemos $q^* > 0$ y, además, como el problema
	\eqref{theory:formulation} es factible, se sigue del Teorema \ref{theory:th:feasibility} que el
	presupuesto $u$ es no negativo. De esto obtenemos que $\vec{v} \geq \vec{0}$. Así también,
	\begin{equation*}
		\vec{q}^T\vec{v} = \left\lfloor \frac{u}{q^*} \right\rfloor q^*
		\leq \frac{u}{q^*}q^* = u,
	\end{equation*}
	y entonces $\vec{v}$ es factible. De aquí se sigue que este vector provee una cota inferior para
	el problema (\ref{theory:formulation}). Así pues, todo vector $\vec{x}$ candidato a ser el
	óptimo del problema satisface
	\begin{equation*}
		\vec{q}^T\vec{x} = \frac{\vec{p}^T\vec{x}}{m} \geq \frac{\vec{q}^T\vec{v}}{m} = 
		\floor{\frac{u}{q^*}} \frac{q^*}{m}.
	\end{equation*}
	Nos interesa determinar el entero $\tau$ más pequeño tal que todo punto sobre la capa
	entera $H_{\vec{q}, k\norm{\vec{q}}^{-2}}$ con $k \in \lbrace \tau, \tau + 1, \ldots \rbrace$
	satisfaga esta desigualdad. Del Lema \ref{phase-1:lemma:layer}, encontramos que $k$ debe satisfacer
	\begin{equation*}
		\frac{k}{\norm{\vec{q}}^{2}} = \frac{\vec{q}^T\vec{x}}{\norm{\vec{q}}^2} \geq
		\left\lfloor \frac{u}{q^*} \right\rfloor \frac{q^*}{m}
		\frac{1}{\norm{\vec{q}}^2},
	\end{equation*}
	equivalentemente,
	\begin{equation*}
		k \geq \floor{\frac{u}{q^*}}\frac{q^*}{m}.
	\end{equation*}
	Consecuentemente,
	\begin{equation*}
		\tau =
		\left\lfloor \left\lfloor \frac{u}{q^*} \right\rfloor \frac{q^*}{m}
			\right\rfloor.
	\end{equation*}
	Finalmente, recordemos del Lema \ref{phase-1:lemma:eta} que $\eta$ es la primera capa en
	satisfacer la restricción presupuestaria. Por lo tanto, el óptimo del problema
	\eqref{theory:formulation} se encuentra en una capa entera parametrizada por $\tau \leq k \leq \eta$.
\end{proof}

\begin{observation}
	Siempre se cumple que $\tau \leq \eta$. En efecto,
	\begin{equation*}
		\left\lfloor \frac{u}{q^*} \right\rfloor q^*
		\leq \frac{u}{q^*} q^* = u,
	\end{equation*}
	como $m > 0$, tenemos
	\begin{equation*}
		\left\lfloor \frac{u}{q^*} \right\rfloor \frac{q^*}{m}
		\leq \frac{u}{m}.
	\end{equation*}
	Aplicando la función piso a ambos lados de la desigualdad y comparando con \eqref{eq:tau} y el
	Lema \ref{phase-1:lemma:eta} encontramos que $\tau \leq \eta$.
\end{observation}
\begin{observation}
	Nuevamente, la suposición de que el escalar $m$ sea positivo ocurre sin pérdida de generalidad.
	Así como mencionamos en el capítulo anterior que si $m$ es negativo entonces existe un parámetro
	$\eta'$ análogo a $\eta$, también existe $\tau' \geq \eta'$ tal que la solución del problema
	\eqref{theory:formulation} se encuentra en una capa parametrizada por $\eta' \leq k \leq \tau'$.
\end{observation}


% Es decir, a pesar de que el óptimo se encuentre sobre la capa entera $k$ que estamos analizando, la
% elección de los primeros parámetros que realicemos puede afectar el tiempo de terminación de nuestro
% algoritmo. Hay dos extremos en las posibles estrategias que podemos adoptar para realizar estas
% elecciones. Para visualizarlo, tenemos que nuestro presupuesto actual $\omega_{i}$ determina el
% siguiente presupuesto a partir de
% \begin{equation*}
% 	\begin{cases}
% 		x_i = \omega_ix_i' + g_{i+1}t_i, \\
% 		\omega_{i+1} = \omega_i\omega_{i+1}' - \frac{q_i}{\prod_{j=1}^{i}g_j}t_i,
% 	\end{cases}
% \end{equation*}
% donde la primera ecuación indica cuántos elementos de $x_i$ decidimos adquirir a partir del
% presupuesto actual $\omega_i$.
% 
% El primer extremo está en buscar agotar todo nuestro presupuesto disponible en las primeras
% elecciones de $x_1, x_2, \ldots, x_i$. Es decir, adquirimos la mayor cantidad que
% podamos de los primeros productos. Bajo esta perspectiva, es razonable imponer un orden en $\vec{q}$
% de manera que
% \begin{equation*}
% 	q_1 \geq q_2 \geq \cdots \geq q_n,
% \end{equation*}
% por lo que primero adquirimos los artículos más caros. En este caso diremos que $\vec{q}$ está en
% orden descendente. Si adoptamos esta estrategia es porque suponemos que el óptimo se concentra en
% una vecindad de los primeros $i$ artículos. Es decir, si tenemos la creencia de que $x_{i +
% 1}, \ldots, x_n$ pueden ser aproximadamente cero.
% 
% % TODO: es mejor argumentar esto en la sección de análisis de resultados.
% % De ser este el caso, entonces es razonable suponer que los tiempos de terminación de esta búsqueda
% % son similares a los de Ramificación y Acotamiento. 
% 
% El segundo extremo es esencialmente lo opuesto. Esto no quiere decir que ahora ordenamos $\vec{q}$
% de manera ascendente y escogemos las primera $t_i$ lo más pequeñas posible\footnote{Si
% 	hiciéramos esto es porque creemos que $x_1, \ldots, x_i$ son aproximadamente cero,
% 	pero entonces podemos permutar estas entradas de manera que se encuentren hasta el final y
% emplear la primera estrategia.}. Consiste en escoger $t_i$ de manera que se encuentre en el
% punto medio de sus cotas inferiores y superiores. Es decir, creemos que el óptimo se encuentra en
% una vecindad del centro de masa de la $k$-ésima capa entera.
% 
% Observemos que, independientemente del caso, si una capa entera no contiene puntos factibles,
% entonces ambas estrategias agotan todas las elecciones posibles de $t_1, \ldots,
% t_{n-1}$. Por lo tanto, los tiempos de terminación de ambas estrategias son iguales para capas
% enteras que no contienen puntos óptimos. La segunda estrategia, no obstante, es candidata ideal para
% realizar una búsqueda binaria. Discutimos más sobre esto último en la sección de análisis de
% resultados.

\begin{lemma}
	\label{lemma:layer-dist}
	Sean $q$ y $m$ enteros distintos de cero. Entonces la función $\Delta \colon \R \to \R$ dada por
	\begin{align*}
		\Delta(x) &\coloneq \left\lfloor \frac{x}{m} \right\rfloor - \left\lfloor \left\lfloor
		\frac{x}{q} \right\rfloor \frac{q}{m} \right\rfloor,
	\end{align*}
	es periódica con periodo $\lcm{q, m}$.
\end{lemma}
\begin{proof}
	Tenemos
	\begin{align*}
		\Delta(x + \lcm{q, m})
		&= \left\lfloor \frac{x}{m} + \frac{\lcm{q, m}}{m} \right\rfloor
		- \left\lfloor \left\lfloor \frac{x}{q} + \frac{\lcm{q,m}}{q} \right\rfloor \frac{q}{m}
			\right\rfloor,
	\end{align*}
	pero $q, m \mid \lcm{q, m}$, por lo que $\lcm{q,m}/m$ y $\lcm{q,m}/q$ son enteros. Por las
	propiedades de la función piso obtenemos:
	\begin{align*}
		\Delta(x + \lcm{q,m})
		&=
		\left\lfloor \frac{x}{m} \right\rfloor + \frac{\lcm{q,m}}{m}
		- \left\lfloor \left\lfloor \frac{x}{q} \right\rfloor\frac{q}{m} + 
			\frac{\lcm{q,m}}{q}\cdot\frac{q}{m} \right\rfloor \\
		&= 
		\left\lfloor \frac{x}{m} \right\rfloor + \frac{\lcm{q,m}}{m}
		- \left\lfloor \left\lfloor \frac{x}{q} \right\rfloor\frac{q}{m}\right\rfloor
		- \frac{\lcm{q,m}}{m} \\
		&= 
		\left\lfloor \frac{x}{m} \right\rfloor
		- \left\lfloor \left\lfloor \frac{x}{q} \right\rfloor\frac{q}{m}\right\rfloor \\
		&= \Delta(x),
	\end{align*}
	que es lo que queríamos demostrar.
\end{proof}
\begin{definition}
	Sea $\vec{p} \in \R^n$ un vector esencialmente entero y sea $\vec{q} \in \Z^n$ su múltiplo
	coprimo. Consideremos los parámetros $\eta$ y $\tau$ (c.f. Lemas \ref{phase-1:lemma:eta} y
	\ref{lemma:tau}) como funciones del presupuesto $u$. Entonces decimos que la función $\Delta^*
	\colon \R \to \R$ dada por
	\begin{equation}
		\label{eq:dist-layers}
		\Delta^*(u) \coloneq \eta(u) - \tau(u)
	\end{equation}
	denota el \textbf{número de capas enteras a revisar} dado el presupuesto $u$.
\end{definition}

Si queremos aplicar el Lema \ref{lemma:layer-dist} a la función de la definición anterior, debemos
reducir nuestra atención a vectores $\vec{p}$ enteros. Esto se debe a que debemos asegurar que el
múltiplo $m$ sea entero\footnote{
	Es la creencia del autor que el Lema \ref{lemma:layer-dist} puede ser generalizado para múltiplos
	$m$ racionales, mas esto no agrega demasiado valor en lo que sigue de la tesis.
}. Independientemente del comportamiento
periódico de $\Delta^*$, tenemos que esta función varía significativamente ante cambios en $m$.
Esto último implica que el número de capas enteras a revisar depende del número de cifras decimales
usadas para especificar $\vec{p}$. Véase la Figura \ref{fig:m:ex} o el Ejemplo \ref{ex:decimals}. 

\begin{example}
	\label{ex:decimals}
	Si tenemos $\vec{p} \coloneq (9.6, 7.2, 5.6)^T$, entonces $m = 0.8$ y por lo tanto el número de
	capas a revisar dado $u \coloneq 119$ es $\Delta^*(u) = 14$. En cambio, si tenemos $\vec{p} \coloneq
	(9.60, 7.28, 5.68)^T$, obtenemos $m = 0.08$, por lo que el número de capas a revisar dado $u$ es
	$\Delta^*(u) = 1499$. Es decir, si usamos una cifra decimal más, entonces $\Delta^*(u)$ se
	multiplica por 100, aproximadamente.
\end{example}

\begin{figure}[htbp]
  \centering

  \begin{minipage}[t]{0.48\textwidth}
    \centering
    \includegraphics[width=\linewidth]{/home/tempdata/repos/thesis/static/misc/delta8m12q.pdf}
  \end{minipage}
  \hfill
  \begin{minipage}[t]{0.48\textwidth}
    \centering
    \includegraphics[width=\linewidth]{/home/tempdata/repos/thesis/static/misc/delta008m960q.pdf}
  \end{minipage}

  \caption{Número de capas a revisar en función del presupuesto. \textit{Izquierda: }Para los
	  parámetros $m = 8$ y $q^* = 12$ encontramos que hay un máximo de una capa a revisar.
	\textit{Derecha: } A medida que $m$ se vuelve fraccionario, las capas a revisar aumentan. En
		este caso tenemos $m = 0.08$ y $q^* = 960$.}
  \label{fig:m:ex}
\end{figure}

Observaremos en el análisis de resultados que el número de capas enteras que nuestro algoritmo
revisa en realidad disminuye a medida que aumenta el presupuesto $u$.

En esta segunda parte de la sección, demostraremos que para un presupuesto $u$ suficientemente
grande, la solución del problema \eqref{theory:formulation} se encuentra en la $\eta$-ésima capa
entera. Este resultado será análogo al Teorema \ref{infinite:th:complexity}. No obstante, para
lograr aquello, necesitamos de un par de definiciones y lemas preliminares.

Para motivar al lector, primero mostramos que existe una vecindad fija de todo punto en $\R^n$ de
manera que esa vecindad contiene al menos un punto entero. Esto es especificado en el Teorema
\ref{lemma:ball-cover}.

Luego, observamos que el ``trozo'' no negativo de una capa entera $H_{\vec{q},
k\norm{\vec{q}}^{-2}}$ crece a medida que $k$ aumenta. Así pues, si $k$ es lo suficientemente
grande, habrá un punto sobre ese trozo no negativo cuya vecindad también se encuentra contenida en
ese trozo y, por lo tanto, habrá un punto entero no negativo sobre ese trozo. Esto es especificado
en el Teorema \ref{th:intsimplex}.

Finalmente, relacionamos el punto entero que se encuentra sobre el pedazo no negativo de
$H_{\vec{q}, k\norm{\vec{q}}^{-2}}$ con el problema \eqref{theory:formulation}. Así pues, concluimos
esta sección con los Teoremas \ref{th:intnonneg1} y \ref{th:intnonneg2}.

\begin{definition}
	\label{fin:def:ball}
	Sea $\vec{q} \in \Z^n$ un vector coprimo y sea $k$ un entero positivo. Entonces definimos la
	\textbf{bola cerrada} sobre la $k$-ésima capa entera $H_{\vec{q}, \norm{\vec{q}}^{-2}}$ con
	radio $r > 0$ y centro $\vec{x} \in H_{\vec{q}, k\norm{\vec{q}}^{-2}}$ como
	\begin{equation}
		\label{eq:k-ball}
		B_r^{(k)}(\vec{x}) \coloneq \lbrace \vec{y} \in \R^n \colon \norm{\vec{y} - \vec{x}} \leq r
		\rbrace \cap H_{\vec{q}, k \norm{\vec{q}}^{-2}}.
	\end{equation}
\end{definition}
% TODO: mostrar una imagen con una bola.
\begin{theorem}
	\label{lemma:ball-cover}
	Sea $\vec{q} \in \Z^n$ un vector coprimo y supongamos que $q_n \neq 0$. Sea $k$ un entero
	positivo. Entonces existe $r > 0$ tal que la familia de bolas
	\begin{equation*}
		\left\lbrace B_r^{(k)}(\vec{x}) \colon \vec{x} \in H_{\vec{q}, k\norm{\vec{q}}^{-2}} \cap
			\Z^n \right\rbrace
	\end{equation*}
	es una cubierta de la $k$-ésima capa entera $H_{\vec{q}, k\norm{\vec{q}}^{-2}}$.
\end{theorem}
\begin{proof}
	Como $q_n \neq 0$, recordemos del Teorema \eqref{th:lattice} que
	\begin{equation*}
		\vec{x} \in H_{\vec{q}, k\norm{\vec{q}}^{-2}} \cap \Z^n \iff \vec{x} = k\vec{\nu} + M\vec{t}
	\end{equation*}
	para algún vector $\vec{t} \in \Z^{n-1}$, donde recuperamos $\vec{\nu}$ y $M$ de
	\eqref{eq:vec-omega} y \eqref{eq:mat-T}, respectivamente. Así, tenemos
	\begin{equation*}
		\left\lbrace B_r^{(k)}(\vec{x}) \colon \vec{x} \in H_{\vec{q}, k\norm{\vec{q}}^{-2}} \cap
			\Z^n \right\rbrace
			=
		\left\lbrace B_r^{(k)}(k\vec{\nu} + M\vec{t}) \colon \vec{t} \in \Z^{n-1} \right\rbrace.
	\end{equation*}
	A partir de esto último sabemos que $B_r^{(k)}(k\vec{\nu} + M\vec{t}) \subseteq H_{\vec{q},
	k\norm{\vec{q}}^{-2}}$ para todo punto entero $\vec{t} \in \Z^{n-1}$. Luego, para cualquier $r >
	0$ tenemos
	\begin{equation}
		\label{eq:ball-cover:1}
		\bigcup_{\vec{t} \in \Z^{n-1}}B_r^{(k)}(k\vec{\nu} + M\vec{t}) \subseteq
		H_{\vec{q}, k\norm{\vec{q}}^{-2}}.
	\end{equation}
	Ahora bien, sea $\vec{y}$ un punto sobre la $k$-ésima capa entera. Por el Lema \ref{lemma:iso2}
	sabemos que las columnas de $M$ son linealmente independientes, y entonces existe $\vec{t} \in
	\R^{n-1}$ tal que
	\begin{equation*}
		\vec{y} = k\vec{\nu} + M\vec{t}.
	\end{equation*}
	Sea $\lfloor \vec{t} \rceil \in \Z^{n-1}$ el vector resultante de redondear cada entrada de
	$\vec{t}$ al entero más cercano. Luego, $\vec{t} = \lfloor \vec{t} \rceil + \vec{\delta}$,
	donde $\vec{\delta} \in \R^{n-1}$ satisface $\norm{\vec{\delta}}_{\infty} \leq 0.5$. Definamos
	\begin{equation*}
		\vec{x} \coloneq k\vec{\nu} + M\lfloor \vec{t} \rceil \in \Z^{n-1},
	\end{equation*}
	de donde se sigue que
	\begin{align}
		\norm{\vec{y} - \vec{x}}^{2} 
		&= \norm{M\vec{\delta}}^{2} \nonumber \\
		&\leq \sum_{i=1}^{n-1}|\vec{\delta}_i|^2 \norm{M\vec{e}_i}^{2} \nonumber \\
		&\leq \frac{1}{4}\sum_{i=1}^{n-1} \norm{M\vec{e}_i}^{2} \label{eq:alt-M-bound} \\
		&= \frac{1}{4}\norm{M}_F^2, \nonumber
	\end{align}
	donde $\norm{M}_F$ denota la norma Frobenius de $M$. Por lo tanto, si definimos
	\begin{equation}
		\label{eq:radius}
		r \coloneq \frac{1}{2}\norm{M}_F,
	\end{equation}
	encontramos que $\vec{y} \in B_r^{(k)}(\vec{x})$. Luego, como $\vec{y} \in H_{\vec{q},
	k\norm{\vec{q}}^2} \cap \Z^n$ fue genérico, se sigue que
	\begin{equation}
		\label{eq:ball-cover:2}
		H_{\vec{q}, k\norm{\vec{q}}^{-2}} \subseteq
		\bigcup_{\vec{t} \in \Z^{n-1}}B_r^{(k)}(k\vec{\nu} + M\vec{t}).
	\end{equation}
	Juntando esto con (\ref{eq:ball-cover:1}) obtenemos lo que queríamos demostrar.
\end{proof}

\begin{definition}
	\label{def:aff}
	Sean $\vec{v}_1, \ldots, \vec{v}_m \in \R^n$ una colección de vectores,
	entonces definimos su \textbf{combinación afina} a partir de
	\begin{equation*}
		\aff\braces{\vec{v}_1, \ldots, \vec{v}_m} \coloneq \braces{\theta_1\vec{v}_1 + \cdots + \theta_k\vec{v}_k
		\colon \theta_1 + \cdots + \theta_m = 1}.
	\end{equation*}
\end{definition}

\begin{lemma}
	\label{lemma:aff}
	Sean $\vec{v}_1, \ldots, \vec{v}_m \in \R^n$ una colección de vectores. Entonces
	\begin{equation*}
		\aff\braces{\vec{v}_1, \ldots, \vec{v}_m} = \vec{v}_j + \gen\braces{\vec{v}_i -
		\vec{v}_j}_{i=1}^{n} = \vec{v}_j + \gen\braces{\vec{v}_i - \vec{v}_j}_{i\neq j}.
	\end{equation*}
\end{lemma}
\begin{proof}
	Puesto que $\vec{v}_j - \vec{v}_j = \vec{0}$, se sigue inmediatamente que
	\begin{equation*}
		\vec{v}_j + \gen\braces{\vec{v}_i - \vec{v}_j}_{i = 1}^{n}
		=
		\vec{v}_j + \gen\braces{\vec{v}_i - \vec{v}_j}_{i \neq j}.
	\end{equation*}
	Sean $\theta_1, \ldots, \theta_m$ escalares tales que $\theta_1 + \cdots + \theta_m = 1$. Por un
	lado, tenemos
	\begin{align*}
		\sum_{i=1}^{m}\theta_i\vec{v}_i
		&= \sum_{i=1}^{m}\theta_i(\vec{v}_i - \vec{v}_j) + \sum_{i=1}^{m}\theta_i\vec{v}_j \\
		&= \sum_{i \neq j}\theta_i(\vec{v}_i - \vec{v}_j) + \vec{v}_j.
	\end{align*}
	De donde se sigue que $\aff\braces{\vec{v}_1, \ldots, \vec{v}_m} \subseteq \vec{v}_j +
	\gen\braces{\vec{v}_i - \vec{v}_j}_{i \neq j}$.

	Ahora bien, sea $\braces{\lambda_i}_{i \neq j}$ un conjunto de $m - 1$ escalares y definamos
	\begin{equation*}
		\lambda_j = 1 - \sum_{i \neq j}\lambda_i.
	\end{equation*}
	Observemos que $\lambda_1 + \cdots + \lambda_m = 1$ y, además,
	\begin{align*}
		\vec{v}_j + \sum_{i \neq j}\lambda_i(\vec{v}_i - \vec{v}_j)
		&= \left(1 - \sum_{i \neq j}\lambda_i\right)\vec{v}_j
		+ \sum_{i \neq j}\lambda_i\vec{v}_i \\
		&= \lambda_j\vec{v}_j + \sum_{i \neq j}\lambda_i\vec{v}_i \\
		&= \sum_{i = 1}^{m}\lambda_i\vec{v}_i.
	\end{align*}
	De donde se sigue que $\vec{v}_j + \gen\braces{\vec{v}_i - \vec{v}_j}_{i \neq j} \subseteq
	\aff\braces{\vec{v}_1, \ldots, \vec{v}_m}$. Puesto que hemos mostrado ambas contenciones,
	obtenemos lo que queríamos demostrar.
\end{proof}
\begin{example}
	\label{ex:aff}
	% Si $\vec{q}$ es un vector coprimo que satisface $q_n \neq 0$, entonces la $k$-ésima capa entera
	% $H_{\vec{q}, k\norm{\vec{q}}^{-2}}$ es la combinación afina de un conjunto de vectores. En
	% efecto, sabemos del Teorema \ref{th:lattice} que el vector $\vec{\nu}$ (ver
	% \eqref{eq:vec-omega}) junto con las columnas $\vec{m}_1, \ldots, \vec{m}_{n-1}$ de M (ver
	% \eqref{eq:mat-T}) forman una base de la red $\Z^n$ y, por extensión, del espacio vectorial
	% $\R^n$.

	% De esta manera tenemos $\vec{x} \in H_{\vec{q}, k\norm{\vec{q}}^{-2}}$ si y solo si $\vec{x} =
	% k\vec{\nu} + M\vec{t}$ para alguna $\vec{t} \in \R^{n-1}$. Así pues,
	% \begin{equation*}
	% 	H_{\vec{q}, k\norm{\vec{q}}^{-2}} = k\vec{\nu} + \gen\braces{\vec{m}_i}
	% 	= k\vec{\nu} + \gen\braces{(\vec{m}_i + k\vec{\nu}) - k\vec{\nu}}.
	% \end{equation*}
	% Por el Lema \ref{lemma:aff} sabemos entonces que la $k$-ésima capa entera es la combinación
	% afina de los vectores $k\vec{\nu}, \vec{m}_1 + k\vec{\nu}, \ldots, \vec{m}_{n-1} + k\vec{\nu}$.

	Si $\vec{q} > \vec{0}$ es un vector coprimo y $k$ un entero positivo, entonces la $k$-ésima capa
	entera $H_{\vec{q}, k\norm{\vec{q}}^{-2}}$ es la combinación afina de un conjunto de vectores.
	En efecto, recordemos de la Definición \ref{phase-1:def:c-layer} que esta capa entera es
	simplemente un hiperplano afino. Como $\vec{q}$ es el vector normal a este hiperplano, se sigue
	que puede ser escrito como $\vec{v} + \ker{\vec{q} \mapsto \vec{q}^T\vec{x}}$ para alguna
	$\vec{v} \in H_{\vec{q}, k\norm{\vec{q}}^{-2}}$.

	Sean $\vec{u}_1, \ldots, \vec{u}_n$ las intersecciones de la $k$-ésima capa entera con cada uno
	de los ejes. Es decir, sean, para cada $i \in \braces{1, \ldots, n}$,
	\begin{equation}
		\label{def:u-basis}
		\vec{u}_i \coloneq \frac{k}{q_i}\vec{e}_i.
	\end{equation}
	Como cada $\vec{u}_i$ está en la $k$-ésima capa entera, se verifica que $\vec{q}^T\vec{u}_i = k$
	y por lo tanto $\vec{u}_i - \vec{u}_j \in \ker{\vec{q} \mapsto \vec{q}^T\vec{x}}$. No es difícil
	ver entonces que el conjunto de vectores $\braces{\vec{u}_i - \vec{u}_j}_{i \neq j}$ forma una
	base del espacio nulo de la transformación lineal $\vec{q} \mapsto \vec{q}^T\vec{x}$, por lo que
	obtenemos
	\begin{equation*}
		H_{\vec{q}, k\norm{\vec{q}}^{-2}} = \vec{u}_j + \gen\braces{\vec{u}_i - \vec{u}_j}_{i \neq
		j}.
	\end{equation*}
	Por el Lema \ref{lemma:aff} concluimos que la $k$-ésima capa entera es la combinación
	afina de los vectores $\vec{u}_1, \ldots, \vec{u}_n$.
\end{example}

\begin{definition}
	\label{def:simplex}
	Sean $\vec{v}_1, \ldots, \vec{v}_m \in \R^n$ vectores linealmente independientes. Entonces
	definimos el \textbf{símplice} $\sigma$ como la combinación convexa de estos vectores:
	\begin{equation*}
		\sigma = \conv\braces{\vec{v}_1, \ldots, \vec{v}_m}
		\coloneq
		\braces{\theta_i\vec{v}_1 + \cdots + \theta_k\vec{v}_m
		\colon \theta_1 + \cdots + \theta_m = 1, \theta_i \geq 0}.
	\end{equation*}
	Decimos entonces que $\sigma$ es generado por $\vec{v}_1, \ldots, \vec{v}_n$. También definimos
	la \textbf{$j$-ésima faceta} $\sigma_j$ de $\sigma$ como el símplice generado por los vectores
	$\braces{\vec{v}_i}_{i \neq j}$.
\end{definition}
\begin{observation}
	Comparando con la Definición \ref{def:aff}, encontramos que todo símplice $\sigma$ generado por
	$\vec{v}_1, \ldots, \vec{v}_m$ está contenido en la combinación afina de estos vectores. Es
	decir,
	\begin{equation}
		\label{aff:contention}
		\conv\braces{\vec{v}_1, \ldots, \vec{v}_m} \subseteq
		\aff\braces{\vec{v}_1, \ldots, \vec{v}_m}.
	\end{equation}
\end{observation}
\begin{observation}
	Si $\sigma$ es un símplice generado por $m$ vectores, entonces tiene $\binom{m}{m-1} = m$
	facetas. Tomaremos por hecho, puesto que de otra manera arriesgamos desviarnos por una tangente,
	que estas facetas constituyen la frontera relativa del símplice. Es decir, tomaremos por hecho
	que las facetas constituyen las ``aristas'' o ``caras'' de $\sigma$.
\end{observation}

\begin{lemma}
	\label{lemma:sigmachar1}
	Sea $\vec{q} > \vec{0}$ un vector coprimo y sea $H_{\vec{q}, k\norm{\vec{q}}^{-2}}$ la
	$k$-ésima capa entera, con parámetro $k$ positivo. Consideremos el símplice $\sigma$ generado por
	los vectores definidos en \eqref{def:u-basis}, entonces
	\begin{equation*}
		\sigma = H_{\vec{q}, k\norm{\vec{q}}^{-2}} \cap \R^n_{\geq \vec{0}}.
	\end{equation*}
\end{lemma}
\begin{proof}
	En el Ejemplo \ref{ex:aff} mostramos que
	\begin{equation*}
		\label{eq:aff-layer}
		H_{\vec{q}, k\norm{\vec{q}}^{-2}} = \aff\braces{\vec{u}_1, \ldots, \vec{u}_n}.
	\end{equation*}
	Sea $\vec{x} \in \sigma$, de la Definición \ref{def:simplex} y de \eqref{aff:contention} encontramos
	que $\vec{x}$ se encuentra en la $k$-ésima capa entera. Además, existen escalares $\theta_1,
	\ldots, \theta_n$ no negativos tales que
	\begin{equation*}
		\vec{x} = \theta_1\vec{u}_1 + \cdots + \theta_n\vec{u}_n
		= k\begin{pmatrix}
			\theta_1 / q_1 \\
			\vdots \\
			\theta_n / q_n
		\end{pmatrix}.
	\end{equation*}
	Como $\vec{q} > \vec{0}$ y $k > 0$ por hipótesis, tenemos que $\vec{x} \geq \vec{0}$, lo que
	implica que $\vec{x} \in H_{\vec{q}, k\norm{\vec{q}}^{-2}} \cap \R^n_{\geq \vec{0}}$.

	El otro lado de la contención se muestra de manera completamente análoga.
\end{proof}

En el contexto del problema \eqref{theory:formulation}, sabemos que si $\sigma$ es generado por los
vectores en \eqref{def:u-basis} entonces, por el Lema anterior, todo punto entero sobre $\sigma$ es
un punto factible siempre que $0 < k \leq \eta$, donde recuperamos $\eta$ del Lema
\ref{phase-1:lemma:eta}. Nos gustaría entonces garantizar la existencia de tal punto entero.

Adoptamos la siguiente estrategia: nos concentramos en un punto $\vec{x} \in \sigma$ y abrimos una
bola (ver Definición \ref{fin:def:ball}) con radio dado por \eqref{eq:radius}. Si esa bola está
contenida en el símplice $\sigma$, entonces el Teorema \ref{lemma:ball-cover} garantiza la
existencia de un punto entero sobre $\sigma$. Por el Lema anterior, garantizaríamos la
existencia de un punto entero no negativo sobre la $k$-ésima capa entera.

Lo que se encuentra a continuación es un análisis para determinar qué tan grande debe ser $k$ para
asegurar que la bola de radio \eqref{eq:radius} esté contenida en el símplice $\sigma$, dado que la
bola está centrada en un punto particular, a saber, en el baricentro del símplice.

\begin{definition}
	Sea $\sigma$ un símplice generado por $\vec{v}_1, \ldots, \vec{v}_m \in \R^n$, definimos su
	\textbf{baricentro} $\est{\sigma}$ como
	\begin{equation*}
		\est{\sigma} \coloneq \frac{1}{m} \sum_{i=1}^{m}\vec{v}_i.
	\end{equation*}
\end{definition}
\begin{observation}
	El baricentro $\est{\sigma}$ es un elemento de $\sigma$. Esto se debe a que $\est{\sigma}$ es la
	combinación convexa de $\vec{v}_1, \ldots, \vec{v}_m$, donde $\theta_1 = \cdots = \theta_m =
	\frac{1}{m}$.
\end{observation}

\begin{definition}
	\label{def:r-sigma}
	Sea $\sigma$ un símplice y sea $\est{\sigma}$ su baricentro. Entonces definimos el \textbf{radio
	de la circunferencia inscrita} en $\sigma$ con centro $\est{\sigma}$ como
	\begin{equation}
		\label{eq:def:r-sigma}
		r_\sigma \coloneq \max \lbrace r > 0 \colon B_r^{(k)}(\est{\sigma})
		\subseteq \sigma \rbrace,
	\end{equation}
	donde $B_r^{(k)}(\est{\sigma})$ está dada en la Definición \ref{fin:def:ball}.
\end{definition}

Encontraremos que el radio de la circunferencia inscrita está dado por el mínimo de las distancias
entre el baricentro del símplice con cada una de sus facetas. Puesto que $\est{\sigma}_j \in
\sigma_j$, sabemos bien por álgebra lineal, bien por optimización, que la distancia entre $\sigma$ y
su $j$-ésima faceta $\sigma_j$ es
\begin{equation}
	\label{eq:dist:v1}
	d(\est{\sigma}, \sigma_j) = |\uvec{\mu}_j^T(\est{\sigma} - \est{\sigma}_j)|,
\end{equation}
donde $\uvec{\mu}_j$ es un vector unitario y normal a la $j$-ésima faceta.

\begin{lemma}
	\label{mu:orth}
	Sean $\vec{q} > \vec{0}$, $k > 0$ y retomemos el símplice $\sigma$ generado por los vectores
	$\braces{\vec{u}_i}_{i=1}^{n}$ en \eqref{def:u-basis}. Definamos, para cada $j \in \braces{1,
	\ldots, n}$,
	\begin{equation}
		\label{eq:normal}
		\vec{\mu}_j \coloneq \vec{u}_j - \frac{\vec{q}^T\vec{u}_j}{\vec{q}^T\vec{q}}\vec{q}
		= \vec{u}_j - \frac{k}{\norm{\vec{q}}^2}\vec{q}.
	\end{equation}
	Entonces $\vec{\mu}_j$ es un vector normal a la $j$-ésima faceta $\sigma_j$ del símplice
	$\sigma$.
\end{lemma}
\begin{proof}
	Debemos mostrar que si $\vec{x} \in \sigma_j$, entonces $\vec{\mu}_j^T\vec{y} = 0$ para todo
	$\vec{y} \in \sigma_j - \vec{x}$. Por la Definición \ref{def:simplex}, tenemos que los vectores
	$\braces{\vec{u}_i}_{i \neq j}$ generan la $j$-ésima faceta $\sigma_j$, y entonces basta mostrar
	que $\vec{\mu}_j^T\vec{y} = 0$ para todo $\vec{y} \in \sigma_j - \vec{u}_m$ con $m \neq j$.

	Sea, pues, $m \in \braces{1, \ldots, n} \setminus \braces{j}$. Tenemos de las definiciones
	\ref{def:aff} y \ref{def:simplex}, así como del Lema \ref{lemma:aff} que
	\begin{equation*}
		\sigma_j = \conv\braces{\vec{u}_i}_{i \neq j} \subseteq
		\aff\braces{\vec{u}_i}_{i \neq j}
		= \vec{u}_m + \gen\braces{\vec{u}_i - \vec{u}_m}_{i \neq j}.
	\end{equation*}
	De donde obtenemos
	\begin{equation*}
		\sigma_j - \vec{u}_m \subseteq \gen\braces{\vec{u}_i - \vec{u}_m}_{i \neq j},
	\end{equation*}
	así que basta mostrar que $\vec{\mu}_j^T(\vec{u}_i - \vec{u}_m) = 0$ para todo $i \neq j$ Cabe
	mencionar que los vectores $\braces{\vec{u}_i}_{i=1}^{n}$ son ortogonales entre sí. Sustituyendo
	con la definición de $\vec{\mu}_j$ en la hipótesis, obtenemos
	\begin{align*}
		\vec{\mu}_j^T(\vec{u}_i - \vec{u}_m)
		&=
		\vec{u}_j^T\vec{u}_i - \vec{u}_j^T\vec{u}_m - \frac{k}{\norm{\vec{q}}^2}(\vec{q}^T\vec{u}_i
		- \vec{q}^T\vec{u}_m) \\
		&= 0 - 0 -\frac{k}{\norm{\vec{q}}^2}(k - k) \\
		&= 0.
	\end{align*}
	De esta manera, concluimos que $\vec{\mu}_j$ es un vector normal a $\sigma_j$.
\end{proof}

Ahora que encontramos vectores normales $\vec{\mu}_j$ para cada faceta $\sigma_j$, podemos
simplificar un poco más \eqref{eq:dist:v1}. Aprovechando el hecho de que
$\braces{\vec{u}_i}_{i=1}^{n}$ son todos ortogonales entre sí, obtenemos cálculos simples:
\begin{align*}
	\vec{\mu}_j^T\est{\sigma}
	&=
	\left(\vec{u}_j - \frac{k}{\norm{\vec{q}}^2}\vec{q}\right)^T \frac{1}{n}\sum_{i=1}^{n}\vec{u}_i \\
	&=
	\frac{1}{n}\sum_{i=1}^{n}\vec{u}_j^T\vec{u}_i - \frac{k}{n\norm{\vec{q}}^2}
	\sum_{i=1}^{n}\vec{q}^T\vec{u}_i \\
	&= \frac{1}{n}\norm{\vec{u}_j}^2 - \frac{k}{n\norm{\vec{q}}^2}\sum_{i=1}^{n}k \\
	&= \frac{k^2}{nq_j^2} - \frac{k^2}{\norm{\vec{q}}^2}.
\end{align*}
A través de un procedimiento similar, encontramos también que
\begin{equation}
	\label{eq:facetaux}
	\vec{\mu}_j^T\est{\sigma}_j = -\frac{k}{\norm{\vec{q}}^2},
\end{equation}
y por lo tanto
\begin{equation}
	\label{eq:signed}
	\vec{\mu}_j^T(\est{\sigma} - \est{\sigma}_j) = \frac{k^2}{nq_j^2}.
\end{equation}
Más adelante normalicaremos $\vec{\mu}$ de manera que este vector sea unitario. Cabe resaltar el
hecho de que el lado derecho \eqref{eq:signed} es positivo. Geométricamente, lo anterior implica
que los vectores normales $\vec{\mu}_j$ de cada faceta $\sigma_j$ apuntan hacia el interior relativo
del símplice $\sigma$. Esto sugiere una caracterización alternativa de $\sigma$ que nos permite
interpretarlo como un poliedro.
\begin{lemma}
	\label{lemma:sigmachar2}
	Sea $\vec{q} > \vec{0}$ un vector coprimo y sea $\sigma$ el símplice generado por los vectores
	definidos en \eqref{def:u-basis}, con $k > 0$. Entonces
	\begin{equation}
		\label{eq:sigmachar2}
		\sigma = \bigcap_{j=1}^{n}
		\braces{\vec{x} \in \R^n \colon \uvec{\mu}_j^T(\vec{x} - \est{\sigma}_j) \geq 0}
		\cap H_{\vec{q}, k\norm{\vec{q}}^{-2}},
	\end{equation}
	donde $\uvec{\mu}_j$ es el vector $\vec{\mu}_j$ definido en \eqref{eq:normal} normalizado.
\end{lemma}
\begin{proof}
	Denotemos por $\braces{\vec{u}_i}_{i=1}^{n}$ los vectores ortogonales definidos en
	\eqref{def:u-basis}. Como $\uvec{\mu}_j$ es el vector $\vec{\mu}_j$ normalizado, se sigue que
	\begin{equation*}
		\braces{ \vec{x} \in \R^n \colon \uvec{\mu}_j^T(\vec{x} - \est{\sigma}_j) \geq 0}
		=
		\braces{ \vec{x} \in \R^n \colon \vec{\mu}_j^T(\vec{x} - \est{\sigma}_j) \geq 0},
	\end{equation*}
	y entonces podemos trabajar con $\vec{\mu}_j$ sin normalizarlo. 

	Sea $\vec{x} \in \sigma$. Por el Lema \ref{lemma:sigmachar1} sabemos que
	$\vec{x}$ se encuentra en la $k$-ésima capa entera. También sabemos que existen escalares no
	negativos $\theta_1, \ldots, \theta_n$ que suman 1 y que satisfacen $\vec{x} = \theta_1\vec{u}_1
	+ \cdots + \theta_n\vec{u}_n$. Tenemos entonces
	\begin{align*}
		\vec{\mu}_j^T\vec{x}
		&=
		\left(\vec{u}_j - \frac{k}{\norm{\vec{q}}^2}\vec{q}\right)^T
		\left(\theta_j\vec{u}_j + \sum_{i \neq j}\theta_i\vec{u}_i\right) \\
		&= \theta_j\norm{\vec{u}_j}^2 - \frac{k}{\norm{\vec{q}}^2}\sum_{i \neq j}
		\theta_i\vec{q}^T\vec{u}_i \\
		&= \theta_j\frac{k^2}{q_j^2} - \frac{k}{\norm{\vec{q}}^2}\sum_{i \neq j}k\theta_i \\
		&= \theta_j\frac{k^2}{q_j^2} - \frac{k^2}{\norm{\vec{q}}^2}(1 - \theta_j) \\
		&= \theta_j\left(\frac{k^2}{q_j^2} + \frac{k^2}{\norm{\vec{q}}^2}\right)
		- \frac{k^2}{\norm{\vec{q}}^2}.
	\end{align*}
	Retomamos de \eqref{eq:facetaux} el valor de $\vec{\mu}_j^T\est{\sigma}_j$, así que obtenemos
	\begin{equation*}
		\vec{\mu}_j^T(\vec{x} - \est{\sigma}_j)
		= 
		\vec{\mu}_j^T\vec{x} - \vec{\mu}_j^T\est{\sigma}_j
		=
		\theta_j\left(\frac{k^2}{q_j^2} + \frac{k^2}{\norm{\vec{q}}^2}\right),
	\end{equation*}
	lo cual es no negativo para todo $j \in \braces{1, \ldots, n}$.

	Mostramos la otra contención por contrapositiva, así que supongamos que $\vec{x} \not\in
	\sigma$. Por el Lema \ref{lemma:sigmachar1} se sigue o bien que $\vec{x} \not\in H_{\vec{q},
	k\norm{\vec{q}}^{-2}}$ o bien que $\vec{x} \not\in \R^n_{\geq \vec{0}}$. En el primer caso
	obtenemos inmediatamente que $\vec{x}$ no se encuentra en el lado derecho de
	\eqref{eq:sigmachar2}.

	Supongamos, pues, que $\vec{x}$ está en la $k$-ésima capa entera pero que tiene al menos una
	entrada negativa con respecto a la base canónica. Como $\braces{\vec{u}_i}_{i=1}^{n}$ es base de
	$\R^n$, existen escalares $\braces{\lambda_i}_{i=1}^{n}$ tales que
	\begin{equation*}
		\vec{x} = \sum_{i=1}^{n}\lambda_i\vec{u}_i.
	\end{equation*}
	Como las entradas de $\vec{u}_1, \ldots, \vec{u}_n$ son todas no negativas y $x_j < 0$ para
	alguna $j \in \braces{1, \ldots, n}$, se sigue que $\lambda_j < 0$. Observemos que
	\begin{align*}
		\vec{\mu}_j^T\vec{x}
		&=
		\sum_{i=1}^{n}\lambda_i\vec{\mu}_j^T\vec{u}_i \\
		&=
		\sum_{i=1}^{n}\lambda_i\left(\vec{u}_j - \frac{k}{\norm{\vec{q}}^2}\vec{q}\right)^T\vec{u}_i \\
		&=
		\sum_{i=1}^{n}\lambda_i\left(\vec{u}_j^T\vec{u}_i -
			\frac{k}{\norm{\vec{q}}^2}\vec{q}^T\vec{u}_i\right) \\
		&=
		\lambda_j\norm{\vec{u}_j}^2 - \frac{k^2}{\norm{\vec{q}}} \sum_{i=1}^{n}\lambda_i.
	\end{align*}
	Pero $\vec{x} \in H_{\vec{q}, k\norm{\vec{q}}^{-2}} = \aff\braces{\vec{u}_1, \ldots, \vec{u}_n}$
	(ver Ejemplo \ref{ex:aff}) y entonces los escalares $\lambda_1, \ldots, \lambda_n$ suman a 1.
	Sustituyendo,
	\begin{equation*}
		\vec{\mu}_j^T\vec{x} = \lambda_j\frac{k^2}{q_j^2} - \frac{k^2}{\norm{\vec{q}}^2},
	\end{equation*}
	retomando el valor de $\vec{\mu}_j^T\est{\sigma}_j$ en \eqref{eq:facetaux}, encontramos que
	\begin{equation*}
		\vec{\mu}_j^T(\vec{x} - \est{\sigma}_j) = \lambda_j\frac{k^2}{q_j^2} < 0
	\end{equation*}
	y entonces $\vec{x}$ no es elemento del semi-espacio $\braces{\vec{x} \colon
	\vec{\mu}_j^T(\vec{x} - \est{\sigma}_j) \geq 0}$, por lo que tampoco es elemento del lado
	derecho de \eqref{eq:sigmachar2}.
\end{proof}

\begin{theorem}
	\label{lemma:sigma-radius}
	Sea $\vec{q} > \vec{0}$ un vector coprimo y sea $\sigma$ el símplice generado por los vectores
	$\braces{\vec{u}_i}_{i=1}^{n}$ definidos en \eqref{def:u-basis}. Entonces el radio $r_\sigma$ de
	la circunferencia inscrita (ver Definición \ref{def:r-sigma}) en $\sigma$ con centro
	$\est{\sigma}$ está dado por
	\begin{equation*}
		r_\sigma = \min_{1 \leq j \leq n} d(\est{\sigma}, \sigma_j)
		= \min_{1 \leq j \leq n} \uvec{\mu}_j^T(\est{\sigma} - \est{\sigma}_j),
	\end{equation*}
	donde $\uvec{\mu}_j$ es el vector $\vec{\mu}_j$ definido en \eqref{eq:normal} normalizado.
\end{theorem}
\begin{proof}
	Como $\est{\sigma} \in \sigma$, tenemos del Lema \ref{lemma:sigmachar2} que
	$\vec{\mu}_j^T(\est{\sigma} - \est{\sigma}_j) \geq 0$ y, por lo tanto, deducimos de \eqref{eq:dist:v1}
	que la distancia entre $\est{\sigma}$ y la $j$-ésima faceta $\sigma_j$ es
	\begin{equation}
		\label{eq:dist}
		d(\est{\sigma}, \sigma_j) = \uvec{\mu}_j^T(\est{\sigma} - \est{\sigma}_j).
	\end{equation}

	Supongamos que $r \leq d(\est{\sigma}, \sigma_j)$ para todo $j \in \braces{1, \ldots, n}$ y sea
	$\vec{x} \in B_r^{(k)}(\est{\sigma})$. Observemos que
	\begin{align*}
		\uvec{\mu}_j^T(\vec{x} - \est{\sigma}_j)
		&= 
		\uvec{\mu}_j^T(\vec{x} - \est{\sigma})
		+
		\uvec{\mu}_j^T(\est{\sigma} - \est{\sigma}_j) \\
		&=
		\uvec{\mu}_j^T(\vec{x} - \est{\sigma}) + d(\est{\sigma}, \sigma_j).
	\end{align*}
	Por la desigualdad de Cauchy-Schwartz, tenemos
	\begin{equation*}
		\uvec{\mu}_j^T(\vec{x} - \est{\sigma}) \geq -\norm{\uvec{\mu}_j}\norm{\vec{x} -
		\est{\sigma}} \geq -r,
	\end{equation*}
	pues $\uvec{\mu}$ es unitario y $\vec{x} \in B_r^{(k)}(\est{\sigma})$. Así pues, tenemos
	\begin{equation*}
		\uvec{\mu}_j^T(\vec{x} - \est{\sigma}_j) \geq -r + d(\est{\sigma}, \sigma_j) \geq 0,
	\end{equation*}
	pues supusimos que $r \leq d(\est{\sigma}, \sigma_j)$ para todo $j \in \braces{1, \ldots, n}$.
	Además, como $\vec{x} \in B_r^{(k)}(\est{\sigma})$, por la Definición \ref{fin:def:ball} tenemos
	que $\vec{x}$ se encuentra en la $k$-ésima capa entera. Así pues,
	\begin{equation*}
		\vec{x} \in
		\bigcap_{j=1}^{n}
		\braces{\vec{x} \in \R^n \colon \uvec{\mu}_j^T(\vec{x} - \est{\sigma}_j) \geq 0}
		\cap H_{\vec{q}, k\norm{\vec{q}}^{-2}} = \sigma,
	\end{equation*}
	donde la última igualdad se sigue del Lema \ref{lemma:sigmachar2}. Así pues,
	$B_r^{(k)}(\est{\sigma}) \subseteq \sigma$ si $r \leq d(\est{\sigma}, \sigma_j)$ para toda $j
	\in \braces{1, \ldots, n}$. De la Definición \ref{def:r-sigma} encontramos entonces que el radio
	$r_\sigma$ de la circunferencia inscrita satisface
	\begin{equation}
		\label{r-sigma:down}
		r_\sigma \geq \min_{1 \leq j \leq n}d(\est{\sigma}, \sigma_j).
	\end{equation}

	Ahora bien, supongamos que $r > d(\est{\sigma}, \sigma_j)$ para alguna $j \in \lbrace 1, \ldots,
	n \rbrace$. Consideremos el punto $\vec{x} \in \sigma_j$ que satisface $d(\est{\sigma}, \sigma_j) =
	d(\est{\sigma}, \vec{x})$. Tal punto existe porque $\sigma_j$ es cerrado. Luego,
	$\norm{\vec{x} - \est{\sigma}} < r$. Entonces existe $\varepsilon > 0$ tal que
	\begin{equation*}
		\norm{(\vec{x} - \varepsilon\est{\mu}_j) - \est{\sigma}} \leq r,
	\end{equation*}
	lo que implica que $\vec{x} - \varepsilon\est{\mu}_j \in
	B_r^{(k)}(\est{\sigma})$. Observemos que
	\begin{equation*}
		\uvec{\mu}_j^T((\vec{x} - \varepsilon\uvec{\mu}_j) - \est{\sigma}_j)
		=
		\uvec{\mu}_j^T(\vec{x} - \est{\sigma}_j) - \varepsilon.
	\end{equation*}
	Pero $\vec{x}, \est{\sigma}_j \in \sigma_j$, así que $\vec{x} - \est{\sigma}_j \in \sigma_j -
	\est{\sigma}_j$. Del Lema \ref{mu:orth} encontramos que
	\begin{equation*}
		\uvec{\mu}_j^T(\vec{x} - \est{\sigma}_j) = 0,
	\end{equation*}
	de donde obtenemos
	\begin{equation*}
		\uvec{\mu}_j^T((\vec{x} - \varepsilon\uvec{\mu}_i) - \est{\sigma}_j)
		= -\varepsilon < 0,
	\end{equation*}
	lo cual implica que $\vec{x} - \varepsilon\uvec{\mu}_j$ no se encuentra en el semi-espacio
	definido por $\braces{\vec{x} \colon \uvec{\mu}^T(\vec{x} - \est{\sigma}_j) \geq 0}$. Así pues,
	por el Lema \ref{lemma:sigmachar2}, encontramos que $\vec{x} - \varepsilon\uvec{\mu}_j \not\in
	\sigma$. Pero $\vec{x} - \varepsilon\uvec{\mu}_j \in B_r^{(k)}(\est{\sigma})$. De aquí se desprende
	que $B_r^{(k)}(\est{\sigma}) \not\subseteq \sigma$ si $r > d(\est{\sigma}, \sigma_j)$ para
	alguna $j \in \braces{1, \ldots, n}$. De la Definición \ref{def:r-sigma} obtenemos entonces
	\begin{equation}
		\label{r-sigma:up}
		r_\sigma \leq \min_{1 \leq j \leq n}d(\est{\sigma}, \sigma_j).
	\end{equation}

	De \eqref{r-sigma:down} y de \eqref{r-sigma:up} concluimos entonces con lo que queríamos
	demostrar.
\end{proof}

De \eqref{eq:dist:v1} tenemos
\begin{equation}
	\label{dist:v2}
	d(\est{\sigma}, \sigma_j) = \uvec{\mu}_j^T(\est{\sigma} - \est{\sigma}_j)
	= \frac{\vec{\mu}_j^T(\est{\sigma} - \est{\sigma}_j)}{\norm{\vec{\mu}_j}}.
\end{equation}
Recordemos de \eqref{eq:signed} que ya contamos con el numerador, así que ahora debemos calcular la
norma de $\vec{\mu}_j$. Tenemos
\begin{align*}
	\norm{\vec{\mu}_j}^2
	&= \vec{\mu}_j^T\vec{\mu}_j \\
	&=
	\left(\vec{u}_j - \frac{k}{\norm{\vec{q}}^2}\vec{q}\right)^T
	\left(\vec{u}_j - \frac{k}{\norm{\vec{q}}^2}\vec{q}\right) \\
	&=
	\norm{\vec{u}_j}^2 - 2\frac{k}{\norm{\vec{q}}^2}\vec{q}^T\vec{u}_j +
	\frac{k^2}{\norm{\vec{q}}^4}\vec{q}^T\vec{q} \\
	&= \frac{k^2}{q_j^2} - 2\frac{k^2}{\norm{\vec{q}}^2} + \frac{k^2}{\norm{\vec{q}}^2} \\
	&= \frac{k^2}{q_j^2} - \frac{k^2}{\norm{\vec{q}}^2}.
\end{align*}
De donde obtenemos
\begin{equation}
	\label{eq:den}
	\norm{\vec{\mu}_j} = k\sqrt{\frac{1}{q_j^2} - \frac{1}{\norm{\vec{q}}^2}}.
\end{equation}

Usando \eqref{eq:signed} y \eqref{eq:den} para sustituir en \eqref{dist:v2}, obtenemos
\begin{equation*}
	d(\est{\sigma}, \sigma_j) = \frac{k}{n} \cdot
	\frac{1}{q_j^2\sqrt{q_j^{-2} - \norm{\vec{q}}^{-2}}}
	= \frac{k}{n} \cdot \frac{1}{Q_j},
\end{equation*}
donde definimos $Q_j$ pertinentemente. Finalmente, del Teorema \ref{lemma:sigma-radius} encontramos que
el radio $r_\sigma$ de la circunferencia inscrita en el símplice $\sigma$ con centro
$\est{\sigma}$ está dado por
\begin{equation}
	\label{eq:sigma-radius}
	r_\sigma = \min_{1 \leq j \leq n} \braces{d(\est{\sigma}, \sigma_j)} = \frac{k}{n} \cdot
	\frac{1}{\max_{1 \leq j \leq n} \lbrace Q_j \rbrace}
\end{equation}

\begin{theorem}
	\label{th:intsimplex}
	Sea $\vec{q} > \vec{0}$ un vector coprimo y sea $k$ un entero positivo suficientemente grande.
	Entonces existe un punto entero sobre el símplice $\sigma$ generado por los vectores en
	\eqref{def:u-basis}.
\end{theorem}
\begin{proof}
	Sea $r$ el radio definido en \eqref{eq:radius} y sea $r_\sigma$ el radio definido en
	\eqref{eq:sigma-radius}. Por el Teorema \ref{lemma:ball-cover} sabemos que existe un punto
	entero $\vec{x}$ en $B_r^{(k)}(\est{\sigma})$, y por el Teorema \ref{lemma:sigma-radius} sabemos
	que la bola $B_{r_\sigma}^{(k)}(\est{\sigma})$ está contenida en $\sigma$. Entonces basta
	mostrar que existe $k$ suficientemente grande tal que $r \leq r_\sigma$, pues esto implicaría la
	contención de en medio en la cadena
	\begin{equation*}
		\vec{x} \in B_r^{(k)}(\est{\sigma}) \subseteq B_{r_\sigma}^{(k)}(\est{\sigma}) \subseteq \sigma.
	\end{equation*}
	De \eqref{eq:radius} y de \eqref{eq:sigma-radius} obtenemos que $r \leq r_\sigma$ si y solo si
	\begin{equation}
		\label{eq:eta-limit}
		k \geq \frac{n}{2}\norm{M}_F\max_{1 \leq j \leq n} \lbrace Q_j \rbrace,
	\end{equation}
	que es lo que queríamos demostrar.
\end{proof}
De \eqref{eq:eta-limit} parece que podemos concluir que hay una dependencia lineal entre la
dimensión $n$ y el parámetro de la capa entera $k$. No obstante, la norma $\norm{M}_F$ depende
implícitamente de $n$. Para ser más explícitos con respecto a esta dependencia, podemos rescatar de
\eqref{eq:alt-M-bound} la siguiente cota:
\begin{equation*}
	\frac{1}{4}\sum_{j=1}^{n-1}\norm{M\vec{e}_j}^2
	\leq
	\frac{n-1}{4}\max_{1 \leq j \leq n} \lbrace \norm{M\vec{e}_j}^2 \rbrace,
\end{equation*}
de donde reemplazaríamos la cota \eqref{eq:eta-limit} en el Teorema \ref{th:intsimplex} por
\begin{equation*}
	k \geq \frac{n\sqrt{n-1}}{2} \max_{1 \leq j \leq n} \lbrace \norm{M\vec{e}_j} \rbrace \cdot
	\max_{1 \leq j \leq n} \lbrace Q_j \rbrace.
\end{equation*}
Esta cota, no obstante, es más grande que la propuesta inicialmente.

Además, el resultado que obtuvimos es más fuerte de lo que aparenta. Hemos encontrado una
cota inferior de manera que podamos asegurar la existencia de puntos enteros en una vecindad del
baricentro $\est{\sigma}$. Este punto no es especial, pues en realidad podemos realizar el mismo
procedimiento enfocándonos en otros puntos del símplice $\sigma$ para asegurar soluciones en sus
respectivas vecindades. Entonces, dependiendo del punto, podemos obtener mejores o peores cotas para
$k$. El punto más interesante es aquel que provee la cota inferior más pequeña\footnote{
	Una hipótesis del autor es que el baricentro $\est{\sigma}$ provee, en efecto, la mejor cota.
}.

De manera inmediata obtenemos también los siguientes Teoremas. Cabe mencionar que estos resultados
solamente muestran la existencia de una solución entera $\vec{x}$ no negativa para la ecuación lineal
diofantina $\vec{q}^T\vec{x} = k$. Será en la Sección \ref{subsec:complex} que discutiremos cómo
encontrar esta solución.
\begin{theorem}
	\label{th:intnonneg1}
	Sea $\vec{q} > \vec{0}$ un vector coprimo. Entonces la ecuación lineal diofantina
	$\vec{q}^T\vec{x} = k$ tiene soluciones enteras no negativas para $k$ suficientemente grande.
\end{theorem}
\begin{proof}
	Consideremos el símplice $\sigma$ generado por los vectores en \eqref{def:u-basis} y supongamos
	que $k$ satisface la cota en \eqref{eq:eta-limit}.
	Por el Teorema \ref{th:intsimplex} existe un punto entero no negativo $\vec{x} \in \sigma$, y
	esto implica que $x \in H_{\vec{q}, k\norm{\vec{q}}^{-2}}$ por el Lema \ref{lemma:sigmachar1}.
	Luego, por el Lema \ref{theory:lemma:utility}, $\vec{x}$ satisface la ecuación lineal diofantina
	$\vec{q}^T\vec{x} = k$.
\end{proof}
\begin{theorem}
	\label{th:intnonneg2}
	Sea $\vec{p} \in \R^n$ un vector esencialmente entero y supongamos que su múltiplo coprimo
	$\vec{q}$ tiene entradas estrictamente positivas. Entonces el problema
	\eqref{theory:formulation} se puede resolver a través de encontrar la solución de una sola
	ecuación lineal en $n$ incógnitas para un presupuesto $u$ suficientemente grande.
\end{theorem}
\begin{proof}
	Por la Definición \ref{theory:def:rational} sabemos que existe un escalar $m$ tal que
	$\vec{p} = m\vec{q}$. Supongamos, sin pérdida de generalidad, que $m$ es positivo. Del Lema
	\ref{phase-1:lemma:eta} tenemos que el entero $\eta$ parametriza la primera capa entera en
	satisfacer el presupuesto y que $\eta = \floor{u/m}$. Por el Teorema \ref{th:intnonneg1} sabemos
	que si $\eta$ es suficientemente grande, entonces la ecuación lineal diofantina
	$\vec{q}^T\vec{x} = \eta$ tiene al menos una solución entera no negativa $\vec{x}$. Luego,
	$\vec{x}$ es factible para el problema \eqref{theory:formulation}, pero por la maximalidad de
	$\eta$ encontramos que $\vec{x}$ también es un punto óptimo. En conclusión, solo deviene
	necesario resolver una ecuación lineal diofantina para determinar la solución del problema
	\eqref{theory:formulation}.
\end{proof}

El Teorema \ref{th:intnonneg1} junto con la cota \eqref{eq:eta-limit} provee, hasta donde llega el
conocimiento del autor, nuevas cotas superiores para los números de Frobenius\footnote{
	Véase el Problema de la Moneda en \url{https://en.wikipedia.org/wiki/Coin_problem}.
}. De manera resumida,
dada una colección de enteros $a_1, \ldots, a_n$ coprimos, el número de Frobenius es el entero $F$
más grande tal que $F$ no pueda ser expresado como una combinación lineal entera no negativa de
$a_1, \ldots, a_n$. Un estudio sobre cómo se compara esta colección de cotas con respecto a la
literatura existente, si bien interesante, queda fuera del propósito de esta tesis.

En último lugar, mencionamos que eventualmente es suficiente con revisar la primera capa entera. No
hemos demostrado, empero, que el número de capas enteras a revisar eventualmente decrece en cuanto
el presupuesto $u$ aumenta. Observaremos en el análisis de resultados que hay un patrón periódico y
decreciente en cuanto al número de capas enteras revisadas. Demostrar, en cambio, que este
comportamiento siempre se cumple es mucho más difícil y queda fuera del propósito de esta tesis.

\section{Construcción de soluciones}
\label{subsec:complex}

\noindent
Sea $\vec{p} \in \R^n$ un vector esencialmente entero y supongamos que las entradas de su múltiplo
coprimo $\vec{q}$ son todas estrictamente positivas. Supongamos, sin pérdida de generalidad, que el
escalar $m$ que satisface $\vec{p} = m\vec{q}$ es también positivo. Bastante hemos discutido sobre
cómo la solución del problema \eqref{theory:formulation} se traduce a la búsqueda de una solución
entera no negativa de la ecuación lineal diofantina $\vec{q}^T\vec{x} = k$ para alguna $k \leq \eta$,
donde $\eta$ es tomada del Lema \ref{phase-1:lemma:eta}.

En esta sección presentamos los algoritmos \ref{algo:fin:helper} y
\ref{algo:fin:dioph}, los cuales se encargan de obtener estas soluciones enteras no negativas que
tanto buscamos y, consecuentemente, también se encargan de resolver el problema original
\eqref{theory:formulation}.

% En la segunda parte de esta sección discutimos de manera un tanto informal la complejidad
% algorítmica del Algoritmo \ref{algo:fin:dioph}. Por lo tanto, también discutimos, así como lo
% hicimos en el capítulo \ref{chap:inf}, sobre la complejidad del problema \eqref{theory:formulation}
% en el caso especial de que $\vec{q} > \vec{0}$.

\begin{theorem}
	\label{th:fin:helper:correct}
	El algoritmo \ref{algo:fin:helper} es correcto.
\end{theorem}
\begin{proof}
	Hacemos la demostración por inducción en la dimensión $n$ del vector $\vec{q}$. Supongamos, para
	el caso base, que $n = 2$. Luego, queremos encontrar soluciones enteras no negativas de la
	ecuación
	\begin{equation}
		\label{lemma:correct:base-case}
		q_1x_1 + q_2x_2 = k.
	\end{equation}
	Por hipótesis sabemos que $q_1$ y $q_2$ son coprimos. Luego, del Teorema
	\ref{prerreq:th:construction}, las soluciones enteras de esta ecuación están dadas por
	\begin{equation}
		\label{lemma:correct:base-case:sol}
		\begin{cases}
			x_1 = kx_1' + q_2t, \\
			x_2 = kx_2' - q_1t, \\
		\end{cases}
	\end{equation}
	donde $t \in \Z$ es una variable libre, y $x_1', x_2'$ son los coeficientes de Bézout (c.f.
	Definición \ref{prerreq:def:bezout}) de $q_1$ y $q_2$, respectivamente. Por claridad, escribimos
	$x_1'$ y $x_2'$ como $x_{n-1}'$ y $x_{n}'$ en la línea \ref{alg:fin:bez}. Despejando de estas
	soluciones, encontramos que existen soluciones no negativas si y solo si existe $t \in \Z$ que
	satisfaga
	\begin{equation*}
		\ceil{-\frac{kx_1'}{q_2}} \leq t \leq \floor{\frac{kx_2'}{q_1}}.
	\end{equation*}
	Los enteros $b_1$ y $b_2$ en las líneas \ref{alg:fin:b1} y \ref{alg:fin:b2} representan el
	lado izquierdo y derecho de estas desigualdades, respectivamente. De esta manera, el algoritmo
	devuelve \NIL~ si solo si este intervalo no está bien definido, es decir, si y solo si no existen
	soluciones enteras no negativas. Supongamos, pues, que este intervalo sí está bien definido.
	Entonces, podemos escoger que la variable libre $t$ sea $b_1$. Sustituyendo en
	\eqref{lemma:correct:base-case:sol} obtenemos una solución entera no negativa de la ecuación
	\eqref{lemma:correct:base-case} (líneas \ref{alg:fin:xprev} y \ref{alg:fin:xlast}) y entonces
	el algoritmo es correcto para $n = 2$.

	Supongamos, inductivamente, que el algoritmo es correcto para alguna $n - 1 \geq 2$. Mostramos
	ahora que el algoritmo también es correcto para $n$. Entonces deseamos encontrar soluciones
	enteras no negativas de la ecuación \eqref{eq:dioph} Haciendo la misma sustitución que en
	\eqref{eq:dioph:first-step}, recordando que $q_1, \ldots, q_n$ son coprimos por hipótesis, que
	definimos $\omega_1 \coloneq k$, y renombrando las variables ($x$ en vez de $x_1$, $g$ en vez de
	$g_2$ y $\omega$ en vez de $\omega_2$), obtenemos la ecuación
	\begin{equation}
		\label{lemma:correct:eq}
		q_1x + g\omega = k.
	\end{equation}
	Observemos que, como $g_1 = 1$, el entero $g = \gcd{q_2/g_1, \ldots, q_n/g_1}$, es equivalente a
	lo que se encuentra en la línea \ref{alg:fin:g}. Por el Teorema \ref{prerreq:th:construction}
	tenemos que las soluciones enteras están dadas por
	\begin{equation}
		\label{lemma:correct:sol}
		\begin{cases}
			x = kx' + gt, \\
			\omega = k\omega' - q_1t,
		\end{cases}
	\end{equation}
	donde $t \in \Z$ es una variable libre, y $x', \omega'$ son los coeficientes de Bézout de $x,
	\omega$. Recordemos de \eqref{eq:dioph:first-step} que
	\begin{equation}
		\label{lemma:correct:eq-omega}
		\omega = \frac{q_2}{g}x_2 + \ldots + \frac{q_n}{g}x_n.
	\end{equation}
	Como $\vec{q} > \vec{0}$ por hipótesis, $g > 0$ porque el máximo común divisor siempre es
	positivo, y exigimos que $x_2, \ldots, x_n$ sean no negativos, debe ser el caso que $\omega$
	también sea no negativo. Luego, despejando de \ref{lemma:correct:sol}, existen soluciones no
	negativas de la ecuación \eqref{lemma:correct:eq} si y solo si existe $t \in \Z$ que satisfaga
	\begin{equation*}
		\ceil{-\frac{kx'}{g}} \leq t \leq \floor{\frac{k\omega'}{q_1}}.
	\end{equation*}
	Los enteros $b_1$ y $b_2$ en las líneas \ref{alg:fin:b11} y \ref{alg:fin:b21} representan el
	lado izquierdo y derecho de estas desigualdades, respectivamente. Si no existe tal variable
	libre $t \in \Z$ es porque el intervalo $[b_1, b_2]$ no está bien definido y por lo tanto $b_2 <
	b_1$. El algoritmo entonces salta a la línea \ref{alg:fin:return} y devuelve \NIL.

	Si el intervalo $[b_1, b_2]$ está bien definido, podemos asegurar la no negatividad de $x$ en
	\eqref{lemma:correct:sol} para cualquier elección de $t$ en $[b_1, b_2$] y en
	la línea \ref{alg:fin:subeq} nos encargamos entonces de encontrar soluciones enteras no negativas
	de la ecuación \eqref{lemma:correct:eq-omega}. Se verifica automáticamente que los coeficientes
	del lado derecho de esta ecuación son coprimos y constituyen justamente las entradas del vector
	$\vec{q}^{\texttt{tail}}$ (c.f. línea \ref{alg:fin:qt}). Como $g > 0$ se sigue que
	$\vec{q}^{\texttt{tail}} > \vec{0}$. Luego, $\vec{q}^{\texttt{tail}}$ satisface las hipótesis
	del algoritmo.

	Por hipótesis inductiva, tenemos o bien que $\vec{x}^{\texttt{tail}}$ es entero no negativo y
	solución de \eqref{lemma:correct:eq-omega}, o bien es \NIL. En el primer caso y definiendo
	$\vec{x}$ como el vector de la línea \ref{alg:fin:vecx} encontramos que
	\begin{equation*}
		\vec{q}^T\vec{x} = q_1x + g\left(\vec{q}^{\texttt{tail}}\right)^T\vec{x}^{\texttt{tail}}
		= q_1x + g\omega = k.
	\end{equation*}
	Pero $x \geq 0$ por construcción y $\vec{x}^{\texttt{tail}} \geq \vec{0}$ por este caso de la
	hipótesis inductiva. Así, $\vec{x}$ también es no negativo.

	Finalmente, en caso de que $\vec{x}^{\texttt{tail}}$ sea \NIL, iteramos sobre otra elección de
	la variable libre $t$ y regresamos al caso pasado. En caso de que este vector sea \NIL~ para
	todas las elecciones posibles de $t$ en el intervalo de factiblidad $[b_1, b_2]$, se sigue por
	hipótesis inductiva que la ecuación \eqref{lemma:correct:eq-omega} no tiene solución entera no
	negativa y por lo tanto tampoco la tiene la ecuación \eqref{eq:dioph}. Una vez agotadas estas
	elecciones finitas, devolvemos \NIL~ en la línea \ref{alg:fin:return}.

	En conclusión, si el algoritmo es correcto para vectores $\vec{q}$ con dimensión $n - 1 \geq 2$,
	entonces también es correcto para vectores $\vec{q}$ con dimensión $n$. Juntando esto con el
	caso base, se sigue por inducción que el algoritmo es correcto para toda $n \geq 2$, que es lo
	que queríamos demostrar.
\end{proof}

Observemos que la elección del parámetro libre $t$ en el intervalo de factiblidad $I$ definido en la
línea \ref{alg:fin:feas} del Algoritmo \ref{algo:fin:helper} es similar a la elección del
subproblema $S_i$ de optimización definido en la línea \ref{p1c9:alg:BB_loop} del Algoritmo
\ref{algo:bb}. La diferencia radica en que, como todos los puntos enteros sobre la $k$-ésima capa
entera tienen el mismo nivel de utilidad $k$, no es necesario desarrollar políticas de poda así como
lo hicimos en el Ejemplo \ref{ex:ilp} en la Sección \ref{sec:prerreq}. De cierta manera, la única
política de poda posible es la de infactibilidad por no respetar la no negatividad de un punto
entero.

\begin{theorem}
	\label{th:fin:dioph:correct}
	El algoritmo \ref{algo:fin:dioph} es correcto.
\end{theorem}
\begin{proof}
	A causa del Teorema \ref{th:fin:helper:correct} basta verificar que el algoritmo termina y no
	devuelve \NIL. Además, obtenemos la maximalidad de $k$ debido a la manera en la que iniciamos el
	ciclo en la línea \ref{alg:fin:loop}. Tenemos $0 \leq \eta$ por hipótesis y observemos que
	$\vec{0}$ es la única solución entera no negativa de la ecuación lineal diofantina
	$\vec{q}^T\vec{x} = 0$. De esta manera, si la ecuación $\vec{q}^T\vec{x} = k$ no tiene solución
	para $0 < k \leq \eta$, entonces el algoritmo devuelve $\vec{0}$ debido al Teorema
	\ref{th:fin:helper:correct}.
\end{proof}

Sabemos, en realidad, por el Lema \ref{lemma:tau} que el parámetro $k$ definido en la línea
\ref{alg:fin:loop} descenderá hasta 0 si y solo si el parámetro $\tau$ definido en \eqref{eq:tau}
es nulo. No obstante, la demostración del Teorema \ref{th:fin:dioph:correct} deviene más simple
cuando en el Algoritmo \ref{algo:fin:dioph} dejamos que $k$ se encuentre en $[0, \eta]$ en vez de
$[\tau, \eta]$. Esta modificación, sin embargo, no afecta en lo más mínimo la correctud o la
complejidad del algoritmo.

Siguiendo la misma directriz, vale la pena mencionar lo siguiente con respecto al Algoritmo
\ref{algo:fin:helper}. Varios lenguajes de programación, tales como Python, cuentan con un límite en
las llamadas de recursión que el usuario puede realizar\footnote{
	En la computadora del autor, por ejemplo, el valor predeterminado de este límite es 3000, y por
	lo tanto, solamente podría el autor resolver problemas con dimensión $n \leq 3000$.
}. Si bien este límite puede modificarse, aumenta la posibilidad de encontranos con un
desbordamiento de pila, pues este algoritmo no está expresado en forma de recursión
terminal\footnote{
	Véase \url{https://en.wikipedia.org/wiki/Tail_call}.
}.

Además, este algoritmo, por ejemplo, no minimiza el número de llamadas para calcular el máximo común
divisor en la línea \ref{alg:fin:g}. En efecto, supongamos que un intervalo de factibilidad $I$
definido en la línea \ref{alg:fin:feas} induce a que $\vec{x}^{\texttt{tail}}$ sea \NIL~ para todo
$t \in I$. Entonces estaríamos haciendo $|I|$ llamadas recursivas a \texttt{NonNegativeIntSolFin} en
la línea \ref{alg:fin:subeq} con el mismo vector $\vec{q}^{\texttt{tail}}$ y, por lo tanto,
estaríamos calculando $|I|$ veces la misma $g$ en la línea \ref{alg:fin:g}. Lo mismo ocurre con el
cálculo de los coeficientes de Bézout $x'$ y $\omega'$ en la línea \ref{alg:fin:bez2}.

A pesar de los puntos anteriores, el autor decidió escribir el Algoritmo \ref{algo:fin:helper} de
esa manera debido a que se simplificaba de manera significativa la demostración del Teorema
\ref{th:fin:helper:correct}. Sin embargo, el autor realizó una implementación equivalente más
eficiente a través de ciclos para obtener los resultados de la siguiente sección.

\begin{algorithm}[ht]
	\LinesNumbered
	\SetKwProg{Fn}{Fn}{\string:}{}
	\SetKwFunction{Bezout}{Bezout}
	\SetKwFunction{length}{length}
	\SetKwFunction{NonNegativeIntSol}{NonNegativeIntSolFin}
		\KwData{\\
			$\vec{q} \in \Z^n_{> \vec{0}}$ coprimo tal que \length{$\vec{q}$} $\geq 2$. \\
			$k \geq 0$.
			}
		\KwResult{\\
			$\vec{x} \in \Z^n_{\geq \vec{0}}$ tal que $\vec{q}^T\vec{x} = k$ o \NIL.
		}
		\Begin{
			$n \leftarrow$ \length{$\vec{q}$}\;
			\If{$n = 2$}{
				$x_{n-1}', x_n' \leftarrow$ \Bezout{$q_1$, $q_2$}\; \label{alg:fin:bez}
				$b_1 \leftarrow \ceil{-k x_{n-1}' / q_2}$\; \label{alg:fin:b1}
				$b_2 \leftarrow \floor{k x_{n}' / q_1}$\; \label{alg:fin:b2}

				\If{$b_2 < b_1$}{
					\Return $\NIL$\;
				}

				$x_{n-1} \leftarrow k x_{n-1}' + b_1q_2$\; \label{alg:fin:xprev}
				$x_{n} \leftarrow k x_{n}' - b_1q_1$\; \label{alg:fin:xlast}
				\Return{$(x_{n-1}, x_n)$}\;
			}

			$g \leftarrow \gcd{q_2, \ldots, q_n}$\; \label{alg:fin:g}
			$x', \omega' \leftarrow$ \Bezout{$q_1$, $g$}\; \label{alg:fin:bez2}
			$b_1 \leftarrow \ceil{-k x' / g}$\; \label{alg:fin:b11}
			$b_2 \leftarrow \floor{k \omega' / q_1}$\; \label{alg:fin:b21}
			$I \leftarrow \braces{b_1, b_1 + 1, \ldots, b_2}$\; \label{alg:fin:feas}

			$\vec{q}^{\texttt{tail}} \leftarrow (q_{i + 1} / g \vcentcolon 1 \leq i \leq n - 1)$\;
			\label{alg:fin:qt}
			\While{$I \neq \emptyset$}{
				elegir $t \in I$\;
				$\omega \leftarrow k\omega' - tq_1$\;
				$\vec{x}^{\texttt{tail}} \leftarrow$ \NonNegativeIntSol{$\vec{q}^{\texttt{tail}}$,
				$\omega$}\; \label{alg:fin:subeq}

				\If{$\vec{x}^{\texttt{tail}} \neq$ \NIL}{
					$r \leftarrow$ \length{$\vec{x}^{\texttt{tail}}$}\;
					$x \leftarrow k x' + tg$\; \label{alg:fin:x}
					\Return{$(x, x^{\texttt{tail}}_1, \ldots, x^{\texttt{tail}}_r$})\;
					\label{alg:fin:vecx}
				}

				$I \leftarrow I \setminus \braces{t}$\;
			}

			\Return $\NIL$\; \label{alg:fin:return}
		}
	\caption{\texttt{NonNegativeIntSolFin}}
	\label{algo:fin:helper}
\end{algorithm}
\begin{algorithm}[ht]
	\LinesNumbered
	\SetKwProg{Fn}{Fn}{\string:}{}
	\SetKwFunction{Bezout}{Bezout}
	\SetKwFunction{length}{length}
	\SetKwFunction{NonNegativeIntSol}{NonNegativeIntSolFin}
	\SetKwFunction{Dioph}{Dioph}
		\KwData{\\
			$\vec{q} \in \Z_{> \vec{0}}$ coprimo tal que \length{$\vec{q}$} $\geq 2$. \\
			$\eta \geq 0$.
		}
		\KwResult{\\
			$\vec{x} \in \Z^n_{\geq \vec{0}}$ tal que $\vec{q}^T\vec{x} = k$ con $0 \leq k
			\leq \eta$ maximal.
		}
		\Begin{
			\For{$k \leftarrow \eta$ \KwTo $0$}{ \label{alg:fin:loop}
				$\vec{x} \leftarrow$ \NonNegativeIntSol{$\vec{q}$, $k$}\;
				\If{$\vec{x} \neq$ \NIL}{
					\Return{$\vec{x}$}\;
				}
			}
		}
	\caption{\texttt{Dioph}}
	\label{algo:fin:dioph}
\end{algorithm}
% Con respecto a la complejidad algorítmica de analizar capas enteras podemos decir lo siguiente.
% Supongamos que deseamos encontrar todas las soluciones de (\ref{theory:formulation}). Definamos
% \begin{equation}
% 	\label{phase-1:def:feasible-layer}
% 	P_k \coloneq H_{\vec{q}, k\norm{\vec{q}}^{-2}} \cap \Z_{\geq \vec{0}}^n
% 	=
% 	\lbrace \vec{x} \in \Z^n \vcentcolon \vec{q}^T\vec{x} = k, \vec{x} \geq \vec{0}
% 	\rbrace,
% \end{equation}
% y sea $T(n)$ el tiempo requerido para encontrar todos los puntos en $P_k$ o determinar que este
% conjunto es vacío. Es razonable suponer que $T(n)$ es exponencial en $n$. En efecto, cada par
% $(x_i, \omega_{i + 1})$ genera un intervalo de factibilidad $[t_i^{\min},
% t_i^{\max}]$. Este intervalo ciertamente depende de las elecciones previas de $t_1, \ldots,
% t_{i - 1}$, aunque suprimimos esta dependencia en la notación para tener mayor claridad. Para
% encontrar todos los puntos en $P_k$, el algoritmo recorre todas las posibilidades:
% \begin{equation}
% 	\label{phase-1:complexity:bounds}
% 	\prod_{i=1}^{\kappa_1} \min_{t_1, \ldots, t_{i-1}} \lbrace t_i^{\max} -
% 	t_i^{\min} \rbrace
% 	\leq T(n) \leq
% 	\prod_{i=1}^{\kappa_2} \max_{t_1, \ldots, t_{i-1}} \lbrace t_i^{\max} -
% 	t_i^{\min} \rbrace,
% \end{equation}
% donde $1 \leq \kappa_1 \leq n$ es el entero más grande que asegura que  $\min_{t_1, \ldots,
% t_{i - 1}}\lbrace t_i^{\max} - t_i^{\min} \rbrace$ sea positivo para todo $i \in
% \lbrace 1, \ldots, \kappa_1 \rbrace$. Definimos $\kappa_2$ de manera análoga. Se cumple que
% $\kappa_1 \leq \kappa_2$. Sean $\ell_{\min}, \ell_{\max}$ las longitudes del intervalo de
% factibilidad más pequeño y del más grande en todos los niveles, respectivamente. Es decir, definimos
% \begin{align}
% 	\ell_{\min} &\coloneq \min_{1 \leq i \leq \kappa_1} \left\lbrace \min_{t_1, \ldots,
% 	t_{i - 1}} \lbrace t_i^{\max} - t_i^{\min} \rbrace \right\rbrace,
% 	\\
% 			\ell_{\max} &\coloneq \max_{1 \leq i \leq \kappa_2} \left\lbrace \max_{t_1,
% 			\ldots, t_{i - 1}} \lbrace t_i^{\max} - t_i^{\max} \rbrace
% 		\right\rbrace.
% \end{align}
% Si $P_k$ es vacío, se sigue que no existe ningún intervalo factible en el nivel $n$, lo que implica
% que $\kappa_2 < n$. En caso contrario, el algoritmo recorre hasta el último nivel, por lo que $\kappa_1
% = \kappa_2 = n$. De (\ref{phase-1:complexity:bounds}), obtenemos
% \begin{equation}
% 	\ell_{\min}^{n} \leq T(n) \leq \ell_{\max}^{n}.
% \end{equation}
% En el peor de los casos, nuestro algoritmo recorre todas las capas enteras parametrizadas por
% $\lbrace k, k - 1, \ldots, \tau\rbrace$. Se sigue que
% \begin{align}
% 	\label{phase-1:time}
% 	\text{Tiempo de ejecución}
% 	= \mathcal{O}((k - \tau) \cdot T(n))
% 	= \mathcal{O}(c^{n}),
% \end{align}
% para alguna $c > 1$.
% 
% Ahora bien, este razonamiento aplica a la modificación de nuestro objetivo en donde decidimos buscar
% todas las soluciones posibles. En realidad solo nos interesa encontrar un punto óptimo, por lo que
% podemos concluir que una cota superior para la complejidad de nuestro algoritmo es
% (\ref{phase-1:time}). Asímismo, en la práctica encontramos que la diferencia $k - \tau$ es crucial
% para determinar cuántas capas enteras recorre nuestro algoritmo en el peor de los casos.
% 
% Hemos mostrado anteriormente que para $u$ suficientemente grande es suficiente con recorrer una
% capa. En caso de que $u$ no sea suficientemente grande, observaremos en el análisis de resultados
% cómo se distribuye el número de capas enteras que en realidad se visitan.

\section{Análisis de resultados}

\section{Aplicaciones}

\chapter{Múltiples restricciones}
\noindent
En esta sección hacemos un análisis extensivo sobre lo resulta de agregar más restricciones al
problema (\ref{theory:formulation}). Sea $\vec{p} \in \R^n$ esencialmente entero y consideremos su
múltiplo coprimo $\vec{q} \in \Z^n$. Sea $A \in \Q^{m \times n}$ una matriz racional con renglones
linealmente independientes y sea $\vec{b} \in \Q^m$ un vector. Consideremos el problema
\begin{subequations}
	\label{formulation:multiple}
	\begin{align}
		\max_{\vec{x} \in \Z^n} \quad
			& \vec{q}^T\vec{x}, \label{formulation:multiple:objective} \\
		\text{s.a.} \quad
			& \vec{q}^T\vec{x} \leq u, \label{formulation:multiple:constraint:budget} \\
			& A\vec{x} = \vec{b}, \label{formulation:multiple:constraints} \\
			& \vec{x} \geq \vec{0}. \nonumber
	\end{align}
\end{subequations}
Ciertamente, la solución no se encuentra necesariamente en la $\eta$-ésima capa entera. Por ejemplo,
si dejamos que $A \coloneq \vec{q}^T$ y $b \coloneq u - m$, la solución se encontrará en la
$\xi$-ésima capa entera, donde
\begin{equation*}
	\xi \coloneq \left\lfloor \frac{u}{m} - 1 \right\rfloor < \eta.
\end{equation*}
No obstante, si el problema (\ref{formulation:multiple}) es factible, sabemos que la solución se
encontrará en alguna capa entera con parámetro $k \in \lbrace \eta, \eta - 1, \ldots \rbrace$, pues
todavía contamos con una restricción presupuestaria que se debe satisfacer.
\begin{observation}
	Recordemos del teorema \ref{theory:th:feasibility} que, si tenemos solamente la restricción
	presupuestaria, entonces la utilidad máxima es $\eta$ si $q_i < 0$ para alguna $i \in
	\lbrace 2, \ldots, n - 1\rbrace$. Al igual que en el caso finito, ahora no somos capaces de
	saber inmediatamente en qué capa entera se encuentra nuestra solución.
\end{observation}

Ahora bien, en el contexto del problema (\ref{formulation:multiple}), el parámetro $k \in \Z$ se
encarga de maximizar la utilidad (\ref{formulation:multiple:objective}), así como de respetar el
presupuesto (\ref{formulation:multiple:constraint:budget}) a través de $k \leq \eta$. Similarmente,
el vector $\vec{t} \in \Z^{n-1}$ se encarga de respetar las otras restricciones
(\ref{formulation:multiple:constraints}).
\begin{theorem}
	El problema (\ref{formulation:multiple}) es equivalente al problema de maximización
	\begin{subequations}
		\label{formulation:lattice}
		\begin{align}
			\max_{k \in \Z, \vec{t} \in \Z^{n-1}}
				& k, \\
			\text{s.a.} \quad
				& k \leq \eta, \label{lattice:c-layer} \\
				& AM\vec{t} = kA\vec{\nu} - \vec{b}, \label{lattice:constraints} \\
				& M\vec{t} \geq -k\vec{\nu}.
		\end{align}
	\end{subequations}
\end{theorem}
\begin{proof}
	Por el teorema \ref{th:lattice}, sabemos que la transformación lineal
	\begin{align*}
		(k, \vec{t}) &\mapsto \vec{x} \coloneq k\vec{\nu} + M\vec{t}
	\end{align*}
	es un isomorfismo entre las redes $\Z \oplus \Z^{n - 1}$ y $\Z^n$. Así, tenemos
	\begin{align*}
		A\vec{x} = \vec{b} &\iff AM\vec{t} = \vec{b} - kA\vec{\nu}, \\
		\vec{x} \geq \vec{0} &\iff M\vec{t} \geq -k\vec{\nu},
	\end{align*}
	y por lo tanto basta mostrar que si un vector es factible para un problema, entonces satisface
	la correspondiente restricción presupuestaria del otro problema. Para ello, es de utilidad
	recordar que $\eta$ parametriza la primera capa entera que satisface el presupuesto.

	Sea $\vec{x} \in \Z^n$ un vector factible de (\ref{formulation:multiple}) Como $\vec{x}$ es
	entero, entonces se debe cumplir $\vec{q}^T\vec{x} \leq \eta$. Ahora bien, existe $(k, \vec{t})
	\in \Z^n$ que satisface $\vec{x} = k\vec{\nu} + M\vec{t}$. Por el lema \ref{lemma:iso1} y el
	corolario \ref{lemma:iso2} encontramos que
	\begin{equation*}
		k = \vec{q}^T\vec{x} \leq \eta,
	\end{equation*}
	y entonces $(k, \vec{t})$ es factible. Como $\vec{x}$ fue arbitrario, se sigue que la solución
	del problema (\ref{formulation:multiple}) es una cota inferior del problema
	(\ref{formulation:lattice}). La demostración de que la solución de (\ref{formulation:lattice})
	es una cota inferior de (\ref{formulation:multiple}) es análoga.

	Finalmente, supongamos que $(k, \vec{t}) \in \Z^n$ es solución de (\ref{formulation:lattice}).
	Si existe $\hat{\vec{x}}$ factible para (\ref{formulation:multiple}) con utilidad
	$\vec{q}^T\hat{\vec{x}} = \hat{k}$ estrictamente mayor, entonces consideramos $(\hat{k},
	\hat{\vec{t}})$ tal que $\hat{\vec{x}} = \hat{k}\vec{\nu} + M\hat{\vec{t}}$. Este vector
	también es factible con utilidad $k < \hat{k} \leq \eta$, y entonces $(k, \vec{t})$ no era la
	solución de (\ref{formulation:lattice}). Obtenemos una contradicción.
\end{proof}

\begin{observation}
	El vector objetivo todavía es ortogonal a la restricción presupuestaria. No obstante, es más
	fácil de manejar en caso de usar cortes como en Ramificación y Acotamiento. Si $k^*$ no es
	entero en la solución al problema relajado, la única manera de ramificar es con el nuevo corte
	$k \leq \lfloor k^* \rfloor$, pues el otro corte $k \geq \lceil k^* \rceil$ generará un
	subproblema infactible. Evidentemente, en la sección de análisis de resultados haremos
	comparaciones de tiempo en los tiempos de terminación entre esta formulación y la original.
\end{observation}

La formulación del problema equivalente en el teorema anterior resulta ser más interesante. Podemos
desacoplar esta nueva formulación de manera que obtengamos un problema de maximización y otro de
factibilidad. Supongamos, sin pérdida de generalidad, que las entradas de $A$ y $\vec{b}$ son
enteras. Como los renglones de $A$ son linealmente independientes, de \cite{alex} sabemos que tiene
una única factorización de Hermite. Es decir, existe una matriz $U \in \Z^{n \times n}$ unimodular
que satisface $AU = [H, \vec{0}]$, donde $H \in \Z^{m \times m}$ es triangular inferior y no
singular.

Consideremos el subproblema de maximización
\begin{subequations}
	\label{subformulation:lattice}
	\begin{align}
		\max_{k \in \Z}
			& ~ k, \\
		\text{s.a.} \quad
		k &\leq \eta, \\
			A\tvec{y} &= kA\vec{\nu} - \vec{b},
	\end{align}
\end{subequations}
donde 
\begin{equation*}
	\tvec{y} \coloneq U \begin{pmatrix} \tvec{y}_m \\ \tvec{y}_{n-m} \end{pmatrix}
	= U_m\tvec{y}_m + U_{n-m}\tvec{y}_{n-m} \in \Z^n,
\end{equation*}
con $\tvec{y}_m \in \Z^m$ y $\tvec{y}_{n-m} \in \Z^{n-m}$. Así también, $U_m$ y
$U_{n-m}$ denotan las primeras $m$ columnas y últimas $n - m$ columnas de $U$, respectivamente.
Observemos que para toda $k \in \Z$ se cumple
\begin{equation}
	AU \begin{pmatrix} \inv{H}\left(kA\vec{\nu} - \vec{b}\right) \\ \tvec{y}_{n-m} \end{pmatrix}
	=
	[H, \vec{0}] \begin{pmatrix} \inv{H}\left(kA\vec{\nu} - \vec{b}\right) \\ \tvec{y}_{n-m} \end{pmatrix}
	= kA\vec{\nu} - \vec{b},
\end{equation}
lo cual sugiere definir $\tvec{y}_m \coloneq \inv{H}(\vec{b} - kA\vec{w})$. No obstante,
también debemos asegurarnos que este vector sea entero. Observemos que $\tvec{y}_{n-m}$ queda
libre, así que en realidad este subproblema tiene dimensión $m + 1$. Definimos el conjunto de
factibilidad
\begin{equation}
	\label{eq:feas-set}
	F \coloneq \lbrace k \in \Z \vcentcolon \inv{H}\left(kA\vec{\nu} - \vec{b}\right) \in \Z^m \rbrace
	\cap \lbrace k \in \Z \vcentcolon k \leq \eta \rbrace.
\end{equation}
\begin{observation}
	Para que $F$ sea no vacío, debe existir $k \in \Z$ tal que $\det(H) \mid (k\vec{a}_j^T
	\vec{\nu} - b_j)$ para todo $j \in \lbrace 1, \ldots, m \rbrace$, donde $\vec{a}^T_j$
	denota el $j$-ésimo vector renglón de $A$. Es decir, una condición suficiente y necesaria para
	la no vacuidad de $F$ es
	\begin{equation*}
		\det(H) \mid \gcd{k\vec{a}_1^T\vec{\nu} - b_1, \ldots, k\vec{a}_m^T\vec{\nu} - b_m}.
	\end{equation*}
	Ahora bien, $H$ es triangular inferior e invertible, por lo que $\det(H) \neq 0$ es el producto
	de los elementos $h_1, \ldots, h_m$ en su diagonal. Entonces $h_j \mid \det(H)$ para todo $j \in
	\lbrace 1, \ldots m \rbrace$ y una condición necesaria para la no vacuidad de $F$ es
	\begin{equation*}
		\lcm{h_1, \ldots, h_m} \mid \gcd{k\vec{a}_1^T\vec{\nu} - b_1, \ldots, k\vec{a}_m^T\vec{\nu} - b_m}.
	\end{equation*}
\end{observation}

Si $F$ es vacío, deducimos que este subproblema es infactible y por lo tanto
(\ref{formulation:lattice}) también lo es. Supongamos, pues, que $F \neq \emptyset$. No es difícil
observar que $F$ tiene un elemento maximal $k^*$ y que este elemento es la solución al subproblema
(\ref{subformulation:lattice}). Luego, dada esta solución $k^* \in \Z$, buscamos resolver el
subproblema de factibilidad
\begin{subequations}
	\label{subformulation:feasibility}
	\begin{align}
		M\vec{t} &= \tvec{y}, \\
		M\vec{t} &\geq -k^*\vec{\nu}.
	\end{align}
\end{subequations}
Observemos que tenemos un sistema de $n$ ecuaciones lineales con $2n - m - 1$ incógnitas, por lo que
tendremos que lidiar con $n - m - 1$ parámetros libres:
\begin{align}
	\label{eq:feasibility-eqs}
	M\vec{t} = \tvec{y} = U_m\tvec{y}_m + U_{n-m}\tvec{y}_{n-m}
   \iff [M, -U_{n-m}] \begin{pmatrix} \vec{t} \\ \tvec{y}_{n-m} \end{pmatrix} = U_m\tvec{y}_m.
\end{align}
Si consideramos ahora la forma normal de Smith de esta matriz por bloques, obtenemos dos matrices
unimodulares $S \in \Z^{n \times n}$ y $T \in \Z^{(2n - m - 1) \times (2n - m -1)}$ que satisfacen
\begin{equation*}
	S[M, -U_{n-m}]T = D \in \Z^{n \times (2n - m - 1)},
\end{equation*}
donde $D$ es una matriz diagonal cuyas $n$ primeras entradas son distintas de cero y las restantes
$n - m - 1$ son cero. Si multiplicamos $S$ por la izquierda en ambos lados de la ecuación
(\ref{eq:feasibility-eqs}), tenemos
\begin{equation*}
	D\inv{T}\begin{pmatrix} \vec{t} \\ \tvec{y}_{n-m} \end{pmatrix}
	= SU_m\tvec{y}_{m}.
\end{equation*}
Si $d_i$ no divide a $(SU_m\tvec{y}_{m})_i$ para alguna $i \in \lbrace 1, \ldots, n \rbrace$,
encontramos que la primera ecuación del subproblema (\ref{subformulation:feasibility}) no tiene
solución en los enteros, lo que implica que la elección de $k^*$ fue la incorrecta para asegurar
soluciones enteras a este subproblema. De ser este el caso, redefinimos $F \leftarrow F \setminus
\lbrace k^* \rbrace$. Si $F$ ahora es vacío, entonces (\ref{formulation:lattice}) es
infactible, de caso contrario escogemos el nuevo elemento de maximal de $F$ y repetimos el proceso.

Supongamos, pues que $d_i \mid (SU_m\tvec{y}_{m})_i$ para todo $i \in \lbrace 1, \ldots,
n\rbrace$, por lo que obtenemos $n$ soluciones enteras $\vec{r} \in \Z^n$ y $n - m - 1$ variables
libres $\vec{s} \in \Z^{n-m-1}$:
\begin{equation*}
	\inv{T}\begin{pmatrix} \vec{t} \\ \tvec{y}_{n-m} \end{pmatrix}
	=
	\begin{pmatrix} \vec{r} \\ \vec{s} \end{pmatrix}.
\end{equation*}
Por lo tanto, nuestro vector $\vec{t}$ es una función lineal de $\vec{s}$, es decir, $\vec{t} =
\vec{t}(\vec{s})$. Hasta este punto el proceso no ha sido complicado, pues nos hemos encargado de
resolver sistemas de ecuaciones lineales diofantinas. En términos del problema original
(\ref{formulation:multiple}), hemos encontrado los vectores $\vec{x}(\vec{s}) \coloneq
k^*\vec{\nu} + M\vec{t}(\vec{s})$ que maximizan la utilidad y que satisfacen todas las
restricciones excepto, posiblemente, las de no negatividad.

La dificultad entra en juego cuando queremos determinar el vector de variables libres $\vec{s} \in
\Z^{n-m-1}$ que hagan que $\vec{t}(\vec{s})$ satisfaga la desigualdad en el subproblema
(\ref{subformulation:feasibility}). Debilitando más esta condición, nos gustaría determinar si el
conjunto
\begin{equation*}
	\lbrace \vec{s} \in \Z^{n-m-1} \vcentcolon M\vec{t}(\vec{s}) \geq -k^*\vec{\nu} \rbrace
\end{equation*}
es vacío o no. En esta versión debilitada no nos interesa saber qué elementos contiene o tan
siquiera cuántos elementos contiene. Es sabido que los programas enteros tales como
(\ref{formulation:multiple}) o (\ref{formulation:lattice}) son problemas difíciles de resolver, en
el sentido de que no es conocido si se pueden resolver en tiempo polinomial. A lo largo de este
capítulo, no obstante, hemos resuelto todos los problemas en tiempo polinomial\footnote{En
	\cite{alex} se muestra que calcular el máximo común divisor, resolver ecuaciones lineales
	diofantinas, y calcular las factorizaciones tanto de Hermite como de Smith son operaciones
	acotadas por tiempo polinomial.}.
La única deducción posible, entonces, es que el problema de determinar las variables $\vec{s}$, o
bien de determinar cuántas hay, o bien de determinar su existencia, son todos problemas difíciles de
resolver.

A pesar de lo anterior, hay dos casos donde la dificultad se reduce drásticamente. El caso menos
interesante es cuando $m = n - 1$, de manera que no hay parámetros libres. Esto se debe a que el
politopo factible resultante es un semirrayo o un segmento de línea. Al momento de escoger la
$k^*$-ésima capa entera, estamos agregando la ecuación $k^* = k$, con lo que obtenemos un sistema
lineal entero de $n$ ecuaciones con $n$ incógnitas, y entonces la solución es única. Basta entonces
verificar que este único vector $\vec{t}$ satisface la desigualdad en el subproblema
(\ref{subformulation:feasibility}). El caso un poco más interesante se obtiene cuando $m = n - 2$.
De esta manera obtenemos un solo parámetro, con lo que podemos determinar rápidamente la existencia
o inexistencia de un intervalo de factibilidad.

\begin{example}
	\label{ex:two-var}
	Consideremos el problema con $n = 2$ variables y $m = 1$ restricciones
	\begin{align*}
		\max
			~& x - y, \\
		\text{s.a.} \quad
			& x - y \leq 12, \\
			& 3x + 5y = 25, \\
			& x, y \geq 0.
	\end{align*}
	En este caso tenemos $A = (3, 5), \vec{b} = 25$, y también $\vec{q} = (1, -1)^T$, al igual que
	$\eta = 12$. De (\ref{eq:vec-omega}) y (\ref{eq:mat-T}) obtenemos
	\begin{equation*}
		\vec{\nu} = \begin{pmatrix} 1 \\ 0 \end{pmatrix},
		M = \begin{pmatrix} -1 \\ -1 \end{pmatrix}.
	\end{equation*}
	De la forma normal de Hermite de $A$ tenemos
	\begin{equation*}
		H = 1, U = \begin{pmatrix} 2 & -5 \\ -1 & 3 \end{pmatrix},
	\end{equation*}
	y de la forma normal de Smith de $[M, -U_m]$,
	\begin{equation*}
		S = \begin{pmatrix} -1 & 0 \\ 1 & -1 \end{pmatrix},
		D = \begin{pmatrix} 1 & 0 \\ 0 & 8 \end{pmatrix},
		T = \begin{pmatrix} 1 & 5 \\ 0 & 1 \end{pmatrix}. 
	\end{equation*}

	Como $H = 1$, se sigue que $\inv{H} (\vec{b} - kA\vec{\nu}) = 25 - 3k$ es entero para todo $k
	\in \Z$. Así, el conjunto factible $F$ (c.f. \ref{eq:feas-set}) está dado por
	\begin{equation*}
		F = \Z \cap \lbrace k \in \Z \vcentcolon k \leq 12 \rbrace
		= \lbrace k \in \Z \vcentcolon k \leq \eta = 12 \rbrace.
	\end{equation*}
	Entonces escogemos $k^* = 12$ por ser el elemento maximal de $F$. Así, encontramos
	\begin{equation*}
		SU_m\tvec{y}_m = SU_m \left(\inv{H} (\vec{b} - k^*A\vec{\nu})\right)
		= \begin{pmatrix} 22 \\ 33 \end{pmatrix}
	\end{equation*}
	Observemos que la segunda entrada de $SU_m\tvec{y}_m$ no es divisible por $D_{22} = 8$.
	Así, el subproblema (\ref{subformulation:feasibility}) no es factible para la elección de $k^*$
	previa. Escogemos el segundo elemento de $F$ más grande, con lo que tenemos $k^* \leftarrow 11$.
	En este caso obtenemos $SU_m\tvec{y}_m = (-16, -24)^T$, por lo que sí hay soluciones
	enteras. Luego, se debe satisfacer,
	\begin{equation*}
		\inv{T} \begin{pmatrix} \vec{t} \\ \tvec{y}_{n-m} \end{pmatrix} =
		\begin{pmatrix} 16 \\ 24 \end{pmatrix},
	\end{equation*}
	de donde se sigue que $(\vec{t}, \tvec{y}_{n-m}) = (1, 3)$. Verificamos factibilidad:
	\begin{equation*}
		M\vec{t} + k^*\vec{\nu}
		= 1 \begin{pmatrix} -1 \\ -1 \end{pmatrix} + 11 \begin{pmatrix} 1 \\ 0 \end{pmatrix}
		= \begin{pmatrix} 10 \\ -1 \end{pmatrix} \not \geq \vec{0}.
	\end{equation*}
	Ahora la elección de $k^*$ dio un punto entero pero con una entrada negativa. Seguimos este
	procedimiento hasta llegar a $k^* \leftarrow 3$. En este caso obtenemos $(\vec{t},
	\tvec{y}_{n-m}) = (-2, -6)^T$, de donde
	\begin{equation*}
		M\vec{t} + k^*\vec{\nu}
		= -2 \begin{pmatrix} -1 \\ -1 \end{pmatrix} + 3 \begin{pmatrix} 1 \\ 0 \end{pmatrix}
		= \begin{pmatrix} 5 \\ 2 \end{pmatrix} \geq \vec{0}.
	\end{equation*}
	Concluimos diciendo que $(k^*, \vec{t}) \coloneq (3, -2)$ es el óptimo del programa
	(\ref{formulation:lattice}) y entonces $(x, y) = (5, 2)$ es el óptimo de
	(\ref{formulation:multiple}).
\end{example}
\begin{example}
	Ahora consideremos el problema con $n = 3$ variables y $m = 1$ restricciones
	\begin{align*}
		\max
			~& x - y + 2z, \\
		\text{s.a.} \quad
			& x - y  + 2z \leq 10 \\
			& 3x + 4y - z = 15 \\
			& x, y, z \geq 0.
	\end{align*}
	En este caso tenemos $A = (3, 4, -1), \vec{b} = 15$, y también $\vec{q} = (1, -1, 2)^T$, al igual que
	$\eta = 10$. De (\ref{eq:vec-omega}) y (\ref{eq:mat-T}) obtenemos
	\begin{equation*}
		\vec{\nu} = \begin{pmatrix} 1 \\ 0 \\ 0 \end{pmatrix},
		M = \begin{pmatrix} 1 & 0 \\ -1 & 2 \\ -1 & 1 \end{pmatrix}.
	\end{equation*}
	De la forma normal de Hermite de $A$ tenemos
	\begin{equation*}
		H = 1, U = \begin{pmatrix} 0 & 0 & 1 \\ 0 & 1 & 0 \\ -1 & 4 & 3 \end{pmatrix},
	\end{equation*}
	y de la forma normal de Smith de $[M, -U_m]$,
	\begin{equation*}
		S = \begin{pmatrix}
			1 & 0 & 0 \\
			-1 & -1 & 0 \\
			3 & 4 & -1
		\end{pmatrix},
		D = \begin{pmatrix}
			1 & 0 & 0 & 0 \\
			0 & 1 & 0 & 0 \\
			0 & 0 & 7 & 0
		\end{pmatrix},
		T = \begin{pmatrix}
			1 & 0 & 0 & 1 \\
			0 & 0 & 1 & 0 \\
			0 & 1 & 2 & -1 \\
			0 & 0 & 0 & 1
		\end{pmatrix}.
	\end{equation*}
	Nuevamente, observemos que $H = 1$ y por lo tanto $F = \lbrace k \in \Z \vcentcolon k \leq 10
	\rbrace$. Seguimos exactamente el mismo procedimiento que en el Ejemplo \ref{ex:two-var} hasta
	llegar a $k^* \leftarrow 5$. Encontramos que se satisface
	\begin{equation*}
		\inv{T} \begin{pmatrix} \vec{t} \\ \tvec{y}_{n-m} \end{pmatrix}
		=
		\begin{pmatrix} 0 \\ 0 \\ 0 \\ s \end{pmatrix}
		\implies
		\begin{pmatrix} \vec{t} \\ \tvec{y}_{n-m} \end{pmatrix}
		=
		s \begin{pmatrix} 1 \\ 0 \\ -1 \\ 1 \end{pmatrix},
	\end{equation*}
	donde $s \in \Z$ es la única variable libre. En este caso podemos determinar rápidamente un
	intervalo de existencia: tenemos $M\vec{t} \geq -k^*\vec{\nu}$ si y solo si
	\begin{equation*}
		s\begin{pmatrix} 1 \\ 0 \\ -1 \end{pmatrix} \geq
		\begin{pmatrix} -5 \\ 0 \\ 0 \end{pmatrix},
	\end{equation*}
	de donde se sigue inmediatamente que $s \in \lbrace -5, -4, \ldots, 0 \rbrace$. Sustituyendo en
	$\vec{t}$ y transformando a $\vec{x}$, encontramos que
	\begin{equation*}
		\left\lbrace
			\begin{pmatrix} 0 \\ 5 \\ 5 \end{pmatrix},
			\begin{pmatrix} 1 \\ 4 \\ 4 \end{pmatrix},
			\begin{pmatrix} 2 \\ 3 \\ 3 \end{pmatrix},
			\begin{pmatrix} 3 \\ 2 \\ 2 \end{pmatrix},
			\begin{pmatrix} 4 \\ 1 \\ 1 \end{pmatrix},
			\begin{pmatrix} 5 \\ 0 \\ 0 \end{pmatrix}
		\right\rbrace
	\end{equation*}
	son las seis soluciones del problema. Todas alcanzan un nivel de utilidad $k^* = 5$.
\end{example}

Si el programa (\ref{formulation:multiple}) es factible, entonces el programa
(\ref{formulation:lattice}) también lo es. A partir de nuestro procedimiento, eventualmente
encontraremos un par $(k^*, \vec{t}^*)$ que resuelva tanto el subproblema de maximización
(\ref{subformulation:lattice}) como el de factibilidad (\ref{subformulation:feasibility}).

Ahora bien, son dos las maneras en las que nuestro problema sea infactible. Puede que nuestro
conjunto de factibilidad $F$ sea vacío y por lo tanto el sistema de ecuaciones lineales
(\ref{formulation:multiple:constraints}) sea inconsistente. O bien, puede ser que $F$ tenga
cardinalidad infinita pero para ninguno de sus elementos se satisfaga el subproblema de
factibilidad.

% TODO: mostrar una imagen.
Esto último puede ocurrir cuando el sistema de ecuaciones siempre tiene solución pero todas ellas
son negativas. En efecto, si en el Ejemplo \ref{ex:two-var} reemplazamos el lado derecho de la
igualdad $\vec{b} = 25$ por $\vec{b} = -4$, nos encontramos en aquella situación.

En conclusión, para asegurar terminación en tiempo finito, cualquier algoritmo basado en este método debe
asegurarse primero que el conjunto de factibilidad $F$ tiene un número finito de puntos. Este caso
lo estudiamos en la siguiente sección.

\subsection{Eliminando la restricción presupuestaria}
\noindent
Consideremos ahora el problema
\begin{subequations}
	\label{formulation:last}
	\begin{align}
		\max_{\vec{x} \in \Z^n} \quad
			& \vec{q}^T\vec{x}, \label{formulation:last:objective} \\
		\text{s.a.} \quad
			& A\vec{x} = \vec{b}, \label{formulation:last:constraints} \\
			& \vec{x} \geq \vec{0}, \nonumber
	\end{align}
\end{subequations}
Evidentemente, si su programa relajado tiene un valor objetivo $u^*$ finito, podemos agregar la
restricción presupuestaria $\vec{q}^T\vec{x} \leq u^*$ a este problema de manera válida. Entonces
podemos suponer sin pérdida de generalidad que este programa es equivalente a
(\ref{formulation:multiple}) siempre que su valor objetivo sea finito. Consecuentemente, podemos
utilizar las herramientas desarrolladas en la sección pasada para resolver este problema entero.

Es más, supongamos que el politopo asociado al problema relajado es acotado y no vacío. Entonces
tanto el problema de maximización como de minimización tienen valores objetivos finitos. Llamemos a
estos valores $\ell^*$ y $u^*$, respectivamente. Ahora la restricción
\begin{equation*}
	\ell^* \leq \vec{q}^T\vec{x} \leq u^*
\end{equation*}
es válida para el problema (\ref{formulation:last}). De la misma manera que $\eta$ parametriza la
primera capa entera que satisface el presupuesto, podemos definir análogamente la última capa que
satisface el presupuesto. Usando el mismo razonamiento que en el lema \ref{phase-1:lemma:eta},
encontramos que esta capa está parametrizada por $\tau \coloneq \lceil \ell^*/m \rceil$ si $m$ es
positiva. Así pues, al definir nuestro conjunto de factibilidad $F$ como
\begin{equation*}
	F \coloneq \lbrace k \in \Z \vcentcolon \inv{H}\left(kA\vec{\nu} - \vec{b}\right) \in \Z^m \rbrace
	\cap \lbrace k \in \Z \vcentcolon \tau \leq k \leq \eta \rbrace,
\end{equation*}
podemos replicar las mismas técnicas que en la sección pasada. Pero además, $F$ es un conjunto
finito y por lo tanto tenemos terminación en tiempo finito para este caso. Es decir, cualquier
algoritmo basado en los métodos desarrollados en la sección pasada podrá decidir en tiempo finito si
el problema es factible o no. En caso de que sí lo sea entonces terminará con la solución óptima.

Existen varios algoritmos para resolver el problema relajado de (\ref{formulation:last}) en su
versión general. Es cierto que el método del simplex es el más utilizado, a pesar de tener una
complejidad algorítmica no acotada polinomialmente. También es cierto que existen métodos
polinomiales para resolver este problema, tales como el método elipsoidal o el algoritmo de
Karmarkar. Pero más interesante es el hecho de que ya existen cotas superiores para ciertas
instancias de estos problemas, por ejemplo, en el caso del Problema de la Mochila, \cite{martello}
provee una cota superior razonable, y ciertamente el valor de 0 es una cota inferior justa. Mucho
hablaremos de este problema en el Capítulo 3. No obstante, el autor considera prudente dedicar el
siguiente capítulo para el caso infinito.

\chapter{Conclusiones}
\noindent
Durante el trabajo de tesis se obtuvieron los siguientes resultados teóricos, los cuales son de
carácter original y fueron desarrollados por el autor.

En los teoremas \ref{theory:th:infeasibility} y \ref{theory:th:feasibility} se simplifica y
estructura el análisis del problema \eqref{theory:formulation}. A partir de ellos podemos, en primer
lugar, deshacernos automáticamente de instancias infactibles y, en segundo lugar, de separar en
casos las instancias factibles.

En la proposición \ref{prop:xint} se muestra una relación lineal entre el vector solución $\vec{x} \in
\Z^n$ de la ecuación lineal diofantina $\vec{q}^T\vec{x} = k$ con un vector de variables libres
$\vec{t} \in \Z^{n-1}$. Esta proposición da entrada para analizar propiedades de la matriz
$M \in \Z^{n \times (n - 1)}$ y del vector $\vec{\nu} \in \Z^n$ definidos en \eqref{eq:vec-omega} y
\eqref{eq:mat-T}, respectivamente. En el teorema \ref{th:lattice} aprovechamos estas propiedades
para descomponer la red $\Z^n$ como la suma directa de dos subredes $\Lambda_p$ y $\Lambda_h$ (ver
\eqref{eq:dec}) que contienen, respectivamente, soluciones particulares y soluciones homogéneas de
la ecuación lineal diofantina $\vec{q}^T\vec{x} = k$. En el teorema \ref{th:isoclass} se sugiere que
esta descomposición no es exclusivamente generada por $\vec{q}$, sino que lo es por su órbita. Esto
último permite que consideremos una clasificación de programas lineales enteros a partir de los
vectores coprimos asociados a un vector esencialmente entero $\vec{p}$.

En los teoremas \ref{th:alg:inf} y \ref{th:fin:dioph:correct}, así como sus respectivos algoritmos
\ref{algo:inf:ext} y \ref{algo:fin:dioph} mostramos cómo las ecuaciones lineales diofantinas son
esenciales para resolver instancias de \eqref{theory:formulation}. Además, en el caso infinito,
tenemos que la complejidad para resolver este tipo de instancias es polinomialmente acotada. La
dificultad radica en instancias cuando todas las entradas del vector coprimo $\vec{q}$ son
estrictamente positivas. A pesar de ello, en el teorema \ref{th:intnonneg2} se indica que esta complejidad
no depende del número de ecuaciones lineales diofantinas a resolver (pues eventualmente es
suficiente con resolver una), sino que radica en cómo se resuelven estas ecuaciones.

En el teorema \ref{th:intnonneg1} se muestra que el lado derecho de \eqref{eq:eta-limit} es una cota
superior para el número de Frobenius $F$. En realidad, lo que obtenemos es una familia de cotas
superiores, pues $F$ está en función del número de enteros coprimos $q_1, \ldots, q_n$, así como de
sus valores respectivos, es decir, $F = F(q_1, \ldots, q_n; n)$, donde $n \in \Z_{> 0}$. Puesto que
podemos calcular la matriz $M$ definida en \eqref{eq:mat-T} en tiempo polinomial, entonces podemos
calcular en tiempo polinomial la cota superior \eqref{eq:eta-limit} de $F(q_1, \ldots, q_n; n)$ para
$n \in \Z_{> 0}$ fija y para cualesquiera $q_1, \ldots, q_n$.

Finalmente, en el teorema \ref{th:multeq} se demuestra que la formulación \eqref{formulation:lattice} es
equivalente al problema \eqref{formulation:multiple}.
% Si comparamos las respectivas restricciones
% \eqref{lattice:c-layer} y \eqref{formulation:multiple:constraint:budget}, encontramos que ambas son
% ortogonales al vector objetivo, no obstante, la primera es mucho más fácil de controlar que la
% segunda, pues esta ortogonalidad ocurre a lo largo de un solo eje y no de muchos.
En el teorema \ref{th:bbspeed} se muestra que el algoritmo de Ramificación y Acotamiento podría
beneficiarse al utilizar esta formulación equivalente, pues podemos priorizar ramificaciones en $k$
para deshacernos rápidamente de subproblemas infactibles.

Ahora presentamos problemas abiertos que fueron descubiertos a lo largo de esta tesis y que podrían
ser de interés para futuras líneas de investigación:
\begin{enumerate}
	\item En el ejemplo \ref{ex:inf} mostramos para una instancia particular que Ramificación y
		Acotamiento genera una sucesión de subproblemas trasladados. Mostrar o refutar que existe
		una clase instancias de \eqref{theory:formulation} que contienen subproblemas homotéticos.
		En caso afirmativo, mostrar o refutar que este conjunto de subproblemas es infinito cuando
		el vector coprimo $\vec{q}$ tiene una entrada negativa.
	\item Generalizar el lema \ref{lemma:layer-dist} para $m$ racional.
	\item Para encontrar la cota inferior en el lado derecho de \eqref{eq:eta-limit} tuvimos que calcular el
		radio de la bola inscrita en el símplice $\sigma$ y centrada en el baricentro $\est{\sigma}$.
		Es cierto que podemos obtener distintas cotas inferiores si centramos la bola inscrita en
		$\sigma$ en distintos puntos de este símplice. Mostrar o refutar que el baricentro
		$\est{\sigma}$ genera la menor de estas cotas.
	\item Realizar un análisis detallado de la cota superior dada en el lado derecho de
		\eqref{eq:eta-limit} para el número de Frobenius $F(q_1, \ldots, q_n ; n)$ y compararla con
		las cotas establecidas en el capítulo 3 de \cite{frob}.
	\item Construir un algoritmo que resuelva uno de los tres problemas descritos en la conclusión del capítulo
		4.
	\item Realizar experimentos numéricos que comparen los tiempos de terminación de Ramificación y
		Acotamiento al utilizar la formulación \eqref{formulation:multiple} contra su forma
		equivalente \eqref{formulation:lattice}.
\end{enumerate}

{

  \setlength{\epigraphwidth}{0.40\textwidth}
	\begin{flushright}
		\epigraph{Would it save you a lot of time if I just gave up and went mad now?}{\textit{Douglas
		Adams,} \emph{The Hitchhiker's Guide to the Galaxy}}
	\end{flushright}
}

\appendix

\chapter{Algoritmo de Ramificación y Acotamiento}
\label{app:bb}

\begin{algorithm}[ht]
	\LinesNumbered
	\KwData{
		Problema de maximización lineal $S_0$.
		}
	\KwResult{
		Solución óptima entera $\vec{x}^*$ y valor óptimo $\optilp{z}$.
	}
	\Begin{
		$\mathcal{L} \leftarrow \braces{S_0}$\;
		$\vec{x}^* \leftarrow -\vec{\infty}$\;
		$\optilp{z} \leftarrow -\infty$\;
		\While{$\mathcal{L} \neq \emptyset$}{
			elegir de $\mathcal{L}$ subproblema $S_i$\; \nllabel{p1c9:alg:BB_loop}
			obtener de $S_i$ valor óptimo $z^*_i$ y solución óptima $\vec{x}^i$\;
			$\mathcal{L} \leftarrow \mathcal{L} \setminus \braces{S_i}$\;
			\If{$S_i = \emptyset$ o $z^*_i \leq \optilp{z}$}{
				ir al paso \ref{p1c9:alg:BB_loop}\;
			}
			\If{$\vec{x}^i \in \Z^n$}{
				$\vec{x}^* \leftarrow \vec{x}^i$\;
				$\optilp{z} \leftarrow z^*_i$\;
				ir al paso \ref{p1c9:alg:BB_loop}\;
			}
		    elegir $x^i_j \not \in \Z$ y generar subproblemas $S_{i0}$ y $S_{i1}$ con
			regiones factibles
			$S_{i} \cup \braces{x_j \leq \floor{x^i_j}}$ y
			$S_{i} \cup \braces{x_j \geq \ceil{x^i_j}}$, respectivamente\;
			$\mathcal{L} \leftarrow \mathcal{L} \cup \braces{S_{i1}, S_{i2}}$.
		}
		\Return{$(\vec{x}^*, \optilp{z})$}
	}
	\caption{Ramificación y Acotamiento (adaptado de \cite{fabs})} \label{p1c9:alg:BB}
	\label{algo:bb}
\end{algorithm}

\chapter{Algoritmo extendido del caso infinito}
\label{app:inf:ext}

% TODO: implement algorithm
\begin{algorithm}[ht]
	\LinesNumbered
	\SetKwProg{Fn}{Fn}{\string:}{}
	\SetKwFunction{Bezout}{Bezout}
	\SetKwFunction{NonNegativeIntSol}{NonNegativeIntSol}
	\Fn{\NonNegativeIntSol{$\vec{q}$, $\eta$}}{
		\KwData{\\
			Vector coprimo $\vec{q}$ tal que $q_i \neq 0$ para todo $i \in \braces{1, \ldots, n}$ y
			$q_{n-1} < 0 < q_n$.\\
			Lado derecho $\eta$.
			}
		\KwResult{\\
			Solución entera no negativa $\vec{x}$ a la ecuación lineal diofantina $\vec{q}^T\vec{x} =
			\eta$.
		}
		\Begin{
			$\vec{x} \leftarrow \vec{0}$\;
			$\omega_1 \leftarrow \eta$\;
			\For{$i \leftarrow 1$ \KwTo $n - 2$}{
				$g_{i+1} \leftarrow \gcd{q_{i+1}, \ldots, q_n}$\;
				$x_i', \omega_{i+1}' \leftarrow$ \Bezout{$q_i$, $g_{i+1}$}\;
				$t_i \leftarrow \ceil{-\omega_i x_i' / g_{i+1}}$\;
				$x_i \leftarrow \omega_i x_i' + g_{i+1}t_i$\;
				$\omega_{i+1} \leftarrow \omega_i \omega_{i+1}' - q_i t_i$\;
				\For{$j \leftarrow i$ \KwTo $n - 2$}{
					$q_{j+1} \leftarrow q_{j+1}/g_{i+1}$\;
				}
			}

			$x_{n-1}', x_n' \leftarrow$ \Bezout{$q_{n-1}$, $q_n$}\;
			$b_1 \leftarrow -\omega_{n-1} x_{n-1}' / q_n$\;
			$b_2 \leftarrow -\omega_{n-1} x_n' / q_{n-1}$\;
			$t_{n-1} \leftarrow \ceil{\max\lbrace b_1, b_2\rbrace}$\;
			$x_{n-1} \leftarrow \omega_{n-1}x_{n-1}' + q_nt_{n-1}$\;
			$x_{n} \leftarrow \omega_{n-1}x_{n}' - q_{n-1}t_{n-1}$\;

			\Return{$\vec{x}$}
		}
	}
	\caption{Algoritmo para obtener soluciones enteras no negativas a la ecuación lineal diofantina
	$\vec{q}^T\vec{x} = \eta$, dados ciertos supuestos sobre $\vec{q}$.}
	\label{algo:inf:ext}
\end{algorithm}


\clearpage
\bibliographystyle{alpha}
\bibliography{refs}
\end{document}
