\documentclass[11pt]{article}

\usepackage[margin=1in]{geometry}
\usepackage{parskip}
\usepackage{amsmath, amssymb}
\usepackage{enumitem}
\usepackage{hyperref}
\usepackage{mathtools}
\usepackage{graphicx}
\usepackage[spanish]{babel}
\usepackage{amsthm}

% \graphicspath{{../figs}}

\title{Reporte}
\author{Iñaki Liendo}
% \date{\today}

\newtheorem{definition}{Definición}
\newtheorem{theorem}{Teorema}
\newtheorem{lemma}{Lema}
\newtheorem{example}{Ejemplo}

\DeclareMathOperator{\lcm}{lcm}
% \renewcommand{\gcd}{m.c.m}

\begin{document}

\maketitle

\begin{abstract}
\end{abstract}

\section*{Introducción}
\section*{Preliminares}  % MAYBE
En términos generales, los prerrequisitos necesarios para un entendimiento suficiente de los
resultados expuestos en esta tesis son de álgebra, en particular de divisibilidad y ecuaciones
lineales enteras o diofantinas; de álgebra lineal y álgebra lineal numérica; y, evidentemente,
de programación lineal. Dividimos, pues, los preliminares en tres subsecciones correspondientes que
servirán para refrescar la memoria al lector, pero también para que seamos capaces de hacer
referencia a resultados establecidos en la medida en que desarrollamos los nuestros.

\subsection*{Álgebra}
Nos enfocamos principalmente en obtener condiciones necesarias y suficientes para encontrar
soluciones enteras a problemas del tipo
\begin{equation}
	\label{prerreq:algebra:eq:dioph}
	ax + by = c
\end{equation}
con $a, b$ y $c$ enteros. Este tipo de ecuación, así como sus variantes con más de dos variables,
recibe el nombre ecuación lineal entera o, también, de ecuación lineal diofantina.

\begin{theorem}
	\label{prerreq:th:dioph:existence}
	Sean $a, b \in \mathbb{Z}$ no ambos iguales a cero. La ecuación \ref{prerreq:algebra:eq:dioph}
	tiene solución en los enteros si y solo si $\gcd\lbrace a, b \rbrace$ divide a $c$.
\end{theorem}

\begin{theorem}
	\label{prerreq:th:dioph:gen}
	Sean $a, b, c \in \mathbb{Z}$ tales que $a, b$ no son ambos iguales a cero, y sea $(x_0, y_0)$ una
	solución particular de la ecuación \ref{prerreq:algebra:eq:dioph}. Entonces todas las soluciones
	de la ecuación están dadas por
	\begin{equation*}
		\begin{cases}
			x = x_0 + \frac{b}{d} \cdot t, \\
			y = y_0 + \frac{a}{d} \cdot t,
		\end{cases}
	\end{equation*}
	donde $d \coloneq \gcd\lbrace a, b \rbrace$ y $t \in \mathbb{Z}$.
\end{theorem}

\subsection*{Álgebra lineal}
\subsection*{Programación lineal}

\section*{Metodología}
En esta sección se desarolla el algoritmo para resolver prácticamente cualquier tipo de problemas de
programación lineal entera (c.f. Definición TODO: mostrar definición). Se divide en tres
subsecciones que desarollan el modelo de forma incremental.

En primer lugar, consideraremos el caso cuando las únicas dos restricciones son de no-negatividad
($x \geq 0$) y una presupuestaria ($p^Tx \leq u$ para algún escalar $u$). A partir de ello,
generaremos una sucesión de ecuaciones lineales enteras cuya solución provee el óptimo para el
problema.

En segundo lugar, nos deshacemos de la restricción de no-negatividad y agregamos $m$ restricciones
de la forma $Ax \leq b$ además de la presupuestaria. Este es el parteaguas donde el algoritmo toma
relevancia, pero donde también aumenta en complejidad y supone ciertas dificultades con la
estabilidad numérica. Discutiremos en extensión posibles modificaciones y/o direcciones que puedan
mejorar significativamente la estabilidad de nuestro método.

En tercer lugar, eliminamos la restricción presupuestaria y, por lo tanto, nuestro algoritmo será
capaz de resolver problemas lineales enteros en su forma general. Ciertamente esta subsección es la
más corta, pues lo único que hacemos es agregar implícitamente una restricción presupuestaria válida
resolviendo el problema lineal relajado. Es de esta manera que podremos hacer uso de los resultados
obtenidos en la segunda fase.

Ahora bien, para que los siguientes resultados sean válidos, debemos agregar una condición a la
clase de vectores objetivo $p \in \mathbb{R}^n$ de tal manera que cumplan con la siguiente
definición:
\begin{definition}
	Sea $v \in \mathbb{R}^n$ un vector. Decimos que $v$ es esencialmente entero si existe un vector
	$w \in \mathbb{Z}^n$ y un escalar $k \in \mathbb{R}$ tal que $v = kw$. De otra forma, decimos
	que $v$ es esencialmente irracional.
\end{definition}
En otras palabras, decimos que $v \in \mathbb{R}^n$ es esencialmente entero si es un múltiplo real
de un vector entero. Por las siguientes razones descartamos a los vectores esencialmente
irracionales de nuestro análisis: en primer lugar, porque la motivación de este trabajo es la
asignación de recursos que tienen un costo asociado, el cual es necesariamente racional; en segundo
lugar, porque todo número representable en cualquier sistema de aritmética finita es necesariamente
racional. Independientemente del caso, solamente son de nuestro interés los vectores racionales,
pero no es difícil ver que todo vector $v$ en $\mathbb{Q}^n$ es esencialmente entero.

\subsection*{Primera fase}
A lo largo de esta subsección hacemos mucho uso de resultados cubiertos en un curso básico de
álgebra. No obstante, si el lector no se siente familiarizado con el tema o si desea refrescar su
memoria, es recomendable que lea la sección de preliminares.

Sean $p \in \mathbb{R}^n$ un vector no negativo esencialmente entero, $u \in \mathbb{R}_{\geq 0}$ un
escalar y consideremos el problema lineal entero
\begin{align}
	\text{máx.} ~& p^Tx \label{phase-1:ip:obj} \\
	\text{s. a.} ~& p^Tx \leq u \label{phase-1:ip:prep} \\
				 & x \in \mathbb{Z}_{\geq 0}^n \label{phase-1:ip:non-negative}.
\end{align}
La idea para resolver este tipo de problemas es la siguiente: cada escalar $t \in \mathbb{R}$ define
un hiperplano afino
\begin{equation}
	\label{phase-1:hyper}
	H_{p, t} \coloneq \ker\left( x \mapsto p^Tx \right) + t \cdot p
\end{equation}
donde se cumple que todo punto $x \in H_{p, t}$ tiene un mismo nivel de utilidad $p^Tx$. Si logramos
encontrar puntos factibles para el problema \ref{phase-1:ip:obj} entonces obtendremos una
cota inferior para el valor objetivo. Además, si somos capaces de determinar un escalar $t \leq u$
cuyo hiperplano afino asociado $H_{p, t}$ es el último\footnote{Con esto nos referimos a que no
existe un escalar $t < t' \leq u$ tal que $H_{p, t'}$ contiene puntos factibles.} en contener puntos
factibles, entonces la colección de esos puntos conforman la solución del problema.

\begin{definition}
	Sean $v \in \mathbb{R}^n$ un vector y $t \in \mathbb{R}$ un escalar. Decimos que su hiperplano
	afino asociado $H_{v, t}$ (c. f. \ref{phase-1:hyper}) es una capa entera si contiene al menos un
	punto entero.
\end{definition}

\begin{lemma}
	Sea $v \in \mathbb{R}^n$ un vector distinto de cero y sea $x \in \mathbb{R}^n$ un punto.
	Entonces $x \in H_{v, t_x}$, donde $t_x \coloneq \frac{v^Tx}{||v_2^2||}$.
\end{lemma}
\begin{proof}
	Ver \cite{sip}.
\end{proof}

\begin{definition}
	Sea $v \in \mathbb{R}^n$ un vector esencialmente entero y sea $(w_1, \ldots, w_n) \in
	\mathbb{Z}^n$ un múltiplo entero de $v$. Decimos entonces que el vector
	\begin{equation*}
		\frac{1}{\gcd\lbrace w_1, \ldots, w_n\rbrace} \cdot (w_1, \ldots, w_n)
	\end{equation*}
	es un múltiplo coprimo de $v$.
\end{definition}

Para asegurar unicidad, forzamos a que la primera entrada del múltiplo coprimo de $v$ sea positivo.
Por el Corolario (TODO: mostrar corolario)  tenemos que el múltiplo coprimo de un vector
esencialmente entero es, en efecto, un vector coprimo. Cualquier vector coprimo define una familia
de capas enteras y, sorprendentemente, esa familia contiene a todos los puntos enteros en el
espacio, como lo indica el siguiente teorema.
\begin{theorem}
	\label{phase-1:th:family}
	Sea $v \in \mathbb{R}^n$ un vector esencialmente entero distinto de cero y sea $w$ su múltiplo
	coprimo. Entonces la familia de capas enteras $\left\lbrace H_{w, k||w||_2^{-2}} \vcentcolon k
	\in \mathbb{Z} \right\rbrace$ cubre a $\mathbb{Z}^n$.
\end{theorem}
\begin{proof}
	Ver \cite{sip}.
\end{proof}

\begin{lemma}
	\label{phase-1:num-layers}
	Sea $p \in \mathbb{R}^n$ un vector esencialmente entero y sea $u \in
	\mathbb{R}_{\geq 0}$ un escalar. Entonces el número de capas enteras entre 0 y u está
	determinado por
	\begin{equation}
		\label{eq:num_layers}
		\eta \coloneq
		\left\lfloor
			u \cdot \frac{\lcm\lbrace p_1, \ldots, p_n \rbrace}{\gcd\lbrace p_1, \ldots, p_n \rbrace}
		\right\rfloor.
	\end{equation}
\end{lemma}
\begin{proof}
	% TODO
\end{proof}

Ahora bien, somos capaces de caracterizar una cubierta de $\mathbb{Z}^n$ a partir de capas enteras
asociadas a nuestro vector esencialmente entero $p$. Además, debido al lema anterior, obtenemos una
enumeración finita que nos permite analizar si la $k$-ésima capa entera contiene puntos factibles
para el problema \ref{phase-1:ip:obj}. Observemos que si $k \in \lbrace 0, \ldots \eta \rbrace$,
entonces la restricción \ref{phase-1:ip:prep} se satisface automáticamente y, por lo tanto, debemos
exigir solamente no-negatividad.

Buscamos resolver la ecuación entera
\begin{equation}
	\label{eq:dioph:complete}
	p_1x_1 + \cdots + p_nx_n = k.
\end{equation}

\begin{lemma}
	\label{phase-1:lemma:invariance}
	Sea $v \in \mathbb{R}^n$ un vector y sea $t \in \mathbb{R}$ un escalar. Sea $H_{v, t}$ su
	hiperplano afino asociado (c. f. \ref{phase-1:hyper}). Entonces $H_{v, t} = H_{r \cdot v, t}$
	para todo escalar $r \in \mathbb{R}$.
\end{lemma}
\begin{proof}
	% TODO (?)
\end{proof}

Sea $q$ el múltiplo coprimo de $p$. Debido al lema anterior, resolver la ecuación
(\ref{eq:dioph:complete}) es equivalente a resolver
\begin{equation}
	\label{eq:dioph:complete:coprime}
	q_1x_1 + \cdots q_nx_n = k.
\end{equation}

Definamos $g_{n-1} \coloneq \gcd\lbrace q_1, \ldots, q_{n-1}\rbrace$. Así, obtenemos la
ecuación equivalente
\begin{equation*}
	g_{n-1} \left(
		\frac{q_1}{g_{n-1}}x_1 + \cdots + \frac{q_{n-1}}{g_{n-1}}x_{n-1}
		\right)
	+ q_nx_n = k.
\end{equation*}
Si dejamos que $\omega_{n-1}$ sea lo que está dentro de los paréntesis en el primer término, encontramos que
podemos reducir el grado de esta ecuación si, en su lugar, resolvemos el sistema de ecuaciones
\begin{equation}
	\label{eq:dioph:system}
	\begin{cases}
		g_{n-1}\omega_{n-1} + q_nx_n &= k, \\
		\frac{q_1}{g_{n-1}}x_1 + \cdots + \frac{q_{n-1}}{g_{n-1}}x_{n-1} &= \omega_{n-1}.
	\end{cases}
\end{equation}
Observemos que $\gcd\lbrace g_{n-1}, q_n \rbrace = 1$ y también $\gcd\lbrace q_1/g_{n-1}, \ldots,
q_{n-1}/g_{n-1}\rbrace = 1$. Por el Teorema \ref{prerreq:th:dioph:existence} se sigue que ambas
ecuaciones tienen una infinidad de soluciones y, por el Teorema \ref{prerreq:th:dioph:gen}, podemos
generarlas a partir de una solución particular. Como el valor de $\omega_n$ depende explícitamente
de la primera ecuación, enfoquémonos en ella primero. Consideremos la ecuación auxiliar
\begin{equation}
	\label{eq:dioph:orig:aux}
	g_{n-1}\omega_n + q_nx_n = 1.
\end{equation}
Si $(\omega_{n-1}^{(0)}, x_n^{(0)})$ es una solución particular de esta ecuación auxiliar, entonces
$(k\omega_{n-1}^{(0)}, kx_n^{(0)})$ es una solución particular de la primera ecuación de
(\ref{eq:dioph:system}). Pero $g_{n-1}$ y $q_n$ son coprimos, y entonces sus coeficientes de Bézout
asociados proveen una solución particular de esta ecuación auxiliar. Debido al Teorema
\ref{prerreq:th:dioph:gen}, obtenemos una enumeración de soluciones de la primera ecuación del sistema
(\ref{eq:dioph:system}):
\begin{equation*}
	\begin{cases}
		x_n = kx_n^{(0)} - g_{n-1}t_{n-1}, \\
		\omega_{n-1} = k\omega_{n-1}^{(0)} + q_nt_{n-1},
	\end{cases}
\end{equation*}
donde $t_{n-1} \in \mathbb{Z}$. Ahora bien, como $x_1, \ldots, x_{n - 1} \geq 0$ por la formulación
del problema (\ref{phase-1:ip:obj}), se debe cumplir $\omega_{n-1} \geq 0$, pues $p$ es un vector
con entradas no negativas. Así también, se debe cumplir $x_n \geq 0$. Es de esta forma que obtenemos
cotas inferiores y superiores para $t_{n-1}$:
\begin{equation}
	\label{eq:bounds}
	\left\lceil -k \cdot \frac{\omega_{n-1}^{(0)}}{q_n} \right\rceil
	\leq
	t_{n-1}
	\leq
	\left\lfloor k \cdot \frac{x_n^{(0)}}{g_{n-1}} \right\rfloor.
\end{equation}
Si no existe $t_{n-1} \in \mathbb{Z}$ tal que se satisfaga esta desigualdad, entonces podemos
concluir que no existen puntos factibles sobre la $k$-ésima capa entera y, por lo tanto, podemos
continuar la búsqueda de puntos factibles en la $k-1$-ésima capa entera.

En caso contrario, fijamos una $t_{n-1}$ factible, de tal forma que $\omega_{n-1}$ está
completamente determinada. Es de esta manera que ahora somos capaces de resolver la segunda ecuación
del sistema (\ref{eq:dioph:system}). Nos encontramos en una situación completamente análoga a cuando
buscábamos resolver (\ref{eq:dioph:complete:coprime}). No obstante, esta ecuación es de un grado
menor. Así pues, continuamos este proceso recursivamente hasta que debamos
resolver la ecuación en dos variables
\begin{equation*}
	ax_1 + bx_2 = \omega_2,
\end{equation*}
para algunos enteros $a, b$ coprimos. Encontramos las soluciones a partir de sus coeficientes de
Bézout asociados y también del Teorema \ref{prerreq:th:dioph:gen}. Acotamos por medio de (\ref{eq:bounds})
y, en caso de que exista una $t_2$ factible, hemos logrado obtener un punto entero factible. A
diferencia de $t_{n - 1}$, si no existe $t_i$ factible para alguna $i \in \{2, \ldots, n - 2\}$, no
podemos concluir que la $k$-ésima capa entera no contiene puntos enteros factibles. Más bien, la
elección del parámetro $t_{i + 1}$ en la iteración anterior fue incorrecta, por lo que debemos
escoger otro $t_{i + 1}$ factible, en caso de que exista.

% TODO: agregar algoritmo? Debido a lo discutido en esta subsección no es difícil mostrar que es
% correcto. Maybe agregarlo como teorema?

\subsection*{Segunda fase}
Sean $p \in \mathbb{R}^n$ un vector esencialmente entero, $A \in \mathbb{R}^{m \times n}$ una matriz
de rango completo con $n \leq m$ y $b \in \mathbb{R}^m$ un vector. Consideramos ahora el problema
lineal entero
\begin{align}
	\text{máx.} ~& p^Tx, \label{phase-2:ip:obj} \\
	\text{s. a.} ~& p^Tx \leq u, \label{phase-2:ip:prep} \\
				  & Ax \leq b, \label{phase-2:ip:rest} \\
				  & x \in \mathbb{Z}^n.
\end{align}
Observemos que este problema se reduce a \ref{phase-1:ip:obj} si $m = n$ con $A = -I_n$ y $b = 0_n$.
Seguiremos la misma lógica que en la subsección pasada en cuanto a establecer ecuaciones lineales
diofantinas. No obstante, esto lo haremos para establecer una relación lineal entre $x \in
\mathbb{Z}^n$ y el vector de parámetros $t \in \mathbb{Z}^{n-1}$. Si procedemos de aquella forma y
nos enfocamos en una capa entera $H_{p, k}$ con $k \leq u$, podremos, en primer lugar, deshacernos
de la restricción \ref{phase-2:ip:prep} y, en segundo lugar, de reducir el problema lineal
\ref{phase-2:ip:obj} a uno de factibilidad.

Sea $q$ el múltiplo coprimo de $p$. Deseamos resolver la ecuación \ref{eq:dioph:complete:coprime}.
Procedemos con la misma estrategia que en la subsección anterior, solamente que ahora vamos hacia
adelante en vez de hacia atrás. Por conveniencia, definimos $g_1 \coloneq \gcd\lbrace q_1, \ldots,
q_n \rbrace = 1$ y también $\omega_1 \coloneq k$, con lo que obtenemos
\begin{equation*}
	\frac{q_1}{g_1}x_1 + g_2
	\underbrace{\left( \frac{q_2}{g_2 \cdot g_1}x_2 + \cdots + \frac{q_n}{g_2 \cdot
	g_1}x_n \right)}_{=\vcentcolon \omega_2} = k = \omega_1.
\end{equation*}

donde $g_2 \coloneq \gcd\lbrace \frac{q_2}{g_1}, \ldots \frac{q_n}{g_1}\rbrace$. Porque $q$ es
coprimo, se sigue que $\gcd\lbrace \frac{q_1}{g_1}, g_2\rbrace = 1$ y por lo tanto la ecuación
anterior tiene solución para todo $\omega_1 \in \mathbb{Z}$. Así también, una solución particular
está constituida por los coeficientes de Bézout asociados a $\frac{q_1}{g_1}$ y $g_2$, de tal forma que la
solución general está dada por
\begin{equation*}
	\begin{cases}
		x_1 &= \omega_1 \cdot x_1^{(0)} + g_2 \cdot t_1, \\
		\omega_2 &= \omega_1 \cdot \omega_2^{(0)} - \frac{q_1}{g_1} \cdot t_1,
	\end{cases}
\end{equation*}
donde $t_1 \in \mathbb{Z}$. Ahora bien, deseamos resolver la ecuación
\begin{equation*}
	\frac{q_2}{g_2 \cdot g_1}x_2 + \frac{q_3}{g_2 \cdot g_1} + \cdots + \frac{q_n}{g_2 \cdot g_1}x_n
	= \omega_2,
\end{equation*}
por lo que definimos $g_3 \coloneq \gcd\lbrace \frac{q_3}{g_2 \cdot g_1}, \ldots \frac{q_n}{g_2
\cdot g_1} \rbrace$, de tal forma que obtenemos
\begin{equation*}
	\frac{q_2}{g_2 \cdot g_1}x_2 + g_3 \underbrace{ \left( \frac{q_3}{g_3 \cdot g_2 \cdot g_1} + \cdots +
	\frac{q_n}{g_3 \cdot g_2 \cdot g_1}x_n \right)}_{=\vcentcolon \omega_3} = \omega_2.
\end{equation*}
Por un razonamiento similar al anterior, existen soluciones para todo $\omega_2 \in \mathbb{Z}$ y
por lo tanto los coeficientes de Bézout asociados a $\frac{q_2}{g_2 \cdot g_1}$ y $g_3$ proveen una
solución particular, de donde podemos generar la solución general
\begin{equation*}
	\begin{cases}
		x_2 &= \omega_2 \cdot x_2^{(0)} + g_3 \cdot t_2, \\
		\omega_3 &= \omega_2 \cdot \omega_3^{(0)} - \frac{q_2}{g_2 \cdot g_1} \cdot t_2,
	\end{cases}
\end{equation*}
donde $t_2 \in \mathbb{Z}$. De manera inductiva, encontramos que, para $i \in \{1, \ldots, n - 2\}$,
la $i$-ésima solución general está dada por
\begin{equation}
	\begin{cases}
		\label{phase-2:recurrence}
		x_i &= \omega_i \cdot x_i^{(0)} + g_{i+1} \cdot t_i, \\
		\omega_{i+1} &= \omega_i \cdot \omega_{i+1}^{(0)} - \frac{q_i}{\prod_{j=1}^{i}g_j} \cdot t_i,
	\end{cases}
\end{equation}
con $t_i \in \mathbb{Z}$. Para $i = n - 1$ obtenemos la ecuación
\begin{equation*}
	\frac{q_{n-1}}{\prod_{j=1}^{n-2}g_j}x_{n-1} + \frac{q_n}{\prod_{j=1}^{n-2}g_j}x_n =
	\omega_{n-1},
\end{equation*}
que, por construcción, también tiene solución para todo $\omega_{n-1} \in \mathbb{Z}$. Finalmente,
las últimas dos soluciones son
\begin{equation}
	\label{phase-2:eq:last-solutions}
	\begin{cases}
		x_{n-1} &= \omega_{n-1} \cdot x_{n-1}^{(0)} + \frac{q_n}{\prod_{j=1}^{n-2}g_j} \cdot t_{n-1}, \\
		x_n &= \omega_{n-1} \cdot x_{n}^{(0)} - \frac{q_{n-1}}{\prod_{j=1}^{n-2}g_j} \cdot
		t_{n-1},
	\end{cases}
\end{equation}
donde $t_{n-1} \in \mathbb{Z}$.

Ahora bien, deseamos expresar al $x = (x_1, \ldots, x_n)$ como una transformación lineal del vector
de parámetros $t \coloneq (t_1, \ldots, t_{n-1})$. Para ello, debemos encontrar una forma cerrada a
la siguiente relación de recurrencia obtenida en \ref{phase-2:recurrence}:
\begin{equation*}
	\begin{cases}
		\omega_1 &= k, \\
		\omega_{i + 1} &= \omega_i \cdot \omega_{i + 1}^{(0)} - \frac{q_i}{\prod_{j=1}^{i}g_j} \cdot t_i.
	\end{cases}
\end{equation*}
Si ``desenvolvemos'' las igualdades, encontramos que
\begin{align}
	\omega_i &= k \cdot \prod_{j=2}^{i} \omega_j^{(0)} -
			 \sum_{j=1}^{i - 1}\frac{q_j}{\prod_{\ell=1}^{j}g_\ell} \cdot
			 \prod_{\ell=j+2}^{i}\omega_\ell^{(0)} \cdot t_j.
	% k\omega_2^{(0)}\omega_3^{(0)}\cdots\omega_i^{(0)}
	% 		 - \frac{q_1}{g_1}\omega_3^{(0)}\omega_4^{(0)} \cdots \omega_i^{(0)} t_1
	% 		 - \frac{q_2}{g_1g_2}\omega_4^{(0)}\omega_5^{(0)} \cdots \omega_i^{(0)} t_2
	% 		 - \cdots 
	% 		 - \frac{q_{i-1}}{g_1g_2 \cdots g_{i-1}}t_{i-1} \\
\end{align}
Donde, por conveniencia, asignamos a la suma vacía el valor de cero y al producto vacío el valor de
uno. Por simpleza, definimos los coeficientes $a_{ij} \in \mathbb{Z}$ con $j < i$ como
\begin{equation}
	\label{phase-2:eq:coeffs}
	a_{ij} \coloneq \frac{q_j}{\prod_{\ell = 1}^{j}g_\ell} \cdot \prod_{\ell = j +
	2}^{i}\omega_\ell^{(0)}.
\end{equation}
Así pues, juntando esto último con \ref{phase-2:recurrence}, obtenemos para $i \in \{1, \ldots, n -
2\}$, 
\begin{align}
	x_i &= w_i \cdot x_i^{(0)} + g_{i + 1}t_i \nonumber \\
		&= k \cdot \prod_{j=2}^{i}\omega_j^{(0)} \cdot x_i^{(0)} - \sum_{j=1}^{i - 1}a_{ij}x_i^{(0)}
		t_j + g_{i + 1}t_i.
\end{align}
Similarmente, sustituyendo en \ref{phase-2:eq:last-solutions},
\begin{align*}
	x_{n-1} &= k \cdot \prod_{j=2}^{n-1} \omega_j^{(0)} \cdot x_{n-1}^{(0)} - \sum_{j=1}^{n-2}
	a_{n-1,j}x_{n-1}^{(0)} t_j + \frac{q_n}{\prod_{j=1}^{n-2}g_j} t_{n-1}, \\
	x_{n} &= k \cdot \prod_{j=2}^{n-1} \omega_j^{(0)} \cdot x_{n}^{(0)} - \sum_{j=1}^{n-2}a_{n,j}x_n^{(0)}t_j -
	\frac{q_{n-1}}{\prod_{j=1}^{n-2} g_j}t_{n-1}.
\end{align*}

\subsection*{Tercera fase}

\section*{Resultados}
\section*{Conclusiones}

\bibliographystyle{alpha}
\bibliography{refs}

\end{document}
