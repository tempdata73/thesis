\documentclass[11pt]{article}

\usepackage[margin=1in]{geometry}
\usepackage{parskip}
\usepackage{amsmath, amssymb}
\usepackage{enumitem}
\usepackage{hyperref}
\usepackage{mathtools}
\usepackage{graphicx}
\usepackage{amsthm}
\usepackage[spanish]{babel}

% \graphicspath{{../figs}}

\title{Reporte}
\author{Iñaki Liendo}
% \date{\today}

\newtheorem{definition}{Definición}
\newtheorem{theorem}{Teorema}
\newtheorem{lemma}{Lema}
\newtheorem{corollary}{Corolario}

\DeclareMathOperator{\lcm}{lcm}

\begin{document}

\maketitle

\section*{1. Resumen}

Diseñé un método que encuentra todas las soluciones a problemas de la forma
\begin{align*}
	\text{máx. } & c^tx, \\
	\text{s.a. } & c^tx \leq s, \\
				 & x \in \mathbb{Z}^n,
\end{align*}
y que corre en tiempo lineal con respecto a la magnitud del lado derecho de la desigualdad pero
exponencial con respecto al número de dimensiones. Además de encontrar todas las posibles
soluciones, es más rápido que \textit{branch \& bound}. A pesar de tener complejidad lineal, los
experimentos que realicé sugieren que en la práctica tiene complejidad constante. Aquellos dos
últimos puntos contrastan significativamente con \textit{B\&B}, pues este algoritmo encuentra
solamente una solución y tiene una complejidad más que lineal con respecto a la magnitud del lado
derecho de la desigualdad (habíamos dicho que era exponencial en nuestras pláticas si mal no
recuerdo, pero los experimentos que hice indican otra cosa).

% TODO

\section*{2. Prerrequisitos}

\begin{definition}
	Dados los enteros $a$ y $b$, diremos que $a$ divide a $b$ (y escribimos $a \mid b$) si existe un
	entero $r$ tal que $a \cdot r = b$.
\end{definition}

\begin{definition}
	\label{def:gcd}
	Dados dos enteros $a$ y $b$ distintos de cero, su máximo común divisor es el entero $d$ que satisface
	\begin{enumerate}
		\item $d \mid a$ y $d \mid b$ ($d$ es un divisor común de $a$ y $b$).
		\item Si $d' \mid a$ y $d' \mid b$, entonces $d' \leq d$ ($d$ es el máximo de los divisores comúnes).
	\end{enumerate}
\end{definition}

Escribimos $d = \gcd\lbrace a, b\rbrace$ para denotar que $d$ es el máximo común divisor de $a$ y $b$. Así
también, si $a_1, a_2, \ldots, a_n$ son enteros distintos de cero, definimos su máximo común
de manera inductiva: para $n = 2$ usamos la definición ~\ref{def:gcd} y si $n > 2$ entonces
\begin{equation*}
	\gcd\lbrace a_1, a_2, \ldots, a_n\rbrace = \gcd\lbrace a_1, \gcd\lbrace a_2, \ldots,
	a_n\rbrace\rbrace.
\end{equation*}

\begin{definition}
	Sean $a, b, c \in \mathbb{Z}$. Decimos que $c$ es combinación lineal entera de $a$ y $b$ si
	existen enteros $x$ y $y$ tales que $c = ax + by$.
\end{definition}

\begin{theorem}
	Sean $a, b, d \in \mathbb{Z}\setminus\lbrace 0 \rbrace$. Entonces $d = \gcd\lbrace a, b \rbrace$
	si y solo si $d$ es la mínima combinación lineal entera positiva de $a$ y $b$.
\end{theorem}

\begin{definition}
	Sea $d$ el máximo común divisor de los enteros $a$ y $b$. Llamamos coeficientes de Bézout
	asociados a $a$ y $b$ a los enteros $x, y$ que satisfacen $d = ax + by$.
\end{definition}

Cabe mencionar que los coeficientes de Bézout no son únicos. Por ejemplo, 1 es el máximo común
divisor de 3 y 5, de tal forma que se cumple $1 = 5(2) + 3(-3) = 5(5) + 3(-8)$. Así, $(2, -3)$ y
$(5, -8)$ son dos pares de coeficientes de Bézout asociados a 3 y 5.

\begin{definition}
	Sean $a, b \in \mathbb{Z}\setminus\lbrace 0 \rbrace$. Decimos que $a$ y $b$ son coprimos si
	$\gcd\lbrace a, b \rbrace = 1$.
\end{definition}

\begin{corollary}
	\label{cor:gcd}
	Si $d = \gcd\lbrace a, b\rbrace$, entonces $\gcd\lbrace \frac{a}{d}, \frac{b}{d} \rbrace = 1$.
\end{corollary}

\begin{definition}
	Sean $a, b, c \in \mathbb{Z}$. Una pareja de enteros $(x_0, y_0)$ es una solución de la ecuación
	$ax + by = 0$ si se satisface $ax_0 + by_0 = c$.
\end{definition}

\begin{theorem}
	\label{th:dioph:existence}
	Sean $a, b \in \mathbb{Z} \setminus\lbrace 0 \rbrace$. La ecuación $ax + by = c$ tiene solución
	en los enteros si y solo si $\gcd\lbrace a, b \rbrace \mid c$.
\end{theorem}

\begin{theorem}
	\label{th:dioph:gen}
	Sean $a, b, c \in \mathbb{Z}$ tales que $a, b \neq 0$ y sea $(x_0, y_0)$ una solución particular
	de la ecuación $ax + by = c$. Entonces todas las soluciones de la ecuación están dadas por
	\begin{equation*}
		\begin{cases}
			x = x_0 + \frac{b}{d} \cdot t, \\
			y = y_0 + \frac{a}{d} \cdot t,
		\end{cases}
	\end{equation*}
	donde $d = \gcd\lbrace a, b \rbrace$ y $t \in \mathbb{Z}$.
\end{theorem}

Dada la generalización del máximo común divisor a más de dos enteros, bien es cierto que podemos
aplicar inducción para generalizar también las definiciones y resultados anteriores. Decimos, por
ejemplo, que un vector $v \in \mathbb{Z}^n \setminus \lbrace 0 \rbrace$ es coprimo si $\gcd\lbrace
v_1, \ldots, v_n \rbrace = 1$. De ser este el caso, existen coeficientes de Bézout $x_1, \ldots,
x_n \in \mathbb{Z}$ asociados a $v$ que satisfacen $v_1x_1 + \ldots + v_nx_n = 1$. Podemos decir, en
forma más compacta, que $x = (x_1, \ldots, x_n) \in \mathbb{Z}^n$ es un vector de Bézout asociado a
$v$ y por lo tanto se cumple $v^Tx = 1$.

Los siguientes teoremas, lemas y definiciones fueron tomadas de \cite{sip}.

\begin{definition}
	Sea $v \in \mathbb{R}^n$ un vector. Decimos que el hiperplano afino
	\begin{equation*}
		H_{v, t} \coloneq \ker(x \mapsto v^Tx) + t \cdot v
	\end{equation*}
	es una capa entera si contiene al menos un punto entero, es decir, si contiene un punto cuyas
	entradas son todas enteras.
\end{definition}

\begin{lemma}
	Sea $v \in \mathbb{R}^n$ un vector distinto de cero y sea $x \in \mathbb{R}^n$ un punto. Sea $t_x
	\coloneq \frac{v^Tx}{||v||_2^2}$. Entonces $x \in H_{v, t_x}$.
\end{lemma}

\begin{definition}
	Sea $v \in \mathbb{R}^n$ un vector. Decimos que $v$ es esencialmente entero si existe un vector
	$w \in \mathbb{Z}^n$ y un escalar $k \in \mathbb{R}$ tal que $v = kw$. De otra forma, decimos
	que $v$ es esencialmente irracional.
\end{definition}

Es decir, decimos que $v$ es esencialmente entero si es un múltiplo real de un vector entero. Por
las siguientes razones descartamos a los vectores esencialmente irracionales de nuestro análisis: en
primer lugar, porque la motivación de este trabajo es la asignación de recursos que tienen un costo
asociado, el cual es necesariamente racional; en segundo lugar, porque todo número representable en
cualquier sistema de aritmética finita es necesariamente racional. Independientemente del caso,
solamente son de nuestro interés los vectores racionales, pero todo vector $v$ en $\mathbb{Q}^n$ es
esencialmente entero.

\begin{definition}
	Sea $v \in \mathbb{R}^n$ un vector esencialmente entero y sea $(w_1, \ldots, w_n) \in
	\mathbb{Z}^n$ un múltiplo entero de $v$. Decimos entonces que el vector
	\begin{equation*}
		\frac{1}{\gcd\lbrace w_1, \ldots, w_n\rbrace} \cdot (w_1, \ldots, w_n)
	\end{equation*}
	es un múltiplo coprimo de $v$.
\end{definition}

Para asegurar unicidad, forzamos a que la primera entrada del múltiplo coprimo de $v$ sea positivo.
Por el Corolario \ref{cor:gcd} tenemos que el múltiplo coprimo de un vector esencialmente entero
es, en efecto, un vector coprimo.

\begin{theorem}
	Sea $v \in \mathbb{R}^n$ un vector esencialmente entero distinto de cero y sea $w$ su múltiplo
	coprimo. Entonces la familia de capas enteras $\left\lbrace H_{w, k||w||_2^{-2}} \vcentcolon k
	\in \mathbb{Z} \right\rbrace$ cubre a $\mathbb{Z}^n$.
\end{theorem}
\begin{proof}
	TODO. Reproducir demostración de \cite{sip} porque esto me ayuda a mostrar el lema
	\ref{eq:num_layers}.
\end{proof}

\section*{Problema entero simple}
Sea $p \in \mathbb{R}^n \setminus\lbrace 0 \rbrace$ un vector esencialmente entero con entradas no
negativas y sea $l \in \mathbb{R}_{\geq 0}$ un escalar. Consideremos el problema de programación
entera
\begin{align}
	\text{máx.} ~& p^Tx, \label{eq:simple:obj} \\
	\text{s. a.} ~& p^Tx \leq l, \label{eq:simple:boundary} \\
				  & x \in \mathbb{Z}_{\geq 0}^n. \nonumber
\end{align}

% TODO: explicar la estrategia/motivación para resolver este problema.

\begin{lemma}
	\label{lemma:num_layers}

	Sea $p \in \mathbb{R}^n \setminus \lbrace 0 \rbrace$ un vector esencialmente entero y sea $l \in
	\mathbb{R}_{\geq 0}$ un escalar. Entonces el número de capas enteras entre 0 y l está
	determinado por
	\begin{equation}
		\label{eq:num_layers}
		\eta \coloneq
		\left\lfloor
			l \cdot \frac{\lcm\lbrace p_1, \ldots, p_n \rbrace}{\gcd\lbrace p_1, \ldots, p_n \rbrace}
		\right\rfloor.
	\end{equation}
\end{lemma}
\begin{proof}
	TODO
\end{proof}

A causa del Lema \ref{lemma:num_layers} podemos enumerar las capas enteras que se encuentran entre
el origen y la frontera de la restricción \ref{eq:simple:boundary}. Consideremos la $k$-ésima capa
entera, donde $k \in \lbrace 0, \ldots, \eta \rbrace$. Queremos caracterizar la colección de puntos
enteros que se encuentran sobre esta capa entera. Es decir, buscamos resolver la ecuación entera
\begin{equation}
	\label{eq:dioph:complete}
	p_1x_1 + \cdots + p_nx_n = k.
\end{equation}

\begin{lemma}
	Sea $v \in \mathbb{R}^n$ un vector y sea $t \in \mathbb{R}$ un escalar. Definamos el $H_{v, t}$
	hiperplano afino como
	\begin{equation*}
		H_{v, t} \coloneq \ker(x \mapsto v^Tx) + t \cdot v.
	\end{equation*}
	Entonces $H_{v, t} = H_{r \cdot v, t}$ para todo escalar $r \in \mathbb{R}$.
\end{lemma}
\begin{proof}
	TODO (?)
\end{proof}

Sea $q$ el múltiplo coprimo de $p$. Debido al lema anterior, resolver la ecuación
(\ref{eq:dioph:complete}) es equivalente a resolver
\begin{equation}
	\label{eq:dioph:complete:coprime}
	q_1x_1 + \cdots q_nx_n = k.
\end{equation}

Definamos $g_{n-1} \coloneq \gcd\lbrace q_1, \ldots, q_{n-1}\rbrace$. Así, obtenemos la
ecuación equivalente
\begin{equation*}
	g_{n-1} \left(
		\frac{q_1}{g_{n-1}}x_1 + \cdots + \frac{q_{n-1}}{g_{n-1}}x_{n-1}
		\right)
	+ q_nx_n = k.
\end{equation*}
Si dejamos que $\omega_{n-1}$ sea lo que está dentro de los paréntesis en el primer término, encontramos que
podemos reducir el grado de esta ecuación si, en su lugar, resolvemos el sistema de ecuaciones
\begin{equation}
	\label{eq:dioph:system}
	\begin{cases}
		g_{n-1}\omega_{n-1} + q_nx_n &= k, \\
		\frac{q_1}{g_{n-1}}x_1 + \cdots + \frac{q_{n-1}}{g_{n-1}}x_{n-1} &= \omega_{n-1}.
	\end{cases}
\end{equation}
Observemos que $\gcd\lbrace g_{n-1}, q_n \rbrace = 1$ y también $\gcd\lbrace q_1/g_{n-1}, \ldots,
q_{n-1}/g_{n-1}\rbrace = 1$. Por el Teorema \ref{th:dioph:existence} se sigue que ambas ecuaciones
tienen una infinidad de soluciones y, por el Teorema \ref{th:dioph:gen}, podemos generarlas a partir
de una solución particular. Como el valor de $\omega_n$ depende explícitamente de la primera
ecuación, enfoquémonos en ella primero. Consideremos la ecuación auxiliar
\begin{equation}
	\label{eq:dioph:orig:aux}
	g_{n-1}\omega_n + q_nx_n = 1.
\end{equation}
Si $(\omega_{n-1}^{(0)}, x_n^{(0)})$ es una solución particular de esta ecuación auxiliar, entonces
$(k\omega_{n-1}^{(0)}, kx_n^{(0)})$ es una solución particular de la primera ecuación de
(\ref{eq:dioph:system}). Pero $g_{n-1}$ y $q_n$ son coprimos, por lo que sus coeficientes de Bézout
asociados proveen una solución particular de esta ecuación auxiliar. Debido al Teorema
\ref{th:dioph:gen}, obtenemos una enumeración de soluciones de la primera ecuación del sistema
(\ref{eq:dioph:system}):
\begin{equation*}
	\begin{cases}
		x_n = kx_n^{(0)} - g_{n-1}t_{n-1}, \\
		\omega_{n-1} = k\omega_{n-1}^{(0)} + q_nt_{n-1},
	\end{cases}
\end{equation*}
donde $t_{n-1} \in \mathbb{Z}$. Ahora bien, como $x_1, \ldots, x_{n - 1} \geq 0$ por la formulación
del problema (\ref{eq:simple:obj}), se debe cumplir $\omega_{n-1} \geq 0$, pues las $p$ es un vector
con entradas no negativas. Así también, se debe cumplir $x_n \geq 0$. Es de esta forma que obtenemos
cotas inferiores y superiores para $t_{n-1}$:
\begin{equation}
	\label{eq:bounds}
	\left\lceil -k \cdot \frac{\omega_{n-1}^{(0)}}{q_n} \right\rceil
	\leq
	t_{n-1}
	\leq
	\left\lfloor k \cdot \frac{x_n^{(0)}}{g_{n-1}} \right\rfloor.
\end{equation}
Si no existe $t_{n-1} \in \mathbb{Z}$ tal que se satisfaga esta desigualdad, entonces podemos
concluir que no existen puntos factibles sobre la $k$-ésima capa entera y, por lo tanto, podemos
continuar la búsqueda de puntos factibles en la $k-1$-ésima capa entera.

En caso contrario, fijamos una $t_{n-1}$ factible, de tal forma que $\omega_{n-1}$ está
completamente determinada. Es de esta manera que ahora somos capaces de resolver la segunda ecuación
del sistema (\ref{eq:dioph:system}). Nos encontramos en una situación completamente análoga a cuando
buscábamos resolver (\ref{eq:dioph:complete:coprime}). No obstante, esta ecuación es de un grado
menor. Así pues, continuamos este proceso inductivamente (TODO: recursivamente?) hasta que debamos
resolver la ecuación en dos variables
\begin{equation*}
	ax_1 + bx_2 = \omega_2,
\end{equation*}
para algunos enteros $a, b$ coprimos. Encontramos las soluciones a partir de sus coeficientes de
Bézout asociados y también del Teorema \ref{th:dioph:gen}. Acotamos por medio de (\ref{eq:bounds})
y, en caso de que exista una $t_2$ factible, hemos logrado obtener un punto entero factible. A
diferencia de $t_{n - 1}$, si no existe $t_i$ factible para alguna $i \in \{2, \ldots, n - 2\}$, no
podemos concluir que la $k$-ésima capa entera no contiene puntos enteros factibles. Más bien, la
elección del parámetro $t_{i + 1}$ en la iteración anterior fue incorrecta, por lo que debemos
escoger otro $t_{i + 1}$ factible, en caso de que exista.

Finalmente, si iniciamos la búsqueda en la $\eta$-ésima (c.f. \ref{eq:num_layers}) capa entera y
vamos descartando uno a uno las capas hasta encontrar un punto factible, se sigue inmediatamente que
ese punto es también óptimo.

% TODO: escribir el algoritmo en pseudocódigo.

Observemos que si $x^{(0)} \coloneq (x_1^{(0)}, \ldots x_n^{(0)})$ es una solución a esta ecuación
auxiliar, entonces $kx^{(0)}$ es una solución particular de (\ref{eq:dioph:complete:coprime}). Pero
$\gcd\lbrace q_1, \ldots, q_n \rbrace = 1$, con lo que $x^{(0)}$.

La idea del método es la siguiente. Para empezar, nos concentramos exclusivamente en vectores de
utilidad $c \in \mathbb{R}^n$ proyectivamente racionales. Sea $c'$ el múltiplo coprimo de $c$,
entonces el número de \textit{c-layers} entre 0 y la frontera de la restricción $p^tx \leq s$ está
dado por $\eta \coloneq \lfloor s \cdot p'_n / p_n \rfloor$.

Ahora bien, buscamos caracterizar la colección de puntos enteros que se encuentran en el $k$-ésimo
\textit{c-layer} con $k \in \{0, 1, \ldots, \eta\}$. Es decir, buscamos resolver la ecuación entera
\begin{equation*}
	p'_1x_1 + p'_2x_2 + \cdots + p'_nx_n = k.
\end{equation*}
Como $\text{gcd}(p'_1, \ldots, p'_n) = 1$, se sigue que para cualquier $k \in \mathbb{Z}$ existe una
infinidad de soluciones enteras parametrizadas por una solución inicial $(x_1^{(0)}, \ldots,
x_n^{(0)})$ y $n - 1$ parámetros $t_1, \ldots, t_{n-1}$. Para obtener la solución inicial notemos
que podemos resolver el sistema auxiliar
\begin{equation*}
	p'_1x_1 + p'_2x_2 + \cdots + p'_nx_n = 1,
\end{equation*}
donde una solución son los coeficientes de Bézout de $p'_1, \ldots, p'_n$. Luego, a esta solución la
multiplicamos por $k$ y de esta forma obtenemos $x^{(0)}$. Finalmente, para obtener una solución
factible (no negativa) ``simplemente'' acotamos los parámetros. La forma en la que acoto actualmente
los parámetros es recursivamente.

Consideremos el caso $n = 2$. La solución a la ecuación entera está dada por
\begin{equation*}
	\begin{cases}
		x_1 = kx_1^{(0)} + p'_2t, \\
		x_2 = kx_2^{(0)} - p'_1t,
	\end{cases}
\end{equation*}
donde $x_1^{(0)}$ y $x_2^{(0)}$ son los coeficientes de Bézout de $p'_1$ y $p'_2$. Como buscamos
factibilidad, se debe cumplir 
\begin{equation*}
	\frac{-k\cdot x_1^{(0)}}{p'_2} \leq t \leq \frac{k \cdot x_2^{(0)}}{p'_1}, ~~ t \in \mathbb{Z}.
\end{equation*}

Luego, para acotar los parámetros en $n$ dimensiones, definimos $w \coloneq p'_1x_1 + \ldots +
p'_{n-1}x_{n-1}$ y resolvemos el sistema $w + p'_nx_n = k$. Con esto regresamos al caso $n = 2$ y
tenemos que el parámetro $n - 1$ debe satisfacer
\begin{equation*}
	\frac{-k\cdot w^{(0)}}{p'_n} \leq t_{n-1} \leq \frac{k \cdot x_n^{(0)}}{1}, ~~ t_{n-1} \in \mathbb{Z}.
\end{equation*}
No es difícil ver que $w^{(0)} = 1$ y $x_n^{(0)} = 0$. Una vez resuelto este sistema, buscamos
resolver
\begin{equation*}
	p'_1 x_1 + \cdots + p'_{n-1}x_{n-1} = w = kw^{(0)} + p'_nt_{n-1}
\end{equation*}
para cada $t_{n-1}$ factible y repetimos el proceso anterior hasta poder acotar $t_1$.
Es de esta forma que la complejidad con respecto a la dimensión $d$ es entonces $n^d$. Notemos que
$t_i = 0$ es factible para todo $i \in \{2, 3, \ldots, n - 1\}$, así que en términos prácticos
valdría la pena resolver primero $p'_1x_1 + p'_2x_2 = k$ y checar si existe una $t_1$ factible. Si
este es el caso entonces obtendríamos inmediatamente una solución factible con $(x_1, x_2, 0,
\ldots, 0)$.

Finalmente, como nos encontramos en un problema de maximización, para obtener el óptimo empezamos
nuestra búsqueda en los \textit{c-layers} más cercanos a la frontera. Es decir, resolvemos las
ecuaciones con $k = \eta$, luego con $k = \eta - 1$, etcétera y terminamos el algoritmo una vez que
encontremos una solución factible.

\section*{3. Experimentos numéricos}
Por el momento solo he realizado experimentos cuando la dimensión $n$ es 2. En ambas figuras
\ref{fig:3d} y \ref{fig:4d} podemos notar que el método que diseñé (``diophantine'') es
relativamente constante con respecto al \textit{slack}, que denota la magnitud del lado derecho en
la desigualdad $p^tx \leq s$.

También se cumple en ambas figuras que el algoritmo de \textit{B\&B} sí tiene una dependencia
inherente con respecto a $s$: en un inicio se mantiene constante pues $s$ es demasiado pequeña y no
hay efecto real en cuanto al tiempo de convergencia; luego, el tiempo aumenta en proporción a $n$
(cabe mencionar que las gráficas están a escala logarítmica, no sé si haya sido la mejor elección
para interpretarlas) y eventualmente aumentos en $s$ no afectan al tiempo. Esto último coincide con
la presencia de múltiples soluciones enteras, lo cual parece razonable: mientras más soluciones
enteras haya, más probabilidad tiene \textit{B\&B} de encontrarse con una y por lo tanto el tiempo
de convergencia esperado es compensado por esta mayor probablidad.

\begin{figure}[h]
    \centering
    \includegraphics[width=0.7\textwidth]{../figs/cmp-expanded-three-digits.png}
	\caption{Comparación de tiempos entre el método de \textit{Branch \& Bound} y el que diseñé
	cuando los precios tienen tres cifras decimales.}
    \label{fig:3d}
\end{figure}

Una consecuencia sorprendente de lo expuesto es la siguiente: \textit{Branch \& Bound} es sumamente
sensible al redondeo numérico. En efecto, mientras más cifras decimales tengan las entradas $p_1,
\ldots, p_n$, los puntos enteros que se encuentran sobre la $k$-ésima $c$-layer están más
distanciados, por lo que el número de puntos enteros factibles se reduce drásticamente (en un orden
de 10), y entonces \textit{B\&B} tomará significativamente más tiempo en encontrar una solución.
Así pues, para compensar de nuevo el tiempo de convergencia necesitamos aumentar el \textit{slack}
$s$, pero esto afecta completamente el problema original.

\begin{figure}[h]
    \centering
    \includegraphics[width=0.7\textwidth]{../figs/cmp-expanded-four-digits.png}
	\caption{Comparación de tiempos entre el método de \textit{Branch \& Bound} y el que diseñé
	cuando los precios tienen cuatro cifras decimales.}
    \label{fig:4d}
\end{figure}


\section*{4. Siguientes pasos}

\begin{itemize}
    \item Implementar el algoritmo ``diophantine'' para más de dos dimensiones y comparar resultados
		con \textit{B\&B}.
	\item Pensar en otras alternativas para generalizar el método a más de dos dimensiones.
		Encontré algo que se llama la forma normal de Smith para resolver sistemas de ecuaciones
		enteras (ecuaciones lineales enteras en más de dos dimensiones se pueden transformar a
		sistemas de ecuaciones). Investigar.
    \item Gráficas, gráficas gráficas: tiempos vs cifras decimales, tiempos vs dimensión, gráficas
		que transmitan mejor el mensaje.
	\item Posibles aplicaciones (?) o exponerlo como casos degenerados del simplex/b\&b (?).
\end{itemize}

\section*{5. Notas}
\begin{itemize}
	\item Siento que este método puede generalizarse un poco más para agregar otro tipo de
		restricciones al problema original. Solo habría que ver cómo determinar $t_i$'s factibles.
		Esto puede verse como un problema lineal no-entero sobre $t$ y por lo tanto puede resolverse
		más rápido (?).
	\item Si el punto anterior es cierto, entonces este método es otro acercamiento para resolver
		problemas simétricos (c.f. \cite{sip}) sin hacer uso de órbitas ni mucho menos de
		transitividad y por lo tanto es un poco más general.
\end{itemize}

\bibliographystyle{alpha}
\bibliography{refs}

\end{document}
